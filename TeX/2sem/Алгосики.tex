\documentclass{book}
%nerd stuff here
\pdfminorversion=7
\pdfsuppresswarningpagegroup=1
% Languages support
\usepackage[utf8]{inputenc}
\usepackage[T2A]{fontenc}
\usepackage[english,russian]{babel}
% Some fancy symbols
\usepackage{textcomp}
\usepackage{stmaryrd}
% Math packages
\usepackage{amsmath, amssymb, amsthm, amsfonts, mathrsfs, dsfont, mathtools}
\usepackage{cancel}
% Bold math
\usepackage{bm}
% Resizing
%\usepackage[left=2cm,right=2cm,top=2cm,bottom=2cm]{geometry}
% Optional font for not math-based subjects
%\usepackage{cmbright}

\author{Коченюк Анатолий}
\title{Конспект~по~алгоритмам~и~структурам~данных\\ II семестр}

\usepackage{url}
% Fancier tables and lists
\usepackage{booktabs}
\usepackage{enumitem}
% Don't indent paragraphs, leave some space between them
\usepackage{parskip}
% Hide page number when page is empty
\usepackage{emptypage}
\usepackage{subcaption}
\usepackage{multicol}
\usepackage{xcolor}
% Some shortcuts
\newcommand\N{\ensuremath{\mathbb{N}}}
\newcommand\R{\ensuremath{\mathbb{R}}}
\newcommand\Z{\ensuremath{\mathbb{Z}}}
\renewcommand\O{\ensuremath{\emptyset}}
\newcommand\Q{\ensuremath{\mathbb{Q}}}
\renewcommand\C{\ensuremath{\mathbb{C}}}
\newcommand{\p}[1]{#1^{\prime}}
\newcommand{\pp}[1]{#1^{\prime\prime}}
% Easily typeset systems of equations (French package) [like cases, but it aligns everything]
\usepackage{systeme}
\usepackage{lipsum}
% limits are put below (optional for int)
\let\svlim\lim\def\lim{\svlim\limits}
%\let\svlim\int\def\int{\svlim\limits}
% Command for short corrections
% Usage: 1+1=\correct{3}{2}
\definecolor{correct}{HTML}{009900}
\newcommand\correct[2]{\ensuremath{\:}{\color{red}{#1}}\ensuremath{\to }{\color{correct}{#2}}\ensuremath{\:}}
\newcommand\green[1]{{\color{correct}{#1}}}
% Hide parts
\newcommand\hide[1]{}
% si unitx
\usepackage{siunitx}
\sisetup{locale = FR}
% Environments
% For box around Definition, Theorem, \ldots
\usepackage{mdframed}
\mdfsetup{skipabove=1em,skipbelow=0em}
\theoremstyle{definition}
\newmdtheoremenv[nobreak=true]{definition}{Определение}
\newmdtheoremenv[nobreak=true]{theorem}{Теорема}
\newmdtheoremenv[nobreak=true]{lemma}{Лемма}
\newmdtheoremenv[nobreak=true]{problem}{Задача}
\newmdtheoremenv[nobreak=true]{property}{Свойство}
\newmdtheoremenv[nobreak=true]{statement}{Утверждение}
\newmdtheoremenv[nobreak=true]{corollary}{Следствие}
\newtheorem*{note}{Замечание}
\newtheorem*{example}{Пример}
\renewcommand\qedsymbol{$\blacksquare$}
% Fix some spacing
% http://tex.stackexchange.com/questions/22119/how-can-i-change-the-spacing-before-theorems-with-amsthm
\makeatletter
\def\thm@space@setup{%
  \thm@preskip=\parskip \thm@postskip=0pt
}
\usepackage{xifthen}
\def\testdateparts#1{\dateparts#1\relax}
\def\dateparts#1 #2 #3 #4 #5\relax{
    \marginpar{\small\textsf{\mbox{#1 #2 #3 #5}}}
}

\def\@lecture{}%
\newcommand{\lecture}[3]{
    \ifthenelse{\isempty{#3}}{%
        \def\@lecture{Lecture #1}%
    }{%
        \def\@lecture{Lecture #1: #3}%
    }%
    \subsection*{\@lecture}
    \marginpar{\small\textsf{\mbox{#2}}}
}
% Todonotes and inline notes in fancy boxes
\usepackage{todonotes}
\usepackage{tcolorbox}

% Make boxes breakable
\tcbuselibrary{breakable}
\newenvironment{correction}{\begin{tcolorbox}[
    arc=0mm,
    colback=white,
    colframe=green!60!black,
    title=Correction,
    fonttitle=\sffamily,
    breakable
]}{\end{tcolorbox}}
% These are the fancy headers
\usepackage{fancyhdr}
\pagestyle{fancy}

% LE: left even
% RO: right odd
% CE, CO: center even, center odd
% My name for when I print my lecture notes to use for an open book exam.
% \fancyhead[LE,RO]{Gilles Castel}

\fancyhead[RO,LE]{\@lecture} % Right odd,  Left even
\fancyhead[RE,LO]{}          % Right even, Left odd

\fancyfoot[RO,LE]{\thepage}  % Right odd,something additional 1  Left even
\fancyfoot[RE,LO]{}          % Right even, Left odd
\fancyfoot[C]{\leftmark}     % Center

\usepackage{import}
\usepackage{xifthen}
\usepackage{pdfpages}
\usepackage{transparent}
\newcommand{\incfig}[1]{%
    \def\svgwidth{\columnwidth}
    \import{./figures/}{#1.pdf_tex}
}
\usepackage{tikz}
\begin{document}
    \maketitle
    \chapter{Новые структуры данных}
    \section{Дерево отрезков}

    Segment Tree, Range Tree, Interval Tree -- можно наткнутся на другие структуры

    Есть массив $a$ на  $n$ элементов
     \begin{enumerate}
         \item set(i,v) $\quad a[i] = v$
         \item sum(l,r)  $\quad \sum_{i=l}^{r-1} a[i]$
    \end{enumerate}

    $\left[ 3,5,2,1,6,4,8,3 \right] $

    Построим дерево, где в каждом узле сумма двух под ним. Каждое число это сумма чисел на каком--то отрезке.

    Заменим 4 на 7. set(5, 7). Нужно пересчитать все узлы, которые выше него. Их логарифм.

    Теперь к сумме. Сумму от  $l $ до  $r$ мы будем раскладывать на суммы, которые мы уже знаем.
     \begin{enumerate}
        \item Как найти эти суммы: обойдём наше дерево рекурсивно, не заходя в плохие поддереве. Идём вниз: если в поддереве нет нужных элементов, выходим, если полностью содержится в нужном, берём эту сумму, иначе идём дальше вниз.
        \item Мы обошли не так много (порядка логарифма) узлов. Посмотрим на узлы, в которых не сработало отсечение. Тогда наш отрезок содержит одну из границ отрезка. Отрезков, которые содержат границу (правую или левую) не более $2\log n$. 

            Запуск рекурсии: узлов, в которых не сработало отсечение -- $2\log n$, всего узлов $\approx 4\log n$.
    \end{enumerate}

    Будем хранить полное двоичное дерево. У узла $i$ дети  $2i+1$ и  $2i+2$

    \begin{verbatim}
        set(i, v, x, lx, rx):
            if rx - lx = 1:
                tree[x] = v
            else:
                m = (lx+rx)/2
                if i < m:
                    set(i,v,2x+1,lx,m)
                else:
                    set(i,v,2x+2,m, rx)
                tree[x] = tree[2x+1] + tree[2x+2]

        sum(l, r, x, lx, rx):
            if l >= rx || lx >= r:
                return 0
            if lx >= l && rx <= r:
                return tree[x]
            m = (lx+rx)/2
            s1 = sum(l, r, 2x+1, lx, m)
            s2 = sum(l, r, 2x+2, m, rx)
            return s1 + s2
    \end{verbatim}

    Пусть мы теперь хотим посчитать минимум. Делаем то же самое, только в каждом узле храним не сумму на отрезке, а минимум. Только в случае пустого дерева вместо 0 в сумме надо вернуть $+\infty $ (нейтральный элемент по операции)

    Если нужно сделать операцию $a \bigotimes b$. Если у неё есть ассоциативность  $(a\otimes b)\otimes c = a\otimes (b\otimes c)$, то её можно встроить в дерево отрезков. 

$max, +, \cdot , \&, |, \text{НОД}, \text{НОК}$ 

\subsection{Снизу вверх}

Занумеруем также, но будем идти снизу вверх и без рекурсии.

\begin{verbatim}
    set(i, v): // n = 2^k
        x = i + n - 1
        delta = v - tree[x]
        while x >= 0:
            tree[x]+=delta
            x = (x+1)/2-1

    sum(l, r):
        l = n-1+r
        r = n-2+r
        res = 0
        while r >= l:
            if l % 2 = 0:
                res += tree[l]
            l = l / 2
            if r % 2 = 1:
                res += tree[r]
            r = r/2-1
        return res
\end{verbatim}

Если нужно посчитать другую функцию, поменяется нейтральный элемент. Если не коммутативная функцию, можно сначала считать левую, потом правую, а потом их сложить.

Такая реализация чуть лучше работает как $o(\log (l-r))$. Каждый раз разница между $ l$ и $r$ уменьшается в 2 раза.  

\section{Персистентное дерево отрезков}

$\begin{array}{c}
     15\\ 8 ~ ~ ~7\\ 3~5~6~1
 \end{array}
$

set(2,8) Сделаем новый узел рядом с 6. Рядом с узлом 7 на верхнем слое приделаем ещё один узел 9. А к корню приделаем узел 17. Так у нас появилось две параллельные версия дерева отрезков.
 



\end{document}

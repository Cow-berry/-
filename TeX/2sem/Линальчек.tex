\documentclass{book}
%nerd stuff here
\pdfminorversion=7
\pdfsuppresswarningpagegroup=1
% Languages support
\usepackage{mmap}
\usepackage[utf8]{inputenc}
\usepackage[T2A]{fontenc}
\usepackage[english,russian]{babel}
% Some fancy symbols
\usepackage{textcomp}
\usepackage{stmaryrd}
% Math packages
\usepackage{amsmath, amssymb, amsthm, amsfonts, mathrsfs, dsfont, mathtools}
\usepackage{cancel}
% Bold math
\usepackage{bm}
% Resizing
%\usepackage[left=2cm,right=2cm,top=2cm,bottom=2cm]{geometry}
% Optional font for not math-based subjects
%\usepackage{cmbright}

\author{Коченюк Анатолий}
\title{Конспект по линейной алгебре\\ II семестр}

\usepackage{url}
% Fancier tables and lists
\usepackage{booktabs}
\usepackage{enumitem}
% Don't indent paragraphs, leave some space between them
\usepackage{parskip}
% Hide page number when page is empty
\usepackage{emptypage}
\usepackage{subcaption}
\usepackage{multicol}
\usepackage{xcolor}
% Some shortcuts
\newcommand\N{\ensuremath{\mathbb{N}}}
\newcommand\R{\ensuremath{\mathbb{R}}}
\newcommand\Z{\ensuremath{\mathbb{Z}}}
\renewcommand\O{\ensuremath{\emptyset}}
\newcommand\Q{\ensuremath{\mathbb{Q}}}
\renewcommand\C{\ensuremath{\mathbb{C}}}
\newcommand{\p}[1]{#1^{\prime}}
\newcommand{\pp}[1]{#1^{\prime\prime}}
\newcommand{\tl}[1]{\widetilde{#1}}
\newcommand{\ov}[1]{\overline{#1}}
\DeclareMathOperator{\rg}{rg}
\DeclareMathOperator{\Image}{Im}
\DeclareMathOperator{\Ker}{Ker}
\let\latexwedge\wedge
\def\wedge{\,\raisebox{0.5ex}{\ensuremath{\scriptscriptstyle\latexwedge}\,}}
\DeclareMathAlphabet{\mymathbb}{U}{BOONDOX-ds}{m}{n}
% Easily typeset systems of equations (French package) [like cases, but it aligns everything]
\usepackage{systeme}
\usepackage{lipsum}
% limits are put below (optional for int)
\let\svlim\lim\def\lim{\svlim\limits}
\let\svsum\sum\def\sum{\svsum\limits}
%\let\svlim\int\def\int{\svlim\limits}
% Command for short corrections
% Usage: 1+1=\correct{3}{2}
\definecolor{correct}{HTML}{009900}
\newcommand\correct[2]{\ensuremath{\:}{\color{red}{#1}}\ensuremath{\to }{\color{correct}{#2}}\ensuremath{\:}}
\newcommand\green[1]{{\color{correct}{#1}}}
% Hide parts
\newcommand\hide[1]{}
% si unitx
\usepackage{siunitx}
\sisetup{locale = FR}
% Environments
% For box around Definition, Theorem, \ldots
\usepackage{mdframed}
\mdfsetup{skipabove=1em,skipbelow=0em}
\theoremstyle{definition}
\newmdtheoremenv[nobreak=true]{definition}{Определение}
\newmdtheoremenv[nobreak=true]{theorem}{Теорема}
\newmdtheoremenv[nobreak=true]{lemma}{Лемма}
\newmdtheoremenv[nobreak=true]{problem}{Задача}
\newmdtheoremenv[nobreak=true]{property}{Свойство}
\newmdtheoremenv[nobreak=true]{statement}{Утверждение}
\newmdtheoremenv[nobreak=true]{corollary}{Следствие}
\newmdtheoremenv[nobreak=true]{question}{Вопрос}
\newtheorem*{note}{Замечание}
\newtheorem*{example}{Пример}
\renewcommand\qedsymbol{$\blacksquare$}
% Fix some spacing
% http://tex.stackexchange.com/questions/22119/how-can-i-change-the-spacing-before-theorems-with-amsthm
\makeatletter
\def\thm@space@setup{%
  \thm@preskip=\parskip \thm@postskip=0pt
}
\usepackage{xifthen}
\def\testdateparts#1{\dateparts#1\relax}
\def\dateparts#1 #2 #3 #4 #5\relax{
    \marginpar{\small\textsf{\mbox{#1 #2 #3 #5}}}
}

\def\@lecture{}%
\newcommand{\lecture}[3]{
    \ifthenelse{\isempty{#3}}{%
        \def\@lecture{Lecture #1}%
    }{%
        \def\@lecture{Lecture #1: #3}%
    }%
    \subsection*{\@lecture}
    \marginpar{\small\textsf{\mbox{#2}}}
}
% Todonotes and inline notes in fancy boxes
\usepackage{todonotes}
\usepackage{tcolorbox}

% Make boxes breakable
\tcbuselibrary{breakable}
\newenvironment{correction}{\begin{tcolorbox}[
    arc=0mm,
    colback=white,
    colframe=green!60!black,
    title=Correction,
    fonttitle=\sffamily,
    breakable
]}{\end{tcolorbox}}
% These are the fancy headers
\usepackage{fancyhdr}
\pagestyle{fancy}

% LE: left even
% RO: right odd
% CE, CO: center even, center odd
% My name for when I print my lecture notes to use for an open book exam.
% \fancyhead[LE,RO]{Gilles Castel}

\fancyhead[RO,LE]{\@lecture} % Right odd,  Left even
\fancyhead[RE,LO]{}          % Right even, Left odd

\fancyfoot[RO,LE]{\thepage}  % Right odd,something additional 1  Left even
\fancyfoot[RE,LO]{}          % Right even, Left odd
\fancyfoot[C]{\leftmark}     % Center

\usepackage{import}
\usepackage{xifthen}
\usepackage{pdfpages}
\usepackage{transparent}
\newcommand{\incfig}[1]{%
    \def\svgwidth{\columnwidth}
    \import{./figures/}{#1.pdf_tex}
}
\usepackage{tikz}

\begin{document}
    \maketitle
    \chapter{Дополнительные главы линейной алгебры}
    \section{Полилинейная формы}

    \begin{note}
        [вспомним]
        Линейное отображение $\varphi(x+\alpha y) = \varphi(x) + \alpha\cdot \varphi(y)$
    \end{note}

    \begin{definition}
        $\sphericalangle X$ -- ЛП, $\dim X = n$

         $X^*$ -- сопряжённое к  $X$ пространство.

         \underline{Полилинейной формой} (ПЛФ) называется отображение:
          \[
         U:X\times X\times \ldots \times X\times X^*\times X^*\times \ldots \times X^* \to K
         ,\] обладающее свойством линейности по каждому аргументу.

         $\sqsupset x_1, x_2, x_3, \ldots, x_p \in X\quad y^1, y^2, \ldots, y^q\in X^*$

         $u\left( x_1, x_2, \ldots,\p x_i+\alpha \pp x_i, \ldots, x_p, y^1, y^2, \ldots, y^q \right) =\\= \left(x_1, x_2, \ldots, \p x_i, \ldots, x_p, y^1, y^2, \ldots  \right) + \alpha u\left( x_1, x_2, \ldots, \pp x_i, \ldots, x_p, y^1, y^2, \ldots, y^q \right)  $
    \end{definition}

    \begin{note}
        Пара чисел $(p,q)$ называется валентностью полилинейной формы
    \end{note}

    \begin{example}
        $\R^n\quad f:\R\to K$ -- ПЛФ $(1,0)$

        $\hat x: \R^{n*} \to  K$ -- ПЛФ(0,1)

        Скалярное произведение $u(x_1, x_2) = \left( \vec x_1, \vec x_2 \right) $ -- ПЛФ(2,0)

        Смешанное произведение $w(x_1, x_2, x_3) = (\vec x_1, \vec x_2, \vec x_3)$ -- ПЛФ(3,0)
    \end{example}

    $\sqsupset u, w$ -- две полилинейные формы валентности $(p,q)$
     \begin{definition}
        \begin{enumerate}
            \item []
            \item $u=w \iff $ \[u(x_1, \ldots, x_p, y^1, \ldots, y^q) = w(x_1, \ldots, x_p, y^1, \ldots, y^q) \quad\forall x_1\ldots x_p y^1 \ldots y^1\] 
            \item Нуль форма $\Theta\quad \Theta\left( x_1, \ldots, x_p, y^1, \ldots, y^q \right)  = 0$
            \item Суммой ПЛФ валентностей $(p,q)\quad u+v$ называется такое отображение  $\omega$, что  $\omega(x_1, \ldots, x_p, y^1, \ldots, y^1) = u\left( x_1, \ldots, x_p, y^1, \ldots, y^1 \right) + v\left( x_1, \ldots, x_p, y^1, \ldots, y^1 \right)  $
                \begin{lemma}
                    $w$ -- ПЛФ  $(p,q)$

                    $w\left( \ldots ,\p x_i+\alpha \pp x_i, \ldots \right)  = w\left( \ldots ,\p x_i ,\ldots\right) +\alpha w\left( \ldots \pp x_i \ldots \right) $
                \end{lemma}
            \item Произведением полилинейной формы на число $\lambda$ называется отображение  $\lambda u$, такое что:  \[
                    (\lambda u)\left( x_1, \ldots, x_p, y^1, \ldots, y^1 \right)  = \lambda \cdot u\left( x_1, \ldots, x_p, y^1, \ldots, y^1 \right) 
            .\] 
                    \begin{lemma}
                        $\lambda u$ -- ПЛФ  $(p,q)$
                    \end{lemma}


        \end{enumerate}
    \end{definition}

    $\sqsupset \Omega_p^q$ -- множество ПЛФ $(p,q)$

     \begin{statement}
        $\Omega_p^q$ -- ЛП
    \end{statement}


    $\sqsupset \{e_j\}$ -- базис $X\quad \sqsupset \{f^k\}$ -- базис $X^*$

    $x_1 = \sum_{j_1=1}^{n} \xi_1^{j_1} e_{j_1} = \xi_1^{j_1}e_{j_1}$. Дальше значок суммы писаться не будет (иначе помрём) (соглашение о немом суммировании).

    $x_2 = \xi_2^{j_2}e_{j_2}\quad \ldots \quad x_p = \xi^{j_p}e_{j_p}$

    $y^1 = \eta_{k_1}^1f^{k_1}\quad y_2 = \eta_{k_2}^2f^{k_2}\quad \ldots \quad y^1 = \eta_{k_q}^qf^{k_q}$ 
    \begin{align*}        
    w\left( x_1, x_2, \ldots, x_p, y^1, y^2, \ldots, y^q \right) = w\left( \xi_1^{j_1}e_{j_1}, \xi_2^{j_2}e_{j_2}, \ldots, \xi_p^{j_p}e_{j_p}, \eta_{k_1}^1f^{k_1}, \eta^2_{k_2}f^{k_2}, \ldots, \eta_{k_q}^qf^{k_q} \right) \\
    = \xi_1^{k_1}\xi_2^{j_2}\ldots\xi_p^{j_p}\eta_{k_1}^1\eta_{k_2}^2\ldots\eta_{k_q}^q \underbrace{w\left( e_{j_1}, e_{j_2}, \ldots, e_{j_p}, f^{k_1}, f^{k_2}, \ldots, f^{k_q} \right) }\limits_{\omega_{j_1j_2 \ldots j_p}^{k_1 k_2 \ldots k_q} \text{ -- тензор ПЛФ}}
    \\ = \xi_1^{k_1}\xi_2^{j_2}\ldots\xi_p^{j_p}\eta_{k_1}^1\eta_{k_2}^2\ldots\eta_{k_q}^q \omega_{j_1 j_2 \ldots j_p} ^{k_1 k_2 ... k_q}
    .\end{align*}

    \begin{lemma}
        Задание полилинейной формы эквивалентно заданию её тензора в известном базисе
        \[
            w \longleftrightarrow \omega_{i_1 i_2 \ldots i_p} ^{j_1 j_2 \ldots j_q} = \omega^{\vec j}_{\vec i}
        .\] 
    \end{lemma}
    \begin{proof}
        (выше)
    \end{proof}

    \begin{lemma}
        $v \longleftrightarrow \upsilon_{\vec i}^{\vec j}\quad w \longleftrightarrow \omega_{\vec i}^{\vec j}$

        $\implies \begin{cases}
            w+v \longleftrightarrow \upsilon_{\vec i}^{\vec j} + \omega_{\vec i}^{\vec j}\\
            \alpha v \longleftrightarrow \alpha \upsilon_{\vec i}^{\vec j}
        \end{cases}$
    \end{lemma}

     \begin{note}
         $_{t_1 t_2 \ldots t_q}^{s_1 s_2 \ldots s_p}w$ -- индексация базиса $\Omega_p^q$

     $_{t_1 t_2 \ldots t_q}^{s_1 s_2 \ldots s_p}w_{i_1 i_2 \ldots i_p}^{j_1 j_2 \ldots j_q}$

     $_{t_1 t_2 \ldots t_q}^{s_1 s_2 \ldots s_p}w\left( x_1, x_2, \ldots, x_p, y^1, y^2, \ldots, y^q \right) = \xi_1^{s_1}\xi_2^{s_2}\ldots\xi_p^{s_p}\eta_{t_1}^1\eta^2_{t_2}\ldots\eta_{t_q}^q$
    \end{note}
    \begin{note}
        $\sphericalangle _{t_1 t_2 \ldots t_q}^{s_1 s_2 \ldots s_p}w_{i_1 i_2 \ldots i_p}^{j_1 j_2 \ldots j_q} = _{t_1 t_2 \ldots t_q}^{s_1 s_2 \ldots s_p}w\left( e_{i_1}, e_{i_2}, \ldots, e_{i_p}, f^{j_1}, f^{j_2}, \ldots, f^{j_q} \right) \\
        =\delta_{i_1}^{s_1}\delta_{i_2}^{s_2}\ldots\delta_{i_p}^{s_p}\delta_{t_1}^{j_1}\delta_{t_2}^{j_2}\ldots\delta_{t_q}^{j_q}$
    \end{note}

    \begin{example}
        $\R_2^2$ 
        
        $a_1 = \begin{bmatrix} 1&0\\0&0 \end{bmatrix} \quad a_2 = \begin{bmatrix} 0&1\\0&0 \end{bmatrix} \quad a_3 = \begin{bmatrix} 0&0\\1&0 \end{bmatrix} \quad a_4 = \begin{bmatrix} 0&0\\0&1 \end{bmatrix} $

        $a_1 = ^{11}a_1\quad a_2 = ^{12}a_2\quad a_3 = ^{21}a_3\quad a_4 = ^{22}a_4$
    \end{example}

    \begin{theorem}
        Набор $\left\{ _{t_1 t_2 \ldots t_q}^{s_1 s_2 \ldots s_p} W \right\} _{s_1 s_2 \ldots s_p}^{t_1t_2 \ldots t_p} $ -- образует базис в $\Omega_q^p$
    \end{theorem}
    \begin{proof}
        \begin{itemize}
            \item []
            \item [ПН] $\sphericalangle u\in \Omega_q^p$
                \begin{align*}
                    &u\left( x_1, x_2, \ldots, x_p, y^1, y^2, \ldots, y^q \right)  = \xi_1^{i_1}\xi_2^{i_2}\ldots\xi_p^{i_p}\eta_{j_1}^1\eta_{j_2}^2\ldots\eta_{j_q}^qu_{i_1 i_2 \ldots i_p}^{j_1 j_2 \ldots j_q}=\\
                    &= _{j_1 j_2 \ldots j_q}^{i_1 i_2 \ldots i_p}w\left( x_1, x_2, \ldots, x_p, y^1, y^2, \ldots, y^q  \right) u_{i_1 i_2 \ldots i_p}^{j_1 j_2 \ldots j_q}\quad \forall x_1, x_2, \ldots, x_p, y^1, y^2, \ldots, y^q
                .\end{align*}

                $\implies u = _{j_1 j_2 \ldots j_q}^{i_1 i_2 \ldots i_p}w\cdot u_{i_1 i_2 \ldots i_p}^{j_1 j_2 \ldots j_q}$

            \item [ЛНЗ] $_{j_1 j_2 \ldots j_q}^{i_1 i_2 \ldots i_p}W\alpha_{i_1 i_2 \ldots i_p}^{j_1 j_2 \ldots j_q} = \Theta$ Посчитаем на наборе $e_{s_1}, e_{s_2}, \ldots, e_{s_p}, f^{t_1}, f^{t_2}, \ldots, f^{t_q}$

                $\delta_{s_1}^{i_1}\delta_{s_2}^{i_2}\ldots\delta_{j_1}^{t_1}\delta_{j_2}^{t_2}\ldots\delta_{j_q}^{t_q}\alpha_{i_1 i_2 \ldots i_p}^{j_1 j_2 \ldots j_q} = 0$

                $\alpha_{s_1 s_2 \ldots s_p}^{t_1 t_2 \ldots t^q} = 0\quad \forall s_1 \ldots s_p t_1 \ldots t_q \implies $ ЛНЗ (альфа 0 на всех, значит она все нули)
        \end{itemize}
    \end{proof}

    \begin{note}
        Размерность пространства полилинейных форм $\dim \Omega_q^p = n^{p+q}$
    \end{note}

    \section{Симметричные и антисимметричные ПЛФ}

    $\sphericalangle \Omega_0^p\qquad u\left( x_1, x_2, \ldots, x_p \right) $ 

    $\sphericalangle \sigma$ -- перестановка чисел от 1 до $p$.  $\sigma\left( 1, 2, \ldots, p \right)  = \left( \sigma(1), \sigma(2), \ldots, \sigma(p) \right) $

    \begin{definition}
    Полилиненйая форма $u$ называется \underline{симметричной}, если \[
            u\left( x_{\sigma(1)}, x_{\sigma(2)}, \ldots, x_{\sigma(p)} \right) = u\left( x_1, x_2, \ldots, x_p \right)
        .\]  
    \end{definition}

    \begin{lemma}
        Симметричные полилинейные формы валентности $(p,0)$ образуют подпространство $\Sigma^p$ линейного пространства $\Omega_0^p$
    \end{lemma}
    \begin{proof}
        $ \sqsupset u, v\in \Sigma^p$

        $\sphericalangle (u+v)\left( x_{\sigma(1)}, x_{\sigma(2)}, \ldots, x_{\sigma(p)} \right) = u\left( x_{\sigma(1)}, x_{\sigma(2)}, \ldots, x_{\sigma(p)} \right)  + v\left( x_{\sigma(1)}, x_{\sigma(2)}, \ldots, x_{\sigma(p)} \right) =\\=u\left( x_1, x_2, \ldots, x_p \right) + v\left( x_1, x_2, \ldots, x_p \right)  = (u+v)\left( x_1, x_2, \ldots, x_p \right)  $

        Так же с умножением на число.
    \end{proof}

    \begin{definition}
        Полилинейная форма $u$ валентности $(p,0)$ называется \underline{антисимметричной}, если:
        \[
            u\left( x_{\sigma(1)}, x_{\sigma(2)}, \ldots, x_{\sigma(p)} \right) = (-1)^{[\sigma]}u(x_1, x_2, \ldots, x_p) 
        .\]
    \end{definition}
    \begin{lemma}
        Антисимметричные полилинейные формы валентности $(p,0)$ образуют подпространство $\Lambda^p$ линейного пространства  $\Omega_0^p$
    \end{lemma}


    \begin{lemma}
        Полилинейная форма $u\in \Lambda^p \iff u = 0$ при любых двух совпадающих аргументах.
    \end{lemma}
    \begin{proof}
        \begin{itemize}
            \item []
            \item [$\implies $]
        $\sqsupset u\in \Lambda^p$ и $x_i = x_j\quad i\neq j$

        $a = \sphericalangle u\left( \ldots x_i \ldots x_j \ldots \right) = - u\left( \ldots x_j \ldots x_i \ldots \right) = -a \implies  a = 0$
    \item [$\impliedby $] Известно, что если $x_i = x_{j\neq i}$, то $u\left( \ldots x_i \ldots x_j \ldots \right)  = 0\quad \forall i, j$

        Докажем, что $u$ принадлежит  $\Lambda^p$

         $x_i = x_j = \p x_i + \pp x_i$

         $u\left( \ldots x_i \ldots x_j \ldots \right)  = u\left( \ldots \p x_i + \pp x_i \ldots \p x_i + \pp x_i\ldots \right) = u\left( \ldots \p x_i \ldots \p x_i \right) + u\left( \p x_i \ldots \pp x_i \right)  + u\left( \ldots \pp x_i \ldots \p x_i \ldots \right)  + u\left( \ldots \pp x_i \ldots \pp x_i \ldots \right)  $

         Правая часть равна 0. В левой части первое и последнее слагаемые тоже нули, а значит получам 

         \[
             u\left( \ldots \p x_i \ldots \pp x_i \right)  = -u\left( \ldots \pp x_i, \ldots \p x_i \right) 
         .\] 

         
        \end{itemize}
        
    \end{proof}

    \begin{lemma}
        Полилинейная форма $u\in \Lambda^p \iff u\left( x_1, x_2, \ldots, x_p \right) =0$ лишь только $\{x_i\}_{i=1}^p$ -- ЛЗ
    \end{lemma}
    \begin{proof}
        \begin{itemize}
            \item []
            \item [$\implies$] $\sqsupset \{x_i\}_{i=1}^p$ -- ЛЗ $\implies x_k = \sum_{j\neq k}\beta^jx_j = \beta^1x_1 + \beta^2x_2 + \ldots + \beta^p x_p$

                $\sphericalangle u\left( x_1, \ldots, x_k, \ldots, x_p \right) = u\left( x_1, \ldots, \beta^1x_1+\beta^2x_2+\ldots+\beta^px_p, \ldots, x_p \right) = 0 $ 

                При раскрытии будут выносится коэффициенты и получится образ от совпадающих аргументов (хотя бы двух), который 0 по пред. лемме, а значит всё выражение, как сумма нулей, будет нулём.
            \item [$\impliedby $] $u\left( x_1, x_2, \ldots, x_p \right) = 0 $, когда $\{x_i\}_{i=1}^n$ -- ЛЗ  $\implies u\in \Lambda^p$

                $u\left( x_1, x_2, \ldots, x_p \right) = u\left( x_1 + \sum \alpha^ix_i, \ldots, x_p+\sum \alpha^ix_i \right)  = u(x_1, x_2, \ldots, x_p) + u\left( x_1, \ldots, \sum \alpha^ix_i \right) + u\left( \sum \alpha^ix_i, \ldots, x_p \right) = u\left(x_1, x_2, \ldots, x_p  \right)  + \sum_{j=2}^{p} \alpha^iu\left( x_1, \ldots, x_i \right) + \sum_{i=1}^{p-1} \alpha^iu(x_i, \ldots, x_p) $
        \end{itemize}
    \end{proof}

        \section{Практика 02.12}

        \subsection{Тензоры}

        $\omega_{i_1 i_2 \ldots i_p}^{j_1 j_2 \ldots j_q}$

        \begin{definition}
            [Соглашение об упорядочивании индексов]

            Слева направо сверху вниз: $(p,q)\quad r = p+q$ -- ранг тензора, сколько значков.

             $r = 0$ -- число  $\omega$, инвариант
            
             $r = 1$:  $a_i$ -- строчка $\begin{bmatrix} a_1&a_2&\ldots&a_n \end{bmatrix} $ $ b^j$ -- столбик  $\begin{bmatrix} b^2\\b^2\\ \vdots\\ b^n \end{bmatrix} $ 

             $r = 2$: $a_{ij}$  $b_{j}^i$ $c^{ij}$ -- первый индекс всегда строка, второй всегда столбец

             $a_{ij} \longleftrightarrow A = \begin{bmatrix} a_{11} & a_{12} & a_{13}\\ a_{21} & a_{22} & a_{23}\\ a_{31} & a_{32} & a_{33} \end{bmatrix} $ 

             $b^i_j \longleftrightarrow B = \begin{bmatrix} b_1^1 & b_2^1&b_3^1\\b_1^2 & b_2^2 & b_3^2\\ b_1^3 & b_2^3 & b_3^3 \end{bmatrix} $ 

             $r = 3:$  $a_{i j k}$  $b^i_{jk}$  $c^{ij}_k$ $d^{ijk}$

             1й -- строка, 2й -- столбец, 3й -- слой

             $a_{ijk} \longleftrightarrow A = \left[\begin{array}{c c | c c} a_{111} & a_{121} & a_{112} & a_{122}\\ a_{211} & a_{221} & a_{212} & a_{222}\end{array}\right] $ 

             \begin{example}
                 Построить тензор $\varepsilon_k^{ij} = \begin{cases}
                     -1 & (i,j,k) \text{-- чётная}\\
                     1 & (i,j,k) \text{-- чётная}\\
                 0 & (i,j,k)\text{ -- не перестановка}
                 \end{cases}$
             \end{example}

             $r = 4:$ строка, столбец, слой, сечение

             $a_{ijkl}$  $b^i_{jkl}$ $c^{ij}_{kl}$  $d^{ijk}_l$  $e^{ijkl}$ -- последний тензор типа 4,0 (число вверху, число внизу)

             $c^{ij}_{kl} \longleftrightarrow C = 
             \left[\begin{array}{c c | c c}
                 c_{11}^{11} & c_{11}^{12} & c_{12}^{11} & c_{12}^{12}\\
                 c_{11}^{21} & c_{11}^{22} & c_{12}^{21} & c_{12}^{22}\\ \hline
                 c_{21}^{11} & c_{21}^{12} & c_{22}^{11} & c_{22}^{12}\\
                 c_{21}^{21} & c_{21}^{22} & c_{22}^{21} & c_{22}^{22}\\
         \end{array}\right]$

        \end{definition}
        \begin{example}
            $c_{kl}^{ij} = \begin{cases}
                1 & i=k\neq j=l\\
            -1 & i = l \neq  j = k\\
        0 & \text{иначе}\end{cases}
            $
        \end{example}

        \subsection{Операции с тензорами}

        \begin{enumerate}
            \item Линейные операции:
                
                $\omega_{i_1 i_2 \ldots i_p}^{j_1 j_2 \ldots j_q}\qquad v_{i_1 i_2 \ldots i_p}^{j_1 j_2 \ldots j_q}$

                $u = v+\omega\quad u_{i_1 i_2 \ldots i_p}^{j_1 j_2 \ldots j_q} = (v+\omega)_{\vec i}^{\vec j} = v_{\vec i}^{\vec j} + \omega_{\vec i}^{\vec j}$ -- матричное сложение.

                $(\lambda v)_{\vec i}^{\vec j} = \lambda \cdot  v_{\vec i}^{\vec j}$
            \item Произведение:

                $u_{\vec i}^{\vec j}\quad v_{\vec s}^{\vec t}$

                $\omega = u \cdot v\quad \omega_{\vec l}^{\vec k} = u_{\vec i}^{\vec j}\cdot  v_{\vec s}^{\vec t} = \omega_{i_1 i_2 \ldots i_{p_1}s_1s_2 \ldots s_{p_2}}^{j_1 j_2 \ldots j_{q_1}t_1 t_2 \ldots t_{q_2}}$

                $\vec l = \vec j\vec t = j_1 \ldots j_{q_1}t_1 \ldots t_{q_2}$

                $\vec l = \vec i \vec s = i_1 \ldots i_{p_1}, s_1 \ldots. s_{p_2}$
        \end{enumerate}

         \begin{example}
             $a^i_j \longleftrightarrow A = \begin{bmatrix} 1 & 2\\ 3 & 4 \end{bmatrix} $

             $b_k \longleftrightarrow \begin{bmatrix} 1\\-1 \end{bmatrix} $

             $a^i_jb_k = \omega^i_{jk}$. То же самое можно записать как  $a\otimes b = \omega$

             $\omega \longleftrightarrow \left[
                 \begin{array}{c c | c c}
                     1 & 2 & -1 & -2\\ 3 & 4 & -3 & -4
                \end{array}
             \right]$

             $v = b\otimes a\quad v^i_{kj} = b_k\cdot a^i_j\longleftrightarrow V = \left[
                 \begin{array}{c c | c c}
                    1 & -1 & 2 & -2\\ 3 & -3 & 4 & -4
                \end{array}
             \right]$
        \end{example}



\begin{lemma}
    $\sqsupset \{x_{i} \}_{i=1}^p$ -- ЛЗ
\end{lemma}
\begin{proof}
    $u(x_1, x_2, \ldots, x_p) = 0$

    $u\left( \alpha x_1, x_2, \ldots, x_p \right) = 0 $ 

    $u\left( \sum_{i=1}^{p-1} \alpha^ix_i, x_2, \ldots, x_p \right)  = 0$ -- равные $x_p$ и первый аргумент
\end{proof}

$\Omega_0^p$ -- хотим делать из произвольной формы симметричную

 $\sqsupset u\in \Omega_0^p$

 \begin{definition}
     $u^{(s)}\left( x_1, x_2, \ldots, x_p \right) = \frac{1}{p!}\sum_{\sigma} u\left( x_{\sigma(1)}, x_{\sigma(2)}, \ldots, x_{\sigma_{p}} \right) $ -- симметричная форма, образованная из $u$

     $u^{(s)}$ называю \underline{симметризацией}  $u$ и пишут  \[
         u^{(s)} = Sym~u
     .\] 
 \end{definition}

 \begin{note}
     $u^{(s)}\in \Sigma^p$
 \end{note}
 \begin{proof}
     $\sqsupset \tl {\sigma}$ -- другая перестановка

     $u^{(s)} \left( x_{\tl{\sigma}(1)}, \ldots, x_{\tl{\sigma}(pa)} \right)  = \frac{1}{p!}\sum_{\sigma} \left( x_{\sigma\circ \tl{\sigma}(1)}, \ldots, x_{\sigma \circ \tl{\sigma}(p)} \right) = u^{(s)}\left( x_1, x_2, \ldots, x_p \right)  $
 \end{proof}

 \begin{note}
     Деление на $p!$ нужно, чтобы выполнялось  \[
     Sym~u = u
     .\] , если $u$ уже симметричная форма
 \end{note}

 \begin{note}
     $Sym~(\alpha u+\beta v) = \alpha Sym~u + \beta Sym~v$
 \end{note}

 \begin{definition}
     \[
     u^{(a)} = \frac{1}{p!}\sum_{\sigma}(-1)^{[\sigma]} u\left( x_{\sigma(1)}, x_{\sigma(2)}, \ldots, x_{\sigma(p)} \right) 
     .\] 

     Эта операция называется \underline{антисимметризацией} или \underline{альтернированием}

     \[
         u^{(a)} = Asym~u
     .\] 
 \end{definition}
 \begin{note}
     $u^{(a)} \in \Lambda^p$
 \end{note}
 \begin{note}
     \[
         \left( \alpha u + \beta v  \right)^{(a)}  = \alpha u^{(a)} + \beta v^{(a)}
     .\] 
 \end{note}

 \begin{note}
     $Sym~Sym  = Sym$

      $Asym~Asym = Asym$

       $Sym~Asym = 0 \quad Asym~Sym = 0$
 \end{note}

 \begin{problem}
        $\Omega_0^p$

     Найдём базис $\Lambda ^p$
 \end{problem}
 \begin{proof}
     $\sphericalangle \left\{ ^{s_1, s_2, \ldots, s_p} W \right\} _{\vec s}$ -- базис

     $\sphericalangle ^{s_1, s_2, \ldots, s_p}F = p!\cdot Asym \left( ^{s_1, s_2, \ldots, s_p}W \right) $

     \begin{lemma}
         Некоторые формы будут повторятся.

         $^{s_1 \ldots s_i \ldots s_j \ldots s_p}F = - ^{s_1 \ldots s_j \ldots s_i \ldots s_p} F$
     \end{lemma}

     \begin{proof}
        \begin{align*}
            ^{s1 ... s_i .. s_j \ldots s_p}F\left( x_1 \ldots x_i \ldots x_j \ldots x_p \right)  &= ^{s_1 \ldots s_j \ldots s_i \ldots s_p}F\left( x_1, \ldots, x_j, \ldots., x_i, \ldots x_p \right)\\
            &  = - ^{s_1 \ldots s_j \ldots s_p}\\
            &= - ^{s_1 \ldots s_j \ldots s_i \ldots s_p}F\left( x_1 \ldots x_i \ldots x_j \ldots x_p \right) 
        \end{align*}     
     \end{proof}

     \begin{note}
         Ненулевых $C_n^p$ штук
     \end{note}
     
     Упорядочивание $\left\{ ^{s_1 s_2 \ldots d_p} F\right\}_{1 \leqslant s_1 < s_2 < \ldots < s_p \leqslant n} $  -- ненулевой набор. Докажем, что он базис


 \end{proof}

 \begin{theorem}
     Набор $\left\{ ^{s_1 s_2 \ldots s_p}F \right\} _{1\leqslant s_1 < s_2 \ldots < s_p\leqslant  n}$ образует базис в $\Lambda^p$
 \end{theorem}
 \begin{proof}
     \begin{itemize}
         \item []
         \item [Полнота]
             $\sqsupset u\in \Lambda^p$

             \begin{align*}
                 u\left( x_1, x_2, \ldots, x_p \right) &= \xi_1^{i_1}\xi_2^{i_2}\ldots\xi_p^{i_p}u_{i_1 i_2 \ldots i_p} \\
                                                       &= ^{i_1 i_2 \ldots i_p}W\left( x_1, x_2, \ldots, x_p \right) u\left( i_1 i_2 \ldots i_p \right)  \\
                 \text{То же самое:}\quad u &= ^{i_1 i_2 \ldots i_p}W\cdot u_{i_1 i_2 \ldots i_p} \\
             .\end{align*}

             \begin{align*}
                 Asym ~ u &= Asym \left( ^{i_1 i_2 \ldots i_p}W\cdot u_{i_1 i_2 \ldots i_p} \right)  \\
                 u &= Asym\left( ^{i_1 i_2 \ldots i_p} W \right)\cdot u_{i_1 i_2 \ldots i_p}  \\
                   &= \frac{1}{p!} ~ ^{i_1 i_2 \ldots i_p}F\cdot u_{i_1 i_2 \ldots i_p} \\
                   &= \frac{1}{p!}\sum_{1\leqslant  i_1 < i_2 < \ldots < i_p \leqslant  n}\sum_{\sigma}~ ^{\sigma(i_1)\sigma(i_2) \ldots \sigma(i_n)}F \cdot u_{\sigma(i_1)\sigma(i_2) \ldots. \sigma(i_p)} \\
                   &= \frac{1}{p!}\sum_{1\leqslant i_1 < \ldots < i_p \leqslant n} \sum_{\sigma} (-1)^{[\sigma]} {}^{i_1 i_2 \ldots i_p} F \cdot  (-1)^{[\sigma]}u_{i_1 i_2 \ldots i_p} \\
                   &= \frac{1}{p!} \sum_{1 \leqslant  i_1 < \ldots < n} p! ^{i_1 i_2 \ldots i_p}F u_{i_1 i_2 \ldots i_p}\\
             .\end{align*}

             \begin{lemma}
                 $u\in \Lambda ^p \implies  \forall \sigma u_{\sigma(i_1)\sigma(i_2) \ldots \sigma(i_p) = (-1)^{[\sigma]}}u_{i_1 i_2 \ldots i_p}$

                 Тензоры это значение $u$ на  $e_{i_1} \ldots e_{i_p}$. А тогда оно выполняется просто по определению антисимметричной формы
             \end{lemma}
         \item[Линейная независимость]
             $\sphericalangle \alpha_{i_1 i_2 \ldots i_p}\quad ^{i_1 i_2 \ldots i_p}F\alpha_{i_1 i_2 \ldots i_p} = \mathbb{•}{0}$. Подействуем на $e_{s_1} e_{s_2} \ldots e_{s_p}$

             $^{i_1 i_2 \ldots i_p}F\left( e_{s_1}, e_{s_2}, \ldots, s_{s_p}  \right) \alpha_{i_1 i_2  \ldots i_p} = 0$

             $p!\left[Asym^{i_1 i_2 \ldots i_p}\right]\left( e_{s_1} e_{s_2} \ldots s_{s_p} \right) \alpha_{i_1 i_2 \ldots i_p} = 0  $ 

             $p!\cdot \frac{1}{p!}\sum_{\sigma} (-1)^{[\sigma]}{}^{i_1 i_2 \ldots i_p}W\left( e_{\sigma(s_1)}, e_{\sigma(s_2)}, \ldots, e_{\sigma(s_p)} \right) \alpha_{i_1 i_2 \ldots i_p}  = 0$ 

                 $\sum_{\sigma} (-1)^{[\sigma]}\delta_{\sigma(s_1)}^{i_1}\delta_{\sigma(s_2)}^{i_2} \ldots \delta_{\sigma(s_p)}^{i_p} \alpha_{i_1 i_2 \ldots i_p = 0}$ 

                 $\sum_{\sigma}(-1)^{[\sigma]}\alpha_{\sigma(s_1)\sigma(s_2) \ldots \sigma(s_2)} = 0$

                 $p!\alpha_{s_1 s_2 \ldots s_p} = 0 \forall  s_1 s_2 \ldots s_p \implies  \alpha = 0$, если $\alpha$ антисимметричный тензор

     \end{itemize}
 \end{proof}

 \begin{note}
     $\dim \Lambda^p = C_n^p$

      \begin{enumerate}
         \item $p = 0 \implies  C_n^0 = 1 \implies  K$
         \item $p = 1 \implies  C_n^1 = n \implies  X^*$
         \item $p = 2 \implies  C_n^2 = \frac{n(n-1)}{2}$ 
             \\ \hrule
          $C_n^p = C_n^{n-p}$ 
      \item[n] $p = n-1 \implies  C_n^{n-1} = C_n^1 = n$
      \item[n+1] $C_n^n = 1$ 

          $\sphericalangle \Lambda^n$ 

          $\left\{^{i_1 i_2 \ldots i_n}F\right\}_{1\leqslant i_1 < i_2 < \ldots < i _n \leqslant  n} = \left\{ ^{123 \ldots n}F \right\} $ 

          $\sqsupset u\in \Lambda^n \implies  \exists \alpha \quad u = \alpha^{1 2 3 \ldots n}F$

        \begin{align*}
        &\sphericalangle ^{1 2 3\ldots n}F\left( x_1, x_2, \ldots, x_n \right)  =\\
        &=p!\cdot \left[Asym^{1 2 3\ldots n}W\right](x_1, x_2, \ldots, x_n) =\\
        &=\sum_{\sigma} (-1)^{[\sigma]_{1 2 3 \ldots n}}W\left( x_{\sigma(1)} x_{\sigma(2)} \ldots x_{\sigma(n)} \right)  =\\
        &=(-1)^{[\sigma]} \xi_{\sigma(1)}^{1} \xi_{\sigma(2)}^{2} \ldots \xi_{\sigma(n)}^{n} \overset{\triangle}=\\
        &=\det\{x_i\}\\
        \end{align*}


          \begin{lemma}
              $\forall u\in \Lambda^n\quad u = \alpha\left( ^{123 \ldots n}F \right) $
          \end{lemma}
          \begin{proof}
              \begin{align*}
                  u\left( x_1, x_2, \ldots, x_n \right) &=  \xi_1^{i_1}\xi_2^{i_2} \ldots \xi_n^{i_n} u_{i_1 i_2 \ldots i_n}\\
                  &= ^{i_1, i_2, \ldots, i_n}W\left( x_1, x_2, \ldots, x_n \right) u_{i_1 i_2\ldots, i_n}\\
                  & = ^{1 2 3 \ldots n}F(x_1, x_2, \ldots, x_n)\underbrace{u_{1 2 \ldots n}}\limits_{\alpha}
              \end{align*}                        
          \end{proof}
  \end{enumerate}

  \section{Произведение полилинейных форм}

$\sqsupset \Omega_p^q$

\begin{definition}
    $u\in \Omega_{q_1}^{p_1}, v\in \Omega_{q_2}^{p_2}$

    $\sphericalangle \omega \left( x_1, x_2, \ldots, x_{p_1}, x_{p_1+1}, \ldots, x_{p_1+p_2}, y^1, \ldots, y^{q_1}, y^{q_1+1}, \ldots, y^{q_1+q_2} \right) =\\ = u\left( x_1, x_2, \ldots, x_{p_1}, y^1, y^2, \ldots, y^{q_1} \right)  \cdot v\left( x_{p_1+1}, x_{p_1+2}, \ldots, x_{p_1+p_2}, y^{q_1+1}, \ldots, y^{q_1+q_2} \right) $ 

    Такая форма называется консолидированной формой $u$ и  $v$

    $u^{i_1 i_2 \ldots i_{p_1}}_{j_1 j_2 \ldots j_{q_1}} \cdot v_{t_1 t_2 \ldots t_{q_2}}^{s_1 s_2 \ldots s_{p_2}} = \omega^{i_1 \ldots i_p, s_1, \ldots, s_{p_2}}_{j_1 \ldots j_{q_1} t_1, \ldots, t_{q_2}}$
\end{definition}

\begin{note}
    $\omega$ -- ПЛФ  $(p_1+p_2, q^1+ q^2)$

    $\omega = u\cdot v \subseteq \Omega_{q_1+q_2}^{p_1+p_1}$

    $\sphericalangle \Omega = \dotplus \sum_{i, j} \Omega_{q_j}^{p_i}$ -- линейное пространство

    $\left( \Omega, +, \cdot \lambda, \cdot  \right) $ Новое умножение называется внешним
\end{note}

\begin{property}
    \begin{enumerate}
        \item $u\cdot (v\cdot w) = (u*v)\cdot w$
        \item $u\cdot v \neq  v \cdot  u$
        \item $u\cdot (v+w) = u\cdot v+u\cdot w$
        \item $\mymathbb{0}\quad u\cdot \mymathbb{0} = \mymathbb{0}$ -- получившийся нооль из бОльшего пространства
        \item $u(\alpha v) = (\alpha u)\cdot v$
    \end{enumerate}
\end{property}

\begin{definition}
    $\Omega$ -- внешняя алгебра полилинейных форм
\end{definition}
 \end{note}

 \section{Практика №2}
 \subsection{Свёртки}
\begin{example}
    $\omega^{j}_i \sim \begin{pmatrix} 1 & 2 & 3 \\ 8 & 7 & 5\\ 1 & -1 & 9 \end{pmatrix} $

    $w_i^i = \sum_i = \omega_1^1 +\omega_2^2 + \omega_3^3$
\end{example}

\begin{example}
    $w^{ij}_k \sim  \left( 
        \begin{array}{c c| c c}
            1 & 2 & 8 & 9\\ 5 & -1 & 10 & 3
        \end{array}
    \right) $

    $w_i^{ij} = \alpha^j\qquad \alpha^0 = 1 + 10 = 11\quad \alpha^1 = 2+(-3) = -1$

    $\omega^{ij}_i = \begin{pmatrix} 11\\-1 \end{pmatrix} $
\end{example}

\begin{example}
    $\omega^{ij}_{kl} \sim \left( 
        \begin{array}{c c| c c}
            3 & -1 & 4 & 7\\ -8 & 1 & -3 & 11\\ \hline -3 & 4 & 13 & 17\\ 6 & 5 & 19 & 23\\
        \end{array}
    \right) $

    $\omega_{ki}^{ij} = \alpha_k^j \sim \begin{pmatrix} 0 & 10\\ 16 & 27 \end{pmatrix} $

    $\omega^{ij}_{ji} = \sum_j\sum_i\omega^{ij}_{ji} = \sum_k \alpha_k^k = \alpha_0^0 + \alpha_1^1 = 27$
\end{example}

\begin{note}
    Сложную свёртку можно считать как последовательность единичных
\end{note}

\subsection{Транспонирование}

$\omega^i_{jk} = \left( 
    \begin{array}{c c c| c c c| c c c}
        1 & 2 & 3 & 7 & 8 & 9 & 4 & 5 & 6\\
        -3 & -2 & -1 & 9 & 8 & 7 & -7 & 11 & -13\\
        2 & 19 & 17 & 14 & 12 & 9 & 21 & 17 & -1\\
    \end{array}
\right) $

$\psi_{jk}^i = \omega_{kl}^i$

$\psi_{jk}^{i} \sim \left( 
    \begin{array}{c c c | c c c| c c c}
        1 & 7 & 4 & 2 & 8 & 5 & 3 & 9 & 6\\
        -3 & 9 & -7 & -2 & 8 & 11 & -1 & 7 & -13\\
        2 & 14 & 21 & 19 & 12 & 17 & 17 & 9 & -1\\
\end{array}
\right) $ 

$\omega^{ij}_{kl} \sim \left( 
    \begin{array}{c c | c c }
        14 & 1 & 1 & 3\\
        -8 & 2 & -7 & -1\\
        18 & 16 & 9 & 11\\
        -14 & -3 & 17 & 19\\
\end{array}
\right) $ 


$\omega^{ij}_{kl} \sim \left( 
    \begin{array}{c c | c c}
        14 & & 18 & \\
           &&& \\
        1 & 3 & 9 &\\
          &&17&\\
\end{array}
\right) $


\subsection{Свёртка и тензорное произведение}

$a^{ij} \sim \begin{pmatrix} 8 & 9 & 1\\7 & -5 & 4\\ 1 & 1 & 1\end{pmatrix} $ 

$b^k_l \sim \begin{pmatrix} 7 & -1 ^ -3\\ 8 & 4 & 5\\ 11 & -9 & 1 \end{pmatrix} $ 

$a \otimes b = \omega^{ijk}_l \implies  \omega^{ijk}_j = \beta^{ik}$

$\beta^{ik} = a^{ij}b^k_l = \begin{pmatrix} 44 & & \\ & 56 & \\ & & &\\ \end{pmatrix} $

$\beta^{00} = a^{00}b^0_0 + a^{01}b^0_1 + a^{02}b^0_2 = $


$a^{ij}_k \sim \left( 
    \begin{array}{c c | c c}
     2 & 1 & -1 & -7\\
     2 & 2 & 8 & 11\\
\end{array}
\right) $ 

$b_{m, n} \sim \begin{pmatrix} 3 & -3\\ 8 & -1 \end{pmatrix} $ 

$a\otimes b = \omega^{ij}_{kmn} \implies  \omega^{ij}_{kji} = \beta_k \sim \begin{pmatrix} 9\\-94 \end{pmatrix} $

$\omega\in \Omega_0^2\quad \omega(x, y)\in \R\quad x, y\in X$

$\omega \sim  a_{ij}\quad x\sim \xi^k\quad y \sim \eta^l$

$\omega(x, y) = a_{ij}\xi^i\eta^j = (a\otimes x \otimes y)^{ij}_{ij} $

\subsection{Симметризация и асимметризация тензоров}

$\omega_{ij} \sim \begin{pmatrix} 1 & 3 & 7\\-1 & 8 & 4\\ 3 & 2 & -1 \end{pmatrix} $

$sym(\omega_{i_1 , \ldots, i_p}) = a_{j_1 \ldots j_p} = \sum_{\sigma}\omega_{i_{\sigma(1)}, \ldots, i_{\sigma(p)}}$

$a_{ij} = w_{(ij)} \sim \begin{pmatrix} 1 & 1 & 5\\ 1& 8 & 3\\5 &3 & -1\\ \end{pmatrix} $ 

$a_{ij} = \frac{1}{2!}(\omega_{ij} + \omega_{ji})$

$\omega_{ijk} \sim \left( 
    \begin{array}{c c c | c c c| c c c}
        1 & 2 & 3 & -7 & 8 & 1 & 9 & 3 & 13\\
        3 & -4 & 5 & 11 & -7 & 13 & 1 & 4 & 2\\
        8 & 9 & 7 & 5 & 6 & 11 & -7 & 8 & -1\\
\end{array}
\right) $


$a_{ijk} = \frac{1}{6}\left( \ldots \right) $ 

$a_{ijk} \sim \left( 
    \begin{array} {c c c| c c c| c c c}
        1 & -\frac{2}{3} & \frac{20}{3} & -\frac{2}{3} & 5 & 4 & \frac{20}{3} &4 & \frac{13}{3}\\
        -\frac{2}{3} & 5 & 4 & 5 & -7 & \frac{23}{3} & 4& \frac{23}{3} & 7 \\
        \frac{20}{3} &4 & \frac{13}{3} &4 &\frac{23}{3} &7& \frac{13}{3} &7 & -1\\
    
\end{array}
\right) $

\section{Свойства произведения полилинейных форм}

\begin{enumerate}
    \item $u\cdot v\cdot w = u\cdot \left( v\cdot w \right)  = \left( u\cdot v \right) \cdot w$
    \item $u\cdot v\neq v\cdot u$
        \begin{example}
            [Конртпример] $\sqsupset u = f^1\quad v = f^2\quad u, v\in \Omega^1_0$

            $\left( u\cdot v \right) \left( x_1, x_2 \right) =\left( f^1f^2 \right) \left( x_1, x_2 \right) = f^1(x_1)\cdot f^2(x_2) \neq f^2(x_1)\cdot f^1(x_2) = \left( f^2f^1 \right) (x_1, x_2) = \left( v\cdot u \right) \left( x_1, x_2 \right) $
        \end{example}
    \item $\forall p, q\exists \mymathbb{0}\in\Omega_q^p \forall u\in \Omega^{p_1}_{q_1}\quad u\cdot \mymathbb{0} = \mymathbb{0}\cdot u = \mymathbb{0}\in\Omega_{q_1+q}^{p_1+p}$
    \item $u\cdot \left( v+w \right)  = u\cdot v+u\cdot w$
    \item $u\left( \alpha v \right)  = \left( \alpha u \right) v = \alpha\left( uv \right) $
    \item $\exists \left\{ f^k \right\} _{k=1}^n$ -- базис $X^*$

        $\sqsupset \left\{ ^{s_1 s_2 \ldots s_p}W \right\} $ -- базис $\Omega_0^p$

        $^{s_1s_2\ldots s_p}W = f^{s_1}f^{s_2}\cdot \ldots\cdot f^{s_p}$
        \begin{proof}
            $\sqsupset x_1, x_2, \ldots, x_p\in X$

            \begin{align*}
                &^{s_1s_2 \ldots s_p}W\left( x_1, x_2, \ldots, x_p \right)  = \xi_1^{s_1}\xi_2^{s_2}\ldots\xi_p^{s_p}\\
                &= f^{s_1}(x_1)\cdot f^{s_2}(x_2)\ldots\cdot f^{s_p}(x_p) \\
                &= f^{s_1}f^{s_2}\ldots f^{s_p}\left( x_1, x_2, \ldots, x_p \right)  \\
           .\end{align*}
        \end{proof}
    \begin{note}
        $\left\{ ^{s_1 s_2 \ldots s_p}_{t_1 t_2 \ldots t_q}W \right\} $ -- базис $\Omega^p_q$

        $^{s_1 s_2 \ldots s_p}_{t_1 t_2 \ldots t_q}W = f^{s_1}f^{s_2}\ldots f^{s_p}\hat e_{t_1}\hat e_{t_2} \ldots \hat e_{t_q}$

        $\hat e_y(y^k) = y^k(e_t)$
    \end{note}
\item $Sym\left( u\cdot v \right)  = Sym\left( Sym~u \cdot  v \right)  = Sym\left( u\cdot Sym~v \right) $ 

$Asym\left( u\cdot v \right)  = Asym\left( Sym~u \cdot  v \right)  = Asym\left( u\cdot Asym~v \right) $ 
\begin{statement}
    $Asym\left( u\cdot v \right)  = Asym\left( Asym~u\cdot v \right) $
\end{statement}
\begin{proof}
    \begin{align*}
        Asym\left( Asym~u\cdot v \right) \\
        &= Asym\left\{ \frac{1}{p!}\sum_{\sigma} (-1)^{\left[ \sigma \right] }u\left( x_{\sigma(1)}x_{\sigma(2)}\ldots x_{\sigma(p)} \right)  \cdot v\left( x_{p+1}\ldots x_{p+q} \right)\right\}   \\
        &= \frac{1}{p!}\sum_{\sigma}(-1)^{\left[ \sigma \right] }Asym\left[ \underbrace{u\left( x_{\sigma(1)}\ldots x_{\sigma(p)} \right) \cdot v\left( x_{p+1}\ldots x_{p+q} \right)}\limits_{w\left( x_{\sigma(1)\ldots x_{\sigma(p) x_{p+1} .. x_{p+q}}} \right) }  \right]  \\
        &= \frac{1}{p!} \sum_{\sigma}\cancel{(-1)^{\left[ \sigma \right] }} \cancel{(-1)^{\left[ \sigma \right] }}Asym~w\left( x_1, x_2, \ldots, x_, x_{p+1} \right) \\
        &= \frac{1}{p!}p!Asym~w\left( x_1, x_2, \ldots, x_{p}, x_{p+1}, \ldots, x_{p+1} \right)  \\
        &= Asym~\left( u\cdot v \right) \left( x_1 \ldots x_{p+q} \right)  \\
    .\end{align*}
\end{proof}

\end{enumerate}


\section{Внешнее произведение ассиметричных полилинейных форм}

$\sqsupset u\in \Lambda^{p_1}\quad v\in \Lambda^{p_2}$

$\sphericalangle u\cdot \not\in \Lambda^{p_1+p_2} $

$\sphericalangle u\wedge v = \frac{(p_1+p_2)!}{p_1!p_2!}Asym\left( u\cdot v \right) $

\begin{align*}    
    Asym\left( u\wedge v \right)  &= Asym~\left( \frac{\left( p_1+p_2 \right) !}{p_1!p_2!}~Asym\left( u\cdot v \right)  \right) \\
                                  &= \frac{\left( p_1+p_2 \right) !}{p_1!p_2!}\cdot \frac{1}{\left( p_1+p_2 \right) !}~ \sum_{\sigma}\left( -1)^{\left[ \sigma \right] } \right) u\left( x_{\sigma(1)} \ldots x_{\sigma(p_1)} \right) v\left( x_{\sigma(p_1+1} \ldots x_{\sigma\left( p_1+p_2 \right) } \right)  \\
                                  &= \sum_{\p \sigma} \left( -1 \right) ^{\left[ \sigma \right] } u\left( x_{\p \sigma(1)} \ldots x_{\p \sigma_{p_1}} \right) \cdot v\left( x_{\sigma\left( p_1+1 \right) }\ldots x_{\p\sigma\left( p_1+p_2 \right) } \right)  \\
.\end{align*}

$\p\sigma(j)>p_1\quad j\leqslant p_1$

$\p\sigma(j)\quad j>p_1$


Свойства операции $u\wedge v$:
 \begin{enumerate}
     \item  $u\wedge\left( v+w \right)  = u\wedge v + u\wedge w$
     \item $(\alpha u)\wedge v = u\wedge (\alpha v) = \alpha \left( u\wedge v \right) $
     \item $u\wedge v \wedge w = u\wedge (v\wedge w) = (u\wedge v)\wedge w$
      \begin{note}
          \begin{align*}
              (u\wedge v)\wedge w &= \frac{(p_1+p_2)!}{p_1!p_2!}Asym~(u\cdot v) \wedge w\\
                                  &= \frac{\cancel{(p_1+p_2)!}}{p_1!p_2!}\frac{\left( p_1+p_2 +p_3\right) !}{\cancel{(p_1+p_2)!}p_3!}Asym~\left( Asym\left( u\cdot v)w \right)  \right)  \\
                                  &= \frac{\left( p_1+p_2+p_3 \right) !}{p_1!p_2!p_3!}Asym\left( u\cdot v\cdot w \right) \\
                                  &= u\wedge v\wedge w \\
         .\end{align*}
     \end{note}
 \item $u\wedge v \overset ? = v\wedge u$

      \begin{note}
          $u\wedge v = (-1)^{p_1p_2}v\wedge u$
     \end{note}
     \begin{proof}
            \begin{align*}                
                &\sphericalangle \left[u\wedge v\right]\left( x_1, x_2, .., x_{p_1}, x_{p_1+1}, \ldots, x_{p_1+p_2} \right)=\\
                 &=\frac{(p_1+p_2)!}{p_1!p_2!}\frac{1}{(p_1+p_2)!}\sum_{\sigma}(-1)^{\left[ \sigma \right] }u\left( x_{\sigma(1)} \ldots x_{\sigma(p_1)} \right) v\left( x_{\sigma(p_1+1)} \ldots x_{\sigma(p_1+p_2)}  \right)  =\\
                                                                                                                          &= \frac{(p_1+p_2)!}{p_1!p_2!}\frac{(-1)^{p_1p_2}}{(p_1+p_2)!}\sum_{\sigma}(-1)^{\left[ \sigma \right] } v\left( x_{\sigma(p_1)} \ldots x_{\sigma(p_1+p_2)} \right) \cdot  u\left( x_{\sigma(1)}, \ldots, x_{\sigma(p_1)} \right)    =\\
                                                                                                                          &= (-1)^{p_1p_2}\left( v\wedge u \right) \left( x_1, \ldots, x_{p_1+p_2} \right)  \\
            .\end{align*}
     \end{proof}    
     \begin{note}
         $\sqsupset f, g\in \Lambda^1\quad f\wedge g = -g\wedge f$
     \end{note}
 \item $\exists \mymathbb{0}\in\Lambda^p \quad u\wedge u\in \Lambda^p = u\wedge \mymathbb{0} = \mymathbb{0}\in\Lambda^{p+q}$
 \item $u\in\Lambda^p\quad v\in \Lambda^q$

     $\sqsupset p+q>n = \dim X \implies u\wedge v = \mymathbb{0}$
 \item $\left\{ f^k \right\} _{k=1}^n$ -- базис $X^*$

     $\sqsupset \left\{ ^{s_1s_2\ldots s_p}F \right\} $ -- базис $\Lambda^p$

     $\implies ^{s_1s_2\ldots s_p} = f^{s_1}\wedge f^{s_2}\wedge \ldots \wedge f^{s_p}$
     \begin{proof}
        \begin{align*}            
         ^{s_1 s_2  \ldots s_p}F &=   p!Asym\left( ^{s_1s_2 \ldots s_p}W \right) \\ 
         &=   p!Asym\left( f^{s_1}f^{s_2} .. f^{s_p} \right)\\
         &=   p!Asym\left(Asym( f^{s_1}f^{s_2}) .. f^{s_p} \right)\\
         &=   p!Asym \frac{1}{2!}\left( f^{s_1}\wedge f^{s_2}\cdot  .. \cdot f^{s_p} \right)\\
         &=   p!Asym \frac{1}{2!}\left( Asym(f^{s_1}\wedge f^{s_2}\cdot f^{s_3})\cdot   .. \cdot f^{s_p} \right)\\
         &=   p!Asym \frac{1}{3!}\left( f^{s_1}\wedge f^{s_2}\cdot f^{s_3}\cdot   .. \cdot f^{s_p} \right)\\
         &= \ldots \\
         &= f^{s_1}\wedge f^{s_2} \ldots f^{s_p} \\
        .\end{align*}

     \end{proof}
 \item $\sphericalangle \Lambda^n\quad \dim \Lambda^n = 1$

     $^{123\ldots n}F$ -- единственный базисный элемент
     
     $\implies ^{123 \ldots n} = f^1\wedge f^2 \wedge \ldots \wedge f^n$

     $\sphericalangle ^{12 \ldots n}F(x_1, x_2, \ldots, x_n = n! Asym\left[ ^{12 \ldots n}W \right] \left( x_1, x_2, \ldots, x_n \right) = n!\frac{1}{n!}\sum_{\sigma}(-1)^{\left[ \sigma \right] } {}^{12 \ldots n}W\left( x_{\sigma(1) \ldots x_{\sigma(2)} \ldots x_{\sigma(n)}} \right) = \sum_{\sigma}(-1)^{\left[ \sigma \right] } \xi^1_{\sigma(1)}\xi^2_{\sigma(2)} \ldots \xi^n_{\sigma(n)}\triangleq \det\{x_1, x_2, \ldots, x_n\} = \sum_{\sigma} (-1)^{\left[ \sigma \right] } f^1\left( x_{\sigma(1)} \right) f^2\left( x_{\sigma(2)} \right) f^n\left( x_{\sigma(n)} \right)  = \left( f^1\wedge f^2 \wedge \ldots \wedge f^n \right) \left( x_1, x_2, ... x_n \right)  $
\end{enumerate}

\section{Основы теории определителей}

$\sqsupset X$ -- ЛП $\dim X = n$

$\left\{ e_j \right\} _{j=1}^n$ --базис X \quad $\left\{ f^k \right\} _{k=1}^n$ -- базис $X^*$

$\left\{ x_i \right\} _{i=1}^n$ -- набор векторов из X

\begin{definition}
    \begin{align*}
        \det\{x_1, x_2, \ldots, x_n\} &= ^{12\ldots n}F\left( x_1, x_2, \ldots, x_n \right)  \\
                                      &= \left( f^1\wedge f^2\wedge \ldots \wedge f^n \right) \left( x_1, x_2, \ldots, x_n \right)  \\
                                      &= \left< x_k = \sum_{j=1}^{n} \xi_k^je_j\right> \\
                                      &= \sum_{\sigma}(-1)^{\left[ \sigma \right] }\xi^1_{\sigma(1)}\xi^2_{\sigma(2)} \ldots \xi^k_{\sigma(k)} \\
    .\end{align*}
\end{definition}

\begin{definition}
    $k$-мерным параллелепипедом в $X$, построенном на векторах $\{x_i\}_{i=1}^k$ называется следующее множество:
    \[
        T = \{\sum_{i=1}^{k} \alpha^ix_i, \quad \alpha^i\in [0,1]\}
    .\] 
\end{definition}

\begin{definition}
    Формой объёма в ЛП $X$ называется отображение  $\omega^i{(n)}$ удовлетворяющееся следующим свойствам:
    \begin{enumerate}
        \item $\omega^{(n)}(T_n)$ -- число  $ = \omega^{(n)}\left( x_1, x_2, \ldots, x_n \right) $ 
        \item $\omega^{(n)}$ -- полилинейное отображение
        \item  $\omega^{(n)}$ -- ассиметричное отображение
    \end{enumerate}
    
\end{definition}

\begin{lemma}
    $\omega^{(n)}\in\Lambda^n \iff \omega^{(n)} = \alpha\cdot {}^{12\ldots n}F$
\end{lemma}

\section{Практика}

\begin{problem}
    $a_{ijk} \longrightarrow \left[ 
        \begin{array}{c c c | c c c | c c c}
            1 & 2 & 3 & 4 & 5 & 6 & 7 & 8 & 9\\
            9& 8&7&6&5&4&3&2&1\\
            1&2&3&2&3&1&3&2&1\\
    \end{array}
    \right] $
\end{problem}
\begin{proof}
    $\sphericalangle a_{i\left( jk \right) }$

    $i = 1 \quad a_{1jk}\rightarrow \begin{pmatrix} 1&4&7\\2&5&8\\3&6&9 \end{pmatrix} \overset{S}{\longrightarrow} \begin{pmatrix} 1 & 3 & 5\\3 & 5 & 7 \\ 5 & 7 & 9 \end{pmatrix} $ 

    $\left[ 
        \begin{array}{c c c| c c c| c c c}
            1 & 3 & 5 & 3& 5 & 7 & 5 & 7 & 9\\
            9&7&5&7&5&3&5&3&1\\
            1&2&3&2&3&\frac{3}{2}&3 & \frac{3}{2}&1
        \end{array}
    \right] $
\end{proof}

\section{Определитель, продолжение}

\begin{example}
    $^{i_1i_2\ldots i_p} = f^{i_1}\wedge f^{i_2}\wedge \ldots \wedge f^{i_p}$
\end{example}

Свойства:
\begin{enumerate}
    \item[] $\left\{ e_j \right\} _{j=1}^k$ -- базис $X \implies x_i = \sum_{j=1}^{n} \xi_i^je_j$

        $\sphericalangle M = \left\{ x_1, x_2, \ldots, x_n \right\}  = \begin{bmatrix} \xi_1^1 & \xi_2^1 & \ldots & \xi_n^1 \\ \xi_1^2 & \xi_2^2 & \ldots &\xi_n^2\\ \vdots & \vdots &\ddots & \vdots\\ \xi_1^n & \xi_2^n & \ldots & \xi_n^n  \end{bmatrix} $ 

        $\sqsupset \det \left\{ x_1,x_2, \ldots, x_n \right\}  = \det M$
    \item $\det M^T = \det M$

        \begin{proof}
            $\det M = \sum_{\sigma}(-1)^{\left[ \sigma \right] }\xi_1^{\sigma(1)}\xi_2^{\sigma(2)} \ldots \xi_n^{\sigma(n)}$

            $\sphericalangle \det M^T = \sum_{\sigma}(-1)^{\left[ \sigma \right] }\xi^1_{\sigma(1)}\xi^2_{\sigma(2)} \ldots \xi^n_{\sigma(n)} = \sum_{\sigma^{-1}}(-1)^{\left[ \sigma^{-1} \right]} \xi_1^{\sigma^{-1}(1)}\ldots \xi_n^{\sigma^{-1}(n)} = \det M$
        \end{proof}

        \begin{lemma}
            $\sigma \in S_n \implies \left[ \sigma \right]  = \left[ \sigma^{-1} \right] $
        \end{lemma}
        \begin{proof}
            $\sigma \circ \sigma^{-1} = e$ -- чётная перестановка. 

            Чётность при композиции складывается, значит у $\sigma$ и  $\sigma^{-1}$ равная чётность (если разная, то  $e$ -- нечётная ?!!)
        \end{proof}
    \item $\det \left\{ x_1, \ldots, x_i, \ldots, x_j, \ldots, x_n \right\} = -\det \left\{ x_1, \ldots, x_j, \ldots, x_i, \ldots, x_n \right\} $
        \begin{proof}
            Основное свойство антисимметричной формы
        \end{proof}
    \item $\det \left\{ x_1+\sum_{i=2}^{n} \alpha^ix_i, x_2, \ldots, x_n \right\} = \det \left\{ x_1, x_2, \ldots, x_n \right\} $
    \item $\left\{ x_i \right\} _{i=1}^n$ -- ЛЗ $\implies \det \left\{ x_1, x_2, \ldots, x_n \right\} = 0$
    \item $\det \left\{ \alpha x_1, x_2, \ldots, x_n \right\} = \alpha \det \left\{x_1, x_2, \ldots, x_n  \right\} $ 

    \item 
        \begin{proof}            
            $M = \left\{ x_1, x_2, \ldots, x_n \right\}  = \begin{bmatrix} \xi_1^1 & \xi_2^1 & \ldots & \xi_m^1 & \ldots & \xi_n^1 \\ \xi_1^2 & \xi_2^2 & \ldots &\xi_m^2 & \ldots & \xi_n^2\\ \vdots & \vdots &\ddots & \vdots & \ddots & \vdots\\ \xi_1^n & \xi_2^n & \ldots & \xi_m^n & \ldots & \xi_n^n \end{bmatrix} $ 

        \begin{align*}
            \sphericalangle \det \left\{ x_1, x_2, \ldots, x_n \right\} & = f^1\wedge f^2 \wedge \ldots \wedge f^n\left( x_1, x_2, \ldots, x_n \right)\\
                                                                        &= f^1\wedge f^2 \wedge \ldots \wedge f^n\left( x_1, x_2, \ldots, \sum_{j=1}^{n} \xi^j_me_j, \ldots, x_n \right)  \\
                                                                        &= \sum_{j=1}^{n} \xi^j_m f^1 \wedge \ldots \wedge f^n\left( x_1, x_2, \ldots, e_j, \ldots, x_n \right)  \\
                                                                        &= \sum_{j=1}^{n} (-1)^{\left| m-j \right| }\xi^j_m f^1 \wedge f^2 \wedge \ldots \wedge \underbrace{f^j(e_j)}\limits_{1}\wedge \ldots \wedge f^n\left( x_1, x_2, \ldots, x_{m-1}, x_{m+1}, \ldots, x_n \right)  \\
                                                                        &= \sum_{j=1}^{n} (-1)^{\left| m-j \right| }\xi^j_m \underbrace{\underbrace{f^1 \wedge \ldots \wedge f^n}\limits_{\scriptsize\cancel{f^j}}\left( x_1, x_2, \ldots, x_{m-1}, x_{m+1}, \ldots, x_n \right)}\limits_{M_j^m\text{ -- дополнительный минор элемента }\xi^j_n}  \\
                                                                        &= \sum_{j=1}^{n} (-1)^{\left| m-j \right| }\xi_n^jM_j^m = \sum_{j=1}^{n} (-1)^{j-m}\xi_m^jM_j^m \\
            =\ldots &=\sum_{j_1}\sum_{j_2}\ldots \sum_{j_k}(-1)^{j_1-m}(-1)^{j_2-m}\ldots (-1)^{j_k-m+k}\underbrace{\xi_{m_1}^{j_1}\xi_{m_2}^{j_2}\ldots \xi_{m_k}^{j_k}}\limits_{L}M_{j_1, j_2, \ldots, j_k}^{m_1, m_2, \ldots, m_k} \\
                    &= (-1)^{j_1-m_1+j_2-m_2+\ldots +j_k-m_k} L^{j_1, j_2, \ldots, j_k}_{m_1, m_2, \ldots, m_k}M_{j_1, j_2, \ldots, j_k}^{m_1, m_2, \ldots, m_k}\\
        .\end{align*}

        \begin{theorem}
            [Лапласа]
            $\det \left\{ x_1, x_2, \ldots, x_n \right\} = (-1)^{j_1+\ldots+j_n-m_1-\ldots-m_n}L_{m_1, \ldots, m_n}^{j_1, j_2, \ldots, j_n}M_{j_1, j_2, \ldots, j_n}^{m_1, m_2, \ldots, m_n}$
        \end{theorem}

        \begin{example}
            $\det
            \begin{vmatrix}
                1&2&3\\0&1&2\\1&2&1
            \end{vmatrix} = \det 
            \begin{vmatrix}
                1&2\\0&1
            \end{vmatrix}\cdot 1 + \det 
            \begin{vmatrix}
                1&2\\1&2
            \end{vmatrix}\cdot 2\cdot (-1)^7 + \det
            \begin{vmatrix}
                0&1\\1&2\\
            \end{vmatrix} = 1-3 = 2$
        \end{example}

        \begin{example}
             
            $ \det
            \begin{vmatrix}
            1&2&5&6\\3&4&7&8\\0&0&1&2\\ 0&0&3&4 
            \end{vmatrix} = \det
            \begin{vmatrix}
                1&2\\3&4
            \end{vmatrix}\cdot \det
            \begin{vmatrix}
                1&2\\3&4
            \end{vmatrix} = 4$
        \end{example}

        \end{proof}
\end{enumerate}

\begin{example}
    $
    \begin{vmatrix}
        A_{11} & A_{12} & \ldots & A_{1n}\\
        0 & A_{22} & \ldots & A_{2n}\\
        0&0&\ldots & A_{nn}
    \end{vmatrix} = \prod_{j=1}^n \det A_{jj}$
\end{example}

\section{Ранг матрицы}

    $\sqsupset \left\{ x_i \right\} _{i=1}^k$ -- набор векторов в $X\quad \left( \dim X = n \geqslant k \right) $ 

    \begin{lemma}
        $\left\{ x_i \right\} _{i=1}^k$ -- ЛЗ $\iff \forall \omega\in  \Lambda^k\quad \omega\left( x_1, x_2, \ldots, x_n \right) = 0$
    \end{lemma}
    \begin{proof}
        \begin{itemize}
            \item []
            \item [$\implies $] $\sqsupset \left\{ x_i \right\} _{i=1}^n$ -- ЛЗ

                $\sphericalangle \omega\left( x_1, x_2, \ldots, x_k \right) = \omega\left( \underbrace{x_1+\sum_{i=2}^{k}}\limits_{\simeq x_k\quad = \beta x_k} \alpha^x_i, x_2, \ldots, x_k \right) = \beta\omega\left( x_1, x_2, \ldots, x_k \right) = 0 $
            \item [$\impliedby $] $\left\{ x_1, x_2, \ldots, x_k \right\} \quad \forall \omega\in\Lambda ^k\omega\left( x_1, x_2, .., x_k \right) =0 \overset ? {\implies } $ ЛЗ

                Допустим оратное. Тогда мы можем достроить этот набор до базиса $X\quad \left\{ x_1, x_2, \ldots, x_k, e_{k+1}, \ldots, e_n \right\} $. К нему есть сопряжённый базис $\left\{ f^1, f^2, \ldots, f^n \right\} $ 

                \begin{align}
                    \sphericalangle f^1\wedge f^2 \wedge \ldots \wedge \left( x_1, x2, \ldots, x_n \right) &= \sum_{\sigma} (-1)^{\left[ \sigma \right] } f^{\sigma(1)}f^{\sigma(2)}\ldots f^{\sigma(k)}\left( x_1, x_2, \ldots, x_k \right) \\
                                                                                                           &= \sum_{\sigma}(-1)^{\left[ \sigma \right] }f^{\sigma(1)}(x_1)f^{\sigma(2)}\ldots f^{\sigma(k)}\neq 0 \\
                \end{align}
        \end{itemize}
    \end{proof}

    \begin{note}
        $\forall \omega\in\Lambda^p\quad \omega = ^{i_1i_2\ldots i_k}F\omega_{i_1i_2\ldots i_k}$

        $\impliedby \left( \left\{ x_i \right\} _{i=1}^n \text{ -- ЛЗ } \iff ^{i_1i_2\ldots i_k}F\left( x_1, x_2, \ldots, x_k \right) =0 \forall i_1 \ldots i_k \right) $
    \end{note}

    $\sphericalangle B =  \begin{bmatrix} \xi_1^1 & \xi_2^1 & \ldots & \xi_n^1 \\ \xi_1^2 & \xi_2^2 & \ldots &\xi_n^2\\ \vdots & \vdots &\ddots & \vdots\\ \xi_1^n & \xi_2^n & \ldots & \xi_n^n  \end{bmatrix} \quad \det B = 0 \implies  x_1,x_2, \ldots, x_n$ -- ЛЗ. Но нам хочется узнать, а сколько там независимых

    $\Lambda^n\quad \Lambda^{n-1}\quad ^{i_1 i_2 \ldots i_{n-1}}F\left( x_1, x_2, \ldots, x_{n-1} \right) $. Если найдётся такой минор, который не равен 0, то у нас есть $n-1$ ЛНЗ векторов. Если все равны 0, то значит их меньше. Мы уменьшаем число и смотрим дальше, а там тоже выбор из двух.

    \begin{definition}
        Ранг матрицы -- максимальный размер её отличного от нуля минора.
    \end{definition}

    \begin{definition}
        Базисные строки -- набор ЛНЗ строк в количестве ранга матрицы.
    \end{definition}

    \begin{theorem}
        Число ЛНЗ строк матрицы равно $rg~A$ (ранг  $A$)
    \end{theorem}


\section{Практика. Вычисление определителей}

\begin{enumerate}
    \item Приведение к треугольному виду.
    \item Метод выделения линейных множителей

        \begin{definition}
            [Определитель Вандермомда (определитель Того-кого-нельзя-называть)]
            $\prod_{i>j}(x_i-x_j$

             $
             \begin{vmatrix}
                 1 & x_1 &x_2^2 & x_1^3 & \ldots & x_1^{n-1}\\
                 1 & x_2 &x_2^2 & x_2^3 & \ldots & x_2^{n-1}\\
                 \ldots&\ldots&\ldots&\ldots& \ddots & \ldots\\
                 1 & x_n & x_n^2 & x_n^3 & \ldots & x_n^{n-1}\\
             \end{vmatrix}$
        \end{definition}
\end{enumerate}

\section{Тензорная алгебра}

\begin{theorem}
    TODO
\end{theorem}
\begin{proof}
    TODO
\end{proof}

$\sphericalangle W \in \Omega_q^p$

$W\quad \underbrace{x_1x_2\ldots x_p}_{\in X}\underbrace{y^1y^2 \ldots y^q}_{\in X^*} \mapsto K$

$\sqsupset \left\{ e_j \right\} _{j=1}^n$ -- базис $K\quad \left\{ f^i \right\} _{i=1}^n$ 

$f^k(e_j) = \delta_j^k$

$x_1 = \sum_{i=1}^{n} \xi_1^{i_1}e_{i_1} \quad \ldots\quad x_p = \sum_{i=1}^{n} \xi_p^{i_p} e_{i_p} $

$y^1 = \sum_{j=1}^{n} \eta_{j_1}^1f^{i_1}\quad \ldots\quad y^q = \sum_{j=1}^{n} \eta^q_{j_q}f^{j_q}$

\begin{align*}    
    W\left( x_1x_2\ldots x_p y^1 y^2 \ldots y^q \right) &= W\left( \xi_1^{i_1} \ldots \right)  \\
                                                        &= \xi_1^{i_1} \ldots \eta_{j_q}^qW\left( e_{i_1}e_{i_2} \ldots f^{j_1}f^{j_2} \ldots \right)  \\
                                                        &\ov{\triangle}= \xi_1^{i_1}\xi_2^{i_2} \ldots \eta^1_{j_1}\eta^2_{j_2} \ldots \omega_{i_1i_2\ldots i_p}^{j^1j^2 \ldots j^q}  \\
.\end{align*}


$\sqsupset \left\{ \tl e_l \right\} _{l=1}^n$ -- базис X (новый)

$\tl e_l = \sum_{j=1}^{n} \tau_l^je_j\qquad \left( \tau = \|\tau_l^j\| \right) $

$\sqsupset \left\{ \tl f^m \right\} _{m=1}^n$ -- базис $X^*$ (сопряжённый к новому)

$\tl f^m = \sum_{j=1}^{n} \sigma_j^mf^j \qquad \left( \sigma_j^m\tau_l^j = \delta_l^m \right) $

$\sqsupset \omega_{i_1i_2\ldots i_p}^{j_1j_2 \ldots j_q} = W\left( e_{i_1} \ldots e_{i_p}f^{j_1}\ldots f^{j_q} \right) $

\begin{align*}    
    \sphericalangle \tl{\omega}^{t_1t_2 \ldots t_q}_{s_1 s_2 \ldots s_p} &= W\left( \tl e_{s_1}\tl e_{s_2}\ldots. \tl e_{s_p}\tl f^{t_1}\ldots \tl f^{t_q} \right)\\
                                                                         &= W\left( \sum_{i_1=1}^{n} \tau_{s_1}^{i_1}e_{i_1} \ldots \sum_{i_p=1}^{n} \tau_{s_p}^{i_p}e_{i_p}, \sum_{j_1=1}^{n} \sigma_{j_1}^{t_1} \ldots \sum_{j_q = 1}^{n} \sigma_{j_q}^{t_q}f^{j_q} \right)  \\
                                                                         &= \tau_{s_1}^{i_1}\tau_{s_2}^{i_2} \ldots \tau_{s_p}^{i_p}\sigma_{j_1}^{t_1} \ldots \sigma_{j_q}^{t_q} \omega_{i_1i_2 \ldots i_p}^{j^1j^2 \ldots j^q} \\
.\end{align*}

\begin{note}
    Вообще аргументы могут идти по-разному и тогда могут писать $\omega^{i_1i_2}{}_{j_1j_2j_3}{}^{i_3i_4i_5}$
\end{note}

\begin{definition}
    Тензором типа $(p,q)$  называется алгебраический объект вида  $\omega_{i_1i_2\ldots i_p}^{j_1j_2 \ldots j_q}$, где:
    \begin{enumerate}
        \item [] $i_1i_2 \ldots i_p$ -- ковариантные индексы
        \item [] $j^1 j^2 \ldots j^q$ -- контравариантные индексы
    \end{enumerate}
    и преобразующийся при замене базиса $(T)$ по закону
    \[
    \tl{\omega}^{t_1t_2 \ldots t_q}_{s_1 s_2 \ldots s_p} = \tau_{s_1}^{i_1}\tau_{s_2}^{i_2} \ldots \tau_{s_p}^{i_p}\sigma_{j_1}^{t_1} \ldots \sigma_{j_q}^{t_q} \omega_{i_1i_2 \ldots i_p}^{j^1j^2 \ldots j^q}
    .\] 
\end{definition}

\begin{note}
    Они, естественно, образуют линейное пространство (можно рассмотреть сложение, домножение на число и доказать свойство)
\end{note}

Операции:
\begin{enumerate}
    \item Сложение \quad $\omega_{\vec i}^{\vec j}\quad v_{\vec i}^{\vec j}\quad \in (p,q)$

         \[
         u_{i_1i_2 .. i_p}^{j^1 j^2 \ldots j^q} = \omega_{i_1i_2 \ldots i_p}^{j_1j_2 \ldots j_q} + v_{i_1i_2 \ldots i_p}^{j^1 j^2 \ldots j^q} \]

             $\tl u_{\vec i}^{\vec j} = \tl{\omega + v}_{\vec i}^{\vec j} = \tl \omega_{\vec i}^{\vec j} + v_{\vec i }^{\vec j}$
         \item Умножение на число $u_{\vec i}^{\vec j} = \alpha \omega_{\vec i}^{\vec j}$

             $\tl u_{\vec i}^{\vec j} = \tl {\left( \alpha \omega \right) }_{\vec i}^{\vec j} = \alpha \tl {\omega}_{\vec i}^{\vec j}$
         \item Тензорное произведение. $\sqsupset \omega_{\vec i}^{\vec j} \left( p_1, q_1 \right) \quad v_{\vec i}^{\vec j}\quad (p_2, q_2)$

             $u = \omega \otimes v \left( p_1+p_2, q_1+q_2 \right) $

    \begin{align*}
    &\tl \omega_{s_1s_2 \ldots s_{p_1}}^{t_1t_2 \ldots t_{q_1}}\tl v_{s_{p_1+1} \ldots s_{p_1+p_2}}^{t_{p_1+1} ... t_{p_1+p_2}} =\\
    &= \tau_{s_1}^{i_1} ..  \tau_{s_{p_1+p_2}}^{i_{p_1+p_2}} \sigma_{j_1}^{t_1} \ldots \sigma_{j_{q_1+q_2}}^{t_{q_1+q_2}} \underbrace{\omega_{i_1i_2 \ldots i_{p_1}}^{j_1j_2 \ldots j_{q_1}}v_{i_{p_1+1}\ldots i_{p_1+p_2}}^{j_{q_1} \ldots j_{q_1+q_2}}}_{u_{i_1 i_2 \ldots i_{p_1+p_2}}^{j_1j_2 \ldots j_{q_1+q_2}}}
    .\end{align*}
\item Транспонирование. Замена двух индексов тянется до замены двух индексов в преобразовании (просто везде один меняется на другой). А тогда преобразование то же самое. Note: менять можно индексы одного типа
    \begin{note}
        Если менять индексы разного типа, то нарушится ко(нтра)вариантность и это не будет тензором
    \end{note}
\item Свёртка. $\omega_{i_1 i_2 \ldots k \ldots i_p}^{j_1 j_2 \ldots k \ldots j_q}$

    $\omega_{s_1s_2 \ldots m \ldots s_p}^{t_1t_2 \ldots m \ldots t_q} = \sigma_{j_1}^{t_1} \ldots \sigma_k^m \ldots \sigma_{j_q}^{t_q}\tau_{s_1}^{i_1} ... \tau_m^k \ldots \tau_m^k \ldots \tau_{s_p}^{i_p}\omega_{i_1i_2 \ldots k \ldots i_p}^{j_1j_2 \ldots k \ldots j_q}$

    $\sigma_k^m\tau_m^k = 1$ -- т.е. эти индексы просто не участвуют в преобразовании.

\end{enumerate}

\section{Практика. Вычисление определеителей}

\begin{enumerate}
    \item Метод рекуррентных отношений

        $D_n = p D_{n-1}$

        $D_n = p_nD_{n-1}$

        $D_n = p-qD_{n-1}$

        $D_n = p_n - q_nD_{n-1}$

        $D_n = pD_{n-1} + qD_{n-2}$
    \item Представление определителя в виде суммы
\end{enumerate}

\section{Ранг матрицы это инвариант}

$\sqsupset A, B $

$AB = C\quad c_{ij} = \sum_{k=1}^{n}a_{ik}b_{kj}$

$
\begin{array}{c@{\,}c}
    &\begin{bmatrix} b_{11} & b_{12} & \ldots\\ b_{21} & b_{22} & \ldots\\ \vdots & \vdots & \ldots \end{bmatrix} \\[8mm]
    \begin{bmatrix} a_{11} & a_{12} & \ldots\\ a_{21} & a_{22} & \ldots\\ \vdots & \vdots & \ldots \end{bmatrix}    & \begin{bmatrix} c_{11} & c_{12} & \ldots\\ c_{21} & c_{22} & \ldots\\ \vdots & \vdots & \ldots \end{bmatrix} \\[8mm]
    \begin{bmatrix} 0 & 1 &  & \\ 1 & 0 & &\\ & & 1 &\\ &&&1 \end{bmatrix} & \begin{bmatrix} b_{21} & b_{22}& &\\ b_{11} & b_{12} & &\\ &&&\\&&& \end{bmatrix} 
\end{array}$


Если поменять две строчки местами, то они поменяются и у результата.

Если умножить строку на число, в итоге та же строка тоже умножится

Если добавим к одной строке другую, то же произойдёт и с результатом.

Матрицы преобразований:
\begin{enumerate}
    \item $\begin{bmatrix} 0 & 1 &  & \\ 1 & 0 & &\\ & & 1 &\\ &&&1 \end{bmatrix} $\\
    \item $\begin{bmatrix} 1 & & & \\ &\lambda & && \\ &&1&&\\ &&&\ddots &\\ &&&&1 \end{bmatrix} $
    \item $\begin{bmatrix} 1 & 1&&\\&1&&\\&&1&\\&&&1 \end{bmatrix} $
\end{enumerate}

\begin{note}
    Произведение матриц некоммутативно. 

    $AB\neq BA\qquad (A+B)^2 \neq A^2+2AB+B^2\quad (A+B)(A-B)\neq A^2-B^2$
\end{note}

\begin{example}
    $A\quad \rg A = a\quad \rg B = b$

     \[
         \rg AB = \min(m, n)
    .\] 

    Про ранг суммы нельзя ничего сказать
\end{example}

\begin{example}
    $A_{m\times r}\quad B_{r\times n}$

    $\rg (AB) = r \implies \rg A = r\quad \rg B = r$
\end{example}

\begin{problem}
    Сформулировать в терминах рангов необходимое и достаточное условие того, чтобы три точки на плоскости не лежали на одной прямой

    $P(x_1, y_1)\quad Q(x_2, y_2)\quad R\left( x_3, y_3 \right) $

    $Ax+By+C = 0$ -- уравнение прямой

    $\begin{cases}
        Ax_1+By_1+C = 0\\
        Ax_2+By_2+C = 0\\
        Ax_3+By_3+C = 0\\
    \end{cases}$ 

    Мы требуем нетривиальное решение. Это некрамеровская система, значит

    $\rg \begin{bmatrix} x_1&y_1&1\\x_2&y_2&1\\x_3&y_3&1\\ \end{bmatrix} <3$
\end{problem}

\begin{problem}
    То же самое, только для четырёх точек в пространстве и плоскость через них.

    $\begin{bmatrix} x_1&y_1&z_1&1\\ x_2&y_2&z_2&1\\x_3&y_3&z_3&1\\ x_4&y_4&z_4&1\\  \end{bmatrix} $

    4 -- не лежт

    3 -- на одной плоскости

    2 -- на одной прямой
\end{problem}

\begin{problem}
$\left[\begin{array}{cc|c} A_1&B_1&C_1\\A_2&B_2&C_2\\A_3&B_3&C_3 \end{array}\right] $ 


    2, 3 -- две прямые параллельны или они пересекаются в трёх разных точках

    2, 2 -- пересекаются в одной точке

    1, 2 -- три параллельных прямые, две из которых совпадают

    1, 1 -- все совпадают
\end{problem}


\section{Линейные операторы}

 \begin{definition}
    $\sqsupset X, Y$ -- ЛП, $\dim X = n\quad \dim Y = m$

     $\sphericalangle \phi: X\to Y$

     \begin{align*}
         \varphi: X &\longrightarrow Y \\
         x &\longmapsto y = \varphi(x) = \varphi x\\
         \varphi(x+y) = \varphi(x) + \varphi(y)\\
         \varphi\left(\lambda x  \right) = \lambda \varphi(x) 
     .\end{align*}    
\end{definition}

\begin{note}
    $\varphi: X\longrightarrow X$

    действует на (биекция) -- автоморфизм

    действует в (сюръекция) -- эндоморфизм
\end{note}

\begin{example}
    \begin{enumerate}
        \item $\mathscr{I}: X\to X\quad \mathscr{I}x = x$ -- тождественный оператор
        \item $\mathbb{O}:X\to X\quad \mathbb{O}x = 0$ -- нулевой оператор
        \item  $\mathscr{P}:X\to X\quad X = L_1 \dotplus L_2 \implies x = x_1+x_2$
            $\begin{cases}
                \mathscr{P}_{L_1}^{\parallel L_2}x = x_1\\
                \mathscr{P}_{L_2}^{\parallel L_1} = x_2
            \end{cases}$ -- проектор
        \item $\varphi: \p C[a,b] \to  \p C[a,b]\quad \left( \varphi f \right) (t) = \int_a^bf(s)K(s,t)ds$
        \item $\mathscr{D}: C^{\infty }(a,b) \to  C^{\infty}(a,b)\quad \left( \mathscr{D}f \right) (t) = \frac{df}{dt}$
    \end{enumerate}
\end{example}

$\sqsupset \mathscr{L}(X,Y)$ -- множество операторов действующих из $X$ в  $Y$

$\sqsupset \varphi, \psi \in \mathscr{L}(X, Y)$

\begin{definition}
    Суммой операторов $\varphi, \psi$ называется отображение  \[
    \chi = \varphi + \psi
    .\] 
    и определяемое как $\chi(x) = \varphi(x) + \psi(x)$
\end{definition}
\begin{lemma}
    $\chi \in \mathscr{L}(X, Y)$

     \begin{align*}
         \chi(x+y) = (\varphi + \psi)(x+y) = \varphi(x+y) + \psi(x+y) = (\varphi + \psi)(x) + (\varphi + \psi)(y) = \chi x + \chi y
    .\end{align*}
\end{lemma}

\begin{definition}
    Умножением оператора на число $\lambda \in K$ называется отображение
     \[
    \omega = \lambda \varphi
    .\] 
    $\omega(x) = \left( \lambda \varphi \right) (x) = \lambda\cdot  \varphi(x)$
\end{definition}
\begin{lemma}
    $\omega\in\mathscr{L}(X, Y)$
\end{lemma}

\begin{theorem}
    $\mathscr{L}(X, Y)$ -- Линейное Пространство
\end{theorem}

\begin{question}
    $\dim(X, Y) = ?$
    
\end{question}

$\sqsupset \left\{ e_j \right\} _{j=1}^n$ -- базис $X\quad \left\{ h_k \right\} _{k=1}^m$ -- базис $Y$

$\sqsupset x\in X \implies x = \sum_{j=1}^{n} \xi^je_j$ 

\begin{align*}
    \varphi x &= \varphi\left( \sum_{j=1}^{n} \xi^je_j \right)  
               =\sum_{j=1}^{n} \xi^j\underbrace{\varphi(e_j)}_{\in Y} \\
              &\overset * = \sum_{j=1}^{n} \xi^j \sum_{k=1}^{m} a_j^kh_k 
              = \sum_{k=1}^{m}\underbrace{ \left( \sum_{j=1}^{n} \xi^ja_j^k \right)}_{\eta^k} h_k
              = \sum_{k=1}^{n} \eta^kh_kx
.\end{align*}


$*\quad\varphi(e_j) = \sum_{k=1}^{m} a_j^kh_k$

\begin{definition}
    Набор $A_{\varphi} = \|a_j^k\|$ образует матрицу, которая называется матрицей линейного оператора $\varphi$ в паре базисов  $\left\{ e_j \right\} $ и $\left\{ h_k \right\} $
\end{definition}

\begin{lemma}
    Задание линейного оператора в пространстве эквивалентно заданию его матрицы при фиксированной паре базисов.
\end{lemma}

\begin{note}
    $\varphi x = y$

    $A_{\varphi}\xi = \eta$

    Тождественный оператор имеет единичную матрицу.

    Нулевой матрицу из нулей

    $X = L_1 \dotplus L_2\quad L_1 \left\{ e_j \right\} _{j=1}^k\quad L_2 \left\{ e_j \right\} _{j=k+1}^n$ 

    $\begin{bmatrix} E&O\\O&O \end{bmatrix} $ -- матрица проектора (единичный квадрат, всё остальное ноль)
\end{note}

$\sphericalangle \left\{ _k^jE \right\}$  -- набор в $\mathscr{L}(X, Y$

$\sqsupset x\in X\quad x = \sum_{j=1}^{n} \xi^je_j\quad _k^jEx = \xi^jh_k$ (единичка в матрице на месте $(j,k)$)

\begin{theorem}
    Набор $\left\{ _k^jE \right\} $ образует базис $\mathscr{L}(X, Y)$
\end{theorem}
\begin{proof}
    \begin{itemize}
        \item []
        \item Полнота. $\sqsupset \varphi\in \mathscr{L}(X, Y)$

        \begin{align*}            
            \varphi(x) &= \sum_{j=1}^{n} \xi^j\varphi(e_j) = \sum_{j=1}^{n} \xi^j \sum_{k=1}^{m} a_j^k \\
            &= \sum_{j=1}^{n} \sum_{k=1}^{m} \xi^ja_j^kh_k = \sum_{j=1}^{n} \sum_{k=1}^{m} {}_k^jE(x)a_j^k\ \ \forall x\in X
        .\end{align*}


            $ \implies  \varphi = \sum_{j=1}^{n} \sum_{k=1}^{m} {}_k^jEa_j^k$
        \item Линейная независимость. 
            $\sqsupset \sum_{j=1}^{n} \sum_{k=1}^{m}{}_k^jE\alpha_j^k = \mathscr{O}\quad \mid e_1$ 

            $\sum_{j=1}^{n} \sum_{k=1}^{m}{}_k^jE\left( e_1 \right) \alpha_j^k = \sum_{k=1}^{m} 1\cdot \underbrace{h_k}_{\text{ЛНЗ}}\cdot \alpha_1^k = 0 \implies \alpha_1^k = 0$ 

            $\alpha_j^k = 0 \forall k, j$
    \end{itemize}
\end{proof}

\begin{note}
    $\dim \mathscr{L}(X, Y) = m\cdot n$

    $\sphericalangle K_n^m$ -- пространство $m \cdot  n$
\end{note}

\begin{lemma}
    $\mathscr{L}(X, Y) \simeq K_n^m$
\end{lemma}

$\sqsupset \left\{ e_j \right\} _{j=1}^n\quad  \left\{ \tl e_k \right\} _{k=1}^n$ -- базис $X$ 

$\sqsupset \varphi:X\to X\quad A_{\varphi}\quad \tl A_{\varphi}$

Как преобразуется матрица оператора при переходе между базисами
\begin{align*} 
    \varphi \tl e_s i&= \varphi\left( \sum_{j=1}^{n} \tau^j_se_j \right) =\sum_{j=1}^{n} \tau^j_s \varphi\left( e_j \right) &= \sum_{j=1}^{n} \sum_{k=1}^{n} \tau^j_s a^k_je_k\\
                     &= \sum_{j=1}^{n} \tl a_s^j\tl e_j = \sum_{j=1}^{n} \tl a^j_s \sum_{k=1}^{n} \tau_j^k e_k &= \sum_{j=1}^{n} \sum_{k=1}^{n} \tl a_s^j\tau_j^ke_k
.\end{align*}

$\implies \sum_{j=1}^{n} \tau^j_sa^k_j = \sum_{j=1}^{n} \tl a^j_s\tau^k_j$ 

$A_{\varphi}T = T\tl A_{\varphi}$

 \begin{lemma}
     $\tl A_{\varphi} = S A_{\varphi}T\quad S = T^{-1}$ -- преобразование SAT
\end{lemma}

$\sqsupset T\quad \det T\neq 0$

$\sqsupset A\in \R_n^n$

$A \longrightarrow T^{-1}AT$ -- преобразование подобия

$A~B \iff B = T^{-1}AT$ -- отношение эквивалентности. Разбивается на непересекающиеся классы

$\sqsupset X, Y, Z$

$\sphericalangle \varphi\in\mathscr{L}(X, Y)$

$\sphericalangle \psi\in\mathscr{L}(Y, Z)$

$X \overset {\varphi}{\longrightarrow} Y \overset{\psi}{\longrightarrow} Z$

 \begin{definition}
     Композицией линейных операторов $\phi$ и  $\psi$ называется отображение  \[
     \chi = \psi \circ \varphi,
     \] 
      которое действует как \[
         \chi(x) = \left( \varphi\circ \psi \right) (x) = \pi\left( \varphi(x) \right) 
     .\] 
\end{definition}
\begin{lemma}
    $\psi\circ \varphi = \chi\in\mathscr{L}(X, Z)$
\end{lemma}
\begin{proof}
    $\sqsupset x, y\in X$

    $\chi(x+y) = \left(\psi\circ \varphi  \right)(x+y) = \psi\left( \varphi(x+y) \right) = \psi\left( \varphi x + \varphi y \right)  = \psi(\varphi (x)) + \psi(\varphi( y)) = \chi (x) + \chi (y)  $

    $\chi\left( \lambda x \right)  = \lambda \chi(x)$
\end{proof}

$\sqsupset \left\{ e_j \right\} _{j=1}^n$ -- базис $X$,  $\left\{ h_k \right\} _{k=1}^m$ -- базис $Y$,  $\left\{ g_l \right\} _{l=1}^s$ -- базис $Z$

В паре базисов  $\varphi \longleftrightarrow A_{\varphi}$

В паре базисов  $\psi \longleftrightarrow B_{\psi}$

Хочется получить матрицу  $\chi \longleftrightarrow C_{\chi}$

$\chi\left( e_j \right) = \sum_{l=1}^{s} C^l_kg_l$ 

С другой стороны 
\begin{align*}  
 &= \left( \psi\circ \varphi \right) (e_j) = \psi\left( \varphi(e_j) \right) \\
 &= \psi\left( \sum_{k=1}^{m} a_j^kh_k \right)  \\
 &= \sum_{k=1}^{m} a_j^k\psi(h_k) \\
 &= \sum_{k=1}^{m} a^k_j \sum_{l=1}^{s} b_k^lg_l \\
 &= \sum_{l=1}^{s} \left( \sum_{k=1}^{m} a^k_jb^l_k \right) g_l \\
.\end{align*}

$\implies c^l_j = \sum_{k=1}^{m} a^k_jb^l_k \iff C_{\chi} = B_{\psi}\cdot A_{\varphi}$

$\sphericalangle \mathbb{K}^n_n\quad$ ``+'' ``$\lambda$'' ``$\cdot $''

\begin{definition}
    Алгеброй $\mathscr{A}$ называется линейное пространство, наделённое операцией умножение, так что выполняются следующие требования (аксиомы):
     \begin{enumerate}
         \item $a_1\left( a_2a_3 \right)  = (a_1a_2)a_3 \forall a_1, a_2, a_3$
         \item $a\left( b+c \right)  = ab+ac$
         
             $(a+b)c = ac+bc$
         \item  $\lambda\left( ab \right)  = \left( \lambda a \right) b = a\left( \lambda b \right) $
    \end{enumerate}
\end{definition}

$\sqsupset \mathscr{A}$ -- некоторая алгебра $\sqsupset x, y\in \mathscr{A}$

$\sqsupset \left\{ e_j \right\}$ -- базис $\mathscr{A} \implies \begin{cases}
    x = \sum_{j=1}^{n} \xi^je_j\\
    y = \sum_{k=1}^{n} \eta^ke_k
\end{cases}$ 

$x\cdot y = \left( \sum_{j=1}^{n} \xi^je_j \right) \cdot \left( \sum_{k=1}^{n} \eta^ke_k \right)  = \sum_{k,j = 1}^{n} \xi^j\eta^k\underbrace{\left( e_j\cdot e_k  \right)}_{\sum_{l=1}^{n} m^l_{jk}e_l } $

$\left\{ m^l_{jk} \right\}$ -- структурные константы алгебры $\mathscr{A}$

\begin{example}
    $\C$

    \begin{tabular}{|c|c|c|}       
    \hline &1&i\\ \hline
    1 & 1 & i\\ \hline
i&i&-1\\ \hline
    \end{tabular}

    Такую же для кватернионов можно сделать. 
\end{example}

\begin{problem}
    Выразить свойство ассоциативности в рамках структурных констанст

    Сделать то же самое с коммутативностью
\end{problem}

\begin{note}
    Алгебра это полугруппа (есть ассоциативности)
\end{note}

$\sphericalangle \mathscr{A}$ -- алгебра $\sqsupset e_L\in \mathscr{A}:\quad \forall x\in \mathscr{A}\quad e_L x = x$ -- левая единица

$e_R\in\mathscr{A}:\quad \forall x\in \mathscr{A}\quad xe_R = x$ -- правая единица

\begin{note}
    Если есть и левая и правая, то они совпадают

    Если нет, то может быть несколько одного типа.
\end{note}

$\sqsupset x\in \mathscr{A}\quad \sqsupset y\in \mathscr{A}: yx = e$, $y$ -- левый обратный

$z\in\mathscr{A}\quad xz = e$ -- правый обратный

 \begin{note}
     Если у $x$ есть правый \underline{или} левый, то он называется обратимым

     Если есть и тот и другой, то они совпадают $y = z = x^{-1}$
      \begin{proof}
         $z = yxz = y$
     \end{proof}
\end{note}

\begin{example}
    $\mathcal{L}(X,X)$ -- алгебра операторов ($\circ$)

    $K_n^n$ -- алгебра матриц
\end{example}

\begin{definition}
    $\sqsupset x\in \mathcal{A}$

    $y:\quad yx = e \implies y$ -- левый обратный

    $z: xz = e \implies z$ -- правый обратный
\end{definition}

\begin{statement}
    Если существует и тот, и другой, то они совпадают и обозначаются $x^{-1}$
\end{statement}

\section{Обратная матрица}

$\sqsupset A \in K_n^n$

\begin{definition}
    Матрица $B\in K_n^n$ Называется левой обратной к  $A$, если
     \[
    BA=E
    .\]

    Матрица $X\in K_n^n$ называется правой обратной к $A$, если  
    \[
    AC=E
    .\] 

    ($E$ -- единичная матрица)
\end{definition}

\begin{lemma}
    Обратная матрица существует тогда и только тогда, когда её определитель не равен нулю

    $\det A\neq 0 \iff \exists A^{-1}:\quad A A^{-1} = A^{-1}A = E$
\end{lemma}
\begin{proof}
    \begin{itemize}
        \item []
        \item [$\implies $] $\det A \neq  0 \ov ? {\implies } \exists B, C:\quad BA = AC = E $

        $\sqsupset A = \|a^i_j\|\quad c = \|c^i_j\|\quad E = \|\delta^i_j\|$

        \begin{align*}
            &\begin{bmatrix} C \end{bmatrix} &\\
            \begin{bmatrix} A \end{bmatrix} & &=\begin{bmatrix} E \end{bmatrix} 
        .\end{align*}

        $\sum_{j=1}^{n} a^i_jc^j_k = \delta^i_k$ 

        Зафиксируем $k = k_0$-ый столбец. $\implies \delta^i_{k_0} = \beta^i\quad c^j_{k_0} = \xi^j$

        $\implies \sum_{j=1}^{n} a^i_j\xi^j = \beta^i$ -- система линейных уравнений. Матрица уравнения -- ровно матрица $A$

        Нам нужно единственное решение, чтобы она была Крамеровской  $\implies \det A \neq 0$

        \begin{note}
            Существования правой ИЛИ левой обратной достаточно, чтобы определитель был не равен нулю.
        \end{note}

        $\implies \exists C:\quad AC = E$

        $ ?\exists B: BA = E$

        $E^T = E\quad (BA)^T = A^TB^T\qquad A^TB^T = E$

        Аналогично нам нужно  $\det A^T \neq 0$, но $\det A^T = \det A$, а значит мы свели к тому же условию

        $\exists B, C \implies  \exists B=C = A^{-1}$
    \item [$\impliedby $] Можно сказать, что уже сделали. Там также сводим к крамеровской системе, а там определитель не ноль.
    \end{itemize}
\end{proof}

\begin{note}
    Можно вычислять обратную матрицу честно выписывая все уравнения на все члены, но такое матричное произвдение у нас определено не случайно, у него уже есть структура и ей следует пользоваться
\end{note}

Методы вычисления обратной матрицы:
\begin{enumerate}
    \item Метод Гаусса

        $\begin{bmatrix} A\mid E \end{bmatrix} \sim \begin{bmatrix} E\mid A^{-1} \end{bmatrix} $

         Мини-описание: делаем прогон, чтобы получить треугольную, потом получаем диагональную, затем домножаем столбцы или строки, чтобы получить $E$
         
         $T_n \ldots T_2T_1A = E \implies  T_n\ldots T_2T_1 E = A^{-1}$ (не доказательство, просто демонстрация)
     \item Союзная матрица.

         $A^{-1} = \frac{1}{\det A}\tl A^T$ 

         $\tl A^i_j$ -- вычёркиваем в  $A$ соответствующие столбцы и строки, считаем определитель оставшегося (название получившегося: алгебраическое дополнение)
         \begin{proof}
             $A = \|\alpha^i_j\| A^{-1} = \|\gamma^i_j\|$

             $A A^{-1} = E \quad \sum_{j=1}^{n} \alpha^i_j\gamma^j_k = \delta^i_k$ 

             $k = k_0$ -- зафиксировали

             $\delta^i_{k_0} = \beta^i\quad \gamma^j_{k_0} = \xi^j\quad \alpha^i_j = a_j$ (вектор по $i$)

              $\sum_{j=1}^{n} a_j\xi^j = b$ 

              $\xi^j = \frac{\Delta_j}{\det A} = \frac{\det \left( A| a_j \to b \right) }{\det A}o$ 

              $b$ -- столбец с нулями и одной единицей на месте  $k_0$. Можем разложить по столбцу определитель 

              $ = \frac{A^{k_0}_j}{\det A} \forall k_0$ 

              $\xi^j = \gamma^j_{k_0} = \frac{A_j^k}{\det A}$ 

              $\implies A^{-1} = \frac{\tl A^T}{\det A}$
         \end{proof}
\end{enumerate}

\section{Операторная алгебра}

$\mathcal{L}(X, X) \simeq K^n_n$

 $\sqsupset \varphi : X\to X$

\begin{definition}
    Ядром оператора $\varphi$ называется множество
     \[
         \Ker \varphi = \{x\in X: \varphi x = 0\}
    .\] 
\end{definition}

\begin{lemma}
    $\Ker \varphi$ -- ЛПП  $X$
\end{lemma}

\begin{definition}
    Образом оператора $\varphi$ называется множество
     \[
         \Image \varphi = \varphi(X)
    .\] 
\end{definition}
\begin{lemma}
    $\Image \varphi $ -- ЛПП $X$
\end{lemma}

\begin{theorem}
    [О ядре и образе]
    $\sqsupset \varphi: X\to X$
    \[
    \dim Ker\varphi + \dim \Im \varphi = \dim X
    .\] 
\end{theorem}
\begin{proof}
    Утверждаем, что $X$ делится на две непересекающиеся части: ядро и образ

    $\dim \Ker \varphi = K\quad \left( \dim X = n \right) $  

    $\sqsupset \Ker\varphi = \mathcal{L}\left\{ e_1, e_2, \ldots, e_k \right\} $ 

    $X = \mathcal{L}\left\{ \underbrace{e_1, e_2, \ldots, e_k}_{\Ker \varphi}, e_{k+1}, \ldots, e_n \right\} $

    $\sqsupset x\in X\quad x = \sum_{i=1}^{k} \xi^ie_i + \sum_{j=k+1}^{n} \xi^je_j$ 

    $\sphericalangle  \varphi x = \varphi\left( \sum_{j=k+1}^{n} \xi^je_j \right)  = \sum_{j=k+1}^{n} \xi^j\varphi\left( e_j \right)\quad \forall x\in X$ 

    Так образов $e_{k+1} \ldots e_n$ хватает, чтобы разложить образ. Пока что есть только полнота: любой образ раскладывается. Надо доказать следующее: 

    $\left\{ \varphi\left( e_j \right)  \right\} _{j = k+1}^n$  -- ЛНЗ

    $\sphericalangle \sum_{j=k+1}^{n} \alpha^i\varphi(e_j) = 0 \implies  \varphi\left( \underbrace{\sum_{j=k+1}^{n} \alpha^je_j}_{z} \right) = 0 \implies z\in \Ker\varphi \implies z = \beta^1e_1 + \ldots \beta^ke_k $. Но тогда есть два разложения по первым $k$ и одновременно по $k+1 \ldots n$, но тогда они были бы линейно зависимы, а они выбраны не такими (они базис $ \implies $ есть ЛНЗ в них)

    $\implies \alpha^j = 0\quad \forall j$


\end{proof}

    \begin{note}
        $\Ker \varphi\cap \Image \varphi = \left( 0 \right) $

        $\dim \Ker\varphi = k\quad \dim \Image \varphi = n-k $

         $\dim \Ker \varphi + \dim \Image \varphi = n = \dim X$

         $\implies X = \Ker\varphi \dotplus \Image \varphi$
    \end{note}

$\sqsupset \varphi: X\to X$

\begin{definition}
    Оператор $\varphi^{-1}$ называется обратным к  $\varphi$, если:
    \[
        \varphi^{-1}\circ\varphi = I
    .\] 
    $\forall x\in X\quad \left( \varphi^{-1}\circ \varphi \right) (x) = \varphi^{-1}\left( \varphi(x) \right) $
\end{definition}

\begin{theorem}
    [Инвариантное условие существования $\varphi^{-1}$] 
    $\exists \varphi^{-1} \iff \begin{cases}
        \Ker\varphi = \{0\}\\
        \Image \varphi = X
    \end{cases}$
\end{theorem}
\begin{note}
    $\varphi(x) = y \implies x = \varphi^{-1}(y)$ 
    
    $x \longleftrightarrow \xi^j\quad y\longleftrightarrow \eta^k\quad \varphi \longleftrightarrow \|a^i_m\|$

     $\eta^k = \sum_{i=1}^{n} a_i^k\xi^i$ -- в такой записи нужно искать решение системы уравнений

     $\eta = A\xi$ 

     Нужно, чтобы система была совместна и определена (крамеровская). ш

     $\sphericalangle \sum_{i=1}^{n} a^k_i = 0$ : Если существует единственно решение такого, то существует обратный оператор (по теорема Фредгольма)
\end{note}

\begin{statement}
    $\exists \varphi^{-1} \iff \Ker\varphi = \{0\}$
\end{statement}
\begin{proof}
    [Доказательство теоремы]

    $\exists \varphi^{-1} \implies \forall y\quad \varphi x = y$ имеет решение $\left[ x = \varphi^{-1}y \right] $

    $\sqsupset \Ker\varphi = \{0\} \implies \dim \Image \varphi = n \implies \Image \varphi \simeq X \implies \varphi$ -- сюръекция

     $\begin{cases}
         \varphi(x_1) = y_1\\
         \varphi(x_2) = y_2\\
     \end{cases}$ 

     $\varphi\left( x_1-x_2 \right) =y_1-y_2\neq 0 \implies  x_1-x_2\not\in \Ker\varphi \implies x_1\neq x_2 \implies $ инъекция

     $\implies $ биекция
\end{proof}

\section{Внешняя степень оператора}

$\Lambda^n\quad \dim X = n\quad \dim X = n$

$\dim \Lambda^n = C^n_n = 1 \implies \left\{ ^{12\ldots n}F \right\} $ 

$^{12 \ldots n}F = f^1\wedge f^2\wedge \ldots \wedge f^n$

$\det \left\{ x_1x_2 \ldots x_n \right\}  = ^{12\ldots n}\left( x_1, x_2, \ldots, x_n \right)  = \sum_{\sigma}(-1)^{\left[ \sigma \right] }\xi^1_{\sigma(1)} \ldots \xi^n_{\sigma(n)} = \det A\quad A = \left[ x_1x_2 \ldots x_{n}  \right] $

$\Lambda_n\quad {}_{12 \ldots n}\hat F = \hat e_1 \wedge \hat e_2\wedge \ldots \wedge \hat e_n = \left<X^{* * } = X \right> = e_1\wedge e_2\wedge \ldots \wedge e_n$ (под негалочками будем понимать то, что с галочками. формально сняли их)

\begin{align*}    
    \sphericalangle x_1\wedge x_2 \wedge \ldots \wedge x_n &= \xi^{i_1}_1e_{i_1}\wedge\xi^{i_2}_2e_{i_2}\wedge \ldots \wedge \xi^{i_n}_ne_{i_n} \\
                                                           &= \sum{\sigma}(-1)^{\left[ \sigma \right] }\xi_1^{\sigma(1)}\xi_2^{\sigma(2)} \ldots \wedge \xi_n^{\sigma(n)} e_1\wedge e_2\wedge \ldots \wedge e_n\\
                                                           &= \det A^T e_1\wedge e_2\wedge \ldots \wedge e_n\\
                                                           &=\det \left[ x_1 x_2 \ldots x_n \right] 
.\end{align*}

\begin{lemma}
    $\det \left[ x_1 x_2 \ldots x_n \right]  = \det \left\{ x_1 x_2 \ldots x_n \right\} $
\end{lemma}


\begin{definition}
    [Внешняя степень оператора $p$]
$\sqsupset \varphi:X\to X$

$\Lambda_p\quad \left\{ _{i_1i_2 \ldots i_p}F = e_{i_1}\ldots e_{i_p} \right\}_{1\leqslant i_1< \ldots< i_p\leqslant n} $

$\sphericalangle \varphi^{\Lambda_p}:\Lambda_p \to \Lambda_p$

$\sqsupset x_1\wedge x_2\wedge \ldots \wedge x_n \in \Lambda_p$

$\varphi^{|lambda^p}\left( x_1\wedge x_2\wedge \ldots \wedge x_n \right)  = \varphi x_1 \wedge \varphi x_2 \wedge \ldots\wedge \varphi x_n$
\end{definition}

$\sqsupset p = n \implies  \implies \varphi^{\Lambda_n}:\Lambda_n \to \Lambda_n$

\begin{lemma}
    $\sqsupset z\in \Lambda_n \implies \varphi^{\Lambda_n}z = \alpha z$
\end{lemma}
\begin{proof}
    $\sqsupset z = e_1\wedge e_2\wedge \ldots \wedge e_n$

    \begin{align*}
        \varphi^{\Lambda_n}z &= \varphi e_1\wedge \varphi e_2 \wedge \ldots \wedge \varphi e_n\\
                             &= a_1^{j_1}e_{j_1}\wedge a_2^{j_2}e_{j_2}\wedge \ldots \wedge a_n^{j_n}e_{j_n} \\
                             &=\underbrace{ \sum_{\sigma}(-1)^{\left[ \sigma \right] }a_1^{\sigma(1)}a_2^{\sigma(2)}\ldots a_n^{\sigma(`n)}}_{\alpha} \overbrace{e_1\wedge e_2\wedge \ldots \wedge e_n}^z = \alpha z
    .\end{align*}

\end{proof}

\begin{definition}
    $\alpha \ov{\triangle} = \det \varphi$ -- определитель Линейного Оператора 
\end{definition}

\begin{note}
    
       $\varphi^{\Lambda^n}\left( x_1\wedge x_2 \wedge \ldots \wedge x_n \right)  = \det\varphi \cdot x_1\wedge x_2\wedge \ldots \wedge x_n$
   \begin{align*} 
       \varphi^{\Lambda^n}\left( x_1\wedge x_2 \wedge \ldots \wedge x_n \right) &= \varphi x_1\wedge \varphi x_2\wedge \ldots \wedge \varphi x_n \\
                                                                                &= \sum_{\sigma}(-1)^{\left[ \sigma \right] } \xi_1^{\sigma(1)}\ldots\xi_n^{\sigma(n)}\varphi\left( e_1\wedge e_2\wedge \ldots \wedge e_n \right) \\
                                                                                &= \det \varphi \cdot  \left( x_1\wedge x_2\wedge \ldots \wedge x_n \right)  \\
                                                                                &= \sum_{\sigma}\left( -1 \right) ^{\left[ \sigma \right] }\xi_1^{\sigma(1)}\ldots\xi_n^{\sigma(n)}\cdot \det \varphi \cdot  \left( e_1\wedge e_2\wedge \ldots \wedge e_n \right)  \\
   .\end{align*}

\end{note}

$\sqsupset \varphi, x\in End(X)$

\begin{theorem}
    $\det\left( \varphi\circ \chi \right)  = \det\varphi \cdot \det \chi$
\end{theorem}
\begin{proof}
     \begin{align*}
         \left( \varphi\chi \right) ^{\Lambda_n}\left( x_1\wedge x_2\wedge \ldots \wedge x_n \right) = \varphi\left( \chi x_1 \right) \wedge \varphi\left( \chi x_2 \right) \wedge \ldots\wedge \varphi\left( \chi x_n \right) \\
         &= \det \varphi \cdot \left( \chi x_1\wedge \chi x_2 \wedge \ldots \wedge \chi x_n \right)  \\
         &= \det \varphi\cdot \det \chi\left( x_1\wedge x_2 \wedge \ldots \wedge x_n \right)  \\
    .\end{align*}

    С другой стороны $\left( \varphi \chi \right) ^{\Lambda_n}\left( x_1\wedge x_2\wedge \ldots \wedge x_n \right)  = \det\left( \varphi\chi \right) \left( x_1\wedge x_2\wedge \ldots \wedge x_n \right) $
\end{proof}

\section{Практика: ядро и образ }

$\varphi: X\to Y$

 \begin{enumerate}
     \item $\Ker\varphi = \{x\in X\mid \varphi(x) = 0\}$ -- ЛПП $X$
     \item  $\Image \varphi = \varphi(X) = \{y\in Y\mid \exists x\in X\quad \varphi(x) = y\}$ -- ЛПП $Y$
\end{enumerate}

$\sqsupset Y = X\quad \varphi$ -- эндоморфизм

$\sqsupset \{e_j\}_{j=1}^n$ -- базис $X\quad \varphi \longleftrightarrow A_{\varphi} = \|a^i_j\|\quad \varphi\left( e_j \right)  = \sum_{i=1}^{n} a^i_je_j$ 

$x\longleftrightarrow \xi^i \implies \varphi(x) \longleftrightarrow A_{\varphi}\xi$

\begin{problem}
    Как найти ядро линейного оператора?
\end{problem}
\begin{proof}
    $\sphericalangle x:\quad \varphi(x) = 0$

    но, у нас есть базис, поэтому эквивалентно можно искать $\xi:\quad A_{\varphi}\xi = 0 \implies $ ФСР -- базис $\Ker\varphi$ 
\end{proof}

\begin{example}
    $\varphi: \R^3\to \R^3$

    $\varphi\left[ \xi^1\xi^2\xi^3 \right] ^T = \left[ \xi^2,\xi^1+\xi^3,\xi^3 \right] $

    $A_3 = \begin{bmatrix} 0&1&0\\1&0&0\\0&1&1 \end{bmatrix}  \implies  \Ker \varphi = \{0\}$
\end{example}

\begin{example}
    $D(p) = 2\cdot \frac{d^2p}{dt^2} - \frac{dp}{dt}$ 

    $A_{\varphi} = \begin{bmatrix} 0&-1&4&0\\0&0&-12&12\\0&0&0&-3\\0&0&0&0 \end{bmatrix} $

    $A_{\varphi}\xi = 0 \implies \begin{cases}
        -\xi^2 + 4\xi^3 = 0\\
        -12\xi^3 + 12\xi^4\\
        -3\xi^4
    \end{cases} \implies \xi^1 - \forall \quad \xi^1 = 1$. Все остальные нули

    В $\Ker\varphi$ лежит  $\begin{bmatrix} 1\\0\\0\\0 \end{bmatrix} $, а значит там только константы, что соответствует нашему оператору
\end{example}

\begin{example}
    $\varphi A = A^T$

    $\begin{bmatrix} 1&0&0&0\\0&0&1&0\\0&1&0&0&\\0&0&0&1 \end{bmatrix} $ 

    $\det \neq 0 \implies \Ker\varphi = \{0\}$
\end{example}

\begin{example}
    $\varphi:E_3\to E_3\quad \varphi(\vec x) = \vec x - \frac{\left( \vec x, \vec n \right) }{(\vec n, \vec n}\vec n $

    $\varphi(\vec x) = 0 \implies \vec x = \frac{\left( \vec x\vec n \right) }{\left( \vec n\vec n \right) }\vec n \implies \vec x\parallel \vec n$ 

    $A_{\varphi} = \begin{bmatrix} \frac{2}{3}&-\frac{1}{3}&-\frac{1}{3}\\-\frac{1}{3} &\frac{2}{3}&-\frac{1}{3}\\-\frac{1}{3}&-\frac{1}{3}&\frac{2}{3} \end{bmatrix} $ 

    $\begin{bmatrix} -\frac{1}{3}&-\frac{1}{3}&\frac{2}{3}\\-\frac{1}{3}&\frac{2}{3}&-\frac{1}{3}\\\frac{2}{3} &-\frac{1}{3}&-\frac{1}{3} \end{bmatrix}
     \sim 
     \begin{bmatrix} -\frac{1}{3}&-\frac{1}{3}&\frac{2}{3}\\0&1&-1\\0&-1&1 \end{bmatrix} 
     \sim  
     \begin{bmatrix} -\frac{1}{3}&-\frac{1}{3}&\frac{2}{3}\\0&1&-1\\0&0&0 \end{bmatrix}  $ 

    $\begin{cases}
        \xi^3 = 1\\
        \xi^2 = 1\\
        \xi^1 = 1
    \end{cases}$ 

    $\vec n = \left( 1, 1, 1 \right) $
\end{example}

\begin{problem}
    Найти ядро и образ ЛОп

    $A = \begin{bmatrix} 1&1&1&2&-1\\2&1&-2&-1&1\\3&-1&0&-1&1\\1&-2&2&0&0\\2&-2&-1&-3&2 \end{bmatrix} $

    $\ldots\ldots\ldots\ldots$

    $x_1, x_2 = \ldots$ -- пространство размерности 2, ядро
\end{problem}
\begin{proof}
    $\varphi(X) = \Image \varphi\quad \varphi(x) = y\quad \sqsupset y = \begin{bmatrix} \eta^1\\ \eta^2 \\ \eta^3 \\ \eta^4 \\ \eta^5 \end{bmatrix} $

     Если дописать сбоку столбец эт и преобразовывать его вместе с матрицей, то в конце когда слева у нас де строки нулей, справа есть две нулевые комбинации эт. Строим ФСР по этой системе и находим базис Образа оттуда

     НО есть способ лучше. Что такое $A$? это матрица образов базисных векторов. Значит они все лежат в образе. Более того, всё сводится к базису, а значит образ это линейная оболочка векторов из матрицы. Обычно просят найти таки базис образа, а не просто описание , но мы просто приводим её к диагональному виду и выделяем ненулевые строки.
\end{proof}

\begin{problem}
    Найти полный прообраз $a$. Это то же самое что найти один прообраз и добавить все линейные комбинации ядра. (потому что по сути мы решаем неоднородную систему линейных уравнений, а она даёт решения в виде многообразия через частное решение и линейную комбинацию
\end{problem}

\begin{problem}
    $\varphi:\R^5 \to \R^3$

    $A_{\varphi} = \begin{bmatrix} 3&1&-2&4\\2&-3&6&-5\\8&-1&2&3 \end{bmatrix} $ 

    $M\subseteq Y\quad \begin{cases}
        \xi_{61} + \xi^2 = 0\\
        \xi^1 - \xi^3 = 0\\
    \end{cases}$ 

    $\sqsupset y\in Y\quad y\in M \iff By = 0\quad B = \begin{bmatrix} 1&1&0\\1&0&-1\\0&0&0 \end{bmatrix} $

     $x\longleftrightarrow \xi$

     $A_{\varphi}x = y\in M$

     $BA_{\varphi}x = By = 0$

    Таким образом:
    \begin{itemize}
        \item Прообраз элемента это многооразие
        \item Прообраз линейного пространство это линейное пространство
    \end{itemize}
\end{problem}


\end{document}

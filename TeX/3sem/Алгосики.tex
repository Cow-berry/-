\documentclass{book}
%nerd stuff here
\pdfminorversion=7
\pdfsuppresswarningpagegroup=1
% Languages support
\usepackage[utf8]{inputenc}
\usepackage[T2A]{fontenc}
\usepackage[english,russian]{babel}
% Some fancy symbols
\usepackage{textcomp}
\usepackage{stmaryrd}
% Math packages
\usepackage{amsmath, amssymb, amsthm, amsfonts, mathrsfs, dsfont, mathtools}
\usepackage{cancel}
% Bold math
\usepackage{bm}
% Resizing
%\usepackage[left=2cm,right=2cm,top=2cm,bottom=2cm]{geometry}
% Optional font for not math-based subjects
%\usepackage{cmbright}

\author{Коченюк Анатолий}
\title{Алгоритмы и Структуры Данных}

\usepackage{url}
% Fancier tables and lists
\usepackage{booktabs}
\usepackage{enumitem}
% Don't indent paragraphs, leave some space between them
\usepackage{parskip}
% Hide page number when page is empty
\usepackage{emptypage}
\usepackage{subcaption}
\usepackage{multicol}
\usepackage{xcolor}
% Some shortcuts
\newcommand\N{\ensuremath{\mathbb{N}}}
\newcommand\R{\ensuremath{\mathbb{R}}}
\newcommand\Z{\ensuremath{\mathbb{Z}}}
\renewcommand\O{\ensuremath{\emptyset}}
\newcommand\Q{\ensuremath{\mathbb{Q}}}
\renewcommand\C{\ensuremath{\mathbb{C}}}
\newcommand{\p}[1]{#1^{\prime}}
\newcommand{\pp}[1]{#1^{\prime\prime}}
% Easily typeset systems of equations (French package) [like cases, but it aligns everything]
\usepackage{systeme}
\usepackage{lipsum}
% limits are put below (optional for int)
\let\svlim\lim\def\lim{\svlim\limits}
\let\svsum\sum\def\sum{\svsum\limits}
%\let\svlim\int\def\int{\svlim\limits}
% Command for short corrections
% Usage: 1+1=\correct{3}{2}
\definecolor{correct}{HTML}{009900}
\newcommand\correct[2]{\ensuremath{\:}{\color{red}{#1}}\ensuremath{\to }{\color{correct}{#2}}\ensuremath{\:}}
\newcommand\green[1]{{\color{correct}{#1}}}
% Hide parts
\newcommand\hide[1]{}
% si unitx
\usepackage{siunitx}
\sisetup{locale = FR}
% Environments
% For box around Definition, Theorem, \ldots
\usepackage{mdframed}
\mdfsetup{skipabove=1em,skipbelow=0em}
\theoremstyle{definition}
\newmdtheoremenv[nobreak=true]{definition}{Определение}
\newmdtheoremenv[nobreak=true]{theorem}{Теорема}
\newmdtheoremenv[nobreak=true]{lemma}{Лемма}
\newmdtheoremenv[nobreak=true]{problem}{Задача}
\newmdtheoremenv[nobreak=true]{property}{Свойство}
\newmdtheoremenv[nobreak=true]{statement}{Утверждение}
\newmdtheoremenv[nobreak=true]{corollary}{Следствие}
\newtheorem*{note}{Замечание}
\newtheorem*{example}{Пример}
\renewcommand\qedsymbol{$\blacksquare$}
% Fix some spacing
% http://tex.stackexchange.com/questions/22119/how-can-i-change-the-spacing-before-theorems-with-amsthm
\makeatletter
\def\thm@space@setup{%
  \thm@preskip=\parskip \thm@postskip=0pt
}
\usepackage{xifthen}
\def\testdateparts#1{\dateparts#1\relax}
\def\dateparts#1 #2 #3 #4 #5\relax{
    \marginpar{\small\textsf{\mbox{#1 #2 #3 #5}}}
}

\def\@lecture{}%
\newcommand{\lecture}[3]{
    \ifthenelse{\isempty{#3}}{%
        \def\@lecture{Lecture #1}%
    }{%
        \def\@lecture{Lecture #1: #3}%
    }%
    \subsection*{\@lecture}
    \marginpar{\small\textsf{\mbox{#2}}}
}
% Todonotes and inline notes in fancy boxes
\usepackage{todonotes}
\usepackage{tcolorbox}

% Make boxes breakable
\tcbuselibrary{breakable}
\newenvironment{correction}{\begin{tcolorbox}[
    arc=0mm,
    colback=white,
    colframe=green!60!black,
    title=Correction,
    fonttitle=\sffamily,
    breakable
]}{\end{tcolorbox}}
% These are the fancy headers
\usepackage{fancyhdr}
\pagestyle{fancy}

% LE: left even
% RO: right odd
% CE, CO: center even, center odd
% My name for when I print my lecture notes to use for an open book exam.
% \fancyhead[LE,RO]{Gilles Castel}

\fancyhead[RO,LE]{\@lecture} % Right odd,  Left even
\fancyhead[RE,LO]{}          % Right even, Left odd

\fancyfoot[RO,LE]{\thepage}  % Right odd,something additional 1  Left even
\fancyfoot[RE,LO]{}          % Right even, Left odd
\fancyfoot[C]{\leftmark}     % Center

\usepackage{import}
\usepackage{xifthen}
\usepackage{pdfpages}
\usepackage{transparent}
\newcommand{\incfig}[1]{%
    \def\svgwidth{\columnwidth}
    \import{./figures/}{#1.pdf_tex}
}
\usepackage{tikz}
\usepackage{listings}

\definecolor{codegreen}{rgb}{0,0.6,0}
\definecolor{codegray}{rgb}{0.5,0.5,0.5}
\definecolor{codepurple}{rgb}{0.58,0,0.82}
\definecolor{backcolour}{rgb}{0.95,0.95,0.92}

\lstdefinestyle{myStyle}{
    backgroundcolor=\color{backcolour},   
    commentstyle=\color{codegreen},
    keywordstyle=\color{magenta},
    numberstyle=\tiny\color{codegray},
    stringstyle=\color{codepurple},
    basicstyle=\ttfamily\footnotesize,
    breakatwhitespace=false,         
    breaklines=true,                 
    captionpos=b,                    
    keepspaces=true,                 
    numbers=left,                    
    numbersep=5pt,                  
    showspaces=false,                
    showstringspaces=false,
    showtabs=false,                  
    tabsize=2,
    extendedchars=\true
}

\lstset{style=myStyle}

\lstset{language=Python}
\begin{document}
    \maketitle
    
    \chapter{Алгоритмы на графах и строках и фане.}
    \section{Графы и обход в ширину}
    кружочки и стрелочки 

\begin{figure}[!ht]
    \centering
    \incfig{fex}
    \caption{fex}
    \label{fig:fex}
\end{figure}

$n$ вершин,  $m$ рёбер,  $T(n, m)$

Связность -- из любой вершины можно дойти до любой другой

$m\geqslant n-1\quad m\leqslant \frac{n(n-1)}{2}$ если связный и нет приколов с кратными рёбрами и петлами

\begin{definition}
    Матрица смежности -- матрица $m(a,b) = \begin{cases}
        1, \text{a и b связаны ребром}\\
        0, \text{иначе}
    \end{cases}$
\end{definition}

\begin{example}
    \begin{enumerate}
        \item 2, 3\\
           \item 4\\
           \item 5\\
           \item 2, 3\\
           \item 4\\
    \end{enumerate}

    более компактный и полезный способ хранить что с чем связано
\end{example}

\begin{definition}
    Компонента связаности -- класс эквивалентности по отношению эквивалентности быть связанным.
\end{definition}

\begin{definition}
    [Поиск в глубину]

    Дали нам граф. Берём вершину и помечаем все вершины, которые из неё достижимы. Всё, что мы пометили это кмпонента связности. Дальше берём непомеченную и аналогично выделяем вторую компоненту и так пока вершины не закончатся
\end{definition}

\begin{lstlisting}
    dfs(v):
        mark[v] = True
        for vu \in out(v):
            if !mark[u]:
                dfs(u)
\end{lstlisting}    
\begin{lemma}
    Мы пометили все достижимые и только их
\end{lemma}
\begin{proof}
    Ходим только по рёбрам, значит все помеченные вершины достижимы из $s$

    Есть вершина  $s$ и достижимая $v$. Предположим, что мы не дошли. Значит на пути до  $v$ была первая вершина, до которой мы не дошли. Но дошли до соседеней с ней, значит запустился оттуда и ведел непомеченную. Он её пометит в цикле, противоречие.
\end{proof}

\section{Что можно делать поиском в глубину в ориентированном графе}

\begin{definition}
    Топологическая сортировка -- сортировка вершин, чтобы все рёбра шли слева направо
\end{definition}

\begin{problem}
    Построить топологическую сортировку у ациклического графа.
\end{problem}
\begin{problem}
    В ациклическом графе есть вершина, в которую ничего не входит. Вставим самой левой в сортировке и уберём из графа.

    Возьмём любую вершину, в которую ничего не входит. Добавляем в сортировку, убираем из графа.
\end{problem}

\begin{lstlisting}
   z = []
   for v = 0 .. n-1:
        if deg[v] = 0
            z.insert(v)
    while !z.empty():
        x = z.remove()
        for y in out(x):
            deg[y]--
            if deg[y] == 0:
                z.insert(y)

\end{lstlisting}

Время алгоритма $O(m)$

\begin{proof}
    \begin{lstlisting}
        mark[v] = True
        for m in out(v):
            if !mark[u]:
                dfs(u)
        topsort.add(v)
    \end{lstlisting}
\end{proof}

\begin{statement}
    Все рёбра идут слева направо. 
\end{statement}

\begin{problem}
    Как понять есть ли циклы
\end{problem}
\begin{proof}
    Запустить topsort. Если получилась фигня, значит есть цикл.
\end{proof}

\begin{figure}[!ht]
    \centering
    \incfig{dfstree}
    \caption{dfstree}
    \label{fig:dfstree}
\end{figure}

\begin{figure}[!ht]
    \centering
    \incfig{xor}
    \caption{xor}
    \label{fig:xor}
\end{figure}


Выберем для всех рёбер, до которых мы не дошли в dfs по циклу из него и рёбер из dfs. Из получившихся циклов можно собрать (ксорами множеств) любой цикл)

\section{Связность в ориентированных графах}

\begin{note}
    Сильная связность является отношением эквивалентности. Можно выбирать классы эквивалентности -- компоненты связности. После этого можно построить конденсацию -- граф на компонентах связности, где обозначается односторонняя связь между компонентами.

\begin{figure}[!ht]
    \centering
    \incfig{dvureb}
    \caption{dvureb}
    \label{fig:dvureb}
\end{figure}
\end{note}

\begin{figure}[!ht]
    \centering
    \incfig{neprimer}
    \caption{neprimer}
    \label{fig:neprimer}
\end{figure}

Aлгоритм:
\begin{enumerate}
    \item запускаем dfs, записываем вершины в порядке выхода. 
    \item Если идти по входящим рёбрам из первой вершины в списке, то мы пометим все вершины в компоненте связности 1. Затем перейдя к следующей не помеченной вершине, мы пометим вторую компоненту и т.д.
\end{enumerate}
\begin{lstlisting}
    dfs(v):
        ...
        p.push_back(v)
        -> 1 6 2 3 7 4 5 -- dvureb
        -> 1 6 7 2 3 4 5 -- neprimer
\end{lstlisting}

\begin{lstlisting}
    for i = 0 .. n-1
        dfs1(i)
    reverse(p)
    for i  = 0 .. n-1:
        if !mark[p[i]]:
            dfs2(p[i])
\end{lstlisting}
\begin{proof}
    Возьмём компоненту связаности. Возьмём первую вершину оттуда, в которую зашёл dfs. Он оттуда не уйдёт, пока всё в ней не пометит. 

    На компонентах сильной связности есть ``топсорт'' по первым вершинам в компонентах, которые рассматривает dfs. Это гарантирует нам, что мы будем запускать dfs по обратным рёбрам в компонентах, все входящие компоменты в которую мы уже пометили. Значит dfs2 останется только идти внутри компоненты, что нам и требовалось.
\end{proof}


\begin{problem}
    [2-SAT]    
$(x\vee y)\land(!y\lor !x)\land(z\lor !x) = 1 $

$x\lor y = !x \to y$

Если есть стрелки в обе стороны между  $x $ и  $\neg x$, то это противоречие.

$A \implies B \iff  \neg B \implies \neg A $ -- кососимметричность импликации

Если есть пусть между $u$ и  $v$, то есть обратный между  $\neg v$ в  $\neg u$.

Алгоритм:
в конденсации строим топсорт (он ациклический). В каждой паре компонент, ту, которая правее, делаем True.
\end{problem}

\begin{figure}[!ht]
    \centering
    \incfig{vershinki}
    \caption{vershinki}
    \label{fig:vershinki}
\end{figure}

\begin{proof}
    Берём левую вершину. Все следствия из неё выполняются. Из неё рёбра только выходят. В ней значение False, значит все седствия выполняются. Рассмотрим симметричную к ней, она возможно где-то в середине. В неё только входят рёбра, у неё всё хорошо. Убираем их их по индукции всё хорошо.
\end{proof}

\end{document}

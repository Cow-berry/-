\documentclass{book}
%nerd stuff here
\pdfminorversion=7
\pdfsuppresswarningpagegroup=1
% Languages support
\usepackage[utf8]{inputenc}
\usepackage[T2A]{fontenc}
\usepackage[english,russian]{babel}
% Some fancy symbols
\usepackage{textcomp}
\usepackage{stmaryrd}
% Math packages
\usepackage{amsmath, amssymb, amsthm, amsfonts, mathrsfs, dsfont, mathtools}
\usepackage{cancel}
% Bold math
\usepackage{bm}
% Resizing
%\usepackage[left=2cm,right=2cm,top=2cm,bottom=2cm]{geometry}
% Optional font for not math-based subjects
%\usepackage{cmbright}

\author{Коченюк Анатолий}
\title{Дифференциальные уравнения}

\usepackage{url}
% Fancier tables and lists
\usepackage{booktabs}
\usepackage{enumitem}
% Don't indent paragraphs, leave some space between them
\usepackage{parskip}
% Hide page number when page is empty
\usepackage{emptypage}
\usepackage{subcaption}
\usepackage{multicol}
\usepackage{xcolor}
% Some shortcuts
\newcommand\N{\ensuremath{\mathbb{N}}}
\newcommand\R{\ensuremath{\mathbb{R}}}
\newcommand\Z{\ensuremath{\mathbb{Z}}}
\renewcommand\O{\ensuremath{\emptyset}}
\newcommand\Q{\ensuremath{\mathbb{Q}}}
\renewcommand\C{\ensuremath{\mathbb{C}}}
\newcommand{\p}[1]{#1^{\prime}}
\newcommand{\pp}[1]{#1^{\prime\prime}}
% Easily typeset systems of equations (French package) [like cases, but it aligns everything]
\usepackage{systeme}
\usepackage{lipsum}
% limits are put below (optional for int)
\let\svlim\lim\def\lim{\svlim\limits}
\let\svsum\sum\def\sum{\svsum\limits}
%\let\svlim\int\def\int{\svlim\limits}
% Command for short corrections
% Usage: 1+1=\correct{3}{2}
\definecolor{correct}{HTML}{009900}
\newcommand\correct[2]{\ensuremath{\:}{\color{red}{#1}}\ensuremath{\to }{\color{correct}{#2}}\ensuremath{\:}}
\newcommand\green[1]{{\color{correct}{#1}}}
% Hide parts
\newcommand\hide[1]{}
% si unitx
\usepackage{siunitx}
\sisetup{locale = FR}
% Environments
% For box around Definition, Theorem, \ldots
\usepackage{mdframed}
\mdfsetup{skipabove=1em,skipbelow=0em}
\theoremstyle{definition}
\newmdtheoremenv[nobreak=true]{definition}{Определение}
\newmdtheoremenv[nobreak=true]{theorem}{Теорема}
\newmdtheoremenv[nobreak=true]{lemma}{Лемма}
\newmdtheoremenv[nobreak=true]{problem}{Задача}
\newmdtheoremenv[nobreak=true]{property}{Свойство}
\newmdtheoremenv[nobreak=true]{statement}{Утверждение}
\newmdtheoremenv[nobreak=true]{corollary}{Следствие}
\newtheorem*{note}{Замечание}
\newtheorem*{example}{Пример}
\renewcommand\qedsymbol{$\blacksquare$}
% Fix some spacing
% http://tex.stackexchange.com/questions/22119/how-can-i-change-the-spacing-before-theorems-with-amsthm
\makeatletter
\def\thm@space@setup{%
  \thm@preskip=\parskip \thm@postskip=0pt
}
\usepackage{xifthen}
\def\testdateparts#1{\dateparts#1\relax}
\def\dateparts#1 #2 #3 #4 #5\relax{
    \marginpar{\small\textsf{\mbox{#1 #2 #3 #5}}}
}

\def\@lecture{}%
\newcommand{\lecture}[3]{
    \ifthenelse{\isempty{#3}}{%
        \def\@lecture{Lecture #1}%
    }{%
        \def\@lecture{Lecture #1: #3}%
    }%
    \subsection*{\@lecture}
    \marginpar{\small\textsf{\mbox{#2}}}
}
% Todonotes and inline notes in fancy boxes
\usepackage{todonotes}
\usepackage{tcolorbox}

% Make boxes breakable
\tcbuselibrary{breakable}
\newenvironment{correction}{\begin{tcolorbox}[
    arc=0mm,
    colback=white,
    colframe=green!60!black,
    title=Correction,
    fonttitle=\sffamily,
    breakable
]}{\end{tcolorbox}}
% These are the fancy headers
\usepackage{fancyhdr}
\pagestyle{fancy}

% LE: left even
% RO: right odd
% CE, CO: center even, center odd
% My name for when I print my lecture notes to use for an open book exam.
% \fancyhead[LE,RO]{Gilles Castel}

\fancyhead[RO,LE]{\@lecture} % Right odd,  Left even
\fancyhead[RE,LO]{}          % Right even, Left odd

\fancyfoot[RO,LE]{\thepage}  % Right odd,something additional 1  Left even
\fancyfoot[RE,LO]{}          % Right even, Left odd
\fancyfoot[C]{\leftmark}     % Center

\usepackage{import}
\usepackage{xifthen}
\usepackage{pdfpages}
\usepackage{transparent}
\newcommand{\incfig}[1]{%
    \def\svgwidth{\columnwidth}
    \import{./figures/}{#1.pdf_tex}
}
\usepackage{tikz}
\DeclareMathOperator{\Dom}{Dom}
\DeclareMathOperator{\Lip}{Lip}
\begin{document}
    \maketitle
    %
    %
    %
    % Сашин блок
    
    Связь по: mvbabushkin@itmo.ru --- просьба писать именно на почту
30 --- экзамен 
70 --- практика (будет уточняться)

Литературу пришлю

\section{Введение}
\subsection{Уравнения первого порядка}
Допустим, $y$ --- неизвестная величина. Заметим, что это не просто число, а некоторая зависимость (например, температура, зависящая от времени), то есть это некоторая искомая функция. Ну, и часто непосредственно, нам не написать чему она равна; явно эту функцию просто так не напишешь, но можно написать некую взаимосвязь между этой функцией и переменной, возможно еще и производной и т.п.

\begin{definition}
	Такая взаимосвязь называется \textbf{дифференциальным уравнением}.
\end{definition}

\begin{example}
Допустим, у нас есть кролики, заведем таблицу и будем считать кроликов каждый день.

Предположем, мы смотрели на эксперемент и обнаружили такую зависимость:
Прирост примерно пропорционален текущему клличеству и времени замерки.

\[y_{k + 1} - y_k \approx \alpha  y_k (t_{k + 1} - t_k)\]

Так же заметим, что если узмельчатьь шаг времени, то зависимость будет все более и более точная.

\[\dfrac{y_{k + 1} - y_k}{t_{k + 1} - t_k} = \alpha y_k\]

Тогда слева производная.
\[y’(t) = \alpha y(t) - g \cdot y\]
Как же получать такие формулы? Все-таки мы не привели ни одного аргумента, что эта формула верна…
Пусть этим занимаются физики, мы лишь будем решать используя эти формулы.

Попробуем поугадывать решения:

\[\varphi(t) = \alpha t \implies (\alpha t)’ = \alpha (\alpha t) \implies \alpha = \alpha ^ 2 t \implies t = \dfrac{1}{\alpha}\]

\[\varphi(t) = e^{\alpha t} \implies (e^{\alpha t})’ = \alpha e^{\alpha t} \implies \alpha e^{\alpha t} \equiv \alpha e^{\alpha t} \text{на} ~ \R \implies \varphi(t) = C e^{\alpha t} ~-~ \text{все решения}\]
\end{example}

\begin{problem}
Пусть дано $m(0) = 25$г, $m(30) = 42$г, $m(t_2) = 2m(0), ~ t_2 - ?$

То есть нам нужно найти точку на плоскости (здесь был рисунок), но она может быть где угодно, так что предположем, что у нас есть еще какие-то данные:
\[m’(t) = \alpha m(t) ~\&~ m(t) = C e^{\alpha t}\]
\end{problem}

\begin{proof}
[Решение]
\[25 = m(0) = C \quad 42 = m(30) = 25 e^{\alpha \cdot 30} \implies \alpha = \dfrac{1}{30} \ln \dfrac{42}{25} \implies 50 = m(t_2) = 25 e^{\frac{1}{30} \ln \frac{42}{25}} \implies t_2 \approx 40\]
\end{proof}

\subsection{Второй закон Ньютона}
\[F = ma \implies a = \dfrac{F(t, x, v)}{m} \implies \pp x  =\dfrac{F(t, x, \p x)}{m}\]
Так что дифференциальные уравнение встречаются очень часто --- мотивируйтесь их решать.

\section{Уравнения первого порядка и его решения}
\begin{definition}
Дифференциальное уравнение первого порядка --- это уравнение вида:
\[F(x, y, y’) = 0\]
\end{definition}

\begin{definition}
Функция $\varphi$, если:
\begin{enumerate}
\item $\varphi \in C^1(a, b)$
\item $F(x, \varphi(x), \varphi’(x)) \equiv 0, ~ x \in \in (a, b)$
\end{enumerate}
\end{definition}

\begin{example}
$y’ = -\dfrac{1}{x^2}$

$y = \dfrac{1}{x} + C, ~ x \in \R \setminus \{0\}$

\[y = \begin{cases}
\frac{1}{x} + A, x < 0 \\
\frac{1}{x} + B, x > 0
\end{cases}\]

Сейчас одна точка разрыва, а если их больше, то было бы больше независимых констант…
Поэтому, решениями являются функции на отрезке.
\end{example}

\begin{definition}
Интегральная кривая --- это график его решения.
\end{definition}

<!-- Опять рисунок -->

\begin{definition}
Общее множество решений для дифференциального уравнения --- это множество всех его решений.
\end{definition}

\begin{definition}
$F(x, y, c) = 0$

Общий интеграл --- это такой интеграл при некотором значении константы в решении которого, соотношение неявно задает все решения.
\end{definition}

\subsection{Уравнение в нормальной форме}
\begin{definition}\label{1.2}
Уравнение в неявной форме:
\[y’ = f(x, y)\]
\end{definition}

Для такого мы определим \textbf{область задания} --- это аналог ОДЗ.
\begin{definition}
Область задания --- это множество $\Dom f$ (domain) --- множество, где уравнение имеет смысл.
\end{definition}

\begin{example}
$y’ = -\dfrac{1}{x^2}, ~ f(x, y) = -\dfrac{1}{x^2}, ~ \Dom f = \R \setminus \{0\} \times \R$
\end{example}

\begin{problem}
$\sphericalangle y’ = x + y$. Пусть $\varphi$ --- решение. У нас есть такая связь: $\varphi’(x) = x + \varphi(x)$, в частности, в точке $(2, 3)$.

$\sphericalangle (2, 3) \quad \varphi’(2) = 2 + 3 = 5 = f(2, 3)$

То есть, если там проходит наша функция, то она проходит там под углом $\arctg 5$.

$\sphericalangle (4, 3) \quad \varphi’(4) = 4 + 3 = 7 = f(4, 3)$

Никто не мешает нам взять какую-то сетку, и в каждой точке этой сетки мы поймем, как примерно ведут себя интегральные кривые. То есть, можно не решая уравнения, можно построить такое поле и увидеть, как ведут себя интегральные кривые.

\begin{definition}
То есть, задать уравнение --- это значит увидеть, как ведет себя поле направлений.
\end{definition}
\end{problem}

Из этого геометрического смысла, мы можем сделать еще один вывод.

Возьмем какую-то точку (потом научимся их находить), посчитаем в ней угол, пойдем по этому направлению, новая точка --- новое направление, и т. д.
Чем мельче шаг, тем ближе ломаная к интегральной кривой.
\[x_{k + 1} = x_k + h\]
\[y_{k + 1} = y_k + f(x_k, y_k) h\]
\[\dfrac{\delta y_k}{\delta x_k} = f(x_k, y_k)\]

Так определяется \textbf{ломаная Эйлера}.

\subsection{Уравнение в дифференциалах}

Давайте запишем производную, как отношение дифференциалов, и перепишем  уравнение \ref{1.2}.
\[dy = f(x, y) dx\]
\[f(x, y) dx - dy = 0\]

\begin{definition}\label{1.3}
Уравнение в дифференциалах: $P(x, y) dx + Q(x, y) dy = 0$
\end{definition}

\begin{definition}
Функция $\varphi$ --- это решение \ref{1.3}, если:
\begin{itemize}
\item $\varphi \in C^1 (a, b)$
\item $P(x, \varphi(x)) + Q(x, \varphi(x)) \varphi’(x) \equiv 0, ~ x \in (a, b)$
\end{itemize}
\end{definition}

\begin{definition}
Область определения \ref{1.3} --- это множество $\Dom P \cap \Dom Q$
\end{definition}

\begin{example}
Пусть $xdx + ydy = 0 \qquad \Dom P \cap \Dom Q = \R^2$

$y = \sqrt{R^2 - x^2}, ~ x \in (-R, R) \quad xdx + \sqrt{R^2 - x^2} \cdot\left(- \dfrac{2x}{2 \sqrt{R^2 - x^2}}\right) dx = 0$
%<!-- Рисунок -->
\end{example}

В чем еще одна идея такого вида уравнения? В том, что $x$ и  $y$ здесь равноправны, то есть $x = k y$ --- это тоже \textit{конечное} решения.

\begin{definition}
Пара или вектор-функция $r(t) = (\varphi(t), \psi(t))$ --- это параметрическое решение уравнения \ref{1.3}, если:
\begin{itemize}
\item $\varphi, \psi \in C^1 (\alpha, \beta), r’(t) \neq 0, ~ \forall t \in (\alpha, \beta)$ --- второе условие, чтобы не было изломов у функции (как у модуля)
\item $P(\varphi(t), \psi(t)) \varphi’(t) + Q(\varphi(t), \psi(t)) \psi’(x) \equiv 0$
\end{itemize}
\end{definition}

\begin{example}
$xdx + ydy = 0 \implies (R \cos t, R \sin t), ~ t \in \R$ --- параметрическое решение.
\end{example}

$P(\varphi, \psi) \varphi’ + Q(\varphi, \psi) \psi’(x) \equiv 0 \implies (P, Q) \cdot (\varphi’, \psi’) = 0$

$F = (P, Q) \quad r = (\varphi, \psi) \quad F \perp r’$

<!-- Рисунок -->

И здесь у нас никакие направления не исключаются, в отличии от поля, где исключались вертикальные направления.
    
    % конец Сашиного блока
    %
    %
    %
    

    \section{Задача Коши и уравнения с разделяющимися переменными}

    \subsection{Задача Коши (ЗК)}

    \begin{definition}
        Задачей Коши или начальной задачей называется задача отыскания решения уравнения в нормальной форме $\p y = f(x,y)$, которая удовлетворяет начальному условию  $y(x_0) = y_0$

        \[\p y = f(x,y)\quad y(x_0) = y_0\]

        $(x_0, y_0)$ -- начальные данные
    \end{definition}

    Вопросы: есть ли решение и может ли их быть несколько?

    \begin{theorem}
        [Теорема о существовании для уравнений 1-го порядка]

        $G$ -- область (открытое связное множество),  $f\in C(G)$,  $(x_0, y_0)\in G \implies \exists $ решение задачи Коши в некоторой окрестности точки $x_0$
    \end{theorem}

    \begin{example}
        $\p y = f(x, y)\quad f(x, y) = \begin{cases}
            1&, y>0\\
            0, & y\leqslant 0\\
        \end{cases}$ 

\begin{figure}[!ht]
    \centering
    \incfig{legko}
    \caption{legko}
    \label{fig:legko}
\end{figure}
    \end{example}

    \begin{theorem}
        [Теорема единственности для уравнения 1-го порядка]

        $G$ -- область,  $f, \p g_y \in C(G), (x_0,y_0)\in G, \varphi_1, \varphi_2$ -- решения ЗК на $(\alpha, \beta) \implies \varphi_1 \equiv \varphi_2$ на $\left( \alpha, \beta \right) $
    \end{theorem}

    \begin{example}
        $\p y = 3\sqrt[3]{y^2} $ 

        $f(x, y) = 3\sqrt[3]{y^2} $ -- непрерывна везде, $G = \R^2$

        По теореме о существовании через любую точку проходит хотя бы одна интегральная кривая.

        $\p f_y = 3\cdot \frac{2}{3}\cdot \frac{1}{\sqrt[3]{y} } = \frac{2}{\sqrt[3]{y} }$ 

        На прямой $y = 0$ нарушаются условия теоремы об единственности, значит в этих точках могут (но не факт, что будут) проходить несколько интегральных кривых.

         \begin{gather*}
             dy = 2\sqrt[3]{y^2}dx\\
             y = (x-c)^3
        \end{gather*}

\begin{figure}[!ht]
    \centering
    \incfig{uzhas}
    \caption{uzhas}
    \label{fig:uzhas}
\end{figure}

Ответ: $y = (x-C)^3\quad C\in \R, x\in \R$

$y = 0, x\in \R$ -- особое решение. Имеются составные решения
    \end{example}

    \begin{definition}
        Решение $\varphi$ на  $(a, b)$ уравнения $\p y = f(x, y)$ называется \underline{особым}, если \[\forall x_0\in (a,b)\ \forall \varepsilon>0 \exists \varphi_1 \text{ -- решение задачи}, \p y = f(x,y)\quad y(x_0) = \varphi(x_0)\] 

        на $(\alpha,\beta)$, где  $\beta - \alpha < \varepsilon, x_0\in \left( \alpha, \beta \right) $, но $\varphi_1 \not\equiv \varphi$ на $\left( \alpha, \beta \right) $
    \end{definition}

    \subsection{Уравнения с разделяющимися переменными}
    ~
    \begin{definition}
        [Уравнения с разделёнными переменными]
        \[P(x)dx + Q(y)dy = 0\]
    \end{definition}

    \begin{theorem}
        [Общее решение уравнения с разделёнными переменными]
        $P\in C(a, b)\quad Q\in C(c, d)\quad (\alpha, \beta)\subset (a, b)$

        Тогда функция $y = \varphi(x)$ -- решение на $(\alpha,\beta) \iff :$
        \begin{enumerate}
            \item $\varphi\in C^1(\alpha, \beta)$
            \item  $\exists C\in \R$, т.ч. $\varphi$ неявно задаётся уравнением  $\int P(x)dx + \int Q(y)dy = C$
        \end{enumerate}
    \end{theorem}
    \begin{proof}
         \begin{itemize}
             \item []
             \item [$\implies $] Дано, что $\varphi$ -- решение  $\implies $ автоматически выполняется первый пункт.

                 $\sqsupset x_0\in (\alpha, \beta)$ -- прозвольно. $y_0 := \varphi(x_0)$, тогда пункт 2 запишется как:
                 \[\int_{x_0}^xP(x)dx + C_1 + \int_{y_0}^y Q(t)dt + C_2 = C\]
                 \[\int_{x_0}^xP(x)dx + \int_{y_0}^{\varphi} Q(t)dt = A\quad \forall x\in (\alpha, \beta)\]
                  
                 Пусть $t  = \varphi(\tau) \implies $ л.ч.

                 \[\int_{x_0}^x P(t)dt + \int_{x_0}^xQ\left( \varphi(\tau \right) \p \varphi(\tau)dt = \int_{x_0}^x\left( P(\tau) + Q\left( \varphi(\tau \right) \p \varphi(\tau) \right) dt \equiv 0 \text{ на } \left( \alpha, \beta \right) \]
             \item [$\impliedby $] Дано: $\varphi\in C^1(\alpha, \beta)$ и  $\int P(x)dx + \left[\int Q(y)dy \right]_{y = \varphi(x)}\equiv C$ на  $(\alpha, \beta)$

                 продиффиренцируем наше тождество (законно, потому что  $\varphi$ непрерывно дифференцируемо)

                 $P(x) + Q\left( \varphi(x) \right) \cdot \p \varphi(x)\equiv 0$ на $(\alpha, \beta)$
            
        \end{itemize}
    \end{proof}

    \begin{example}
        $xdx + ydy = 0$

         \begin{gather*}
            \int xdx + \int ydy = C\\
            x^2 + y^2 = 2C\\
            x^2 + y^2 = A
        \end{gather*}

        $A>0\quad 
        \begin{matrix}
            y = \pm \sqrt{A-x^2}&x\in (-\sqrt{A}, \sqrt{A})\\
            x = \pm \sqrt{A - y^2}&, y\in \left( -\sqrt{A}, \sqrt{A} \right)  
        \end{matrix}
        $               
    \end{example}

    \begin{definition}
        Уравнение с разделяющимися переменными:
        \[p_1(x)q_1(y)dx + p_2(x)y_2(y)dy = 0\]

        $p_2(x_0) = 0 \implies x \equiv x_0$ -- решение

        $q_1(t_0) = 0 \implies  y\equiv y_0$ -- решение

        Далее отдельно рассматриваем на каждой из областей (4 здесь):

        \[\frac{p_1(x)}{p_2(x)}dx + \frac{q_2(y)}{q_1(y)}dy = 0\]
    \end{definition}

    \begin{example}
        \[2ydx - xdy = 0\]

        $x = 0, y = 0$ -- решения. Нужно отдельно смотреть все четверти

         \begin{gather*}
            \frac{2dx}{x} = \frac{dy}{y}\\
            2\ln |x| = \ln |y| + C\\
            y = Ax^2, A>0, x>0
        \end{gather*}

        В остальных четвертях аналогично. Они все стыкуются в нуле и общее решение -- всевозможные стыковки.
\begin{figure}[!ht]
    \centering
    \incfig{gohan1}
    \caption{gohan1}
    \label{fig:gohan1}
\end{figure}
    \end{example}

    \begin{example}
        \[ydx - xdy = 0\]

        \begin{gather*}
            \int \frac{dx}{x} = \int \frac{dy}{y}  + C\\
            \ln |x| = \ln |y| + C\\
            y = Ax
        \end{gather*}

\begin{figure}[!ht]
    \centering
    \incfig{gohan2}
    \caption{gohan2}
    \label{fig:gohan2}
\end{figure}
    \end{example}

    \begin{definition}
        Два уравнения называют \underline{эквивалентными}, если они имеют одинаковую область задания и одинаковый набор интегральных кривых.
    \end{definition}
    \begin{note}
    \[p_1(x)q_1(y)dx + p_2(x)q_2(y)dy = 0\]
    не экивалентно \[\frac{p_1(x)}{p_2(x)}dx +\frac{q_1(y)}{q_2(y)}dy = 0\]
    \end{note}

    \begin{theorem}
        [Теорема о существовании и единственности для уравнений с разделёнными переменными]

        $P\in C(a, b)\quad Q\in C(c, d)$

        $(x_0, y_0)$ -- не особая точка уравнения (т.е.  $P(x_0)\neq 0, Q(y_0)\neq 0) \implies $ уравнение \[\int_{x_0}^xP(t)dt + \int_{y_0}^yQ(t)dt = 0\])

        определяет единственное решение в некоторой окрестности точки $x_0$
    \end{theorem}
    
    \section{Некоторые типы уравнений, интегрируемых в квадратурах}

    решение уравнения -- конечное число арифметических действий, суперпозиции и взятия интегралов от обеих частей

    \subsection{Линейные уравнения}

    \begin{definition}
        $\p y = p(x)y + q(x)$ -- линейное уравнения (ЛУ)

        $\p y = p(x)y$ -- однородное линейное уравнения (ЛОУ)
    \end{definition}
    \begin{lemma}
        [Общее решение линейного однороднго уравнения]

        $\sqsupset p\in C(a,b) \implies y = Ce^{\int p}, C\in \R, x\in (a,b)$ -- Общая запись решения ЛОУ
    \end{lemma}
    \begin{proof}
        $dy = p(x)ydx$

        $y=0$ -- решение

        $\int \frac{dy}{y} = \int p(x)dx + C$

        $\ln |u| = \int p + C$

        При $y>0\quad y = Ae^{\int p}, A >0$

        При $y<0\quad y = Ae^{\int p}, A <0$

        $y = 0$ не особое по теореме об единственности из прошлой секции.
    \end{proof}

    \begin{theorem}
        [Общее решение линейного уравнения]

        $\sqsupset p, q\in C(a,b)$

        $\implies y = \left( C + \int q e^{-\int p} \right) e^{\int p}\quad C\in \R\quad x\in (a,b) $
    \end{theorem}
    \begin{proof}
        Подстановкой убеждаемся, что  все эти функции -- решения.

        Пусть есть ещё решения $\varphi$ -- решение ЛУ на  $\left( \alpha, \beta \right) $, и оно не задаётся формулой из условия теоремы.

        Возьмём любую точку, через которую проходит это решение, $(x_0, \varphi(x_0))$. Подставим в общую формулу эту точку, то выразим $C$

        $y_0 = (C + E(x_0))F(x_0)\quad C = \frac{y_0}{F(x_0)} - E(x_0)$

        По теореме об единственности новое решение совпадает с решение с константой $C$ везде, где оба определены --  на $(\alpha, \beta)$, а тогда  $\varphi$ задаётся формулой, противоречие
    \end{proof}

    \subsection{Метод вариации постоянной или Метод Лагранжа}

    \begin{enumerate}
        \item вместо ЛУ решаем соответствующее ЛОУ

            $y\p y = p(x)y\quad t = Ce^{\int p}$

        \item заменяе  $C$ на  $C(x)$ и подставляем $y = C(x)e^{\int p}$ в исходное ЛУ уравнение
            
            \begin{gather*}
                \left( Ce^{\int p} \right) ^{\prime} = pCe^{\int p} + q\\
                \p C e^{\int p} + Ce^{\int p}\cdot p = pCe^{\int p} + q\\
                \p C = qe^{-\int p}
           \end{gather*}                    
       \item находим $C = \int qe^{-\int p} + C_1$
       \item Подставляем $C(x)$ вместо  $C$  в решение (ЛОУ)

           $y = \left( C_1 + \int q e^{-\int p} \right)e^{\int p} $
    \end{enumerate}

    \subsection{Уравнение Бернулли}

    \begin{definition}
        $\p y = p(x)y + q(x)y^{\alpha}\quad \alpha\not\in \left\{ 0,1 \right\} $
    \end{definition}

    Замена на $z = y^{1-\alpha}$ сводит его к ЛУ

     \begin{gather*}
         \frac{\p y}{y^{\alpha}} = p(x)y^{1-\alpha} + q(x)\\
         \p z = (1-\alpha) \frac{\p y}{y^{\alpha}}\\
         \p z = (1-\alpha)p(x)z + (1-\alpha)q\left( x \right) \\
         f(x,y) = py + qy^{\alpha}\\
         \p f_y = p + q y^{\alpha-1}\\
    \end{gather*}
    $\alpha \geqslant 1 \implies $ непрерывна, следовательно по теореме об единственности $y = 0$ -- не особое

    \subsection{Уравнение Риккати}
    \begin{definition}
        $\p y = p(x)y^2 + q(x)y + r(x)$ -- квадратичная функция от $y$
    \end{definition}
    \begin{statement}
        [Лиувилль] Уравнение \[\p y = y^2+x^{\alpha}\]
        интегрируется в квадратурах $\iff \frac{\alpha}{2\alpha + 4}\in \Z $ или $\alpha = -2$
    \end{statement}

    Если известно решение $\varphi$, то подставновка  $y = z + \varphi$ сводится к уравнению Бернулли

    \subsection{Уравнение в полных дифференциалах}

    \begin{definition}
        \[P(x,y)dx + Q(x,y)dy = 0\]

        Если $\exists u: \p u_x = P\quad \p u_y = Q$
    \end{definition}
    \begin{theorem}
        [Общее решение УПД]

        $P, Q\in C(G)\quad G\subseteq \R^2$ -- область, $\p u_x = P, \p u_y = Q$

        $y = \varphi(x)$ -- решение УПД  на $(a, b) \iff $ 
        \begin{enumerate}
            \item $\varphi\in C^1(a,b)$ 
            \item $\exists C: \quad U(x, \varphi(x)) \equiv C$ на $(a,b)$ (т.е.  $\varphi$ неявно задано уравнением  $u(x,y) = C$)
        \end{enumerate}
    \end{theorem}
    \begin{proof}
        \begin{itemize}
            \item []
            \item [$\implies $]
                \begin{enumerate}
                    \item Выполняется по определению решения
                    \item Имеем $P(x,\varphi(x)) + Q(x, \varphi(x))\p \varphi(x)\equiv 0$

                        $P(x,\varphi(x)) + Q(x, \varphi(x))\p \varphi(x) = \left( u\left( x, \varphi(x) \right)  \right) ^{\prime} = 0 \implies u(x, \varphi\left( x \right) ) = C$

                \end{enumerate}
            \item[$\impliedby $] Имеем $u(x, \varphi(x)) \equiv C$, дифференцируем:
                 \begin{gather*}
                     \p u_x(x, \varphi(x)) + \p u_y(x, \varphi(x))\p \varphi(x) \equiv 0\\
                     P(x, \varphi(x)) + Q(x, \varphi(x))\p \varphi(x) \equiv 0
                \end{gather*}
        \end{itemize}
    \end{proof}

    Предположим $u\in C^2(G)\quad \p u_x = P\quad \p u_y = Q\quad \p P_y = \p u_{xy} = \p u_{yx} =\p Q_x $
    \begin{statement}
    Условие $\p P_y = \p Q_x$  -- достаточное для того, чтобы уравнение было УПД, если $G$ -- односвязная область
    \end{statement}

    \begin{definition}
        Область односвязна, если любая замкнутая кривая стягивается в точку (гомотопна точке)
    \end{definition}

    \begin{definition}
        Функция $u:\quad \p u_x = P, \p u_y = Q$ называется потенциалом уравнения в полных дифференциальных (= потенциал поля  $(P,Q)$)
    \end{definition}

    Потенциал УПД находится по формуле:

    $u(x,y) = C + \int_{\gamma}P(x,y)dx + Q(x,y)dy$

    Если кривая  $\gamma \begin{cases}
        x = x(t)\\
        y = y(t)\\
        t\in [\alpha,\beta]\\
\end{cases} \implies  \int_{\gamma}P(x,y)dx + Q(x,y)dy = \int_{\alpha}^{\beta}\left( P( x(t), y(t))\p x(t) + Q(x(t), y(t))\p y(t)  \right) dt $

\subsection{Интегрирующий множитель}

\begin{definition}
    $\sqsupset \mu(x,y)\neq 0\quad \forall x, y$ и \[\mu Pdx + \mu Qdy = 0\] -- УПД $\implies \mu$ -- интегральный множитель уравнения $Pdx + Qdy = 0$
\end{definition}

\begin{statement}
    [Необходимое условие]

    (если $\mu\in C^1(G))$ \[\left( \mu P \right)^{\prime}_y = \left( \mu Q \right) _x^{\prime} \])
\end{statement}

\begin{example}
     \begin{gather*}
         \p y = p(x)y + q(x) = 0\\
         \left( p(x)y + q(x) \right) dx - dy = 0\\
         \begin{matrix}
             \p P_y = p(x)& \p Q_x = 0
         \end{matrix}
    \end{gather*}
    Будем искать $\mu$ в виде  $\mu = \mu(x) \implies  0 \cdot  P + \mu P = \p \mu (-1) + 0 \implies \p \mu = -\mu p$

    $\mu = Ce^{-\int p}$, пусть $C = 1\quad\mu = e^{-\int p}$

    Умножим исходное уравнение на  $\mu = e^{-\int p}$
     \begin{gather*}
         \p y e^{-\int p} = pye^{-\int p} + qe^{-\int p}\\
         \p ye^{-\int p} - pye^{-\int p} = qe^{\int p}\\
         \left( ye^{-\int p} \right)^{\prime} = qe^{-\int p}\\
         ye^{-\int p} = \int qe^{-\int p} + C\\
         y = \left( C + \int qe^{-\int p} \right) e^{\int p}
    \end{gather*}
\end{example}

\section{Уравнения неразрешённые относительно производной}

\begin{definition}
    $F(x,y,\p y) = 0$ -- разрешённые относительно производной
\end{definition}

\subsection{Уравнения разрешимые относительно производной}

\begin{example}
    $\left( \p y - f_1(x,y) \right) \left( \p y - f_2(x,y) \right)  = 0$

    Если $\varphi$ -- решение $\p y = f_1(x,y)$ или $\p y = f_2(x,y)$, то $\varphi$ -- решение исходного

    Обратное неверно
\end{example}

\begin{example}
    $y^{\prime 2} -4x^2 = 0$

    \begin{align*}
        \left( \p y - 2x \right) (\p y + 2) &= 0 \\
        \p y = 2x &\implies y = x^2+C \\
        \p y = -2x & \implies -x^2+C \\
    .\end{align*}

\begin{figure}[!ht]
    \centering
    \incfig{obrnev}
    \caption{obrnev}
    \label{fig:obrnev}
\end{figure}

Есть составные решения, где интегральные уравнения стыкуются в точках $\left( x_0, y_0 \right) \quad \p f_1(x_0,y_0) = f_2(x_0,y_0)$.
\end{example}

\subsection{Метод введения параметра}

$x = \varphi(t)\quad y = \psi(t)\quad t\in \left( \alpha, \beta \right) $

$\exists \varphi^{-1} \implies $ функция $\psi\circ\varphi^{-1}$ задана параметрически

\subsubsection{Неполные уравнения}

$\sphericalangle F(x,\p y) = 0$ -- уравнения множества на плоскости $\R^2_{x,\p y}$

Пусть $x = \varphi(t)\quad \p y = \psi(t)$ -- гладкая параметризация  $\gamma$  и $\exists \varphi^{-1}$

\begin{statement}
    $x = \varphi(t), y = \int \psi(t)\p \varphi(t)dt$ -- параметрически заданное решение  $F(x,\p y) = 0$
\end{statement}
\begin{proof}
    $F(x,\p y_x(x)) = F(\varphi(t), \frac{(\int \psi(t)\p \varphi(t)dt + C)^{\prime}}{\p \varphi(t)} = F\left( \varphi(t), \psi(t) \right) \equiv 0$
\end{proof}

Правило нахождения решений (некоторых) уравнения $F(x, \p y) = 0$:
 \begin{enumerate}
     \item Подобрать параметризацию множества $F(x, \p y)$ в  $\R^2_{x,\p y}$ \[x = \varphi(t)\quad \p y = \psi(t)\quad t\in \left( \alpha, \beta \right) \]
     \item В основном соотношении метода введения параметризации \[dy = \p y_x dx\] сделать подстановки \[dy = \p y_t dt \quad \p y_x = \psi(t)\quad dx = \p x_t dt = \varphi^{-1} dt\]
         \begin{align*}
             \implies \p y_t dt &=  \psi(t)\p \varphi(t) dt\\
             \p y_t &= \psi(t)\p \varphi (t)\\
             \implies y &= \int \psi(t)\p \varphi(t)dt + C \\
             x &= \varphi(t) \\
         .\end{align*}
        Так мы получили решения, заданное параметрически
\end{enumerate}         

\begin{example}
    $e^{\p y_x} + \p y_x = x$

    Пусть  $\p y_x = t$

     $x = e^t + t$

      $\sphericalangle dy = \p y_xdx$, заменим:
      \begin{gather*}
          dy = \p t_t dt\\
          \p t_x = t\\
          dx = \p x_t dt = (e^t+1)dt\\
          \implies \p y_t = t\left( e^t +1 \right) \\
          \implies y = \int t\left( e^t + 1 \right) dt = e^t\left( t-1 \right) +\frac{t^2}{2}+C
      \end{gather*}

      Ответ: $\begin{cases}
          x = t + e^t\\
          y = e^t\left( t-1 \right)  + \frac{t^2}{2}+C
      \end{cases}$
\end{example}
            
$\sphericalangle F(y,\p y) = 0$

Сделать подстановки 
\begin{align*}
    dy &= \p ydt = \p \varphi dt \\
    \p y_x &=  \psi(t)\\
    dx &=  \p x_t dt\\
    \implies \p \varphi(t) dt = \psi(t)\p x_{t} dt\\
    \p\varphi(t) = \psi(t)\p x_t\\
    \p x_t = \frac{\p \varphi(t)}{\psi(t)}\\
    x = \int \frac{\p \varphi(t)}{\psi(t)}dt + C\\
    \implies \begin{cases}
        x = \int \frac{\p \varphi}{\psi} + C\\
        y = \varphi\\
    \end{cases}
.\end{align*}
-- решение заданное параметрически

\subsection{Полное уравнение}

$F(x,y,\p y) = 0$ --уравнение множества $\sigma$ в пространстве  $\R^3_{x, y, \p y}$.

Пусть $\begin{cases}
    x = \varphi(u,v)\\
    y = \psi(u,v)\\
    \p y = \chi(u,v)\\
\end{cases}$ -- гладкая параметризация $\sigma$

Правило нахождения решений (некоторых) полного уравнения:
\begin{enumerate}
    \item Подобрать параметризацию множества $F(x,y,\p y) = 0$ в  $\R^3_{x, y, \p y}$
    \item В основном соотношении $dy = \p y_x dx$ поставить $\begin{cases}
            dy = \p y_u du + \p y_v dv = \p \psi_u du + \p \psi_v dv\\
            \p y_x = \chi(u,v)\\
            dx = \p x_u du + \p x_v dv = \p \varphi_u du + \p\varphi_v dv
    \end{cases}$ 
    (Цель -- получить уравнение, содержащее только $u,v $)
    \[\p \psi_u du  + \p \psi_v dv = \chi\cdot \left( \p \varphi_u du + \p \varphi_v dv \right) \]
\item Если $v = g(u,C)$ -- решение  уравнения сверху, то $\begin{cases}
        x = \varphi(u, g(u, C))\\
        y = \psi(u, g(u,C))\\
\end{cases}$ -- решение исходного уравнения, заданное параметрически

\end{enumerate}

\begin{example}
    $x\p y - y = \frac{\p y}{2}\ln \frac{\p y}{2}$
\end{example}
\begin{proof}
    Пусть $\begin{cases}
        x = u\\
        \p y = v\\
        y = uv - \frac{v}{2}\ln \frac{v}{2}
    \end{cases}$ 

    $\sphericalangle $ основное соотношение $dy = \p y_x dx$
     \begin{align*}
         dy &= \p y_u du + \p y_v dv = vdu  + \left( u - \frac{1}{2}\ln  \frac{v}{2} - \frac{1}{2} \right) dv\\
         \p y_x &=  v\\
         dx &=  \p x_u du + \p x_v dv = du\\
         vdu + \left( u - \frac{1}{2} \ln  \frac{v}{2} - \frac{1}{2} \right) dv = vdu\\
    .\end{align*}

    $\sphericalangle $ 2 уравнения:
     \begin{align*}
         u - \frac{1}{2}\ln v_2 - \frac{1}{2} = 0\\
         dv = 0\\
     .\end{align*}

     \begin{align*}
         v &= 2e^{2u-1} \\
         v &= c \\
     .\end{align*}

     $x=\varphi\left( u, (u,C) \right)  = u$

     $y = \psi\left( u, g(u,C) \right)  = u\cdot \left( 2 e^{2u-1} \right)  - \left( \frac{2 e^{2u-1}}{2} \right) \ln  \frac{2 e^{2u-1}}{2} = 2u e^{2u-1} - e^{2u-1}(2u-1) = e^{2u-1} =y = e^{2x-1}$ -- решение

      \begin{align*}
          x &= u \\
          y&= u\cdot C - \frac{C}{2} \ln \frac{C}{2} \\
          \implies y &=  Cx - \frac{C}{2}\ln \frac{C}{2} \\
      .\end{align*} -- решение
\end{proof}

\subsection{Задача Коши для уравнения, разрешимого относительно производной}

\begin{definition}
    Задачей Коши для этого уравнения называются задачу нахождения его решений, удовлетворяюих начальным условиям:
    $\begin{cases}
        y(x_0) = y_0\\
        \p y(x_0) = \p y_0
    \end{cases}$
\end{definition}

Чтобы задача Коши имела хотя бы одно решение необходимо $F\left( x_0, y_0, \p y_0 \right)  = 0$ (согласование начальных данных)

\begin{theorem}
    [существование и единственность решени уравнения, разрешимого относительно производной]

    $G\subset \R^3$ -- область

    $F\in C^1(G)\quad \left( x_0, y_0, \p y_0 \right) \in G\quad F\left( x_0, y_0, \p y_0 \right)  = 0\quad \p F_{\p y}\left( x_0, y_0, \p y_0 \right) \neq 0$

    $\implies $ в некоторой окрестности точки $x_0 \exists !$ решение задачи Коши. \[F\left( x,y, \p y \right) = 0 \quad y(x_0) = y_0\quad \p y(x_0) = \p y_0\]
\end{theorem}

\begin{definition}
    Решение $\varphi$ на $\left( a,b \right) $ уравнения $F(x,y,\p y) = 0$ называется \underline{особым}, если $\forall x_0\in (a,b)\quad \exists \psi$ -- решение \[F(x,y,\p y) = 0\quad y(x_0) = \varphi(x_0)\quad \p y(x_0) = \varphi^{-1}(x_0)\], отличающееся от $\varphi$ в  $\forall $ сколь-угодно малой окрестности $x_0$
\end{definition}

\begin{definition}
    Множество $D = \left\{ (x,y)|\exists \p y\in \R\quad F(x,y,\p y) = 0 \text{ и } \p F_{\p y}\left( x, y, \p y \right) = 0 \right\} $ называется дискриминанной кривой
\end{definition}

Алгоритм нахождения особого решения:
\begin{enumerate}
    \item найти общий интеграл
    \item Найти дискриминантную кривую $D$, исключив $\p y$ из системы  $\begin{cases}
            F\left( x, y, \p y \right)  = 0\\
            \p F_y\left( x, y, \p y \right)  = 0
    \end{cases}$ 
\item Найти интегральные кривые, проходящие внутри.
\item Проверить найденное решение на соответствие определению особого решения 
\end{enumerate}

\begin{example}
    [Продолжение]

    \begin{enumerate}
        \item $y = e^{2x-1}\quad y = Cx - \frac{C}{2}\ln \frac{C}{2}$ 
        \item $\begin{cases}
            x\p y -y - \frac{\p y}{2}\ln  \frac{\p y}{2} = 0\\
            x - \frac{1}{2}\ln \frac{\p y}{2} - \frac{1}{2} = 0\\
        \end{cases}$ 
    
        $\ldots\implies t = e^{2x-1} \implies D = \left\{ (x,y) | y = e^{2x-1}\right\} $
    \item Интегральная кривая $y = e^{2x-1}$ лежит в  $D$
    \item Возьмём  $x_0\in \R\quad \exists ? C$ $\begin{cases}
            Cx_0 - \frac{C}{2} \ln \frac{C}{2}  = e^{2x_0-1}\\
            C = 2 e^{2x_0-1}
        \end{cases} \implies  \psi = 2e^{2x_0-1}x - e^{2x_0-1}(2x_0-1)$ 

        $\psi \not\equiv e^{2x-1}$
    \end{enumerate}
\end{example}

\section{Уравнения высшего порядка}
\subsection{Основные понятия}
\begin{definition}
    [Уравнение n-го порядка]
    \[F(x,y,\p y, \ldots, y^{(n)} = 0\]
\end{definition}

\begin{definition}
    Функция $\varphi$ на интервале $\left( a, b \right) $ -- решение такого уравнения, если:
    \begin{enumerate}
        \item $\varphi\in C^n(a,b)$
        \item Подстановка обращает уравнение в тождество \[F(x,\varphi, \p \varphi, \ldots, \varphi^{(n)}) \equiv 0\]
            на $(a,b)$
    \end{enumerate}
\end{definition}

\begin{definition}
    [Каноническое уравнение $n$-го порядка]
    \[y^{(n)} = f\left( x, y, \p y, \ldots, y^{n-1} \right) \]
\end{definition}

\begin{definition}
    Задаче Коши для канонического уравнения $n$-го порядка называют задачу нахождения его решения, удовлетворяющего начальным условиям:
     $\begin{cases}
         y(x_0) = y_0\\
         y\left( x_0 \right) =y_1\\
         \vdots\\
         y\left( x_{0} \right) =y_{n-1}\\
     \end{cases}$

     Числа $\left( x_0, y_0, \ldots, y_0^{n-1} \right) $ -- начальные данные 
\end{definition}

\begin{note}
    [Геометрический смысл задачи Коши на уравнениях второго порядка]
    $\begin{cases}
        \pp y = f\left( x, y, \p y \right)\\
        y(x_0) = y_0\\
        \p y(x_0) = \p y_0\\
    \end{cases}$        

\begin{figure}[!ht]
    \centering
    \incfig{geokasha}
    \caption{geokasha}
    \label{fig:geokasha}
\end{figure}
\end{note}
\begin{note}
    [Механический смысл]

    $x$ -- координата точки,  $t$ -- время

     $\frac{dx}{dt} = \overset \cdot x$

     $\begin{cases}
     \ddot x = f\left( t, x, \dot x \right)\\
     x\left( t_0 \right)  = x_0\\
     \dot x \left( t_0 \right)  = v_0\\
     \end{cases}$
\end{note}

\begin{theorem}
    [Сущетсвование решения Задачи Коши для канонического уравнения $n$-го порядка]

    $G$ -- область в  $\R^{n+1}\quad f\in C(G)\quad \left( x_0, y_0, \ldots, y_0^{(n-1)} \right) \implies  $ в некоторой окрестности точки $x_0\ \exists $ решение
\end{theorem}

\begin{theorem}
    [Единственность решения Задачи Коши для канонического уравнения $n$-го порядка]

    $G$ -- область в  $\R^{n+1}_{x, y, \p y, \ldots, y^{n-1}}$

    $f, \p f_y, \p f_{\p y}, \ldots, \p g_{y^{(n-1)}}\in C(G)\quad \left( x_0, y_0, \ldots, y_0^{n-1} \right) \in G$

    $\varphi_1, \varphi_2$ -- решение Задачи Коши на $(a,b) \implies \varphi_1 \equiv \varphi_2$ на $\left( a,b \right) $
\end{theorem}

\begin{definition}
    Решение $\varphi$ на $(a,b)$ называется \underline{особым}, если для любой точки $x_0\in (a,b)$ найдётся другое решение $\psi$, т.ч. \[\psi(x_0) = \varphi(x_0)\quad \p \psi(x_0) = \p \varphi(x_0), \ldots, \psi^{(n-1)}(x_0) = \varphi^{n-1}(x_0)\], при этом $\varphi\neq \psi$ в любой сколько угодно малой окрестности $x_0$
\end{definition}
\subsection{Методы понижения}

 \begin{enumerate}
     \item $F(x, y^{(k)}, \ldots, y^{(n)} = 0$ (нет $y, \p y, \ldots, y^{(k-1)}$ )

         Простейший случай $y^{(n)} = f(x)\quad \left( y^{(n-1)} \right) ^{\prime} = f(x) \implies y^{(n-1)} = \int f(x)dx + C$

         \begin{example}
             \[\pp y = \sin x\]
             \begin{gather*}
                 \p y = \int \sin xdx + C_1 = -\cos x + C_1\text{ -- первый интеграл }\\
                 y = \int\left( -\cos x+C_1 \right) dx + C_2 = -\sin x + C_1x + C_2 \text{ -- второй интеграл }\\
             \end{gather*}
         \end{example}

         В общем случае делаем замену $z = y^{(k)}$, понижаем порядок уравнения на $k$ единиц
     \item  \[F(y, \p y, \ldots, y^{(n)} = 0\] (нет $x$) Подстановка  $\p y = z(y)$ понижает порядок уравнения на  $1$

         Допустим  $y$ -- решение такого уравнения и сущесвует  $y^{-1} \implies \p y(x) = \p y\left( y^{-1}\left( y(x) \right)  \right) $, т.е. $\p y (x) = z(y(x))\quad z = \p y \circ y$

         Получим уравнение, которому удовлетворяет функция  $z$. Далее не пишем  $x$ для краткости

         $\p y  = z(y)\quad \pp y = \p {z(y)} = \p z(y)\p y \implies y^{(3)} = \ldots = \pp z(y) \cdot  z(y)^2 + \left( \p z(y) \right) ^2 \cdot  z(y)$
        
         $F(y, z(y), \p z(y)z(y), \pp z(y)z(y)^2 + \p z(y)^2z(y))\equiv 0$ при $x\in (a,b)$ или при  $y\in (A,B)$

         Таким образом  $z$ -- решение уравнения  $F(y, z, \p z z, \pp zz^2+ z^{\prime 2}z = 0$

         \begin{example}
             $\pp y + y\p y = 0\quad y(0) = 2\quad \p y(0) = -2$

             решаем  $\pp y = -y\p y$

             т.е.  $f(x,y, \p y) = -y\p y $

             $f\in C\left( \R^3_{x, y, \p y} \right) \implies \forall $ Задачи Коши имеется решение

             $\p f_y = -\p y\quad \p f_{\p y} = -y$

             $\p f_y, \p f_{\p y}\in C\left( \R^3_{x, y, \p y} \right) \implies \forall $ Задачи Коши имеет единственно решение

             Сделаем подстановку $\p y = z(y) \implies \pp y = \p z(y)\p y = \p zz \implies \p zz+yz = 0\quad z\left( \p z + y \right) = 0$

             $z = 0$, т.е.  $\p y = 0$ не даёт решения Задачи Коши

             $\sphericalangle \p z = -y\quad z = \int (-y)dy + C = -\frac{y^2}{2}+C_1$

             Воспользуемся начальными данными:

             $\p y(0) = \frac{-y_0^2}{2}+C_1\quad -2 = -\frac{2^2}{2} + C_1 \implies C_1 = 0$ 

             $\p y = - \frac{y^2}{2}$ и $y = 0$ -- не решение,  $-\int \frac{2dy}{y^2} = \int dx \implies \frac{2}{y} = x+C_2$ 

             начальные данные $\implies \frac{2}{2} = 0+C_2 \implies C_2 = 1$ 

             $y = \frac{2}{x+1}$ 
         \end{example}
\end{enumerate}
\subsection{}

\begin{definition}
    \[F(x, y, \p y, \ldots, y^{(n)} = 0\]-- уравнение однородное отночительно $y, \ldots, y^{(n)}$, если $\forall t\quad F(x,ty, t\p y, \ldots, ty^{(n)} = t^{\alpha} F(x, y, \p y, \ldots, y^{(n)})$

    Замена $z = \frac{\p y}{y}$ понижает порядок уравнения
\end{definition}
\subsection{ Уравнение в точных производных}

\begin{definition}
    Уравнение $F(x,y, \p y, \ldots, y^{(n)}) = 0$ -- уравнение в точных производных, если $\exists \phi = \phi(x,y,\p y, \ldots, y^{(n-1)}$, что \[\frac{d}{dx}\phi(x,y,\p y, \ldots, y^{(n-1)}) = F\left( x, y, \p y, \ldots, y^{(n)} \right)  \]
\end{definition}

\begin{statement}
    Пусть $F(x,y, \ldots, y^{(n)})$ -- уравнение в точных производных и функция $\phi$ удовлетворяет определению. Тогда  $y$ -- решение  \[\iff \exists C:\quad \phi\left( x, y, \p y, \ldots, y^{(n-1)} \right) = C\]
\end{statement}


\section{Системы дифферецниальных уравнений}

\subsection{Нормальная система}

\begin{definition}
    [Нормальная система дифферецниальных уравнений порядка $n$]

    \[\begin{cases}
        \p y_1 = f_1\left( x, y_1, \ldots, y_{n}  \right)\\
        \p y_2 = f_2\left( x, y_1, y_2, \ldots, y_{n}  \right) \\
        \ldots\\
        \p y_{n}  = f_n\left( x, y_1, \ldots, y_{n}  \right) \\
    \end{cases}\]

    Положим $Y = \left( y_1 \ldots y_{n}  \right) ^T\quad F(x,Y) = \left( f_1(x,Y), \ldots, f_n\left( x, Y \right) \right)^T $
\end{definition}

\begin{definition}
    [Нормальное $n$-мерное уравнение]

    \[\p Y = F(,Y)\]
\end{definition}

\begin{definition}
    Вектор-функция $\varphi$ называется решением нормального  $n$-мерного уравнения, если
     \begin{enumerate}
         \item $\varphi\in C^1\left( (a,b)\to \R^n \right) $
         \item $\p\varphi(x) \equiv F(x,\varphi(x))$ на  $\left( a, b \right) $
    \end{enumerate}
\end{definition}

\begin{definition}
    Интегральная кривая нормального $n$-мерного уравнения -- графики его решений
\end{definition}

\begin{definition}
    Задача Коши для нормального уравнения $n$-мерного порядка называется задача нахождения его решени, удовлетворяющего начальному условию  $Y(x_0) = Y_0\quad \left( Y_0\in \R^n \right) $
\end{definition}

\subsection{Сведение уравнения к системе}

\begin{definition}
    Пусть $\Lambda_n: C^{n-1} \to \left( C(a,b) \right) ^n$ \[\Lambda_n y = \left( y, \p y, \ldots, y^{(n-1)} \right) ^T\]
\end{definition}

\begin{theorem}
    [об эквивалентной системе]

    Пусть $\varphi$ -- решение на  $(a,b)$ уравнения  \[y^{(n)} = f\left( x, y, \p y, \ldots, y^{(n-1)} \right) \]

    Тогда функция $\Phi = \Lambda_n \varphi$ -- решение на  $(a,b)$ системы
    $\begin{cases}
        \p y_1 = y_2\\
        \p y_2 = y_3\\
        \ldots\\
        \p y_{n-1} = y_{n} \\
        \p y_n = f\left( x,y_1,\ldots, y_n \right) 
    \end{cases}$ 

    И наоборот: Пусть $\Phi = \left( \varphi_1, \ldots, \varphi_n \right) ^T$ -- решение на $\left( a,b \right) $ системы уравнений сверху. Тогда $\varphi_1$ -- решение на $(a,b)$ уравнения \[y^{(n)} = f\left( x, y, \p y, \ldots, y^{(n-1)} \right) \]
\end{theorem}
\begin{proof}
    Пусть $\varphi$ -- решение уравнения.

    $\sphericalangle \varphi_k := \varphi^{(k-1)}\quad k = \overline{1,n}$

    Подставсим их в систему:
    $\begin{cases}
        \p \varphi_1 = \p \varphi = \varphi_2\\
        \p \varphi_2 = \left( \p \varphi \right)^{\prime}  = \pp \varphi = \varphi_3\\
        \ldots\\
        \p \varphi_{n-1} = \left( \varphi^{(n-2)} \right)^{\prime} = \varphi^{(n-1)}=\varphi_n\\
        \p \varphi_n = \left( \varphi^{(n-1)} \right) ^{\prime} = \varphi^{(n)} = f\left( x, \varphi(x), \ldots, \varphi^{(n-1)}\left( x \right)  \right) = f\left( x, \varphi_1(x), \ldots, \varphi_n(x) \right) 
    \end{cases}$

    И в обратную сторону: Пусть дано решение системы $\Phi = \left( \varphi_1, \ldots, \varphi_n \right) ^T \implies $

    $\begin{cases}
        \p \varphi_1 = \varphi_2\\
        \p \varphi_2 = \varphi_3\\
        \ldots\\
        \p \varphi_{n-1} = \varphi_n\\
        \p\varphi_n = f\left( x, \varphi_1, \ldots, \varphi_n \right) 
    \end{cases}$ 

    \begin{gather*}
        \pp \varphi_1 = \p \varphi_2 = \varphi_3\\
        \varphi^{(3)} = \p \varphi_3 = \varphi_4\\
        \ldots\\
        \varphi^{(n-1)} = \p \varphi_{n-1} = \varphi_n\\
        \varphi^{(n)} = \p \varphi_n = f\left( x, \varphi_1, \ldots, \varphi_n \right) = f\left( x, \varphi_1, \p \varphi_1, \pp \varphi_1 , \ldots, \varphi_1^{(n-1)} \right) 
    \end{gather*} $\implies \varphi_1$ -- решение уравнения и $\Phi = \Lambda_n \varphi_1$
\end{proof}

\subsection{Вспомогательные сведения}

\begin{definition}
    Пусть $r = \left( x_1, \ldots, x_{n}  \right) ^T\in \R^n$

    Тогда $|r| = \max_{i\in [1:n]} |x_i|$
\end{definition}

\begin{lemma}
$f\in C\left( \left[ a,b \right] \to \R^n \right) \implies $ \[\left| \int_a^b f\right| \leqslant \int _a^b |f| \]
\end{lemma}
\begin{proof}
    $\int_a^b f = \left( \int_a^bf_1, \ldots, \int_a^b f_n \right)^T $

    Левая часть: $\left| \int_a^bf \right| = \max_i \left| \int_a^bf_i \right|  $

    $\left| \int _a^b f_i \right| \leqslant \int_a^b |f_i|$ 

    Правая часть: $\int_a^b \left| f(t) \right| dt = \int_a^b \max_j \left| f_j(t) \right| dt$

    Рассмотрим неравенство для левой части для $t\in[a,b]\quad \left| f_i(t) \right| \leqslant \max_j \left| f_j(t) \right| \implies \int_a^b \left| f_i(t) \right| dt \leqslant \int_a^b \left| \max_j|f_j(t)| \right| dt = \int_a^b |f| \implies \max_i \int_a^b |f_i(t)|dt \leqslant \int |f|$

    \[\max_i \left| \int_a^b f_i\right| \leqslant \max_j \int_a^b |f_i| \leqslant \int_a^b|f| \]
\end{proof}

\begin{definition}
    пусть $A = \left( \alpha_{ij} \right) \in M_{n\times m}\left( \R \right) \quad \left( i\in [1:n]\quad j\in [1:m] \right) $

    Тогда $|A| = \max_{ij} |\alpha_{ij}|$
\end{definition}

\begin{lemma}
    $A\in M_{n\times m}(\R)\quad B\in M_{m\times k}(\R) \implies |AB| \leqslant m\cdot |A|\cdot |B|$
\end{lemma}
\begin{proof}
    $A\cdot B = \left( \gamma_{ij} \right) \quad \gamma_[ij] = \sum_{l = 1}^{m} \alpha_{il} \cdot  \beta_{lj} $ 

    $\left| \gamma_{ij} \right| \leqslant \sum_{l=1}^{m} |\alpha_{il}| \cdot  |\beta_{lj}| \leqslant \sum_{k=1}^{m} |A| \cdot  |B| = m|A|\cdot |B|$ 

    $|AB| = \max \left| \gamma_{ij} \right| \leqslant m\cdot |A|\cdot |B|$
\end{proof}

\begin{definition}
    Пусть $X,Y$ -- нормированные пространства и дана функция  $f:X\to Y$

    Тогда $\omega\left( f, h \right)  = \sup_{\|x_1-x_2\|_X\leqslant h}\|f(x_1) - f(x_2)\|_Y$ -- модуль (непрерывности) функции $f$  с шагом  $h$
\end{definition}

\begin{property}
    Пусть $X,Y$ -- нормированные пространства.  $f:X\to Y$. Тогда:
     \begin{enumerate}
         \item $\omega(f,0) = 0$
         \item  $\omega(f,h)\uparrow $ по  $h$
         \item  $f$ равномерно непрерывна на  $X \iff w\left( f,h \right) \to_0\quad h\to 0$
    \end{enumerate}
\end{property}

\begin{definition}
    Множество функций $\left\{ f_n \right\} $, где $f_n: [a,b] \to \R^m$ равностепенно непрерывное $\iff  \sup_n\omega(f_g,h)\to 0\quad h\to 0$ 
\end{definition}

\begin{note}
    Если $\left\{ f_n \right\} $ равностепенно непрерывно, то функции в нём равномерно непрерывны

    Обратное неверно. $\sphericalangle f_k(x) = kx, k>0$, они равномерно непрерывны, но супремум это $\infty $
\end{note}

\begin{theorem}
    [Арцела-Асколи]

    Пусть функции последовательности $\left( f_n \right) _{n\in \N }$  равностепенно непрерывны $\left( f_n\in C[a,b] \right) $ и $\exists M\forall n\forall x\quad \left| f_n(x)\leqslant M \right| $ 

    Тогда $\exists f\in C[a,b]$ И $\exists \left( f_{n_k} \right) $

    $f_{n_k}\rightrightarrows f$ при $k\to \infty $
\end{theorem}

\section{Теорема о существовании}
\subsection{Теорема существования}
\begin{definition}
    Функция $\varphi$ -- решение на отрезке  $[a,b]$ уравнения \[r(t) = r_0 + \int_{t_0}^t f\left( \tau, r\left( \tau \right)  \right) d\tau\] -- Эквивалентное интегральное уравнение

    Если:
    \begin{enumerate}
        \item $\varphi\in C\left( [a,b]\to \R^n \right) $
        \item $\varphi(t) \equiv r_0 + \int_{t_0}^t f\left( \tau, \varphi(\tau) \right) d\tau $ на $[a,b]$
    \end{enumerate}
\end{definition}


\begin{lemma}
    [О равносильном интегральном уравнении]
    $f\in C\left( G \to \R^n \right) , G$ -- область в $\R^{n+1}$

    $\left( t_0, r_0 \right) $

    Тогда $\varphi$ -- решение на $[a,b]$ Задачи Коши \[\dot r = f(t,r)\quad r(t_{0}) = r_0\] $\iff \varphi$ -- решение на $[a,b]$ эквивалентного интегрального уравнения
\end{lemma}
\begin{proof}
    \begin{itemize}
        \item []
        \item [$\implies $]
            $\varphi$ -- решение Задачи Коши. Проинтегрируем условие и получаем  \[\int_{t_0}^t \dot \varphi(\tau)d\tau = \int_{t_0}^t f\left( \tau, \varphi\left( \tau \right)  \right) d\tau\]

            $\varphi(t) = \underbrace{\varphi(t_0)}_{r_0} + \int _{t_0}^t f\left( \tau, \varphi(\tau) \right) d\tau$ -- верно на $[a,b]$

            $\varphi$ -- непрерывная, потому что является решением Задачи Коши.
        \item [$\impliedby $] $\varphi$ -- решение эквивалентного интегрального уравнения ($\implies \varphi\in C^1\left( [a,b] \to \R^n \right) $). Продифференцируем условие и получим $\dot \varphi(t) = f\left( t, \varphi(t) \right) $ на $[a,b]$

            И кроме того  $\varphi(t_0) = r_0 + \int_{t_0}^{t_0} = r_0$

            А тогда $\varphi$ -- решение Задачи Коши
    \end{itemize}
\end{proof}
\begin{figure}[!ht]
    \centering
    \incfig{peano}
    \caption{peano}
    \label{fig:peano}
\end{figure}
\begin{definition}
    [Отрезок Пеано]



$\Pi\subseteq G\quad M = \max_{(t,r)\in \Pi} \left| f\left( t, r \right)  \right| $

$\Pi = \left\{ \left( t, r \right) \in G \mid~ \left| t-t_0 \right| \leqslant a\quad \left| r-r_0 \right| \leqslant b \right\} \quad h = \min\left\{ a, \frac{b}{M} \right\} $ 

$[t_0-h, t_0+h]$ -- отрезок Пеано
\end{definition}

\begin{definition}
    [Ломаная Эйлера $E_N$]

    $E_N(t_0) = r_0$

    $E_N(t) = E(t_K) + f\left( t_K, E_N(t_K) \right) \left( t-t_K \right) $, если $t\in  \left( t_K, t_{K+1} \right] $ 

    $t_K = t_0 + \frac{kh}{N}$
\end{definition}

\begin{lemma}
    [Свойства $E_N$]

    $\sqsupset t\in [t_0, t_0+h]$

    \begin{enumerate}
        \item $E_N$ определена в  $t$
        \item  $\left| E_N(t) - r_0 \right| \leqslant M\left( t-t_0 \right) $
    \end{enumerate}
\end{lemma}
\begin{proof}
    База индукции: Пусть $k=1 \implies E_N$ определена на $[t_0,t_1]$ и \[\left| E_N(t) - r_0 \right|  = \left| f\left( t_0, E_N(t_0) \right) \cdot \left( t - t_K \right)  \right| \leqslant M\left( t-t_0 \right) \]

    Переход: Пусть в выполнено на $[t_0, t_K]$ для $k\in [1,N-1]$ 

    $\sphericalangle [t_K, t_{K+1}]$

    Верно ли, что $[t_K, E_N(t_K)]\in G$

    \[\left| E_N(t_K) - E_N(t_0) \right| \leqslant M\left( t_K - t_0 \right) \] по индукционному предположению

    $E_N(t) = E_N(t_K) + f\left( t_K, E_N(t_K) \right) \left( t-t_K \right) $

    $\implies \left| t_K - t_0 \right| \leqslant h\leqslant a;\quad \left| E_n(t_K) - E_N(t_0) \right| \leqslant b \implies \left( t_K, E_N(t_K)\in \Pi \subseteq G \right) $

    $t\in (t_K, t_{K+1}] \implies $

    \begin{align*}
        \left| E_N(t) - r_0 \right| \leqslant \left| E_N(t) = E_N(t_k) \right| + \left| E_N(t_K) - r_0 \right| \\
        &= \left| f\left( t_K, E_N(t_K)\left( t-t_K) \right)  \right)  \right|  + \left| E_N(t_K) - r_0 \right|  \\
        &\leqslant M\left( t-t_K \right)  + M\left( t_K - t_0 \right)  = M\left( t-t_0 \right)  \\
    .\end{align*}

\end{proof}

\begin{theorem}
    [Пеано, Существования решения Задачи Коши]

    $G \subseteq \exists ^{n+1}$ -- область, $f\in C\left( G \to \R^n \right) \quad \left( t_0, r_0 \right) \in G \implies $ Задача Коши \[\dot r = f\left( t, r \right) \quad r(t_0) = r_0\] имеет решение на отрезке Пеано 
\end{theorem}
\begin{proof}
    План:
    \begin{itemize}
        \item Последовательность ломанных Эйлера $\left( E_N \right) _{N\geqslant 1}$
        \item По теореме Арцела-Асколи выберем подпоследовательность $E_{N_m} \to \varphi$ 
        \item Докажем, что $\varphi$ -- решение эквивалентного интегрального уравнения
        \item По лемме будет следовать, что  $\varphi$ одновременно и решение задачи Коши
    \end{itemize}

    Перенесём точку в начало координат (всегда можно сделать заменой перменной). НУО $t_0 = 0, r_0 = 0$

    Также будем рассматривать только правую половину отрезка Пеано. Слева доказательство аналогичное.

    На $[0,h]$ построим последовательность  $\left( E_N \right) _{N\geqslant 1}$ ломанных Эйлера

    $\left( E_N \right) $ равностепенно ограничены, $\left( E_N \right) $ равностепенно непрерывны?

    $\left| E_N(t) \right| \leqslant \left| E_N(t) - r_0 \right|  + \left| r_0 \right| \leqslant M\left( t-t_0 \right)  + \left| r_0 \right| \leqslant Mh + 0 = Mh$. Переход к M по лемме.

    $\left| E_N(t_2) - E_N(t_1) \right| = \left| \int_{t_1}^{t_2}\p E_N(\tau)d\tau \right| \leqslant \int _{t_1}^{t_2} \left| \p E_N(\tau) \right| \tau \quad \left( t_1, t_2\in [0,h\, t_1 < t_2 \right) $

    $\p E_N\left( \tau \right)  = f\left( t_K, E_N(t_K) \right) $, если $\tau \in \left( t_K, t_{K+1} \right) \implies \left| \p E_N(\tau) \right| \leqslant M$

    $\implies \int_{t_1}{t_2} \left| \p E_N(\tau) \right| d\tau \leqslant  \int _{t_1}^{t_2}Md\tau = M\left( t_2-t_1 \right) $

    $\implies \omega\left( E_n, u \right) \leqslant Mu$

    $\implies \sup_{n\in \N } \omega\left( E_N, u \right) \leqslant M\cdot u \to 0\quad u\to 0$

    $\left( E_N \right) $ -- равномерно ограничено и равностепенно непрерывно, следовательно, по теореме Арцела-Асколи $\exists \left( E_{N_m} \right) _{m\geqslant 1}:\quad E_{N_m}\rightrightarrows \varphi$ на $[0,h]$ 

    $\varphi$ непрерывна на  $[a,b]$ по теореме Стокса-Зейделя

    Надо:  $\varphi(t) \equiv \int_0^t f\left( \tau, f\left( \tau \right)  \right) d\tau$ на $[a,b]$

    Имеем  $E_{N_m}(t) = \int_0^t \p E_{N_m}(\tau)d\tau$ 

    при  $m\to \infty \quad \varphi(t) = \lim_{m \to \infty} \int_0^t \p E_{N_m}(\tau)d\tau  $

    НУО $N_m = m$

    \begin{align*}
    \left|\int_0^t \p E_N(\tau)d\tau - \int_0^tf\left( \tau, \varphi(\tau) \right) d\tau \right|\\
    &= \left| \int_0^t\left( \p E_N(\tau) - f\left( \tau, \varphi(\tau) \right)  \right) d\tau \right|  \\
    &\leqslant \int_0^t \left| \p E_N(\tau) - f\left( \tau, \varphi(\tau) \right) d\tau \right|   \\
    &\leqslant \int_0^h \left| \p E_N(\tau) - f\left( \tau, \varphi(\tau) \right)  \right| d\tau  \\
    &= \sum_{k=0}^{N-1} \int_{t_K}^{t_{K+1}} \left| f\left( t_K, E_N(t_K)) - f\left( \tau, \varphi(\tau) \right)  \right)  \right| d\tau \\
    &\leqslant  \sum_{k=0}^{N-1} \int_{t_K}^{t_{K+1}} \omega\left( f, \left| \left( t_K, E_N(t_K) \right)  - \left( \tau, \varphi(\tau) \right)  \right|  \right) d\tau  \\
    .\end{align*}

    \begin{align*}
        \sphericalangle \left| \left( t_K,E_N(t_K) \right)  - \left( \tau, \varphi(\tau) \right)  \right| \\
        &= \left| \left( t_K - \tau, E_N\left( t_K \right)  - \varphi(\tau) \right)  \right|  \\
        &= \max\left\{ \left| t_K - \tau \right| , \left| E_N(t_K) - \varphi(t) \right|  \right\}  \\
    .\end{align*}

    $\left| t_K - \tau \right| \leqslant \frac{h}{N}$

    $\left| E_N(t_K) - \varphi(\tau) \right| \leqslant \left| E_N\left( t_K \right)  - \varphi\left( t_K \right)  \right|  + \left| \varphi(t_K) - \varphi(\tau) \right| \leqslant \|E_N - \varphi\| + \omega\left( \varphi, \frac{h}{N} \right) $ 

    $\implies \left| \left( t_K, E_N(t_K) \right)  - \left( \tau, \varphi(\tau) \right)  \right| \leqslant \max \left\{ \frac{h}{N}, \|E_N - \varphi\| + \omega\left( \varphi, \frac{h}{N} \right)  \right\}  = S_N \to 0$ при $N\to \infty $

    А тогда звершая цепочку
    \begin{align*}
    &\leqslant  \sum_{k=0}^{N-1} \int_{t_K}^{t_{K+1}} \omega\left( f, \left| \left( t_K, E_N(t_K) \right)  - \left( \tau, \varphi(\tau) \right)  \right|  \right) d\tau  \\
    &\leqslant \sum_{k=0}^{N-1} \int_{t_K}^{t_{K+1}}\omega(f, S(N)) d\tau  \\
    &= \omega\left( f, S(N) \right)  \sum_{k=0}^{N-1} \frac{h}{N} \\
    &= \omega\left( f, S(N) \right) \cdot h \\
    .\end{align*}

    $f$ равномерно непрерывна на компакте  $\Pi$

    $\implies \omega\left( f, S(N) \right) \to 0\quad N\to \infty $, т.к. $S(N)\to 0$
\end{proof}

\subsection{Условие Липшица по части переменных}

\begin{definition}
    Полный модуль непрерывности функции $f:G \to \R^n,\quad G \subseteq \R^{n+1}_{t,r}$ \[\omega(f, u, v) = \sup_{\substack{\left| t_2-t_1 \right| \leqslant u\\ \left| r_2-r_1 \right| \leqslant v\\ \left( t_1,r_1 \right) , \left( t_2,r_2 \right) \in G}} \left| f\left( t_2, r_2 \right) - f\left( t_1, r_1 \right)  \right| \]
\end{definition}

\begin{definition}
    Частный модуль непрерывности: $\omega\left( f, 0, v \right) \quad \omega\left( f, u, 0 \right) $
\end{definition}

\begin{definition}
    $f\in \Lip_r(G) \iff  \exists L\quad \omega\left( f, 0, h \right) \leqslant Lh$
\end{definition}

\begin{definition}
    $f\in \Lip_{r, loc}(G) \iff \forall (t,r)\in G \quad \exists u$ -- окрестность точки $\left( t,r\quad f\in \Lip_r\left( U\cap G \right)  \right) $
\end{definition}

\begin{lemma}
    $G\subseteq \R^{n+1}_{t,r}$

    $\p f_r\in M_n\left( C(G) \right) ,\ f\in C\left( G \to \R^n \right) $

    $K\subseteq G$ -- выпыклый компакт $\implies \omega(f,0,h)\leqslant h\cdot \|\p f_r\|h$
\end{lemma}
\begin{proof}
    $\sphericalangle g_t(r) = f(t,r)$

    При фиксированном $t$ функция  $g_t$ удовлетворяет условию предыдущей леммы  $\implies \omega\left( g_t, h \right) \leqslant h \|\p f_r\|h$

    $\sup_t \omega\left( g_t, h \right)  = \omega\left( f, 0, h \right) $

    $\p f_r = \left( g_t \right) ^{\prime}$

    (додоказать в качестве упражнения)
\end{proof}

\begin{lemma}
    $G\subseteq \R^{n+1}_{t,r}$ -- область, $\p f_r\in M_n\left( C(G) \right) , f\in C\left( G \to \R^n \right) $

    $\implies f_7in \Lip_{r,loc}(G)$
\end{lemma}

\begin{lemma}
    $G\subseteq \R^{n+1}_{t,r}$ -- область

    $f\in C\left( G \to \R^n \right) \cap  \Lip_{r,loc}(G),K\subseteq G$ -- компакт $f\in \Lip_r(K)$
\end{lemma}

\section{Теорема Единственности}
\subsection{Теорема Единственности}
\begin{lemma}
    [Тромуом]

    $\varphi\in C\left[a,b \right] \quad t_0\in \left[ a,b \right]\quad \lambda, \mu \geqslant 0  $ \[\forall t\in [a,b]\quad 0\leqslant \varphi(t) \leqslant \lambda + \mu \left| \int_{t_0}^t\varphi(\tau)d\tau \right| \] $\implies \forall t\in [a,b]\quad \varphi(t)\leqslant \lambda e^{\mu\left( t-t_0 \right) }$
\end{lemma}
\begin{proof}
    Докажем при $t \geqslant  t_0$ (при $t<t_0$ аналогично)

    $\varphi(t) \leqslant \lambda + \mu \int_{t_0}^t \varphi(\tau)d\tau =: v(t)$

    $\p v(t) = \mu \varphi(t) \leqslant \mu v$

    $\lambda > 0 \implies v(t) >0$, разделим неравенство на $v$

    $\frac{\p v(t)}{v(t)}\leqslant M$ -- проинтегрируем по $[t_0,t]$

    $\int_{t_0}^t\frac{\p v(\tau)}{v(\tau)}d\tau \leqslant \mu\left( t-t_0 \right) $

    $\ln \frac{v(t)}{v(t_0)} \leqslant  \mu(t-t_0)$ 

    $\frac{v(t)}{v(t_0)} \leqslant e^{\mu\left( t-t_0 \right) }$ 

    $v(t) \leqslant \lambda e^{\mu\left( t-t_0 \right)}$ 

    $\varphi(t) \leqslant v(t) \implies \varphi(t) \leqslant \lambda e^{\mu\left( t-t_0 \right) }$

    $\lambda = 0\quad \varphi(t) \leqslant \mu \int_{t_0}^t\varphi(\tau)d\tau<\varepsilon + \mu \int_{t_0}^t \varphi(\tau)d\tau$

    $\implies $ по доказанному $\varphi(t) \leqslant \varepsilon e^{\mu\left( t-t_0 \right) }$

    При $\varepsilon \to 0$ будет $\varphi(t) \leqslant 0$
\end{proof}

\begin{theorem}
    [Пикар, Существование единственного решения Задачи Коши]

    $G\subseteq \R^{n+1}_{t,r}$ -- область, $\left( t_0,r_0 \right) \in G$

    $f\in C\left( G \to \R^n \right) \cap  \Lip_{r, loc}(G)$

    $\implies $
    \begin{enumerate}
        \item [(i)] на отрезке Пеано $\exists $ решение \[r = f(t,r)\quad r\left( t_0 \right) =r_0\]
        \item [(ii)] $\psi_1, \psi_2$ -- решение Задачи Коши на $\left( a,b \right) \implies  \psi_1\equiv \psi_2$ на $\left( a,b \right) $
    \end{enumerate}
\end{theorem}
\begin{proof}
\begin{figure}[!ht]
    \centering
    \incfig{plankapkan}
    \caption{plankapkan}
    \label{fig:plankapkan}
\end{figure}

НУО $\left( t_0,r_0 \right)  = 0$

$\Pi = \left\{ \left( t,r \right) | \left| t \right| \leqslant A\quad \left| r \right| \leqslant B \right\} $ 

$h = \min \left\{ A, \frac{B}{\|f\|_{C(\Pi)}} \right\} $ 

$\|f\|_{C(\Pi)} = \sup_{(t,r)\in \Pi} \left| f(t,r) \right| $

Докажем существование на $[a,b]$

$\varphi_0(t)\equiv 0\quad \varphi_{k+1}(t) = \int_0^t f\left( \tau, \varphi_k(\tau) \right) d\tau, k\in \Z ^+ $

План:
\begin{enumerate}
    \item Докажем, что $\left( t, \varphi_k(t) \right) \in \Pi\quad \forall k, \forall t\in [0,h]$
    \item $\exists \varphi:\quad \varphi_k\rightrightarrows\varphi, k\to \infty $ на $[0,h]$
    \item  $\varphi$ -- решение Задачи Коши
    \item Единственность
\end{enumerate}

\begin{enumerate}
    \item База: $\varphi_0(t)\in \Pi$ -- верно

        Пусть $\forall t\in [0,h]\quad \left( t, \varphi_k(t) \right) \in \Pi$

        $t\in [0,h] \implies |t|\leqslant A$

        \begin{align*}
            \left| \varphi_{k+1}(t) \right| &\leqslant \int_0^t \left| f\left( \tau, \varphi_k(\tau) \right)  \right| d\tau\\
            &\leqslant \int_0^t \|f\|d\tau = \|f\|\cdot t \leqslant \|f\|h \leqslant\\
            &\leqslant \|f\|\frac{B}{\|f\|}\leqslant B \implies \left( t, \varphi_{k+1}(t) \right) \in \Pi  \\
        .\end{align*}
    \item Докажем: \[\forall \varepsilon > 0\quad \exists N\in \N \quad \forall m\geqslant N, k\in \N \quad \|\varphi_{m+k} - \varphi_m\|\leqslant \varepsilon\]

        Докажем индукцией по $m$, что \[\forall t\in [0,h]\quad \left| \varphi_{m+k}(t) - \varphi_{m}(t) \right| \frac{\|f\|_{C(\Pi)}L^mt^{m+1}}{(m+1)!}\] 

        $\left( f\in \Lip_r(\Pi) \text{ по лемме } \implies 7omega\left( f|_{\Pi}, 0, h \right) \leqslant Lh \right) $

        При $m = 0\quad \left| \varphi_k(t) \right|  \leqslant  \|f\|t$

        $\varphi_k(t) = \int_0^t f\left( \tau, \varphi_{k-1}(\tau) \right) d\tau$

        $\left| \varphi_k(t) \right|  \leqslant \|f\|\cdot \int_0^{t}d\tau = \|f\|t$ 

        Пусть при некотором $m\in \Z _+$ верно.
        \begin{align*}
            \sphericalangle \left| \varphi_{m+1+k} - \varphi_{m+1}(t) \right| =\\
            &= \left| \int_0^t f\left( \tau, \varphi_{m+k}(\tau) \right)d\tau - \int_0^t f\left( \tau, \varphi_m(\tau) \right) d\tau  \right| = \\
            &= \int_0^t \left| f\left( \tau, \varphi_{m+k}\left( \tau \right)  \right)  - f\left( \tau, \varphi_m(\tau) \right)  \right| d\tau \leqslant  \\
            &\leqslant \int_0^t \omega\left( f, 0, \left| \varphi_{m+k}(\tau) - \varphi\left( \tau \right)  \right|  \right) d\tau \leqslant  \\
            &\leqslant \int_0^t L \left| \varphi_{m+k}\left( \tau \right)  - \varphi_m\left(\tau  \right)  \right| d\tau \leqslant \text{ по индукции }  \\
            &\leqslant \int_0^t \frac{L\|f\cdot L^n\tau^{m+1}\|}{(m+1)!}d\tau = \\
            &= \frac{L^{m+1}\|f\|t^{m+1}}{(m+2)!} \\
        .\end{align*}

        $\implies \left| \varphi_{m+k}\left( t \right)  - \varphi_m(t) \right|  \leqslant \frac{\|f\|\cdot L^mh^{m+1}}{(m+1)!}$ 

        $\implies \|\varphi_{m+k} - \varphi_m\| \leqslant \left( \frac{\|f\|h}{m+1} \right) \cdot \frac{\left( Lh \right) ^m}{m!} -), m\to \infty $

        $\implies \left( \varphi_k \right) _{k\geqslant 1}$ фундаментальная

        $\varphi:= \lim_{k \to \infty} \varphi_k\quad \varphi\in C[0,h]$

    \item $\varphi_{k+1}(t) \equiv \int_0^t f\left( \tau, \varphi_k(t) \right) d\tau$

        $\lim_{k \to \infty} \varphi_{k+1}(t) = \lim_{k \to \infty} \int_0^t f\left( \tau \varphi_k(\tau \right) d\tau$

        $\varphi(t) = \lim_{k \to \infty} \int_0^t f\left( \tau, \varphi_K(\tau) \right) d\tau $

        \begin{align*}
            \sphericalangle \left| \int_0^t f\left( \tau, \varphi(\tau) \right)  - f\left( \tau, \varphi_k(\tau) \right)d\tau  \right| \leqslant \\
            &\leqslant  \int_0^t \left| f\left( \tau, \varphi(\tau) - f\left( \tau, \varphi_k(\tau) \right)  \right)  \right| d\tau \leqslant \\
            &\leqslant \int_0^t \omega\left( f, 0, \left| \varphi(\tau) - \varphi_k(\tau) \right|  \right) d\tau \leqslant   \\
            &\leqslant \int_0^t L \left| \varphi(\tau) - \varphi_k(\tau) \right| d\tau\leqslant   \\
            &\leqslant \|\varphi - \varphi_k\|\int_0^t Ld\tau \leqslant   \\
            &\leqslant Lh \|\varphi - \varphi_k\| \to 0, k\to \infty   \\
            \implies \varphi(t) = \int_0^t f\left( \tau, \varphi(\tau) \right) d\tau \text{ эквивалентное интегральное уравнение }
        .\end{align*}

    \item $\psi_1, \psi_2$ -- решения на $(a,b)$ 

        $\implies \begin{cases}
            \psi_1(t) = \int_0^t f\left( \tau, \psi_1(\tau) \right)d\tau \\
            \psi_2(t) = \int_0^t f\left( \tau, \psi_2(\tau) \right) d\tau\\
        \end{cases}$ 

        $\sphericalangle \chi = \left| a \right| $ 

        .....
\end{enumerate}
\end{proof}










\end{document}

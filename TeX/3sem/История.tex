\documentclass{book}
%nerd stuff here
\pdfminorversion=7
\pdfsuppresswarningpagegroup=1
% Languages support
\usepackage[utf8]{inputenc}
\usepackage[T2A]{fontenc}
\usepackage[english,russian]{babel}
% Some fancy symbols
\usepackage{textcomp}
\usepackage{stmaryrd}
% Math packages
\usepackage{amsmath, amssymb, amsthm, amsfonts, mathrsfs, dsfont, mathtools}
\usepackage{cancel}
% Bold math
\usepackage{bm}
% Resizing
%\usepackage[left=2cm,right=2cm,top=2cm,bottom=2cm]{geometry}
% Optional font for not math-based subjects
%\usepackage{cmbright}

\author{Коченюк Анатолий}
\title{Наука и технологии в истории цивилизации}

\usepackage{url}
% Fancier tables and lists
\usepackage{booktabs}
\usepackage{enumitem}
% Don't indent paragraphs, leave some space between them
\usepackage{parskip}
% Hide page number when page is empty
\usepackage{emptypage}
\usepackage{subcaption}
\usepackage{multicol}
\usepackage{xcolor}
% Some shortcuts
\newcommand\N{\ensuremath{\mathbb{N}}}
\newcommand\R{\ensuremath{\mathbb{R}}}
\newcommand\Z{\ensuremath{\mathbb{Z}}}
\renewcommand\O{\ensuremath{\emptyset}}
\newcommand\Q{\ensuremath{\mathbb{Q}}}
\renewcommand\C{\ensuremath{\mathbb{C}}}
\newcommand{\p}[1]{#1^{\prime}}
\newcommand{\pp}[1]{#1^{\prime\prime}}
% Easily typeset systems of equations (French package) [like cases, but it aligns everything]
\usepackage{systeme}
\usepackage{lipsum}
% limits are put below (optional for int)
\let\svlim\lim\def\lim{\svlim\limits}
\let\svsum\sum\def\sum{\svsum\limits}
%\let\svlim\int\def\int{\svlim\limits}
% Command for short corrections
% Usage: 1+1=\correct{3}{2}
\definecolor{correct}{HTML}{009900}
\newcommand\correct[2]{\ensuremath{\:}{\color{red}{#1}}\ensuremath{\to }{\color{correct}{#2}}\ensuremath{\:}}
\newcommand\green[1]{{\color{correct}{#1}}}
% Hide parts
\newcommand\hide[1]{}
% si unitx
\usepackage{siunitx}
\sisetup{locale = FR}
% Environments
% For box around Definition, Theorem, \ldots
\usepackage{mdframed}
\mdfsetup{skipabove=1em,skipbelow=0em}
\theoremstyle{definition}
\newmdtheoremenv[nobreak=true]{definition}{Определение}
\newmdtheoremenv[nobreak=true]{theorem}{Теорема}
\newmdtheoremenv[nobreak=true]{lemma}{Лемма}
\newmdtheoremenv[nobreak=true]{problem}{Задача}
\newmdtheoremenv[nobreak=true]{property}{Свойство}
\newmdtheoremenv[nobreak=true]{statement}{Утверждение}
\newmdtheoremenv[nobreak=true]{corollary}{Следствие}
\newtheorem*{note}{Замечание}
\newtheorem*{example}{Пример}
\renewcommand\qedsymbol{$\blacksquare$}
% Fix some spacing
% http://tex.stackexchange.com/questions/22119/how-can-i-change-the-spacing-before-theorems-with-amsthm
\makeatletter
\def\thm@space@setup{%
  \thm@preskip=\parskip \thm@postskip=0pt
}
\usepackage{xifthen}
\def\testdateparts#1{\dateparts#1\relax}
\def\dateparts#1 #2 #3 #4 #5\relax{
    \marginpar{\small\textsf{\mbox{#1 #2 #3 #5}}}
}

\def\@lecture{}%
\newcommand{\lecture}[3]{
    \ifthenelse{\isempty{#3}}{%
        \def\@lecture{Lecture #1}%
    }{%
        \def\@lecture{Lecture #1: #3}%
    }%
    \subsection*{\@lecture}
    \marginpar{\small\textsf{\mbox{#2}}}
}
% Todonotes and inline notes in fancy boxes
\usepackage{todonotes}
\usepackage{tcolorbox}

% Make boxes breakable
\tcbuselibrary{breakable}
\newenvironment{correction}{\begin{tcolorbox}[
    arc=0mm,
    colback=white,
    colframe=green!60!black,
    title=Correction,
    fonttitle=\sffamily,
    breakable
]}{\end{tcolorbox}}
% These are the fancy headers
\usepackage{fancyhdr}
\pagestyle{fancy}

% LE: left even
% RO: right odd
% CE, CO: center even, center odd
% My name for when I print my lecture notes to use for an open book exam.
% \fancyhead[LE,RO]{Gilles Castel}

\fancyhead[RO,LE]{\@lecture} % Right odd,  Left even
\fancyhead[RE,LO]{}          % Right even, Left odd

\fancyfoot[RO,LE]{\thepage}  % Right odd,something additional 1  Left even
\fancyfoot[RE,LO]{}          % Right even, Left odd
\fancyfoot[C]{\leftmark}     % Center

\usepackage{import}
\usepackage{xifthen}
\usepackage{pdfpages}
\usepackage{transparent}
\newcommand{\incfig}[1]{%
    \def\svgwidth{\columnwidth}
    \import{./figures/}{#1.pdf_tex}
}
\usepackage{tikz}
\begin{document}
    \maketitle
    
    \chapter{Наука и технологии в истории цивилиации}

    Для 5 нужно в проектах участвовать.

    Рубежный и Итоговый тесты. 

    Ещё тесты в цдо, к котоорым готовиться по методичке там же

    Проект: в течение сеестра готовим работу, посвящённую аспекту научно-технического наследия Петра I. Предполагает научную литературу.

    исследовательская яработа и выстепление по ней

    инф. технологии


    Треки:
    \begin{itemize}
        \item Инженерное дело. Теория и правктика. Развитие инженерного дела: армия, флот, фонтанные сооружения, инж. уч. зав., огненные потехи, артиллерия, 
        \item нам Пётр I вёл науки.создание академии наук, пересаживание науки из западной Европы в Россию, кунсткамера, музеи, освоение теорриторий, географические экспедиции: Беринг, 
    \item Историческая память. Сохранение памяти, образа в скульптуре, топонимике и т.п. Здесь тоже форма: текст или видео или ещё что. устно всё равно нужно представить.
    \item Петровский парадиз и новая культура. Создание Снкт-Петербурга, регулярная планировка, уникальность, культура СПб и новая культура России, которая возникаа здесь, балы, театры, фейерверки
    \end{itemize}

    \section{Наука и техника в истории первобытного мира на древнем востоке.}

    Учебник: История науки и техники, учебно-метолическое пособие 

    Темы: мегалиты на территории Северной Европы, Тайны Египетских пирамид, Астрономия в древнем Вавилоне, математика древней Индии: десятичная запись, отличие от геометрической Греции, ввод Алгебры, медицина древнего Китая

    \subsection{Счёт времени и появление письменности}

    Как люди считали время? 


    \begin{enumerate}
        \item 
    первое, самое простое -- сутки, смена дня и ночи. Первобытный человек мог это заифксировать и вести отсёт 
\item Лунный цикл (29,5 суток)
\item Времена года и год (лунные месяца поначалу) 
\item Египтяне разработали первый солнечный календарь, 365 суток. Разливался Нил и сезона было 3 по разливам Нила. Очередной год -- Сириус восходил вместе с солнцем, место восхода солнца начали отслеживать.
\item Юлианский календарь (46 до н.э.) Метонов цикл, вставные месяцы, солнечный и лунный цикл привести в соответствие, Астроном Созиген прибыл в Рим и разработал более точный календарь, который учитывал 365/25? 4-х летний цикл
\item Григорианский календарь. Случился сдвиг на 13 дней за полторы тысячи лет. Там, пасха стала нетогда и вручную откорректировали более точно с 1582 г. Пришёл в Россию только в 1918. Появилась разница между нашим и католическим рождеством, потому что церковь осталась на Юлианском. Старый новый год тоже с этим связан.
    \end{enumerate}

    первобытные люд отсчитывали годы от какого-то значимого события для коллектива.

    \subsection{Хронология}

    Эра -- точка отсчёта. Первые системы их не имели
    \begin{enumerate}
        \item Древневосточные: цикличные как 12-летние в Китае, либо по правителям или династиям как в Египте.
        \item По выборным должностным лицам. Но они выбирались каждый год и года превращались в списки имён
        \item Датировка по олимпиадам (с 776 г до н.э.) Номер олимпиады и 1,2,3 год после неё
        \item Древняя Иудея. Связана с верхним заветом. Сотворение мира -- точка отсчёта. 5508 г до н.э.
        \item Рим -- 753 г. до н.э. Аb Urbe Condita AUC
        \item 525 г  -- Дионисий расчитал дату рождения Христа по некоторым астрономическим явлениям в Библии. Нулевого года нет. За 1 до н.э. идёт 1 н.э. Чуть позже Астрономы посчитали, что Дионисий ошибся и Иисус родился в 4 г до н.э.
    \end{enumerate}

    \section{Введение в историю науки и техники}
    \subsection{Что такое наука?}

    \begin{definition}
        Наука -- любое познание, которое ведёт человек, но это слишком широкое определение. Человек начинает познавать мир, как только появляется.

        Наука -- проявления действия в человеческом обществе совокупной человеческой мысли (В.И. Вернандский)

        Наука -- есть познание с рефлексией (изучает в том числе и само себя) и доказательством. (В.Е. Еремеев) Не раньше древней Греции -- первые научные парадигмы с доказательствами

        Наука -- сфера человеческой деятельности, функцией которой является выработка и теоретическая систематизация объективных знаний о действительности (Философский энциклопедический словарь)

        Наука -- способ удовлетворить своё любопытство за государственные деньги
    \end{definition}

    Происхождение науки:
    \begin{itemize}
        \item По Вернандскому наука произошла из религии. ``В религиозных сознаниях отливались добытые в практической деятельности людей знания, который за счёт этого входили в сознания людей'
        \item Проиходит из магии
    \end{itemize}

    Обе исходят из следующих принципов:
    \begin{enumerate}
        \item Одно событие следует за другим
        \item Порядок и единообразие природных явлений
        \item Стремление к установлению повторяющейся последовательности событий 
    \end{enumerate}

    \subsection{Проблема возникновения науки}
    Три точки зрения:
    \begin{itemize}
        \item Наука как любая познавательная деятельность человека рождается вместе с человеком -- каменный век. Сложный быт $\to $ арифметика, Агрокультура -- исследования в эту сторону
        \item Наука как первая программа исследования природы. Древняя Греция, Индия, Китай. VI-V вв до н.э. В Индии и Китае связано с духовными течениями и появлением научных парадигм. Осевое время ~800 лет, появление философии и различных религий
        \item Наука в современном смысле. Экспериментальное подтверждение
    \end{itemize}

    \section{Что такое техника}

    \begin{definition}
        Техника -- обобщающее наименования сложных устройств, механизмов, систем, а также методов, процессов и технологий упорядоченной искусной деятельности.
    \end{definition}

    Техника свойственна не только человеку, но и животным. Пауки плетут паутины, грачи используют водоизмещение для добывания пищи..

    Наука и Технология очень долгое время были не связаны.

    методы научного познания:
    \begin{itemize}
        \item Индуктиыный -- от частного к общему, т.е. от единичных фактов в обощению
        \item Дедуктивный -- от общего к частному -- выдвижение гипотезы и затем её проверка эмпирическими данными.
    \end{itemize}

    Международное научное сообщество и объём научной информации удваиваются раз в 10 лет. Это считается одной из причин замедление развития теоретической науки.

    Цель курса -- кроме научных результатов описать социокультурных и мировоззренческий контекстом творчества учёных и факторов, тормозивших развитие научных идей.

    Модели развития науки:
    \begin{enumerate}
        \item кумулятивисткая. 3 стадии по О. Конту:
            \begin{enumerate}
                \item Теологическая (причины явления -- сверхъестественные)
                \item Метафизическая (причины явления -- абстрактные сущности, вода, воздух, земля)
                \item Позитивная (привила явления -- неизменные законы природы). Наука -- критерий истинности.
            \end{enumerate}
            Научные знания накапливаются и постоянно идёт прогресс. Линейная модель, но наука не шла ровно, поэтому другие модели:
        \item революционная: Алексадр Койре

            Начная революция -- такой вид новаций в науке, который кардинально меняет основные научные традиции. 3 вида научных революций:
            \begin{enumerate}
                \item Возникновение новых фундаментальных теорий
                \item Внедрение новых методов исследования
                \item Открытие новых ``миров'' (атомов и молекул, галактик, кристаллов, вирусов и др.)
            \end{enumerate}
        \item Ситуационная

            В 1970 приобретает значимость понятие ``ситуационные исследования'' Case Studies. В таких исследованиях ставится задача понять прошлое событие не как вписывающееся в единый ряд развитии, не как обладающее какими-то общими с другим событиями чертами, а как своеобразное, невозпроизводимое в других условиях.
    \end{enumerate}

    Факт и источник в истории:
    \begin{itemize}
        \item Исторический факт -- действительное, невымышленное происшествие, событие, явление в истории, которое может быть использовано для какого-либо заключения, вывода и является проверкой для предположения.
        \item Исторический источник -- письменный памятник или материальный памятник, на основе которого строится историческое исследования.
    \end{itemize}

    наличие источников, сносок с ними -- то, что определяет книги по историю написанную учёным.

    Источник, особенно письменный, -- всегда авторское произведение. Чтобы выделить в нём факты учёные используют инструмент критики.

    \begin{example}
        Доказательство подлинности Слова о полку Игореве. Существовала в единственном экземляре. Она сгорела в Москве. Была единственная копия, которая считалась подделкой. Но в берястеных грамотах, когда их обнаружили нашли обороты речи, которые были использованы в слове о Полку Игореве. Чтобы их использовать автор должен был бы знать об этих грамотах, которые откопали через полтора века.
    \end{example}

    Периодизация человеческой истории:
    \begin{itemize}
        \item Появление человка -- 2 млн. назад
        \item Ранний Паеолит -- 2 млн. назад -- 200 тыс назад
        \item Средний Палерлит -- 200 тыс -- 40 тыс
        \item Возникновение первобытного общества
            \item Нижний Палеолит -- 40 тыс -- 12 ты назад
            \item Мезолит -- 12 тыс --  6 тыс назад
            \item Неолит -- 6 тыс -- 4 тыс -- производство продуктов питания, земледелие и скодовоство
            \item Появление государственности, с которой принято связывать понятие цивилизации -- около 6 тыс назад
    \end{itemize}

    \subsection{Доцивиализационный период}

    эолиты -- затачивали один камень другими, чтобы получить инструмент. В первую очередь делались из кремня, а также яшма, роговик, халцедон, гранитный валун

    Прошло  много сотен тысяч лет между поддерживанием огня и его искусственном добывании. (высечении, выскабливании, выпиливании, высверливании)

    В позднем палеолите появляются составные орудия, каменные топоры с деревянной рукояткой, усиливающе в 2-3 раза сиул и эффективность орудия. Позднее появляются лук и стрелы.

    \subsection{Первый глобальный продовольственные кризис}

    Относительное перенаселение планеты. Чтобы прокормит одного человека охотой и собирательством необходимо 2 км${}^2$ площади. При эффективном земледелии достаточно $100m^2$

    \subsection{Неолитическая революция}

    ``Неолитическая революция'' (термин Гордона Вир Чайлда) -- переход от присваивающих к производящим формам хозяйствам. Появление технологии регулируемого обжига глины, шлифовальных каменных орудий, ткачества, сельскохозяйственных орудий (мотыга, серп). Стало возможно хранить еду в глиняных изделиях

    Развитие металлургии:
    \begin{itemize}
        \item Схема Р. Форбеса
            \begin{enumerate}
                \item Самородный металл как камень
                \item Самородный метал, обработанный ковкой (золото, серебро, метеоритное железо)
                \item Рудная металлургия (из руд получали медь, свинец, серебро, сурьму, сплавы меди)
                \item Металлургия железа.
            \end{enumerate}
        \item Схема Г. Коглена
            \begin{enumerate}
                \item Холодная, а в дальнейшем горячая ковка меди
                \item Плавление самородной меди в открытые формы простых изделий
                \item <..>
            \end{enumerate}
    \end{itemize}

    \subsection{Бронзовый век 2-3 тыс лет до н.э.}

    Металлургия бронзы Появление кочевого скотоводства и поливного земледелия. Появление письменности и первых цивилизаций. С появлением государственности развивается техника, водоснабжение, орошение, методы производства.

    В Китае не было металлургии меди.

    С 1 тыс до н.э. Сразу начался железный век
\end{document}

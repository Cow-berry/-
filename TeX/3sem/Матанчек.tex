\documentclass{book}
%nerd stuff here
\pdfminorversion=7
\pdfsuppresswarningpagegroup=1
% Languages support
\usepackage[utf8]{inputenc}
\usepackage[T2A]{fontenc}
\usepackage[english,russian]{babel}
% Some fancy symbols
\usepackage{textcomp}
\usepackage{stmaryrd}
% Math packages
\usepackage{amsmath, amssymb, amsthm, amsfonts, mathrsfs, dsfont, mathtools}
\usepackage{cancel}
% Bold math
\usepackage{bm}
% Resizing
%\usepackage[left=2cm,right=2cm,top=2cm,bottom=2cm]{geometry}
% Optional font for not math-based subjects
%\usepackage{cmbright}

\author{Коченюк Анатолий}
\title{Математический анализ, 3 семестр}

\usepackage{url}
% Fancier tables and lists
\usepackage{booktabs}
\usepackage{enumitem}
% Don't indent paragraphs, leave some space between them
\usepackage{parskip}
% Hide page number when page is empty
\usepackage{emptypage}
\usepackage{subcaption}
\usepackage{multicol}
\usepackage{xcolor}
% Some shortcuts
\newcommand\N{\ensuremath{\mathbb{N}}}
\newcommand\R{\ensuremath{\mathbb{R}}}
\newcommand\Z{\ensuremath{\mathbb{Z}}}
\renewcommand\O{\ensuremath{\emptyset}}
\newcommand\Q{\ensuremath{\mathbb{Q}}}
\renewcommand\C{\ensuremath{\mathbb{C}}}
\newcommand{\p}[1]{#1^{\prime}}
\newcommand{\pp}[1]{#1^{\prime\prime}}
\newcommand{\tl}[1]{\widetilde{#1}}
\newcommand{\tll}[1]{\widetilde{\widetilde{#1}}}
\newcommand{\ov}[1]{\overline{#1}}
% Easily typeset systems of equations (French package) [like cases, but it aligns everything]
\usepackage{systeme}
\usepackage{lipsum}
% limits are put below (optional for int)
\let\svlim\lim\def\lim{\svlim\limits}
\let\svsum\sum\def\sum{\svsum\limits}
%\let\svlim\int\def\int{\svlim\limits}
% Command for short corrections
% Usage: 1+1=\correct{3}{2}
\definecolor{correct}{HTML}{009900}
\newcommand\correct[2]{\ensuremath{\:}{\color{red}{#1}}\ensuremath{\to }{\color{correct}{#2}}\ensuremath{\:}}
\newcommand\green[1]{{\color{correct}{#1}}}
% Hide parts
\newcommand\hide[1]{}
% si unitx
\usepackage{siunitx}
\sisetup{locale = FR}
% Environments
% For box around Definition, Theorem, \ldots
\usepackage{mdframed}
\mdfsetup{skipabove=1em,skipbelow=0em}
\theoremstyle{definition}
\newmdtheoremenv[nobreak=true]{definition}{Определение}
\newmdtheoremenv[nobreak=true]{theorem}{Теорема}
\newmdtheoremenv[nobreak=true]{lemma}{Лемма}
\newmdtheoremenv[nobreak=true]{problem}{Задача}
\newmdtheoremenv[nobreak=true]{property}{Свойство}
\newmdtheoremenv[nobreak=true]{statement}{Утверждение}
\newmdtheoremenv[nobreak=true]{corollary}{Следствие}
\newtheorem*{note}{Замечание}
\newtheorem*{example}{Пример}
\renewcommand\qedsymbol{$\blacksquare$}
% Fix some spacing
% http://tex.stackexchange.com/questions/22119/how-can-i-change-the-spacing-before-theorems-with-amsthm
\makeatletter
\def\thm@space@setup{%
  \thm@preskip=\parskip \thm@postskip=0pt
}
\usepackage{xifthen}
\def\testdateparts#1{\dateparts#1\relax}
\def\dateparts#1 #2 #3 #4 #5\relax{
    \marginpar{\small\textsf{\mbox{#1 #2 #3 #5}}}
}

\def\@lecture{}%
\newcommand{\lecture}[3]{
    \ifthenelse{\isempty{#3}}{%
        \def\@lecture{Lecture #1}%
    }{%
        \def\@lecture{Lecture #1: #3}%
    }%
    \subsection*{\@lecture}
    \marginpar{\small\textsf{\mbox{#2}}}
}
% Todonotes and inline notes in fancy boxes
\usepackage{todonotes}
\usepackage{tcolorbox}

% Make boxes breakable
\tcbuselibrary{breakable}
\newenvironment{correction}{\begin{tcolorbox}[
    arc=0mm,
    colback=white,
    colframe=green!60!black,
    title=Correction,
    fonttitle=\sffamily,
    breakable
]}{\end{tcolorbox}}
% These are the fancy headers
\usepackage{fancyhdr}
\pagestyle{fancy}

% LE: left even
% RO: right odd
% CE, CO: center even, center odd
% My name for when I print my lecture notes to use for an open book exam.
% \fancyhead[LE,RO]{Gilles Castel}

\fancyhead[RO,LE]{\@lecture} % Right odd,  Left even
\fancyhead[RE,LO]{}          % Right even, Left odd

\fancyfoot[RO,LE]{\thepage}  % Right odd,something additional 1  Left even
\fancyfoot[RE,LO]{}          % Right even, Left odd
\fancyfoot[C]{\leftmark}     % Center

\usepackage{import}
\usepackage{xifthen}
\usepackage{pdfpages}
\usepackage{transparent}
\newcommand{\incfig}[1]{%
    \def\svgwidth{\columnwidth}
    \import{./figures/}{#1.pdf_tex}
}
\usepackage{tikz}
\usepackage{pgfplots}
\pgfplotsset{compat = newest}

\DeclareMathOperator{\Int}{Int}
\DeclareMathOperator{\rang}{rang}
\DeclareMathOperator{\Cl}{Cl}


\begin{document}
    \maketitle
    \chapter{Анализ нескольких переменных. Функциональные ряды. Теория меры, Криволинейные интегралы.}
    литература -- Виноградов, Виноградов-Громов
    \section{Вспоминаем}

    $O\subseteq \R^n$ -- открытое, $f:O \to R, f \in C^{N+1}(O), N\in \N $

    $[a,x]$ -- замкнутый отрезок  $\subset O, a\neq x \implies \exists x\in (a,x):$

    $f(x) = \underbrace{\sum_{k=0}^{N} \frac{d^k_af(x-a)}{k!}}_{\text{Тейлоровский многочлен порядка N}} +\, \frac{d^k_c f(x-a)}{(N+1)!}$

    $T_{N, a, f}$

     \begin{definition}
        $\alpha\in \Z _+^n$ -- пространство мультииндексов

        $\alpha = \left( \alpha_1, \alpha_2 \ldots \alpha_n \right) $ 

        $\left| \alpha \right| = \alpha_1 + \ldots + \alpha_n$

        $\alpha! = \alpha_1 \cdot  \ldots \cdot  \alpha_n$

        $f^{(\alpha)}_{(a)} = \frac{\partial^{|\alpha|}f}{\partial^{\alpha_n}x_n \ldots \partial^{\alpha_1}x_1}$
    \end{definition}

    $h = \left( h_1, \ldots, h_n \right) \quad d_a^k f(h) = \sum_{\substack{\alpha \in \left( Z_+ \right) ^n\\ |\alpha|=k}} \frac{f^{(\alpha)}(a)}{\alpha!}h^{\alpha}$

    \section{Полиномиальная форма Ньютона}

    $N\in \N \quad x = \left( x_1, \ldots, x_n \right) \in \R^n$ 

    $\left( x_1 + \ldots + x_{n}  \right) ^N = \sum_{\substack{\alpha\in \left( \Z _+ \right) ^n\\ |\alpha| = N}} \frac{N!}{\alpha!}x^{\alpha}$
    \begin{proof}
        $p(x) = \left( x_1 + x_2 + \ldots + x_{n}  \right) ^N$

        $\p p_{x_1} = N\left( x_1 + \ldots + x_{n}  \right) ^{N-1} = \p p_{x_2} = \ldots = \p p_{x_{n} }$

        $p^{\alpha} = N(N-1) \ldots \left( N - |\alpha|+1 \right)\left( x_1+\ldots+x_{n}  \right) ^{N - |\alpha|} $

        $|\alpha|\leqslant N+1 \implies p^{(\alpha)} \equiv 0$

        $|\alpha|<N \implies p^{(\alpha)}(0) = 0$

        Неноль получается, только если $|\alpha| = N\quad p^{(\alpha)}(0) = N!$

        Если подставить, то получим:

        $\sum \frac{N!}{\alpha!}x^{\alpha}\quad h = x-a = x-0$
    \end{proof}
    \section{Оценка однородных многочленов}
    \begin{definition}
        $\sum_{\substack{\alpha\in \left( Z_+ \right) ^n \\ |\alpha|=N}}^{n} b_{\alpha}x^{\alpha}$ -- однородный многочлен степени $N$

        В более широком смысле $b_{\alpha}\in \R^m$
    \end{definition}


    $\forall t\in \R\quad p\left( tx \right)  = t^Np(x)$, т.е. однородный многочлен является однородной функцией.

    \begin{statement}
        $\sqsupset p(x) = \sum_{\substack{\alpha\in \left( Z_+ \right) ^n \\ |\alpha|=N}} \frac{N!}{\alpha!}C_{\alpha}x^{\alpha}\quad C_{\alpha}\in \R^m, \sqsupset M: \|C_{\alpha}\|\leqslant M$

        Тогда $\|p(x)\|\leqslant M\left( \sqrt{n} \|x\| \right) ^N$
    \end{statement}
    \begin{proof}
        $\|p(x)\|\leqslant \sum_{\substack{\alpha\in \left( Z_+ \right) ^n \\ |\alpha|=N}}\frac{N!}{\alpha!} |x^{\alpha}|\underbrace{\|C_{\alpha}\|}_{\leqslant M}\leqslant M \sum_{\substack{\alpha\in \left( Z_+ \right) ^n \\ |\alpha|=N}}\frac{N!}{\alpha!}\underbrace{|x^{\alpha}|}_{|x_1|^{\alpha_1} \ldots |x_{n} |^{\alpha_n}} = M \cdot  \left( |x_1| + \ldots + |x_{n} | \right) ^{N}\leqslant M\left( \sqrt{n} \|x\| \right) ^N$

            $\sum_{k=1}^{n} |x_k|\cdot 1 \leqslant \sqrt{\sum_{k=1}^{n} x_k^2} \cdot \sqrt{\sum_{k=1}^{n} 1} = \|x\| \sqrt{n}  $ -- неравенство Коши, что сумма скаларяных произведений меньше произведения норм
    \end{proof}

    Формула Тейлора-Лагранжа на отображения буквально не переносится
    $f\in C^1(O)\quad f(x) - f(a) = d_c f(x-a)$

    для отображений нарушается

    $f(t) = \begin{pmatrix} \cos t\\ \sin t \end{pmatrix} a = 0, $ ``х''$ = 2\pi \quad f(x) - f(a) = 0$

    $\p f(t) = \begin{pmatrix} -\sin t\\ \cos t \end{pmatrix} $


    \begin{theorem}
        Открытое $O\subseteq  \R^n, n, m \in \N \quad N \in \Z_+, f\in C^{N+1}\left( O \to  \R^m \right)$

        $ [a,x]\in O, a\neq x$

        Тогда $f(x) - T_{N, a, f}(x)$ -- остаточный член, который оценивается так:

        $\|f(x) - T_{N, a, f}(x)\| \leqslant \frac{1}{(N+1)!}\sup_{x\in (a,x)}\| d_c^{N+1}f(x-a)\|$

        $\sum_{\substack{\alpha\in \left( \Z _+ \right) ^n \\ |\alpha|\leqslant N}} \frac{f^{(\alpha)}(a)}{\alpha!}(x-a)^{\alpha}$
    \end{theorem}
    
    \section{Отступление}

    $f, g: O \to  \R^m$, дифференцируемые в $O$

     $d \left<f, g \right> = \left<df, g \right> + \left<f, dg \right>$

     $d \left<f, g \right>(h) = \left<df(h), g \right> + \left<f, dg(h) \right>$


     $d^2\left<f, g \right> = \left<d^2f, g \right> + 2\left<df, dg \right> + \left<f, d^2 g \right>$

     $d^N\left<f, g \right> = \sum_{k=0}^{N} C_N^k\left< d^kfd^{N-k}g\right>$ (проверка как в одномерном случае)

     Тогда, если $v\in \R^m, v = const$

     $d^N\left<f, v \right> = \left<d^Nf, v \right>$
    
     \begin{proof}
         [Доказательство теоремы 1]
         Если $v\in \R^m, v$ -- фиксирована

         $g(x) = \left<f(x), v \right>:O \to \R, g\in C^{N+1}\left( O\to \R \right) $

         $g(x)-T_{N,a,g}(x) = \frac{1}{(N+1)!}d_c^{N+1}g(x-a)$ по уже установленной теореме Тейлора-Лагранжа для функций
\def \mulinl {\substack{\alpha\in \left( \Z _+ \right) ^n \\ |\alpha|\leqslant N}}
\def \muline {\substack{\alpha\in \left( \Z _+ \right) ^n \\ |\alpha|\neq  N}}

$T_{N, a, g}(x) = \sum_{k=0}^{N} \frac{\left<d^kf, v \right>(x-a)}{k!}$

$\left<f(x) - \sum_{k=0}^{N} \frac{d^kf(x-a)}{k!}, v \right> = \frac{1}{(N+1)!}\underbrace{\left<d_c^{N+1}f(x-a), v \right>}_{\leqslant \|d_c^{N+1} f \ldots\|\|v\|}$ 

$\left| \text{левая часть} \right| \leqslant\underbrace{ \frac{1}{(N+1)!}\sup_{c\in [a,x]} \|d_c^{N+1}f(x-a)\|}_{\text{не зависит от выбора $v$}} \cdot  \|v\|$

Если мы возьмём $v$ остаточным членом, то мы получим оценку остаточного члена, не зависящую от  $v$

$v = f(x) - \sum_{k=0}^{N} \frac{d^kf(x-a)}{k!}$ 

$\|v\|^{\cancel{2}} \leqslant \frac{1}{(N+1)!}\sup \ldots \cancel{\|v\|}$  (сократили на $\|v\|$ )

$\|f - T_{N, a, f}\| \leqslant  \frac{1}{(N+1)!}\sup \ldots.$
     \end{proof}

     \section{Частный случай: формула конечных приращений}

     $O\subseteq \R^n, f\in C^1\left(O\to \R^m  \right) , [a,x]\subseteq O \quad a\neq x$, тогда 

     \[\|f(x) - f(a)\|\leqslant \sup_{c\in [a,x]} \|d_c f\| \|x-a\|\]


     \begin{corollary}
         Пусть $f\in C^1\left( O \to \R^m \right) $, где $O$ -- открытое множество.
         Пусть  $K$ -- выпуклый компакт,  $K\subseteq O$.
         Тогда \[\forall a, b\in K\quad \|f(b)-f(a)\| \leqslant \sup\limits_{c\in K}\|d_cf\| \|b-a\| \]
     \end{corollary}

    \section{Об оценке нормы дифференциала}

    $p(x) = \sum_{\substack{\alpha\in \left( \Z _+ \right) ^n \\ |\alpha|\neq  N}} \frac{N!}{\alpha!}C_{\alpha} x^{\alpha}\quad \forall \alpha \,\|C_{\alpha}\|\leqslant M$

    $\|p(x)\|\leqslant M\left( \sqrt{n} \|x\| \right) ^N$

    $d_c^Nf = \sum \frac{N!}{\alpha!}\underbrace{\frac{f^{(\alpha)}}{N!}}_{=C_{\alpha}} x^{\alpha}$ 

    $M = \max\limits_{\substack{\alpha\in \left( Z_+ \right) ^n\\c\in[a,x]\\|\alpha|=N}}\|\frac{f^{\alpha}(x)}{N!}\| \implies  \|d_c^Nf(x-a)\|\leqslant  M\left( \sqrt{n} \|x-a\| \right) ^N$


         Пусть $O$ $\subseteq \R^n$ открыто, $ f\in C^1\left( O \to  \R^m \right) $

        $K$ -- компактно в  $O \implies  f$ липшецево на $K$, т.е.  
        \[\exists C\in \R: \|f\left( \p x \right)  - f\left(\pp x  \right)  \|\leqslant C\|\p x - \pp x\|\ \forall \p x , \pp x \in K\]
        Следует из формулы конечных приращений и оценки дифференциала и теоремы Вейерштрасса: $\frac{\partial f}{\partial x_1}, \ldots, \frac{\partial f}{\partial x_{n} }\in C(K)$

        $\implies \exists x_1: \quad \|\frac{\partial f}{\partial x_i}(x)\|\leqslant C_1 \forall x\in K$
        
        $M = C_1,  \implies \forall \p x, \pp x\in K\quad \|d_c f\left( \p x - \pp x) \right) \|\leqslant  M\left( \sqrt{n}\|\p x - \pp x\|  \right) \quad C = M\sqrt{n}  $ 

        \section{Экстремум функции нескольких переменных}

        \begin{definition}
            $\sqsupset O\subseteq \R^n\quad f:O\to \R$

            $a\in O\quad a$ называется точкой (локального) максимума, если  $\exists $ окрестность $V_a: \forall x\in V_a \cap O\quad f(x) \leqslant f(a)$

            Экстремум -- максимум или минимум
        \end{definition}

        \begin{statement}
            [Необходимое условие экстремума (безусловного)]
            Пусть $E \subseteq \R^n\quad f$, такое что $E\to \R$.
            $a\in \Int E$ -- точка локального экстремума для  $f$, $f$ дифференцируема в точке  $a$.
        Тогда  \[d_af = \mathbb{O} \left( \iff  \nabla f = \mathbb O \iff  \frac{\partial f}{\partial x_1}(a) = 0, \ldots, \frac{\partial f}{\partial x_{n} }(a) = 0\right) \].
        \end{statement}
        \begin{proof}
            Пусть $a$~--- точка максимума. 
            Фиксируем $h\in \R^n\quad g(t) = f(a + th), t\in \R.$ 
            Для $g$ точка 0 это точка максимума.
            Существует окрестность нуля $\p V(0): \forall t\in \p V(0)\quad g(t) \geqslant  g(0) = f(a)$, $g$ дифференцируема в 0 как композиция, 0 -- внутренний для  $D(g) \implies \p g(0) = 0$.

            $g(t) = f\left( \varphi(t) \right)$.

            $\p g(t)  =\p f\left( \varphi(t) \right) \cdot \p \varphi(t)$.

            $0 = \left<\nabla f, h  \right> = \left( \p f_{x_1}, \ldots, \p f_{x_{n} } \right) \begin{pmatrix} h_1\\\ldots\\ h_n \end{pmatrix}$.
        \end{proof}
            
        \section{Квадратичные формы}
        Если $Q(x)$ допускает представление в виде  $ Q(x) \equiv \sum_{i, j=1}^{n} c_{ij}x_i x_j, \quad c_{i, j}\in \R$. 
        Тогда $Q(x)$ называется квадратичной формой в  $\R^n$.

        \begin{note}
            Любая квадратичная форма есть однородная функция степени 2.
        \end{note}
        \begin{note}
            Не умаляя общности, матрицу коэффициентов $c_{ij}$ можно считать симметричной.
            Если это не так, можно перейти в такой форме  $\p c_{ij} = \p c_{ji} = \frac{c_{ij} + c_{ji}}{2}$.
        \end{note}

        \begin{definition}
            Квадратичная форма $Q(x)$ в  $\R^n$ называется положительно-определённой (положительной) (Q > 0), если 
            \[\forall x\in \R^n \setminus \left\{ 0 \right\} \quad Q(x) >0.\]
        неотрицательно определённой, если неравенство нестрогое. 

        $Q \geqslant 0\quad \ldots Q<0, Q\leqslant 0$


        неопределённая, если $Q \gtrless 0\quad \exists x^1 и x^2 \in \R^n: Q\left( x^1 \right) >0, Q\left( x^2 \right) <0$
        \end{definition}
        \begin{example}
            \begin{enumerate}
                \item $n=2$ $Q(x) = 2x_1^2 - 3x_2^2 \gtrless 0$
                    
                    $Q(x) = 2x_1^2 + 3x_2^2 >0$

                    $Q(x)  =Ax_1^2 + 2Bx_1^2x_2^2 + Cx_2^2$, если $B^2-AC$, то форма знакопеременная

                    Если $A\geqslant 0\quad B^2-AC\leqslant 0$, то $Q \geqslant 0$

                    $A\leqslant 0 \ldots$
            \end{enumerate}
        \end{example}

        \begin{lemma}
            $\sqsupset Q(x)$ -- положительная квадратичная форма в $\R^n$

            Тогда $\exists \gamma >0: \forall x\in \R^n\qquad Q(x) \geqslant \gamma \|x\|^2$
        \end{lemma}
        \begin{proof}
            $\gamma  = \min_{\|x\| = 1}Q(x) = Q(x_0) >0\quad \|x_0\|=1$

            $\forall x\in \R^n \setminus \left\{ 0 \right\} Q(x) = Q\left( \|x\| \frac{x}{\|x\|}\right) = \|x\|^2 \cdot  Q\left( \frac{x}{\|x\|} \right)  \geqslant  \gamma \|x\|^2$
        \end{proof}

        \begin{statement}
            [Достаточое условие экстремума]

            $\sqsupset f: E\to \R, \quad \underline{a\in \Int E}, d_a f = 0\quad \exists  d^2_a f$ 

            Тогда, если $d^2_af>0$ (положительная квадратичная форма как функция дифференциалов  $dx_1, \ldots, dx_{n} $), то $a$ это точка минимума (строгого)

            Если  $d^2_af <0 \ldots.$

            Если $d^2_af \gtrless$, то $a$ \underline{не} точка экстремума

            Это не все случаи, есть нестрогие, в которых ДУЭ не применимо
        \end{statement}

        \begin{example}
            $f(x, y) = x^4 - y^4$

            $g(x, y) = x^4 + y^4$

            для $f$ точка $(0, 0)$ не точка экстремума, а для $g$ -- да 

            $\nabla f = \left( 4x^3, -4y^3 \right)  = 0 \iff  \left( x,y \right)  = \left( 0, 0 \right) $
        \end{example}

        \begin{proof}
            ДУЭ. Пеано в точке $a$:

            $f(x) = f(a) + d_af(x-a) + \frac{1}{2} d^2_af(x-a) + o\left( \|x-a\|^2 \right) $ 

            $f(x) - f(a) = \frac{1}{2} d^2_af(x-a) + \underbrace{\varepsilon(x)}_{\to 0, x\to a}\|x-a\| $

            Если $Q>0$, то по лемме $\exists \gamma >0: Q(x-a) \geqslant  \gamma \|x-a\|^2$

            Т.к. $\varepsilon(x) \to 0, x\to a$, то $\exists V(a): \left| \varepsilon(x) \right| <\frac{\gamma}{8} \forall x\in V(a) \implies  \forall x\in V(a)\quad f(x) - f(a) \geqslant \frac{1}{2} \gamma \|x-a\|^2 - \frac{\gamma}{8} \|x-a\|^2 = \|x-a\|^2\left( \frac{3}{8}\gamma \right) >0\quad x\neq a \implies a$ -- точка строгого минимума

            $Q<0$, рассмотреть  $-f$

            $Q\gtrless 0 \implies  \exists h_+, h_- \in \R^n: Q\left( h_+ \right) >0\quad Q\left( h_- \right) <0$, не умаляя общности $\|h_+\| = \|h_-\| = 1$ 

            $\delta = \min \left\{ \left| Q\left( h_+ \right)  \right| , \left| Q\left( h_- \right)  \right|  \right\} $ 

            Т.к. $\varepsilon(x)\to 0, x\to a$, то $\exists $ окрестность $V_r(a): \left| \varepsilon(x) \right| <\frac{\delta}{4}$ 

            $|t|<r\quad f\left( a + th_+ \right)  - f(a) = \frac{1}{2} t^2 Q\left( x_+ \right) + \varepsilon(x)t^2 \geqslant  \frac{1}{2} t^2\cdot \delta - \cdot \frac{\delta}{4}t^2= t^2 \frac{\delta}{4}>0$ 

            $\left| Q(h_-) \right| \geqslant \delta$

            $Q(h_-) = - |Q(h_-)| \leqslant -\delta$

            $f\left( a + th_- \right) - f(a) \leqslant - \frac{\delta}{2}tse + \frac{\delta}{4}t^2 \leqslant -\frac{\delta}{2}t^2<0 $

            Таким образом в любой окрестности точки $a\quad f(x) - f(a)$ знакопеременная

        \end{proof}

        \section{Практика. Теорема о существовании}

        \begin{theorem}
            [Теорема о неявной функции]
            $F(x, y, z) = 0$

            $(x_0, y_0, z_0):\quad F\left( x_0, y_0, z_0 \right) = 0\quad \p F_z(x_0, y_0, z_0) \neq 0$ и все функции непрерывны в $(x_0, y_0, z_0)$

            $\implies \exists z = z(x, y)\quad z_0 = z(x_0, y_0)\quad F(x, y, z(x, y)) = 0$ в окрестности $(x_0, y_0)$
        \end{theorem}

        \begin{example}
            $x^2 + y^2 - 1 = 0$

            $F(x, y)\quad \left( \frac{\sqrt{2} }{2}, \frac{\sqrt{2} }{2} \right) \quad y = \sqrt{1-x^2} $

            $\p F_y = 2y\quad \p F_y(x_0, y_0) = \sqrt{2}\neq 0 $ 

            $y = y(x)\quad y_0 = y(y_0)\quad x^2+y^2(x)-1=0$

        \end{example}
\begin{figure}[!ht]
    \centering
    \incfig{exex}
    \caption{exex}
    \label{fig:exex}
\end{figure}

        $F(x, y, z)$ И все условия выполняются. как найти  $\frac{\partial}{\partial x}z(x, y)$
        
        $F(x, y, z(x, y)) = 0$

        $\frac{\partial}{\partial x}F(x, y, z(x, y)) = 0$

        $\begin{cases}
            \p F_x + \p F_z\cdot \frac{\partial z(x, y)}{\partial x} = 0\\
            \p F_y + \p F_z\cdot \frac{\partial z(x, y)}{\partial y} = 0\\
        \end{cases}$ 

        Отсюда выражается

        \begin{definition}
            Многозначная функция $f$ -- соответствие  $x \mapsto f(x)$ -- множество
        \end{definition}
        \begin{example}
            $x\mapsto \pm \sqrt{1-x^2} = \left\{ \sqrt{1-x^2} , -\sqrt{1-x^2}  \right\}  $ 

            $x^{\frac{1}{2}} = y\qquad x = y^2$ -- задаёт неявную функцию $y(x)$
        \end{example}

        \begin{figure}[!ht]
                \centering
                \begin{tikzpicture}
                    \begin{axis}[
                        xmin= -1, xmax= 10,
                        ymin= -5, ymax = 5,
                        axis lines = middle,
                    ]
                    \addplot[domain=-1:10, samples=100]{sqrt(x)};
                    \addplot[domain=-1:10, samples=100]{-sqrt(x)};
                    \end{axis}
                \end{tikzpicture}
                \caption{$x=y^2$}
                \label{$x=y^2$}
            \end{figure}

        Пусть $y(x)$ -- многозначная функция. Тогда выбор единственного  $y\in y(x)$ для  $\forall x$ задаёт явную функцию (однозначная функция)

        Каждый такой выбор задаёт однозначную функцию называемую ветвью. Ветви могут быть непрерывными, например в $x=y^2$ бесконечность ветвей. Чтобы уточнить, нужно проговаривать непрерывная ветвь. дифференцируемая ветвь и т.д.

        \begin{example}
            $x^2+y^2 = x^4 + y^4$

            Задаёт многозначную функцию. Определить для каких $x$ она будет 1-, 2-, 3-, 4-значной
        \end{example}


        \section{Лекция 2}
    
        $d^2_af \longleftrightarrow \begin{pmatrix} 
            \pp f_{x_1x_1}(a) & \pp f_{x_1x_2}(a) & \pp f_{x_1x_3}(a) & \ldots \\
            \pp f_{x_2x_1}(a) & \pp f_{x_2x_2}(a) & \pp f_{x_2x_3}(a) & \ldots \\
            \pp f_{x_2x_1}(a) & \pp f_{x_2x_2}(a) & \pp f_{x_2x_3}(a) & \ldots \\
            \ldots & \ldots & \ldots & \ldots \\
        \end{pmatrix} $ 

        \begin{theorem}
            [Критерий Сильвестра]
            \begin{itemize}
                \item Если все главные миноры положительны, то соответствующая квадратичная форма положительна ($a$ -- точка минимума).
                \item Если  $\Delta_1 <0\quad \Delta_2>0\quad \Delta_3 <0 \ldots$, то квадратичная форма отрицательна ($a$ -- точка максимума) 
                \item $\Delta_k\neq 0$, и не реализуется ни первый случай, ни второй, то квадратичная форма неопределённая (т.е. экстремума нет)
            \end{itemize}
        \end{theorem}

        \begin{note}
            $Q(h) = d^2_a f(h)$.
            $Q(h) = \left<A\cdot h, h \right>\quad A = \left( \pp f_{x_ix_j} \right)_{ij} $
        \end{note}

        \begin{problem}
           $z = x^2-xy+y^2-2x+y$ -- исследовать на экстремум 
        \end{problem}
        \begin{proof}
            Необходимое условие экстремума $\p z_x = 0\quad \p z_y = 0$.

             $\p z_x = 2x - y - 2\quad \p z_y = -x + 2y +  1$

             $(x, y) = (1, 0)$, других нет, потому что определитель хороший.

             Достаточное условие экстремума:
             $\pp z_{x x} = 2, \quad \pp z_{xy} = -1,\quad \pp z_{yy}=2$.
             $\left[ d^2_{\left( 1, 0 \right) }f \right] \longleftrightarrow \begin{pmatrix} 2&-1\\-1&2 \end{pmatrix}$. 

             $\Delta_1 = 2>0\quad \Delta_2 = 2*2 - (-1)^2 = 3>0$, таким образом $(1, 0)$~--- строгий минимум.
        \end{proof}

        \begin{note}
            $d^2_{(1,0)}f = 2dx^2 + 2(-1)dxdy + 2dy^2 = dx^2 + dy^2 + (dx-dy)^2 >0$, если $\begin{pmatrix} dx\\dy \end{pmatrix} \neq \begin{pmatrix} 0\\0 \end{pmatrix}$.

            $d^2f>0$
        \end{note}

        \begin{problem}
            [без привлечения $d^2f$]
            $z = x^2y^3(6-x-y)$ -- исследовать на экстремум. 

            $\begin{cases}
                0 = \p z_x = y^3(6x^2-x^3-yx^2\p )_x = y^3(12x-3x^2-2yx) = xy^3(12-3x-2y)\\
                0 = \p z_y = x^2(6y^3-xy^3-y^4 \p )_y = x^2(18y^2-3xy^2-4y^3)= x^2y^2(18-3x-4y)\\
            \end{cases}$ 

           Либо $x=0$, либо  $y=0$, либо  $\begin{cases}
               3x + 2y = 12\\ 3x+4y=18
           \end{cases}$ -- $(2, 3)$.

$\left\{ (0, t) \right\} _{\left\{ t<0 \right\} \cup \left\{ t>6 \right\} }$ -- максимум (нестрогий)

$\left\{ (0, t) \right\} _{\left\{ t\in (0,6) \right\} }$ -- максимум (нестрогий)

$(0,6)$ -- не экстремум

$\sphericalangle K = \left\{ (x, y)|x\geqslant 0\quad y\geqslant 0\quad x+y\leqslant 6 \right\} $ -- компакт

По теореме Больцано-Вейерштрасса существует $\left( x_{\pm}, y_{\pm} \right)\in K $ 

$f\left( x_+, y_+ \right)  = \max\limits_K f$

$f\left(x_-, y_-  \right) = \min\limits_Kf $ 

из распределения знаков следует, что точки границы -- точки минимума.
Тогда $(x_+, y_+)\in \Int K,$ значит $(x_+, y_+)$ удовлетворяет необходимому условии экстремума. 
Такая точка у нас одна~---  $(x_+, y_+) = (2, 3)$.
        \end{problem}

%\begin{figure}[!ht]
%    \centering
%    \incfig{znaki}
%    \caption{znaki}
%    \label{fig:znaki}
%\end{figure}
        \section{Экстремумы и замена переменных}
        Определение точки экстремума непосредственно переносится на случай метрических пространств

        \begin{lemma}
            Пусть $\left( X, \rho_X) \right) , \left( Y, \rho_y \right) $ -- метрические пространства и $f:X \to \R$.
            Пусть $g(b) = a\quad g$ непрерывна в точке  $b$.  $a $ -- точка максимума (минимума) для  $f$.
            Тогда  $b$ -- точка максимума (минимума) для  $f \circ g$.

        \end{lemma}
\begin{figure}[!ht]
    \centering
    \incfig{kartinkalemmi}
    \caption{kartinkalemmi}
    \label{fig:kartinkalemmi}
\end{figure}
        \begin{proof}
            По условию $a$ -- точка локального максимума, т.е.  существует окрестность $U(a)\subseteq X:\quad f(x)\leqslant f(a)\ \forall x\in U(a)$.
            По определению непрерывности существует окрестность $V(b)\subseteq Y:$
            \[g(y)\in U(a)\forall y\in V(b)\implies f(g(y))\leqslant f(a) = f(g(b)).\]
        \end{proof}

        \begin{corollary}
            Если в условии леммы $g$ -- гомеоморфизм  $X$ на  $Y$, то  $a$ -- точка максимума (минимума) для $f\circ g$.
        \end{corollary}
        \begin{corollary}
            Если $g$ -- локальный гомеоморфизм (существует окрестность $V(b)$, такая что в точке $b$ сужение  $g|_{V(b)}$ -- гомеоморфизм на образ ( $g(V(b))$)), то сохраняется вывод предыдущего следствия. 
        \end{corollary}

        \begin{problem}
            $z = xy\sqrt{1 - \frac{x^2}{a^2} - \frac{y^2}{b^2}}\quad (a, b>0) $~--- исследовать на экстремум.
        \end{problem}
        \begin{proof}

            $z(x, y) = -z(-x, y) = -z(x, -y)$

            $z$ -- нечётная по  $x$ и по  $y$. 
            Значит достаточно рассматривать функцию только в первой четверти.
            
            $\begin{cases}
            x = a\rho \cos \alpha\\
            y = b\rho \sin \alpha\\
            \end{cases}$

            $(\rho, \varphi) \to (x, y)$

            $(0,1) \times (0, \frac{\pi}{2}) \to \Int K$ -- гомеоморфизм $\left( \varphi = \arctg \frac{y}{x}\quad r = \sqrt{x^2+y^2}  \right) $ 
            
            $z(\rho, \varphi) = ab\rho^2\cos \varphi \sin \varphi \sqrt{1-\rho^2} \overset{\rho^2=t} = \frac {ab} 2 \sin 2\varphi t\cdot \sqrt{1-t} $

            Необходимое условие экстремума: $\begin{cases}
                \frac{2}{ab} \p z_{\varphi} = 0 = 2\cos 2\varphi\cdot t\sqrt{1-t}\\
                \frac{2}{ab}\p z_{t} = 0 = \sin 2\varphi\left( \sqrt{1-t} -\frac{1}{2} \frac{t}{\sqrt{1-t} } \right) = \frac{\sin 2\varphi \left( 2(1-t)-t) \right) }{2\sqrt{1-t} } 
            \end{cases}$ 

            $\begin{cases}
                 \cos 2\varphi \cdot  t\sqrt{1-t} = 0\\
                 \sin 2\varphi \cdot  \frac{(2-3t)}{\sqrt{1-t} } = 0
            \end{cases} \iff  \begin{cases}
            \cos 2\varphi = 0\quad \varphi \in (0, \frac{\pi}{2})\\
                3t = 4
            \end{cases} \iff  \begin{cases}
                \varphi = \frac{\pi}{4}\\
                t = \frac{2}{3}
            \end{cases} \iff  \begin{cases}
                x = a\cdot \frac{4}{9}\cdot \frac{1}{\sqrt{2} } = \frac{2\sqrt{2}b }{9}\\
                y = b \cdot  \frac{4}{9}\cdot \frac{1}{\sqrt{2} } = \frac{2\sqrt{2} b}{9}
            \end{cases}$

%\begin{figure}[!ht]
%    \centering
%    \incfig{primer}
%    \caption{primer}
%    \label{fig:primer}
%\end{figure}
        \end{proof}

        \begin{problem}
            $f: \underbrace{O}_{\subseteq \R^n} \to  \R$

            $E = \left\{\varphi_1(x)=0, \ldots, \varphi_n(x) = 0  \right\} $

            Исследование $f_{E}$ на экстремум называется задачей об условном экстремуме
        \end{problem}

        \begin{example}
            $f(x, y) = x+y\quad E = \left\{ x+2y = 1 \right\} $
        \end{example}
        \begin{proof}
            $x = 1-2y$

             $f = 1-2y+y = 1-y$

             $\tl f(x, y) = x^2+y^2\quad E = \left\{ x+y=1 \right\} $
        \end{proof}
        \begin{example}
            $f = Ax^2 + 2Bxy + Cy^2\quad E = \left\{ x^2+y^2=1 \right\} $

            $\begin{cases}
                x = \cos \varphi\\
                y = \sin \varphi\\
            \end{cases}$ 


        \end{example}

        \section{Дифференцирование обратного отображения}

        $x\in \R, y\in \R\quad$\[ f(x) = y\]

        $x\in \R^n\quad y\in \R^n\quad f(x) = y\quad A \cdot  x = y, \ A = [f]$  -- линейна

        $f(x) = y$ имеет единственное решение  $\forall y\in \R^n \iff \det A \neq  0$
        
        \begin{theorem}
            [об обратной функции для случая одной переменной]

            $f: (A, B) \to  \R, f\in C^1((A, B)), a\in (A, B), \p f(a)\neq 0$, тогда существует окрестность $V(a):$
            \begin{enumerate}
                \item $\forall x\in V(a)\quad \p f(x)\neq 0$ -- локальная новорожденность производной
                \item $f|_{(A, B)}$ -- инъекция. -- локальная обратимость
                \item  $f(V(a))$ - откр. -- локальная открытость отображения
                \item $\left( f|_{V(a)} \right)^{-1} $ дифференцируема в точке $f(a)$ и  $\left( \left( f|_{V(a)} \right)^{-1}  \right)^{\prime} = \frac{1}{\p f(a)} $ -- дифференцируемость локально обратного
            \end{enumerate}
        \end{theorem}

        \begin{definition}
            $f:\left( X, \Omega_X \right) \to \left( Y, \Omega_Y \right)  $

            Если для любого $O\in \Omega_X\quad f(O)\in \Omega_Y$, то  $f$ называется открытым отображением
        \end{definition}
        \begin{example}
            $f(x) = x^2$ не открытое на $(-1, 1) \to [0,1)$, но открыто на $(-1, 0) \cup (0, 1)$, потому что нет точек, где $\p f(x) = 0$
        \end{example}
        \begin{proof}
            [доказательство теоремы]
            По следствию теоремы Дарбу, если $\p f(a) >0 (<0)$, то существует окрестность  $V(a):\forall x\in V(a)\quad \p f(x) >0 (<0)$

            $\sqsupset \p f(x)>0$ всюду на  $V(a)$, то $f$ строго возрастает, значит  $f|_{V(a)}$ -- инъекция

            $V(a) = \left( a-\delta, a + \delta \right) \implies f\left( a-\delta, a+\delta \right) = \left( f(a-\delta), f(a+\delta) \right) $ 

            $4 \impliedby $ теоремы о дифференцируемости обратимой функции
        \end{proof}

        \begin{theorem}
            [об обратном отображении]
            Пусть $\underbrace{O}_{\text{открытое}} \subseteq \R^n\quad f:O\to \R^n$ и $\forall x\in O\quad d_xf$ -- обратим (якобиан не обращается в ноль в  $O$)

            Тогда  $f$ -- открытое отображение


        \end{theorem}
        \begin{proof}
            См. доказательства утверждения 3 в теореме о дифференцировании обратного отображения в книжке Виноградов-Громов.
        \end{proof}
        \begin{theorem}
            [теорема об обратном отображении]

            $n\in \N, $ $O$ -- открытое,  $O\subseteq \R^n$

            $f\in C^1(O \to  \R^n)$ $a\in O$. Пусть  $d_af$ обратим ($\iff  \mathcal{J}_af\neq 0$) , тогда существует окрестность $V(a):$
            \begin{enumerate}
                \item $\forall x\in V(a)\quad d_xf$ обратим -- локальная новорожденность производной
                \item $f|_{(A, B)}$ -- инъекция. -- локальная обратимость
                \item  $f(V(a))$ - откр. -- локальная открытость отображения
                \item $\left( f|_{V(a)} \right)^{-1} $ дифференцируема в точке $f(a)$ и  $d_{f(a)} \left( f|_{V(a)} \right)^{-1}  = (d_{a}f)^{-1} $ -- дифференцируемость локально обратного
            \end{enumerate}
        \end{theorem}

        \begin{lemma}
            Пусть $n\in \N $, $O\subseteq \R^n$ открыто, $f:O\to \R^n, f\in C^1(O), a\in O$ и $d_af$ обратим. Тогда  $\forall \sigma >0 $ существует окрестность $V(a):$
             \begin{enumerate}
                 \item $\forall x\in V(a)$ \[\|d_xf - d_af\| < \sigma\]
                 \item $\forall p, q\in V(a)$ \[\|f(p) - f(a) - d_af(p-q)\|\leqslant C_1\|p-q\|\]
                 \item $\forall p, q\in V(a)$ \[C_3 \|p-q\| \leqslant  \|f(p) - f(q)\| \leqslant C_2 \|p-q\|\], такое свойство называется билипшецевость.

                     Здесь конкретно $C_2 = \|d_af\| + \sigma\qquad C_3 = \frac{1}{\|\left( d_af \right) ^{-1}\|} - \sigma$
            \end{enumerate}
        \end{lemma}
        \begin{proof}
            $f\in C^1(a) \implies$ существует окрестность $V(a):$ 1 верно

            $\sphericalangle F(x) = f(x) - d_af(x): O \to \R^n$

            $d_xF(h) = d_xf(h) - d_af(h), \quad F\in C^1(O)$

            $\|f(p) - f(q) - d_ad(p-q)\| = \|F(p) - F(q)\|\leqslant\underbrace{ \sup\limits_{c\in V(a)} \|d_cF\|}_{\leqslant \sigma}$ по теореме о конечных приращениях, т.к. $V(a)$ выпуклое

            $\|d_cF\| = \|\underbrace{d_Cf - d_af}_{<\sigma}\|$

            $\forall p, q\in V(a)$

            $\|f(p) - f(q)\| \leqslant  \sup\limits_{c\in V(a)} \|d_cf\| \|p-q\|$ 

            $\|d_c f\| = \|d_af + \left( d_cf - d_af \right) \| \leqslant  \|d_af\| + \underbrace{\|d_cf - d_af\|}_{<\sigma \text{в силу 1}} \leqslant  C_2$ 

            $\|f(p) - f(q)\| = \|d_af(p-1) - \left( f(p) - f(q)   - d_af(p-q) \right)\| \geqslant  \underbrace{\|d_af(p-q)\|}_{\geqslant  \frac{1}{\|\left( d_af \right) ^{-1}\|} \|p-q\|} - \underbrace{\|f(p) - f(q) - d_af(p-q)\|}_{\leqslant C_1\|p-q\|} \geqslant  C_3 \|p-q\|$
        \end{proof}

        \begin{proof}
            [доказательство (часть) теоремы об обратном отобржаении]

            Существует $\left( d_xf \right) ^{-1} \iff  \mathcal{J} f\neq 0$, но $\mathcal{J} f\in C\left( O\to \R \right) \underset{\text{по неперывности}} {\implies}  $ существует окрестность $V(a): \forall x\in V(a)\quad \mathcal{J}_xf !+0 \implies 1$ 

            $C_0 = \frac{1}{\|\left( d_af \right) ^{-1}\|}, \quad \sigma = \frac{C_0}{4}$, применим лемму к такому $\sigma$

            Не умаляя общности  $V(a) \subseteq V_0(a)$. Т.к. $\sigma < C_0\quad \forall p, q\in V(a)$ в силу  неравенства 3 из леммы $f(p)\neq f(q)$ ($f|_{V(a)}$ -- инъекция.  $\implies  f|_{V(a)}$ -- биекция на  $f(V(a))$, т.е.  $g = f|_{V(a)}$ обратимо и  $4 \impliedby $ правило дифференцирования обратного отображения
        \end{proof}
        \section{Практика}
        \begin{theorem}
            $f(x_1, x_2, \ldots, x_{n} )$  дифференцируема в точке P




            $f$ достигает экстремума в точке  $P \implies  \frac{\partial f}{\partial x_i}(P) = 0$
        \end{theorem}
        
            $d^2 f = \left( \frac{\partial}{\partial x_1}dx_1 + \ldots + \frac{\partial}{\partial x_n}dx_{n}  \right)^2 f $

            $H(f)= \begin{pmatrix} 
                \frac{\partial^2}{\partial x_1^2} & \frac{\partial^2}{\partial x_1\partial x_2} & \ldots\\ \ddots & \ddots & \ddots\\
            \end{pmatrix} $
        \begin{theorem}
            Если $H(f)$ положительно определена в точке $P$, то $P$ -- точка минимума. Если она отрицательно определена, то это точка максимума.
        \end{theorem}

        $d^2f = \frac{\partial^2f}{\partial x_1^2}(dx_1)^2 + \frac{\partial^2 d}{\partial x_1\partial x_2} + \ldots = \lambda_1(dy_1)^2 + \ldots + \lambda_n(dy_{n} )^2$

        $f(dx_1 \ldots dx_{n} ) = f(0) + \underbrace{=0}{\p f(0)dx} + d^2f + o\left(  \right) $ 

        $f(dx_1 \ldots dx_{n} ) - f(0) = d^2f + o\left(  \right) $

        $x^TAx>0 \forall x\neq 0 \overbrace{def}{\iff }$

        $(x_1 \ldots x_{n} )\begin{pmatrix} a_{11} & a_{12} & \ldots a_{1n}\\ a_{21} & a_{22} & \ldots & \ldots\\&&&\\&&& \end{pmatrix} \begin{pmatrix} x_1\\x_2\\ \vdots\\ x_{n}  \end{pmatrix} = \left( x_1 \ldots x_{n}  \right) \begin{pmatrix} \sum a_{1i}x_i\\ \sum a_{2i}x_i\\ \vdots \end{pmatrix} = \sum a_{ji}x_jx_i  $ -- квадратичная форма

        Пусть $Q$ -- квадратичная форма.  $Q = \sum a_{ji}x_jx_i$, где  $a_{ij} = a_{ji} \forall i, j\quad Q = x^TAx$, где $A^T=A$

        Такая матрица $A$ называется  матрицей квадратичной формы $Q$

        $x = Cy\quad Q = \left( Cy \right) ^TACy = y^T\underbrace{B}{\left( C^TAC \right)} y = \lambda_1y_1^2 + \ldots + \lambda_ny_{n} ^2\quad B = \begin{pmatrix} \lambda_1 & 0 & \ldots & 0\\ 0 & \lambda_2 & 0 & 0\\&&&\\&&& \end{pmatrix} $

        \begin{statement}
            Симметричная матрица подобна диагональной матрице.
        \end{statement}
        \begin{proof}
            $A = A^T$

             \begin{example}
                 $\begin{pmatrix} \cos \varphi & \sin \varphi\\-\sin \varphi&\cos \varphi \end{pmatrix}  = \begin{pmatrix} 0&1&-1&0 \end{pmatrix} $
            \end{example}

            \begin{enumerate}
                \item Докажем, что существует собственный вектор.

                    Пусть $Q = x^TAx = \sum_{ij}a_{ij}x_ix_j$

                    Рассмотрим  $Q$ на сфере  $x_1^2+ \ldots +x_{n} ^2 - 1 =0$

                    $Q$ диффрецнируема на сфере $\implies $ достигает максимума в точке $(v_1, \ldots, v_n)$ 

                     $Q$, максимум с ограничением  $F = 0$, то  $\sphericalangle \mathcal L:=Q - \lambda F$

                     У $\mathcal L$ частные производные равны 0 в максимуме

                     $\frac{\partial}{\partial x_1}Q = \frac{\partial}{\partial x_i}\left( \sum_j a_{ij}x_ix_j + \sum_{k\neq i} a_{kj}x_kx_j \right) = \sum_j a{ij}x_j + \sum_j a_{ji}x_j = 1\sum a_{ij}x_j $ 

                     $\frac{\partial F}{\partial x_i} = 2x_i\quad \frac{\partial}{\partial x_i}\mathcal L = 2\sum_ja_{ij}x_j - \lambda 2xi = 0 \forall i$ для $x_j=v_j$

                      $Av = \lambda v \implies v$ -- вещественный собственный вектор

                      Так мы для симметричной матрицы нашли вещественный собственный вектор
                  \item Достроим наш вектор $v$ до базиса  $(v, e_2, e_3, \ldots, e_n)$

                      Запишем $A$ в этом базисе:  $\begin{pmatrix} \lambda& 0& \ldots & 0\\0&&&\\ \vdots&&&\\0\end{pmatrix} $. Дальше делаем по индукции

                      Следовательно существует базис из собственных векторов, где на диагонали стоят собственные числа
            \end{enumerate}
            Итого, любую квадратичную форму $Q$ можно заменой переменных свести к каноническому виду  $Q = \sum_i \lambda_iy_i^2$
        \end{proof}

        \section{Лекция}

        Макаров, Подкорытов: Гладкие отображения и функции

        \begin{theorem}
            [Об открытом отображении]

            $\sqsupset O\subseteq \R^n$ -- открыток, $f\in C^1(O\to \R^n), \quad \forall a\in O\ d_af$ обратим. Тогда $f(O)$ -- открыто.
        \end{theorem}
        \begin{example}
            $x = \left( x_1, \ldots, x_n \right) \quad f(x) = x_1$ -- необратимое, схлопывает шар и там нет открытости
        \end{example}
        \begin{proof}
            $\sqsupset \sigma:$ в лемме: $C_3>0, \sigma = \frac{1}{2} \frac{1}{\|\left( d_af \right) ^{-1}\|}$ 

            $\delta$ -- это половины от  ``$\delta$ из леммы''

            \[\|f(p)-f(q)\| \geqslant C_3\|p-q\|\quad \forall p, q\in B_{\delta}[a]\]

            $r = \frac{1}{2} C_3\cdot \delta\quad ?:B_r(b)\subset f(O) \iff \forall y\in B_r(b) \exists x\in O: f(x) = y$ 

            $\sphericalangle \varphi(x) = \|f(x) - y\|\in C*(O\to \R$

            $\varphi(a) = \|f(a) - y\| = \|b-y\|\leqslant r$


            Если $\|x-a\| = \delta$, то $\varphi(x) = \|f(x) - b\| = \|f(x) - f(a) + f(a) - y\|\geqslant \underbrace{\|f(x) - f(a)\|}_{\geqslant C_3\|x-a\| = 2r} - \underbrace{\|f(a) - y\|}_{<r} > 2r-r = r$

            По теореме Вейерштрасса $\varphi(x)$ достигает на  $B_{\delta}[a]$ своего минимума

            Из оценки  $\varphi(x)$ следует, что  $\min\limits{B_{\delta}[a]}\varphi$ Достигается внутри шара

            $\psi(x) = \varphi^2(x) = \|f(x) - y\|^2$. У функции $\psi(x)$ экстремумы в тех же точках, что и у  $\varphi(x)$

            Необходимое условие экстремума $\exists x_*\in B_{\delta}(a):\quad f_{x_*}\psi = \mathbb O$

            $\varphi(x) = \left<f(x) - y, f(x) - y \right>$

            $d\psi_{x_*} = 2\left<\underbrace{d(f\left( x \right) -y)}_{d_{x_*}f}, f(x) - y \right>$

            Т.к. $df$ обратим в  $B_{\delta}(a) \implies f(x_*)-y=0 \implies f(x_*) = y_0$

        \end{proof}

        \begin{theorem}
           [теорема об обратном отображении]

           $\sqsupset $ открытое $O\subseteq \R^n\quad f\in C^1(O\to \R^n), a\in O, d_af$ обратим

           Тогда существует окретстность $V(a):$
            \begin{enumerate}
                \item [I] $\forall x\in V(a)\quad d_xf$ обратим
                \item [II] $f_{V(a)}$ -- инъекуий (т.е. обратимо как отображение из  $V(a)$ в  $f(V(a))$
                \item [III]  $f(V(a))$ -- открыто
                \item [IV]  $\left( f_{V(a)} \right)^{-1}\in C^1\left( f(V(a))\to \R^n \right)  $ 

                    $ \left( f|_{V(a)} \right)^{\prime}(f(a)) = (\p f)(a) $ (или $d\left( f_{V(a)} \right) ^{-1} = \left( d_af \right) ^{-1}$ 
           \end{enumerate}
        \end{theorem}

        $\|B-A\|<\varepsilon\quad \|B^{-1} - A^{-1}\|<\varepsilon C(A)$

        $\|(d_af)^{-1} - (d_xf)^{-1}\| <\varepsilon C\cdot \left( d_af \right) \implies \left( d_xf \right) ^{-1}$ непрерывно зависит от $x$

        второе объяснение: элементы матрицы $\left( d_xf \right) ^{-1}$ -- результат арифметических действий над частными производными отображения $f; f\in C^1(O) \implies $ элементы $[\left( d_xf \right) ^{-1}]\in C(0)$

        По аналогичным рассуждения, если в условии теоремы $f\in C^r\left( O \to \R \right) $, то локально обратное также из $C^r$



        \begin{definition}
            $\sqsupset r\in Z_+\quad O\subseteq \R^n, O$ -- открытое, $f\in C^r\left( O\to \R^n \right) $

            $f$ Называется диффеоморфизмом класса  $C^r$, если:
             \begin{enumerate}
                 \item $f$ -- биекция на  $f(O)$
                 \item $f(O)$ -- открытое
                 \item  обратное отображение  $f^{-1}\in C^r\left( f(O) \to O \right) $
            \end{enumerate}
        \end{definition}

        \begin{definition}
            $r, O$ --||--,  $a\in O\quad f$ называется локальным диффеоморфизмом в точке $ a$, если существует такая окрестность $V(a)$, что  $f_{V(a)}$ -- гомеоморфзим
        \end{definition}

        \begin{example}
            $y = e^x$ -- диффеоморфизм (глобальный)

             $y = x^2$ -- локальный диффеорморфизм отдельно либо на положительных, либо на отрицательных числах

             $y = \sin x$ -- локальный диффеоморфизм в точках не вида  $\frac{\pi}{2} + \pi k$
        \end{example}

        \begin{theorem}
            [Об обратном отображении ``на языке диффеоморфизмов'']

            Открытое $O\subseteq \R^n, f\in C^r\left( O\to \R^n \right) $
            \begin{enumerate}
                \item Если $a\in O\quad d_af$ обратим, тогда  $f$ локальный диффеоморфизм в точке  $a$ класса  $C^r$\
               \item Если $f$ -- инъекция и  $d_axf$ Обратим всюду в  $O$, то  $f$ -- глобальный диффеоморфизм
            \end{enumerate}
        \end{theorem}

        \begin{example}
            $f(x,y) = \begin{pmatrix} e^y\cos x\\ e^y\sin  x \end{pmatrix} $ 

            $\p f = \begin{pmatrix} -\sin x e^y & e^y\cos x\\
            \cos x e^y & e^y\sin x\end{pmatrix} $

            $\det \p f = e^{2y}(-1)\neq 0$

            $f(0,y) = f(2\pi k)$

            В каждой точке невырожденный дифференциал, но глобальный инъективности нет
        \end{example}

        \begin{example}
            [Важные примеры локальных диффеоморфизмов]

            \begin{enumerate}
                \item Полярные координаты $\phi(r, \varphi) = \left( r\cos \varphi, r\sin \varphi \right) $ 

                    $\phi: [0,+\infty )\times \R \to \R^2$

                    $\p \phi = \begin{pmatrix} \cos \varphi&-r\sin \varphi\\ \sin \varphi & r\cos \varphi \end{pmatrix} $ 

                    $\det \phi = \cos \varphi\cdot r\cos\varphi - \sin \varphi\left( -r\sin\varphi  \right) = r>0$ в $O$, значит  $\phi$ -- локальный ддиффеоморфзм в  $O$ класса  $C^{\infty }$

                    но не глобальный! $\phi(r, \varphi+2\pi k)\equiv \phi(r, \varphi)$

                    $O_1= \left( 0, +\infty  \right) \times \left( -\frac{\pi}{2}, \frac{\pi}{2}  \right) \quad \phi$ -- инъекция d в $O_1 \implies \phi$  глобальный диффеоморфизм в $O_1$

                    $r = \sqrt{x^2+y^2} \quad \varphi = \arctg \frac{y}{x}$
                \item Цилиндрические координаты.

                    $x = r\cos \varphi\quad y = r\sin\varphi\quad z = t$

                    $\p \phi  = \begin{pmatrix} \p x_r & \p x_{\varphi} & \p x_t\\&\nabla y&\\&\nabla z \end{pmatrix}  = \begin{pmatrix} \cos \varphi&-r\sin \varphi&0\\ \sin \varphi & r\cos \varphi &0\\0&0&1 \end{pmatrix}  = r$

                    Таким образом $\phi$ -- локальный диффеорморфизм в области $\left( 0, +\infty  \right) \times \R\times \R$

                    глобальный диффеоморфизм в областях $(0, +\infty ) \times \left( -\pi , \pi  \right) \times x\R$ И $\left( 0, +\infty  \right) \times \left( 0, 2\pi  \right) \times \R$

                    $x^2+y^2=R\quad r=|R|$

                     $x^2+y^2=z^2C \leftrightarrow r^2 = Ct^2\quad \pm r = \tl C = t$
                \item Сферические координаты

                    меряем широту и долготу. $\begin{cases}
                        x=r\cos \varphi\cos \psi\\
                        y=r\sin \varphi \cos \psi\\
                        z=r\sin \psi\\
                    \end{cases}$ 

                    $(r, \varphi, \psi) \to \left( x,y, z \right) \quad C^{\infty }:\R^3 \to  \R_3$

                    $\left| \psi \right| = \begin{pmatrix} \cos \varphi\cos \psi&-r\sin \varphi\cos \psi&-r\cos \varphi\sin \psi\\ \sin \varphi\cos \psi & r\cos \varphi\cos \psi & -r\sin \varphi\sin \psi\\ \sin \psi & 0 & r\cos \psi \end{pmatrix}  = r^2\cos \psi\left( \sin ^2\psi \begin{pmatrix} -\sin \varphi & -\cos \varphi\\cos\varphi&-\sin \varphi \end{pmatrix}  + \cos ^2\psi \begin{pmatrix} \cos \varphi & -\sin \varphi\\ \sin \varphi&\cos \varphi \end{pmatrix}  \right)  = r^2\cos \psi $

                    $\phi$ локальный диффеоморфизм в  $\left( 0, +\infty  \right) \times \R\times \left( -\frac{\pi}{2}, \frac{\pi}{2}   \right) $ 

                    глобальный в $\left( 0, +\infty \right) \times \left( -\pi , \pi  \right) \times \left( -\frac{\pi}{2}, \frac{\pi}{2}   \right)  $ 

                    $x^2+y^2+z^2=R^2 \leftrightarrow r=R$

                    \begin{problem}
                        Записать уравнение сферы $(x-a)^2 + (y-b)^2 + (z-c)^2 = R$ в сферических координатах

                        $x^2+y^2 = Cz^2$ тоже.
                    \end{problem}
                    \begin{problem}
                        В области $x>0, y>0$  $u = \frac{y}{x}\quad v = xy$

                        $\psi:(x, y) \to (u, v)$

                        Вопросы:
                        \begin{enumerate}
                            \item $\psi(O) = ?$
                            \item явялется ли  $\psi$ диффеоморфизмом или локальным диффеоморфизмом 
                            \item Выписать явно функции для обратного к $\psi$ или локально обратного
                        \end{enumerate}
                    \end{problem}
            \end{enumerate}
        \end{example}

        \section{Теорема о неявном отображении}

        Если у нас есть явное выражение $x = g(y)$, то можно явно исследовать функцию от одной переменной $f_E = f\left( h(y), y \right) $

         \begin{definition}
             Говорят, что уравнение $f(x,y) = 0$ неявно задаёт функцию  $y = g(x)$ или  $x = h(y)$, если условия  $F(x,y) = 0$ и  $\begin{cases}
                 x\in D(g)\\
                 y = g(x)\\
             \end{cases}$ равносильны.
         \end{definition}

         \begin{definition}
             $F:E\to \R\quad D\subseteq E$

             $F$ задаёт  $y = g(x)$ или  $x = h(y)$ в $D$, если \[\begin{cases}
                 F(x,y) = 0\\(x,y)\in D
             \end{cases} \iff \begin{cases}
             x\in D(g)\\
             y = g(x)\\
             \end{cases}\]
         \end{definition}

         $\begin{cases}
             F_1\left( x_1, \ldots, x_{k} ,y_1, \ldots, y_m  \right)  = 0\\
             \vdots\\
             F_m\left( x_1, \ldots, x_k, y_1, \ldots, y_m \right)  = 0
         \end{cases} \iff  F(x,y) = 0\quad \begin{cases}
         x = \left( x_1, \ldots, x_k \right) \\
         y = (y_1, \ldots, y_m)\\
         F - \left( F_1, \ldots, F_m \right) 
         
         \end{cases}$

         ``$y = g(x)$''  $\quad \begin{cases}
             y_1 = g_1\left( x_1, \ldots, x_k \right) \\
             \vdots\\
             y_m = g_m\left( x_1, \ldots, x_k \right) 
         \end{cases}$

        \begin{theorem}
            [Теорема о неявном отображении]
            $\sqsupset O$ -- открытое в $\R^{k+m}, F\in C^1\left( O \to  \R^k \right) $

            $(a, b)\in O\quad \left( a = \left( a_1, \ldots, a_k \right) , b = \left( b_1, \ldots, b_k \right)  \right) $ и 
            \begin{enumerate}
                \item $F(a,b) = 0$ 
                \item $\det \p F_y\left( a,b \right) \neq 0$ \[\p F_y = \begin{pmatrix} \frac{\partial F_1}{\partial y_1} &\ldots& \frac{\partial F_1}{\partial y_m}\\ \ldots&\ldots&\ldots\\ \frac{\partial F_m}{\partial y_1}& \ldots& \frac{\partial F_m}{\partial } \end{pmatrix} \]

                    Тогда:
                    \begin{enumerate}
                        \item [I] Существует открытое множество $U\times V$ в $\R^{k+m}$, где $U$ -- окрестность  $a$, а $V$ -- окрестность $b$: в  $U\times V$ уравнение $F(x,y) = 0$ неявно задаёт единственную функцию  $y = g\left( x \right) $
                        \item [II] $g$ дифференцируема в точке  $a$
                        \item [III]  $\p g(x) = -\left( \p F_{y}(a,b) \right) ^{-1} \cdot  \p F_x(a,b)$
                    \end{enumerate}
            \end{enumerate}
        \end{theorem}

    \section{Практика}

    $f(x_1, x_2, \ldots, x_{n} )\quad \varphi_1, \ldots, \varphi_n$

    $S = \left\{ \left( x_1, \ldots, x_{n}  \right) | \begin{cases}
            \varphi_1(x_1, \ldots, x_{n} ) = 0\\
            \vdots\\
            \varphi_n(x_1, \ldots, x_{n} ) = 0\\
    \end{cases} \right\} $ 

    Задача -- найти экстремум $f$ на  $S$

     \begin{theorem}
         [Необходимое условие]

         Пусть для $
         \begin{vmatrix}
             \frac{\partial \varphi_1}{\partial x_1}& \ldots & \frac{\partial \varphi_1}{\partial x_{n}}
                 \vdots&&\\
                 \frac{\partial \varphi_m}{\partial x_1} & \ldots & \frac{\partial \varphi_m}{\partial x_{n} }\\
         \end{vmatrix}$ ранга $m$ 

         $L :=f + \sum \lambda_i \varphi_i$

         Тогда  $x^*$ -- условный экстремум, если  $\begin{cases}
             \varphi_i(x^*) = 0 \forall i = 1, \ldots, m\\
             \frac{\partial f}{\partial x_i}(x^*) = 0 \forall i = 1, \ldots, n\\
         \end{cases}$
\end{theorem}

$f(x) - f(P) = (x-P)^TD^2f(x-P) + o\left( \ldots \right) $ 

В случае поиска на поверхности, рассматриваем касатальные к поверхности, если она гладкая

\begin{theorem}
    [Достаточное условие]

    Пусть $f, \varphi_i = C_2(x^*\in U)$

    $\sum\limits_k \frac{\partial \varphi_i(x^*)}{\partial x_k}dx_k = 0\quad \forall i\quad \sum (dx_k)^2 >0$

    $d^2\Lambda(x^*)$ знакоопределна для  $dx_k$, то  $x^*$ -- экстремум

    Если  $\gtrless 0$, то экстремума нет
\end{theorem}



\section{Лекция}

\begin{theorem}
    [О неявном отображении]

    $F(x, y) = \mathbb O \iff  \begin{cases}
        F_1\left( x_1, \ldots, x_k, y_1, \ldots, y_m \right)  = 0\\
        \vdots\\
        F_m\left( x_1, \ldots, x_k, y_1, \ldots, y_m \right) = 0
    \end{cases}$ 

    $x = \left( x_1, \ldots, x_k \right) , y = \left( y_1, \ldots, y_m \right) , F = \left( F_1, \ldots, F_m \right) $

    Если $F\in C^r(O), O$ -- открытое в  $\R^{k+m}$

    ($r\in \Z ^+$)
    \begin{enumerate}
        \item $F(x^0, y^0) = 0$
        \item $\p F_y(x^0, y^0)$ -- обратима ($\det \p F_y\neq 0$)
    \end{enumerate}

    Тогда $\exists $ открытое $U_{x_0}$ и $V_{y_0}$ и $g:U_{x_0}\to V_{y_0}$:
    \begin{enumerate}
        \item [I] $ \begin{cases}
                (x,y)\in U_{x_0}\times V_{y_0}\\
                F(x,y) = 0
        \end{cases} \iff \begin{cases}
        x\in U_{x_0}\\
        y = g(x)
        \end{cases}$ 
    \item [II] $g\in C^r(U_{x_0})$
    \item [III] $\p g(X) = \left( \p F_y(x,y) \right) ^{-1}\cdot \p F(x,y)$
    \end{enumerate}
\end{theorem}

\begin{definition}
    [Уточнение определения функции (отображения), заданной уравнением неявно]

    $\sqsupset D\subseteq R^k\quad g: D\to \R^m$ Скажем, что отображение $g$ задаётся уравнением 
    \begin{enumerate}
        \item $F(x^0, y^0) = 0$
        \item $\p F_y(x^0, y^0)$ -- обратима ($\det \p F_y\neq 0$)
    \end{enumerate}
    неявно, если \[F(x,g(x)) = 0 \quad \forall x\in D\]

    (аналогично для $x = h(y)$)
\end{definition}

\begin{example}
    $x^2+y^2+z^2=a^2\quad a>0$

    $z = \sqrt{a^2-x^2-y^2} \quad z = -\sqrt{a^2-x^2-z^2}\quad y = \sqrt{a^2-x^2-z^2}\quad \ldots  $

    $\begin{cases}
        \p z_x = ?\\
        \pp z_{xy} = ?\\
    \end{cases}$ 

    $x^2+y^2+z(x,y) \equiv a^2$

    $g(x,y) = z(x,y)\quad F(x,y,z) = x^2+y^2+z^2-a^2$

    $\p F_z = 2z\neq 0 \iff x^2+y^2<a^2$

    $2\left( x+z\cdot \p z_x \right)  = 0 \implies \p z_x = -\frac{x}{z}$ 

    $0+\p z_y \cdot \p z_x + z\cdot \pp z_{xy} = 0\qquad \p z_y = -\frac{y}{z} \implies \pp z_{xy} = -\frac{\p z_y \cdot  \p z_x}{z} = -\frac{xy}{z^3} = -\frac{xy}{\sqrt{a^2-x^2-y^2}^3 }$
\end{example}
Примеры нарушения условия 2 теоремы
\begin{example}
    $x = y^3\quad F(x,t) = x-y^3$

    $\p F_y = -3y^2 = 0\quad y=0$
\end{example}

\begin{example}
    $\left( x^2+y^2 \right) ^2 = x^2-y^2\quad \begin{cases}
        x = r\cos\varphi\\
        y = r\sin\varphi\\
    \end{cases}$ 

    $r^4 = r^2\left( \cos ^2\varphi - \sin^2 \varphi \right) \quad r = \sqrt{\cos 2\varphi} $

\begin{figure}[!ht]
    \centering
    \incfig{lemniscat}
    \caption{lemniscat}
    \label{fig:lemniscat}
\end{figure}

$F(x,y) = \left( x^2+y^2 \right) ^2-(x^2-y^2)$

$\p F_y = 2\left( x^2+y^2 \right)\cdot 2y + 2y = 2y\left( 2(x^2+y^2)+1 \right)  $ 

$\p F_y = 0 \iff y = 0$

$\int g(x) = y$ -- функция, график которой -- график  $F$ в I-III четвертях

 $\begin{cases}
     y = r(\varphi)\sin \varphi\\
     x = r\left( \varphi \right) \cos \varphi\\
 \end{cases}$ 

 $\varphi = \frac{\pi}{4} $ 

 $\p x_{\varphi} = \p r\cos \varphi + r\sin \varphi$

 $\p x(\frac{\pi}{4} = \frac{\sqrt{2} }{2}\left( \p r + r \right) = \frac{\sqrt{2} }{2}\left( \underbrace{\frac{-\sin 2\varphi}{\sqrt{\cos 2\varphi} }}_{\text{не опр. при} \varphi=\frac{\pi}{4} } + \sqrt{\cos 2\varphi}  \right)  $
\end{example}

\begin{note}
    Если $f\in C\left( [a,b] \right) $ и $\exists \lim_{x \to a+} \p f(a)$, то $\exists \p f(a)$, и $\p f(a) = \lim_{x \to a+} \p f(x)$
\end{note}

\begin{example}
    $y(x) = x^2\sin \frac{1}{x}$ 

    $\varphi\in \left( 0, \frac{\pi}{4} \right) $

    $\p y_x = \frac{\p y_{\varphi}}{\p x_{\varphi}} = \frac{\p r\sin \varphi + r\cos \varphi}{\p r\cos \varphi - r\sin \varphi}= \frac{\left( -\frac{\sin 2\varphi}{\sqrt{\cos 2\varphi} }\cdot \sin \varphi + \sqrt{\cos 2\varphi}\cdot \cos \varphi  \right) }{-\frac{\sin \varphi}{\sqrt{\cos 2\varphi}} \cdot \cos \varphi - \sqrt{\cos 2\varphi} \sin \varphi} \to \frac{-\frac{\sqrt{2} }{2}}{-\frac{\sqrt{2} }{2}} = 1\implies \p y_x(x_{\varphi = \frac{\pi}{4}) }$ 

    $\p y_{+}(0) = 1$
\end{example}

\section{Поверхности в $\R^n$ ($k$-мерные)}

Способы задания поверхности:
\begin{enumerate}
    \item Поверхность уровня

        $F: \R^m \to \R\quad C\in \R$

        $\left\{ x\in D(F):\quad F(x) = C \right\} $ -- линия уровня $C$

        $F:D \text{ -- открытое }\subseteq \R^n \to \R^k\quad F = \left( F_1, \ldots, F_k \right) , C\in \R^k$

        $\left( x\in D(F): F(x) = C \right) $ -- поверхность уровня $C$
\end{enumerate}

\begin{definition}
    $\sqsupset f: O\to \R^m\quad O$ -- открытое в $\R^k\quad k, n\in \N $

    $f$ называется регулярным в  $O$, если  $f$ Дифференцируема в  $O$ и в каждой точке  $x\in O\quad \p f(x)$ Имеет максимальный ранг ($\rang \p f(x) = \min(k,n)$)
\end{definition}

\begin{definition}
    $\sqsupset C\in \R^m\quad F\in C^r(O\to \R^m), O$ -- открытое в $\R^n$

    $r\in \Z _+$

    $F$ регулярно в  $O$, тогда поверхность уровня  $C$ называется $r$-гладкой (класса $C^r$) $n-m$-мерной поверхностью
\end{definition}

\begin{example}
    $x^2+y^2+z^2=a^2, a>0$

    $F(x,y,z) = a^2\quad m = 1\quad n =3\quad \p F = \left( 2x, 2y, 2z \right) $

    $S$ --  $3-1=2$-мерная поверхность
\end{example}

\begin{enumerate}
    \item [2] Поверхности-графики

        $\sqsupset g: D\subseteq \R^k \to \R^m\quad k<n\quad n = k + m$

        $\Gamma_g = \left\{ (x,y) : x\in D, y = g(x) \right\}\in \R^n $

        Если $D$ Ограничено в  $\R^k\quad g\in C^r\left( D \to \R^m \right) $, то говорят о графике отображения гладкости $r$ (класса  $C^r$)
\end{enumerate}

\begin{example}
    Сфера -- объединение бесконечно гладких графиков
\end{example}

\begin{enumerate}
    \item [3] Параметрическое задание

        $\Phi: D\subseteq \R^k\to \R^n$

        Если $D$ -- открытое,  $\Phi\in C^r(D)$ и $\Phi$ Регулярна в  $D$, то  $\Phi(D)$ -- $k$-мерная поверхность с параметризацией класса  $C^r$  (параметризованной поверхности)
\end{enumerate}

\begin{example}
    $S \cap \left\{ x>0, y>0, z>0 \right\}  = \left\{ (x,y,z): 
    \begin{matrix}
        x = a\cos \varphi\cos \psi\\
        y = a\sin\varphi\cos \psi\\
        z = a\sin \psi\\
\end{matrix}\quad \varphi, \psi\in (0, \frac{\pi}{2} \right\} $ 

$\Phi: \left( \varphi, \psi \right) \to \left( x, y, z \right) $
\end{example}

\begin{theorem}
    [О способах задания гладких поверхностей]

    $\sqsupset r\in \Z _+, m, k\in \N  n = m+k$

    $S \subseteq \R^n\quad a\in S$. Следующие утверждения равносильны:
    \begin{enumerate}
        \item $\exists$ окрестность $U_a: U_a \cap S$ -- $k$-мерный график класса  $C^r$
        \item  $\exists $ окрестность $U_a: U_a\cap S$ -- $k$-мерная поверхность уровня класса  $C^r$
        \item  $\exists $ окрестность $U_a: U_a\cap S$ -- $k$-мерная поверхность класса  $C^r$ заданная параметрически
    \end{enumerate}
\end{theorem}
\begin{proof}
    \begin{itemize}
        \item []
        \item [$1\implies 2$] $U_a\cap S = \left\{ (x,y): x\in D\quad y = g(x) \right\} $ (посе перенумерации, если требуется)

            $=\left\{ (x,y): F(x,y) = 0 \right\} \quad F = y-g(x)$

            $F$ определена на  $D\times \R^m\quad F: D\times \R^m \subseteq \R^n \to \R^m\quad F\in C^r\left( D\times R^m \right) $

            $\p F_y = E_m\quad \p F $ содержит  $E$ как минор  $\implies $ ранг $\p F$ максимальный

        \item [$2\implies 1$] $F(x,y) = C\quad C\in \R^m\quad F: O\subseteq \R^n \to  \R^m$

            $U_a\cap S = \left\{ (x,y): (2) \right\} $ 

            $\p F$ имеет максимальный ранг  $=m$. С точностью до нумерации координат можно считать, что  $\det \p F_y(a) \neq  0 $. В некоторой окрестности  $\det \p F_y(x,y) \neq 0 \implies $ по теореме о неявной функции $y = g(x)$
        \item [$1\implies 3$] $S\cap U_a = \left\{ (x,g(x)):x\in D \right\} $

            $\Phi: x\in D \subseteq \R^k\to  (x,g(x))\in \R^n$

            $\p \Phi = \begin{bmatrix} &&\\&E_k&\\&&\\ \p g_{1x_1} & &\p g_{1x_k}\\ &&\\ &&\\ \end{bmatrix} \implies \rang \p \Phi = k \implies \Phi$ - регулярна в $D$
        \item [$3\implies 1$] $ S\cap U_a = \left\{ (x,y) = \Phi(u):u\in D \text{ -- открытое в } \R^k \right\} \quad \Phi\in C^r(D), \rang \p\Phi=k$ -- максимальный 

            Не умаляя общности $\det \p \Phi_x\neq 0\quad \tl \Phi(n) = \begin{bmatrix} \Phi_1(u)\\ \vdots\\ \Phi_k(u) \end{bmatrix} \qquad \Phi(n) = \begin{bmatrix} \Phi_1(u)\\ \vdots\\ \Phi_n(u) \end{bmatrix}$

            $\det \begin{bmatrix} \nabla \Phi_1\\ \vdots\\ \nabla \Phi_k \end{bmatrix} \neq 0$

            $\tl{\p \Phi} = \begin{bmatrix} \nabla \Phi_1 \\ \vdots \\ \nabla \Phi_k \end{bmatrix} $ -- обратима

            $\Psi  = \tl{\Phi}^{-1}\quad \tl \Phi :\left( u_1, u_2, \ldots, u_k \right) \to \left( x_1, \ldots, x_k \right) $

            $(x,y) = \Phi(u) = \left( \tl \Phi(u), \tll \Phi(u) = (x, \tll \Phi(\Psi(x))  \right) \in C^r \text{ по теореме об обратном отображении }\quad u = \Psi(x)\quad \Phi(u) = \Phi(\Psi(x))$

    \end{itemize}

    \begin{definition}
        $r$-гладках  $k$-мерная поверхность -- поверхность, для которой справедливо одно из утверждений 1-3 предыдущей теоремы
    \end{definition}

    \begin{example}
        $\gamma(t) = \begin{bmatrix} \gamma_1(t)\\ \vdots \\ \gamma_n(t) \end{bmatrix} :[a,b] \to \R^n$

        $\gamma |_{(a,b)}\quad \gamma\in C^r(a,b)$

        $\rang \p \gamma$ максимален  $\iff \p \gamma(t)\neq 0$
    \end{example}

    \begin{example}
        в  $\R^n\quad D = \left\{ \left<v_1, x \right> + v_2 = 0 \right\} $, где $v_1, v_2\in \R^n$

        $\p F = v_1$. Если $v_1\neq 0$ , то $S$ --  $n-1$-мерная поверхность (гиперплоскость, гперпространство в  $\R^n$)
    \end{example}

    \section{Условный экстремум функций нескольких переменных}

    $f:E\subseteq R^n \to \R\quad E_0\subseteq E$

    Условный экстремум $f$ на  $E_0\equiv$ экстремум $f |_{E_0}$

    $\sqsupset E_0 = \left\{ x\in \R^n: \begin{cases}
            F_1(x) = 0\\
            \vdots\\
            F_m(x) = 0\\
    \end{cases}, m\in \N , m\leqslant n \right\} $ 

    Эти уравнения называются уравнениями связи , функции $F_i$ называются функциями связи

    Экстремум  $F|_{E_0}\equiv $ условный экстремум $F$ при условии  $\left<\text{система уравнений} \right>$
\end{proof}

\begin{theorem}
    [Необходимое условие условного экстремума, геометрическая формулировка]

    $\sqsupset O$ -- открыток $\subseteq \R^n\quad f, F_1, \ldots, F_m\in C^1(O\to \R), m<n$ и $F = \left( F_1, \ldots, F_m \right) $ -- регулярно в $O$.  $\sqsupset $ a -- точка локального условного экстремума для $f$ относительно $F(x) = 0$ 

    Тогда в точке $a\quad \nabla f$ представимо в виде линейной комбинации $\nabla _aF_1, \ldots, \nabla F_m\quad \left( \nabla _1f\in \mathcal L_{in} \left\{ \nabla _aF_1, \ldots, \nabla _af_m \right\}  \right) $
\end{theorem}
\begin{proof}
    \begin{enumerate}
        \item [случай 1:] $m = n-1$

            От противного :  $\rang \begin{bmatrix} \p f(a)\\
                \p F_1(a)\\
                \vdots\\
                \p F_m(a)\end{bmatrix} <n\quad \left( \iff \det \begin{bmatrix} \p f(a)\\ \p F_1(a)\\ \vdots \\ \p F_{n-1}(a) \end{bmatrix}  \right)  = 0 \impliedby  \p f(a), \ldots, \p F_{n-1}(a)$ -- линейно зависимы $ \implies  \exists \lambda_0, \ldots, \lambda_{n-1}:\quad \lambda_0\p f(a) + \sum_{k=0}^{\infty} \lambda_k\p F_k(a) = 0$ 

                Если бы $\lambda_0 = 0 \implies \left\{ \p F_k(a) \right\} _{k=1}^{n-1}$ -- линейно зависимо $\rang\left( \p F_k(a) \right) _{k=1}^{n-1}<n-1$ -- не максимально

                Значит $\lambda_0\neq 0 \implies \p f(a) = -\sum \frac{\lambda_k}{\lambda_0}\p F_k(a)$
    
                $\sphericalangle h(x) = \begin{bmatrix} f(x)\\ F_1(x) \\ \vdots \\ F_{n-1}(x) \end{bmatrix} : Oc\R^n \to \R^n$

                Если неверно, что определитель матрицы производных  этого столбца функций равен нулю, то по теореме об обратном отображении $h(U_a)$ -- открытое множество  $\implies \exists \delta_0>0:\quad \forall \delta\in (0, \delta_0)\quad \begin{bmatrix} f(a)\pm \delta\\ F_1(a) \\ \vdots \\ F_{n-1}(a) \end{bmatrix} \in h(u_n)$

                $\implies \forall $ Открытого $\tl U_a\quad \exists $ точка $x: F_1(x) = F(a) = 0, \ldots, F_{n-1} = F_{n-1}(a) = 0 \qquad f(x) > f(a)$ или $f(x) < f(a)$

                Противоречие --  $x$ -- точка условного локального экстремума
        \item [$m=1, \ldots, n-2$]

            $\p F$ -- максимального ранга  $\sphericalangle \tl h(a) = \begin{bmatrix} f(x) \\ F_1(x)\\ \vdots \\  F_m\left( x \right)  \end{bmatrix} \quad $ Предположим, что $\rang \tl{\p h(x)}<m+1$

            С точностью до нумерации координат $\tl {\p h_{x_1, \ldots, x_{m+1}}}(a)\neq 0$

                $h(x) = \begin{bmatrix} \tl h(x)\\ x_{m+2} \\ \vdots \\ x_{n}  \end{bmatrix} \quad \p h(a) = \begin{bmatrix} \p h_{x_1, \ldots, x_{m+1} }&\ldots\\ \mathbb O\\ E_{n-(m+1)} \end{bmatrix} (a) \implies \exists $ окрестность $U_a\quad h|_{U_a}$

                $a \ov h {\to } \begin{bmatrix} f(a)\\ 0 \\ \vdots \\ 0 \\ a_{m+2} \\ \vdots \\ a_n \end{bmatrix} $. Т.к. $h$ -- открытое, то  $\exists  \sigma>0 :\forall y\in \R^n: \|y - b\|<\sigma \implies y\in h(U_a)$

                $\sphericalangle y = \begin{bmatrix} f(a)\pm \frac{b}{2}\\ 0 \\ \vdots \\ o \\ a_{m+2} \\ \vdots \\ a_n \end{bmatrix} \implies \exists x\in U_a: f(x) = y\quad F_1(x) = \ldots = F_m(x) = 0\quad f(x) f(a) \pm \frac{b}{2}$ 

               Противоречие, ранг не меньше максимального
    \end{enumerate}
\end{proof}



\section{Практика}

$u$ -- непрерывная функция на компакте -- достигает наибольшего и наименьшего значения


\section{Лекция. Дополнение : теорема об открытом отображении (общение)}

\begin{theorem}
    $\sqsupset k, n\in \N \quad k \leqslant  n\quad O$ -- открытое $\in \R^n\quad F:O\to \R^k$, регулярная ($\in C^1\quad$ ранг матрицы Якоби  $=k$)

    Тогда  $F$ является открытом отображением.
\end{theorem}

Случай $n = k$ был установлен.

\begin{proof}
    $ n < k$

     \begin{enumerate}
         \item $F$ -- проекция.  \[F(x_1, x_2, \ldots, x_{n} ) = \left( x_{i_1}, \ldots, x_{i_k} \right) \] -- очевидно открытое

\begin{figure}[!ht]
    \centering
    \incfig{очевидно}
    \caption{очевидно}
    \label{fig:очевидно}
\end{figure}

$x, a\in \R^n\quad \|x-a\|<r \implies \|F(x) - F(a)\|\leqslant \|x-a\|<r$, т.е. $B_{r(a)} \overset F {\to } B_r\left(F(a)  \right) $ 
        \item  $F$ -- регулярно.  $\p F = \begin{bmatrix} \nabla F_1\\ \vdots\\ \nabla F_k \end{bmatrix} $, матрица $n\times k$

            После удаления $n-k$ столбцоы возникает ненулевой минор . Не умаляя общности удаляем последние столбцы (иначе перенумеруем переменные  $x_{1}, \ldots, x_{n} $) $\implies \p F_{\left( x_1, \ldots, x_{n}  \right) } = \begin{bmatrix} \p F_{1x_1} &\ldots & \p F_{1x_k}\\ \ldots & \ldots & \ldots\\ \p F_{kx_1} &\ldots & \p F_{kx_k} \end{bmatrix}\quad \det \p F_{\left( x_1, \ldots, x_{n}  \right) } \neq 0$

            $\sphericalangle \phi(x) = \left( F_1, F_2, \ldots, F_k, x{k+1}, \ldots, x_{n}  \right)^T $

            $\p \phi(x) = \begin{bmatrix} \p F_{\left( x_1, \ldots, xk \right) }& \p F_{\left( x_{k+1}, \ldots, x_{n}  \right) }\\ 0 & E_{n-k} \end{bmatrix} $ 

            $\det \p \phi(x) = \det \p F_{\left( x_1, \ldots, x_k \right) } \cdot  \det \left( E_{n-k} \right) \neq 0\quad \phi$ регулярно в некоторой окрестности фиксированной точки $\implies F = \pi\circ \phi$ -- открытое в точке $a$  в силу произвольности  $a \implies F$ -- окткрыто
    \end{enumerate}     
\end{proof}

\begin{theorem}
    [Необходимое условие условного экстремума (геометрическая формулировка)]

    $\sqsupset k, n\in \N\quad k < n\quad  O$ -- открытое в $\R^n$ 

    $f, F_1, \ldots, F_k\in C^1(O\to \R), \quad F = \left( F_1, \ldots, F_k \right) $ -- регулярно в $O$

    $E = \left\{ x\in O| F(x) = 0 \right\} ,  a\in E$

    Если $a$ -- точка условного экстремума для  $f$ Относительно  \[F(x) = 0\], то \[\nabla_a f\in \mathcal Lin\left\{ \nabla _a F_1, \ldots, \nabla _a F_k \right\} \]

    В частности, если $k=1$, то условие  $\iff \nabla_af $ -- коллинеарен $\nabla _aF$
\end{theorem}

\begin{example}
    Найти наибольшее и наименьшее значение функции \[f(x,y) = (x^2  + (y-1)^2)\sqrt{x^2+y^2-2y} \] на криволинейном треугольнике -- границе множества \[ D = \left\{ (x,y)\mid (x+1)^2 + y^2\geqslant 1,\quad (x-1)^2+y^2\geqslant 1, x^2+y^2\leqslant 2\quad y\geqslant 0 \right\}  \]

    $r = x^2+(y-1)^2$

    $f = \underbrace{r\cdot \sqrt{r+1}}_{g(r)} $ 

    $\nabla f = \p g(r) \cdot  \nabla r$

    при $r = \sqrt{2} - 1 $ касание окружности уровня границы окружности 

    Подозрительные точки: вершины. точки $T_1, T_2, T_3:\quad f(T_1) = f(T_2) = f(T_3)$
\end{example}

\begin{figure}[!ht]
    \centering
    \incfig{krivoyprimer}
    \caption{krivoyPrimer}
    \label{fig:krivoyprimer}
\end{figure}

$\p g(r) = \sqrt{r+1}  + \frac{r}{2\sqrt{r+1}} = \frac{2(r+1) + r}{2\sqrt{r+1} } = \frac{3r+2}{2\sqrt{r+1} }$ не обращается в ноль при $r\geqslant 0$

$K$ -- граница  $D$, компакт $\implies \max, \min$ достижимы

т.к. $g(r)$ стремится вверх и  $r(T_1) = r(T_2) = r(T_3) < g(V_1) = g(V_2) = g(V_3) \implies \min_k g = g(T_1)\quad \max_k g = g(V_1)$



\begin{proof}
    Не умаляя общности $a$ -- точка глобального условного экстремума для  $f$ относительно  $F(x) = 0$

     \begin{note}
         Если $v, v^{1}, \ldots, v^{k}\in \R^k$ и набор $v, v^{1}, v^{k}$ -- линейно зависим, а $v^{1}, \ldots, v^{k}$ -- ЛНЗ, то $v\in \mathcal Lin\left\{ v^1, \ldots, v^k \right\} $ 

         $\implies \exists \lambda_1, \lambda_1, \ldots, \lambda_k:\quad \lambda v + \underbrace{\sum_{j=1}^{k} \lambda_jv_j}_{\neq 0} = 0\implies \lambda\neq 0 \implies v = -\sum_{j=1}^{k} \frac{\lambda_j}{\lambda}v_j$
    \end{note}

    По замечанию достаточно проверить, что $\left\{\nabla _af, \nabla _a F_1, \ldots, \nabla _a F_k\right\}$ -- линейно зависим, т.е. $\tl F = \left( f, F_1, \ldots, F_k \right) ^T$ -- не регулярна в точке $a$.

    От противного: пусть  $\tl F$ -- регулряна в точке $a$, тогда существует окрестность  $U(a): \tl F$ -- регулярно в $U(a)$ 

    По теореме об открытом отображении $\tl F$ -- открыто в  $U(a) \implies \exists \varepsilon >0: \tl F(U(a)) \supset B_{\varepsilon}\left( \tl F(a) \right)\quad \tl F(a) = \left( f(a), F_1(a), \ldots, F_k(a) \right)  = (f(a), 0, \ldots, 0)\quad F_i(a) = 0 $, т.к. $a$ -- точка, удовлетворяющая формулам связи.

    $y_{\pm } = \left( f(a) \pm \frac{\varepsilon}{2}, 0, \ldots, 0 \right) $

    $\|y_{\pm} - \tl F(a)\| = \frac{\varepsilon}{2} \implies y_{\pm}\in B_{\varepsilon}\left( \tl F(a) \right) \subseteq \tl F(U(a)) \implies \exists x_1\in U(a): \tl F(x_{\pm}) = y_{\pm} \iff  f(x_1) = f(a) \pm \frac{\varepsilon}{2}\quad F_1(x_{\pm}) = \ldots = F_k\left( x_{\pm} \right) \implies x_{\pm}\in E?!!  \implies $ точка $a$ не экстремум, что противоречит нашем предположению
\end{proof}

\section{Функция Лагранжа}

\begin{definition}
    [ ``большая'' функция Лагранжа]
    $
    \begin{matrix}
        f\left( x_1, \ldots, x_{n}  \right) \\
        F_1\left( x_1, \ldots, x_{n}  \right) =0\\
        \ldots\ldots\ldots\\
        F_k\left( x_1, \ldots, x_{n}  \right) =0
    \end{matrix}\quad \mathcal L\left( x, \lambda \right)  = f(x) - \sum_{j=1}^{k} \lambda_iF_j(x)\quad \lambda = \left( \lambda_1, \ldots, \lambda_k \right) $

    \[\nabla _af = \sum_{j=1}^{k} \lambda_j \nabla _aF_j\]
\end{definition}

\begin{theorem}
    [Необходимое условие условного экстремума через дифференациал функции Лагранжа]

    В условиях последней теоремы $\exists \lambda_1^*, \ldots, \lambda_k^*\in \R\quad \mathrm d_{\left( a, \lambda^* \right) } \mathcal J = 0$

    В скалярной записи это запишется как:
    $\begin{cases}
        \frac{\partial \mathcal L}{\partial x_1}\left( a, \lambda^* \right)  = \p f_{x_1}(a) - \sum_{j=1}^{k} \lambda_j \p F_{jx_1}(a)\\
        \vdots\\
        \frac{\partial \mathcal L}{\partial x_n}\left( a, \lambda^* \right)  = \p f_{x_{n}}(a) - \sum_{j=1}^{k} \lambda_j^*\p F_{jx_{n} }(a)\\ \hline
        \frac{\partial \mathcal L}{\partial \lambda_1}\left( a, \lambda^* \right) = 0 = F_1(a)\\
        \vdots\\
        \frac{\partial \mathcal L}{\partial \lambda_k}\left( a, \lambda^* \right) = 0 =F_k(a)
    \end{cases} \iff  \begin{cases}
        \nabla f = \sum_{j=1}^{k} \lambda_j \nabla _aF_jё
    \end{cases}$
\end{theorem}

\section{Примеры приложения необходимого условия условного экстремума}

\begin{problem}
    Найти максимум и минимум квадратичной формы на сфере

    \[Q(x) = \sum_{ki,j=1}^{n} q_{ij}x_ix_j\] -- $\forall $ квадратичная форма в $\R^n$

    $x = \left( x_1, \ldots, x_{n}  \right) $

    Продолжаем $[Q] = \left( q_{ij} \right) ^n_{i,j=1}$ -- симметричная

    $S = \left\{ x\in \R^n:\|x\| = 1 \right\} \iff \sum_{i=1}^{n} x_i^2 = \|x\|^2 = 1 $ 

    $F(x) = \|x\|^2-1$

    $\mathcal L(x, \lambda) = Q(x) - \lambda F(x)$

    $\nabla F = 2x$ ($F$ регулярно всюду в  $\R^n\setminus \left\{ 0 \right\} $)

    <...>

    Вывод: $\max_{x\in S}Q(x) = \max \left\{ \lambda: \lambda_i \text{ -- собственные числа } Q \right\} $ 


\end{problem}
$\sqsupset E\subseteq \R^n \quad f\in C\left( \Cl E \to  \R \right)  $. Тогда $\sup_E f = \sup _{\Cl E}$
\end{document}
\begin{proof}
    $A = \sup _Ef\quad B = \sup_{\Cl E}f \implies B\geqslant A$ как $\sup$ по большему множеству.

    $B\leqslant A$. От противного: существует последовательность $\left\{ y_k \right\} _{k=1}^{\infty } \subseteq f\left( \Cl E \right) $ 

    $y_k\to B \implies \exists \left\{ x_k \right\} \subseteq \Cl E$ и $f(x_k) = y_k$

     $\forall k\in \N  \exists \tl x_k \subseteq E\quad \|\tl x_k - x_k\|<\frac{1}{k}$

     <..> вернёмся к этому доказательству позже
\end{proof}

\begin{statement}
    $\sqsupset E$ -- замкнутое множество в $\R^n\quad f\in C(E)$ и $\lim_{\|x\| \to \infty} f = L$

    Тогда \[\sup_E f \times f(E) \cup \{L\}\]
\end{statement}
\begin{proof}
    По определению супремума существует $\left\{ y_j \right\}_{j=1}^{\infty }\subseteq f(E): \p y_j\to \sup_Ef \implies \exists  $ последовательность $\left\{ x_j \right\} \subseteq E: f(x_j) = y_j$. В силу обощённого принципа Больцано-Вейерштрасса $\exists $ подпоследовательность $\left\{ x_{j_i} \right\} _{i=1}^{\infty }: x_{j_i}\to x_*\in \hat \R_r$:
    \begin{enumerate}
        \item Если $x_* = \infty \quad \underbrace{f{x_{j_i}}}_{\to L} = y_{j_i} \underset{i\to \infty } \to \sup_E f$. Теорема о пределе композции
        \item $x_*\neq \infty\quad x_*$ -- предельная точка для $E, E$ замкнуто   $\implies x_*\in E$

             Т.к. $f$ непрерывно на  $E$

             \[f(x_*) = \lim_{i \to \infty} f(x_{j_i} = \lim_{i \to \infty} y_{j_i} = \sup_Ef \implies \sup_E f \subseteq f(E)\]
    \end{enumerate}
\end{proof}

\section{Касательные пространства к поверхностям}

$\sqsupset S$ -- гладкая $k$-мерная поверхность пространства  $\R^n\quad q\in S$

вектор $\tau\in \R^n$, для которого существует гладкий путь $\gamma: [a,b]\to S$ и существует $c\in [a,b]: \p \gamma(c) = \tau$ называется касательным в  $S$ в точке $q$ \[\gamma(c) = q\]

$\Tq S$ -- набор всех касательных векторов к  $S$ в точке  $q$

$\sqsupset \phi$ -- гладка локальная параметризация $S$ вблизи  $a$ ($\exists $ окрестность $U(0)$ в  $\R^n$ и $\exists $ Окрестность $V(q): \Phi(U(0)) = V(q)\cap S\quad \Phi(0) = q$) 

$\gamma^1(t) = \left( t, 0, \ldots, 0 \right) ^T\quad \gamma^2(t) = \left( 0, t, 0, \ldots, 0 \right) ^T\quad \gamma^j:[0,1]\to \R^k\quad \tl \gamma^j = \phi(\gamma^j)$ -- гладкий путь в $S$

$\tau^j = \left( \tl \gamma^j \right) ^{\prime}_t = \p \Phi \cdot  \left( \gamma^j \right) ^{\prime} = j$-ый столбец $\p\Phi$ 

$\tau^1, \ldots, \tau^k$ -- канонические касательные векторам

\begin{statement}
    $\Tq S$ -- линейное пространство
\end{statement}

Локально $S$ также задаётся системой уравнений:

 $\begin{cases}
     F_1(x) = 0\\ \vdots \\ F_{n-k}(x) = 0\\
 \end{cases}\quad \sqsupset \tau\in Tq(S)\quad \tau = \p\gamma(c)$, где $\gamma: [a,b]\to S\quad \gamma(c) = q$

 \section{Лекция}

 \begin{note}
     $\exists \subseteq \R^n\quad f\in C\left( \Cl E \to \R \right) $

     Тогда $A := \sup_E f = \inf_{\Cl E} f =: B$
 \end{note}
 \begin{proof}
     Очевидно $A \leqslant B.\quad B\leqslant A$?

     $\sqsupset x\in \Cl E \implies \exists \left\{ x_j \right\} _{j=1}^{\infty }\subseteq E: x_j\to x, j\to \infty $

     $\forall j\in \N \quad f(x_j)\leqslant A, j\to \infty $

     $f(x) = f\left( \lim_{j \to \infty} x_j \right)  = \lim_{j \to \infty} f(x_j) \leqslant A$

     Перейдём к супремуму по $x: x\in \Cl E \implies B\leqslant A$
 \end{proof}

\begin{figure}[!ht]
    \centering
    \incfig{poverh}
    \caption{poverh}
    \label{fig:poverh}
\end{figure}

$S$-гладкая,  $k$-мерная поверхность.  $\gamma:[a,b]\to S\quad $-- гладкий путь в $S$

$\p \gamma(c)$ для  $c\in [a,b]\quad \gamma(c) = p$

$u = \left( u_1, u_2, \ldots, u_n \right) $

$\p \phi_{u_1}(q)\quad \p \phi_{u_k}(q)\in \Tp S$ -- какнонические касательные векторы

$\p \phi_{u_j}(q) = d_q\phi(e_j)$

 $S$ локально задана как поверхность уровня

 $\exists $ окрестность $U(p):$

 \[S\cap \left( U(p) \right)  = \left\{ x\in U(p): F_1(x) =0, \ldots, F_{n-k}(x) = 0 \right\} \] 

 $F = \left( F_1, \ldots, F_{n-k} \right) $ -- регулярно в $U(p)$

  $\forall \tau\ini \Tp S\quad \forall j=1, \ldots, n-k\quad \tau \perp \nabla _p F_j$

  \begin{statement}
      Если $s$ -гладкая  $k$-мерная поверхность (в $\R^n$), $p\in S$, то  $\Tp S$ -- линейное подпространство  $\R^n$
  \end{statement}
  \begin{proof}
\begin{figure}[!ht]
    \centering
    \incfig{skpoverh}
    \caption{skpoverh}
    \label{fig:skpoverh}
\end{figure}

Если $v_1, v_2\in \Tp S\quad c_1, c_2\in \R \implies v = c_1v_1 + c_2v_2\in \Tp S$

Дополним набор $\nabla _pF_1, .., \nabla _pF_{n-k}$ до базиса ортогонального дополнения к $v\in \R^n$

Добавляем $h^1 \ldots, h^{k-1}\quad \forall j\quad g^j\perp v$

$\sphericalangle $ систему $n-1$ уравнений  $\begin{cases}
    F_1(x) = 0\\
    \vdots\\
    F_{n-k}(x) = 0\\
    \left<x-p, h^1 \right> = 0\\
    \vdots\\
    \left<x-p, h^{k-1} \right> = 0\\
\end{cases}$

Матрица $\begin{bmatrix} 
        \nabla _pF_1\\
        \vdots\\
        \nabla _p F_{n-k}\\
        h^1\\
        \vdots\\
        h^{k-1}
    \end{bmatrix} $  имеет максимальный ранг $\implies $ система задёт (локально) гладкую $n - (n-1) = 1-$мерную поверхность 

    По теореме о способах задания поверхности $\exists $ параметризация $\gamma: t\in \left( \alpha, \beta \right) \to \Gamma\quad 0\in \left( \alpha, \beta \right) ]quad \gamma(0) = p$

    $\p \gamma \perp \nabla_p F_1, \ldots, \nabla _p F_{n-k}, h_1, \ldots, h_{k-1}$

    $\implies \p \gamma(0)$ коллинеарна $v$

    $\sphericalangle \tl \gamma(t) = \gamma(\Theta t)\quad \Theta \in \R\setminus \{0\}$

    $\tl {\p \gamma}(t) = \Theta \p \gamma(\Theta t)$

    $\tl {\p \gamma}(0) = \Theta \cdot  \p \gamma(0)$

    Подбор $\Theta \implies \Theta \cdot  \p \gamma(0) = v$

  \end{proof}

  \begin{corollary}
      $\dim \Tp S = k$

      Т.к. канонические касательные линейно независимы,  $\in \Tp S, \implies \dim \Tp S \geqslant k$

      С другой стороны $\dim \left( \Tp S \right) ^{\perp} \geqslant n-k$ (т.к. $\nabla _p F_j\in \left( \Tp S \right) ^{\perp} $)
  \end{corollary}

  $v\in \Tp S \iff  \begin{cases}
      \left<\nabla _pF_1, v \right> = 0\\
      \left<\nabla _p F_{n-k}, v \right> = 0\\
  \end{cases} \iff  \begin{cases}
  \frac{\partial F_1}{\partial x_1}(p)(v_1) + \ldots + \frac{\partial F_1}{\partial x_{n} }(p)v_n = 0\\
  \ldots\\
  \frac{\partial F_{n-k}}{\partial x_1}(p)(v_1) + \ldots + \frac{\partial F_{n-k}}{\partial x_{n} }(p)v_n = 0\\
  \end{cases}\iff d_p F(v) = 0$

  Можно считать, что дифференциалы $dx_1, \ldots, dx_{n} $ на касательном пространстве связаны системой возникающей при формальном дифференцировании системы $F(x) = 0$

  $d_p F = 0\quad \begin{cases}
  \frac{\partial F_1}{\partial x_1}(p)dx_1 + \ldots + \frac{\partial F_1}{\partial x_{n} }(p)dx_{n}  = 0\\
  \ldots\\
  \frac{\partial F_{n-k}}{\partial x_1}(p)dx_1 + \ldots + \frac{\partial F_{n-k}}{\partial x_{n} }(p)dx_{n}  = 0\\
  dx_i(v) = v_i\\
  \end{cases}$

 \begin{corollary}
     $\Tp S = d_q \Phi\left( \R^k \right) $ 

     Оба объекты линейные подмножества $\R^n$ размерности $k$ и содержат  $d_q\Phi(e_1), \ldots, d_q\Phi(e^k)$
 \end{corollary}

 \begin{example}
     $x^2+y^2+z^2 = R^2\quad S_2(\R)$

     уравнение касательного пространства $2\left( xdx, ydy, zdz \right)  = 0$

     $v\in T_{(x,y,z)}S_2(\R) \iff  xv_1, yv_2, zv_3 = 0$
 \end{example}

 $\Tp S$ -- касательное линейное пространство

  $\Lp S$ -- афинное касательное пространство  $\Lp S = \Tp S + P$

   $x\in \Lp S\quad x - p\in \Tp S$

   $\Lp S = \left\{ x\in \R^n: d_pF(x-P) = 0 \right\} $ 

   $d_pF(x-p) = 0 \iff \begin{cases}
       \frac{\partial F_1}{\partial x_1}(p)(x_1-p_1) + \ldots + \frac{\partial F_1}{\partial x_{n} }(p)(x_{n} -p_n) = 0\\
       \ldots\\
       \frac{\partial F_{n-k}}{\partial x_1}(p)(x_1-p_1) + \ldots + \frac{\partial F_{n-k}}{\partial x_{n} }(p)(x_{n} -p_n) = 0\\
   \end{cases}$ 

   Если $k = n-1$, то  $\Tp S$ и  $\Lp S$ называются касательными (гипер)плоскостями

   Касательная гиперплоскость к графику функции?

   $S = \Gamma_f = \left\{ \left( x_1, \ldots, x_{n-1}, y \right) \in \R^n, \quad \left( x_1, .., x_{n-1}\inO \subseteq \R^{n-1} \right)  \right\} \quad y = f\left( x_1, \ldots, x_{n-1} \right) $ 

   $f\in C^1(O)$

   $F\left( x_1, \ldots, x_{n-1}, y \right)  = f\left( x_1, \ldots, x_{n-1} \right)  - y$

   $\sphericalangle F$ в $O\times \R$

   $\nabla_{p,y}F = \left( \p f_{x_1}(p), \ldots, \p f_{x_{n-1}}(p), -1 \right) $ 

   $x\in \Lp \Gamma_f \iff \p f_{x_1}(p)(x_1-p_1) + \ldots + \p f_{x_{n-1}}(p)(x_{n-1}-p_{n-1}) - \left( y - p_y \right)  = 0$

   $y = f\left( p_1, \ldots, p_{n-1} \right)  + \left<\nabla _{p_1, \ldots, p_{n-1}}f, (x-p_x) \right>$

   \begin{statement}
       [Достаточное условие условного экстремума]

       $\sqsupset O$ -- открытое в $\R^n\quad f, F_1, \ldots, F_{n-k}\in C^2\left( O\to \R \right) \quad F = \left( F_1, \ldots, F_{n-k} \right) , p$ -- решение системы \[F(x) = 0\]

       $\sqsupset $ в точке $p$ выполнено необходимое условие условного экстремума для  $f$ относительно $F(x) = 0$.  $L\left( x_1, .., x_{n}  \right)  = \mathcal L\left( x_1, \ldots, x_{n}, \lambda_1^*, \ldots, \lambda_{n-k}^*  \right) $

       Тогда:
       \begin{itemize}
           \item  Короткий вариант

               Если  $d^2_pL >0 $, то $p$ -- условный минимум

               Если  $d^2_pL<0$, то  $p$ -- условный максимум

               Из $d^2_pL \gtrless 0$, то НЕ следует, что $p$ не точка условного экстремума
           \item Подробный вариант:

               Если $d^2_pL|_{\Tp S}>0$, то $p$ -- точка условного минимума

               Если  $d^2_pL|_{\Tp S}<0$, то $p$ -- точка условного максимума

               Если  $d^2_pL|_{\Tp S}\gtrless$, то $p$ -- седловая точка (условного экстремума нет)
       \end{itemize}
   \end{statement}

   \begin{example}
       $f(x,y) = x^2+y^2$ на $x + y = 1$. найти условный экстремум.

       1-ый способ:  $y = 1-x\quad g(x) = f(x,1-x) = x^2+(1-x)^2 = 2x^2-2x+1$

       $g(x)$ имеет минимум в точке $x = \frac{1}{2}$ (глобальный) $\implies f(x,y)$ имеет условный (глобальный) минимум в точке $(\frac{1}{2}, \frac{1}{2})$ 

       2-ой способ: $\mathcal J\left( x, y, \lambda \right)  = x^2+y^2-\lambda \left( x+y-1 \right) $

       $\begin{cases}
           0 = \mathcal J_x^{\prime} = 2x - \lambda\\
           0 = \mathcal J_y^{\prime} = 2y-\lambda\\
           x+y = 1\\
       \end{cases} \implies   \begin{cases}
           x = y = \frac{1}{2}\\
           \lambda = 2x = 1\\
       \end{cases}$

       $L(x,y) = x^2+y^2-(x+y-1)$

       $\pp L \iff  \begin{bmatrix} 2&0\\0^2\\ \end{bmatrix} $

       $d^2 L > 0$ по короткому  $\quad p = \left( \frac{1}{2}, \frac{1}{2} \right) $ -- условный минимум
   \end{example}

   Наводящие соображения по достаточному условию условного экстремума:
   \begin{enumerate}
       \item $S = \left\{ x\in O\quad F(x) = 0 \right\} $. Исседуется $f|_S$

           $\mathcal L(x) = f(x) - \sum \lambda_kF_k(x)$

           $L = \mathcal L(x, \lambda^*)$

            $L|_S = f_L$,  таким образом условный экстремум  $f \leftrightarrow $ условный экстремум $L$

             $\phi$ -- локально гомеомофризм

             $p$ -- точка условного экстремума для  $L|_S \iff  $ Для $L\circ \phi$ точка  $\phi^{-1}(p) = q$ -- безусловный экстремум

             $d^2(L\circ \phi) \leftrightarrow d^2_pL(d_q\phi) + d_pL\left( d^2_q \phi \right) $ -- одномерная формула 

   \end{enumerate}

   \section{О задача отыскания наибольшего и наименьшего значения функции на множестве.}

   $K$ -- компакт,  $f\in C(K)$

   $f_k$ не имеет экстремума на $K \setminus  K^* \implies \max_K f = \max _{K^*}f, \min..$


    Если $K = S_1\cup \ldots\cup S_N$, где $S_1, \ldots, S_N$ -- поверхности различных размерностей, $S_j\cap S_k \neq 0$

    $\left( S_j \right) _*$ -- множество точек, в которых выполняется НУУЭ (необходимое условие условного экстремума) относительно $x\in S_j$ (либо нарушается ранг системы, задающей нашу поверхность  $S_j$)

    $K_* = \left( S_1 \right) _* \cup \ldots \cup \left( S_N \right) _*$ -- если это конечное множество, то можно их уже просто сравнить

    \begin{example}
        Найти наибольшее и наименьшее значение функции $f(x,y,z)$ на компактном множестве $K$ ограниченном поверхностями  $z = 1$ и  $z = x^2+y^2$

\begin{figure}[!ht]
    \centering
    \incfig{paraplane}
    \caption{paraplane}
    \label{fig:paraplane}
\end{figure}

$S_0 = \Int K\quad S_1 = \left\{ z = 1, x^2+y^2<1 \right\} \quad S_2 = \left\{ \left( x, y, z \right) : z = x^2+y^2, x^2+y^2<1 \right\} \quad S_3 = \left\{ \left( x,y,z\right): z = 1, x^2+y^2=1  \right\} $ 

на $S_0$ необходимое условие безусловного экстремума: \[dF = 0 \iff  \begin{cases}
     0 = \p f_x = y\\
     0 = \p f_y = x+z\\
     0 = \p f_z = y\\
\end{cases} \implies   \begin{cases}
    y = 0\\
    z = -x\\
\end{cases}\quad \left( x,0,-x \right) \quad f\left( x,0,-x \right)  = 0\]

На $S_1\quad g(x,y):= f\left( x,y,1 \right)  = y(x+1)\quad  \begin{cases}
    \p g_x = 0 = y\\
    \p g_y = 0 = x+1\\
    x^2+y^2<1
\end{cases} \iff (-1,0)??\quad \O $

на $S_2\quad f(x,y,z) = xy + y\left( x^2+y^2 \right)  = x^2y+xy+y^3 = h(x,y)\quad x^2+y^2 <1$

\[\begin{cases}
    \p h_x  = 0 = 2xy+y\\
    \p h_y = 0 = x^2+x+2y^2\\
    x^2+y^2<1
\end{cases} \]

на $S_3\quad f = (xy+y)_{x^2+y^2=1} = \sin \varphi\left( \cos \varphi + 1 \right) = \varkappa(\varphi)\quad \varphi\in[-\pi, \pi ] $

\begin{align*}
    \p \varkappa(\varphi) &= \cos \varphi\left( \sin \varphi \right)  -\sin ^2\varphi\\
                          &= \cos ^2\varphi + \cos \varphi - \left( 1-\cos ^2\varphi \right)  \\
                          &= 2\cos ^2\varphi + \cos \varphi - 1  \\
                          &= 2t^2+t-1 =: \mu(t) \\
.\end{align*}

$(2t^2+t)^{\prime} = 4t+1$

$\varkappa(t=-1) = 0\quad \varkappa(t = \frac{1}{2} = \pm \frac{\sqrt{3} }{2}\left( \frac{1}{2} + 1 \right) = \pm \frac{3\sqrt{3} }{4}$

    Подозрительные значения: $0, \pm \frac{\sqrt{12} }{72}\quad \pm \frac{3\sqrt{3} }{4}$ 

    $\max_K f = \frac{3\sqrt{3} }{4}\quad \min _K f = -\frac{3\sqrt{3} }{4}$
    \end{example}

    \begin{example}
        Найти $\sup_Ef\quad E = \left\{ (x,y,z)\in \R^3\quad x,y,z>0 \right\} $ 

        $f(x,y,z) = (x+2y+3z)e^{-(x+y+z)}$

        $g(t) = te^{-t}$

        $f(x,y,z)\leqslant (3x+3y+3z)e^{(x+y+z)} = 3g(t)\leqslant 3\cdot \sup_{t\geqslant 0} g(t) = \frac{3}{e}$ 

        $f(0,0,1) = 3 \cdot  e^{-1} = \frac{3}{e}\leqslant \sup_{\Cl E}f \implies \sup_{\Cl E}f = \sup_Ef = \frac{3}{e}$
    \end{example}

    \section{Функциональная последовательности и ряды}

    $\exists \subseteq \R^n\quad f: E\to \R$ (или $f:E\to \C$)

    $\left\{ f_k(x) \right\} _{k=1}^{\infty };\quad f_k: E\to \C$

    \begin{definition}
        Говорят, что функциональная последовательность $\left\{ f_k(x) \right\} $ сходиться к $f(x)$ поточечно на множестве  $E$, если  $\forall x_0\in E\quad$ числовая последовательность $\left\{ f_k(x_0) \right\} $ сходиться к $f(x_0)$ 
    \end{definition}

    Примеры:
    \begin{enumerate}
        \item $f_k(x) = x^k\quad E = [0,1]\quad f(x) = \begin{cases}
                0, x \in [0,1)\\
                1, x = 1\\
        \end{cases}$
    \end{enumerate}

    \begin{definition}
        [Поточечная сходимость] $d_k(x) \to  f\left(x \right) $ на $E \iff \forall x\in E\forall \varepsilon>0 \exists N = N(\varepsilon, x)>0:\quad \forall k\geqslant N\quad f_k(x) - f(x)<\varepsilon $
    \end{definition}

    \begin{definition}
        [Равномерная сходимость]

        $\sqsupset f, f_k: E\to \C$. Говорят, что $\left\{ f_k \right\} $ сходится к $f$ \underline{равномерно}, если \[\forall \varepsilon>0 \exists N = N(\varepsilon): \forall k\geqslant N \forall x\in E\quad |f_k(x)-f(x)| <\varepsilon\]
    \end{definition}

    \begin{note}
        $\rho_k = \sup_E \left| f_k(x \right| $, то $d_k(x) \rightrightarrows f(x)$ на  $\iff \rho_k\to 0$ при $k\to \infty $
    \end{note}
    \begin{proof}
        \begin{itemize}
            \item []
            \item [$\implies $] $\forall k\geqslant n \implies \rho_k \leqslant \varepsilon \implies \rho_k\to 0$ при $k\to \infty $
            \item [$\impliedby $] $\rho_k\to 0 \implies \forall \varepsilon > 0\exists  N: \forall k\geqslant N\quad 0\leqslant \rho_k<\varepsilon \implies \forall x\in E\quad \left| f_k(x) - f(x) \right| \leqslant \rho_k<\varepsilon$
        \end{itemize}
    \end{proof}

    \begin{example}
        $f_k(x) = x^k\quad \rho_k = \sup_{x\in [0,1]} \left| f_k(x) - f(x)\right|= \max\left\{ \sup_{x\in [0,1)}|f_k - f|, |f_k(0) - f(0)| \right\} = \sup_{x\in [0,1)} \left| f_k(x) - f(x) \right|  = \sup_{[0,1)} x^k = \sup_{[0,1]}x^k = 1$

        $\rho_k \equiv 1 \not\to 0 \implies  f_k\not\rightrightarrows f(x)$ на $E$

        \begin{note}
            $f_k\not\rightrightarrows f$ и на  $[0,1)$

        \end{note}
            $E_{\delta} = [0,\delta], \delta < 1\quad \rho_k = \sup_{x\in [0,\delta]} \left| f_k(x) - f(x) \right| = \sup _{x\in [0,\delta]}x^k = \delta^k \implies  \rho_k\to 0, k\to \infty  \implies  f_k\rightrightarrows f$ на $E_{\delta}$
            \begin{note}
                Если $\forall j\in \N \quad f_k\rightrightarrows f$ на $E_j$, то из этого не слудет, что  $f_k\rightrightarrows f$ на  $\bigcup\limits_{j=1}^{\infty }E_j $, но это становится верным при конечном количестве $E$
            \end{note}
    \end{example}

 \end{document}

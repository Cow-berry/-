\documentclass{book}
%nerd stuff here
\pdfminorversion=7
\pdfsuppresswarningpagegroup=1
% Languages support
\usepackage[utf8]{inputenc}
\usepackage[T2A]{fontenc}
\usepackage[english,russian]{babel}
% Some fancy symbols
\usepackage{textcomp}
\usepackage{stmaryrd}
% Math packages
\usepackage{amsmath, amssymb, amsthm, amsfonts, mathrsfs, dsfont, mathtools}
\usepackage{cancel}
% Bold math
\usepackage{bm}
% Resizing
%\usepackage[left=2cm,right=2cm,top=2cm,bottom=2cm]{geometry}
% Optional font for not math-based subjects
%\usepackage{cmbright}

\author{Коченюк Анатолий}
\title{Алгоритмы и структуры данных}

\usepackage{url}
% Fancier tables and lists
\usepackage{booktabs}
\usepackage{enumitem}
% Don't indent paragraphs, leave some space between them
\usepackage{parskip}
% Hide page number when page is empty
\usepackage{emptypage}
\usepackage{subcaption}
\usepackage{multicol}
\usepackage{xcolor}
% Some shortcuts
\newcommand\N{\ensuremath{\mathbb{N}}}
\newcommand\R{\ensuremath{\mathbb{R}}}
\newcommand\Z{\ensuremath{\mathbb{Z}}}
\renewcommand\O{\ensuremath{\emptyset}}
\newcommand\Q{\ensuremath{\mathbb{Q}}}
\renewcommand\C{\ensuremath{\mathbb{C}}}
\newcommand{\p}[1]{#1^{\prime}}
\newcommand{\pp}[1]{#1^{\prime\prime}}
% Easily typeset systems of equations (French package) [like cases, but it aligns everything]
\usepackage{systeme}
\usepackage{lipsum}
% limits are put below (optional for int)
\let\svlim\lim\def\lim{\svlim\limits}
%\let\svlim\int\def\int{\svlim\limits}
% Command for short corrections
% Usage: 1+1=\correct{3}{2}
\definecolor{correct}{HTML}{009900}
\newcommand\correct[2]{\ensuremath{\:}{\color{red}{#1}}\ensuremath{\to }{\color{correct}{#2}}\ensuremath{\:}}
\newcommand\green[1]{{\color{correct}{#1}}}
% Hide parts
\newcommand\hide[1]{}
% si unitx
\usepackage{siunitx}
\sisetup{locale = FR}
% Environments
% For box around Definition, Theorem, \ldots
\usepackage{mdframed}
\mdfsetup{skipabove=1em,skipbelow=0em}
\theoremstyle{definition}
\newmdtheoremenv[nobreak=true]{definition}{Определение}
\newmdtheoremenv[nobreak=true]{theorem}{Теорема}
\newmdtheoremenv[nobreak=true]{lemma}{Лемма}
\newmdtheoremenv[nobreak=true]{problem}{Задача}
\newmdtheoremenv[nobreak=true]{property}{Свойство}
\newmdtheoremenv[nobreak=true]{statement}{Утверждение}
\newmdtheoremenv[nobreak=true]{corollary}{Следствие}
\newtheorem*{note}{Замечание}
\newtheorem*{example}{Пример}
\renewcommand\qedsymbol{$\blacksquare$}
% Fix some spacing
% http://tex.stackexchange.com/questions/22119/how-can-i-change-the-spacing-before-theorems-with-amsthm
\makeatletter
\def\thm@space@setup{%
  \thm@preskip=\parskip \thm@postskip=0pt
}
\usepackage{xifthen}
\def\testdateparts#1{\dateparts#1\relax}
\def\dateparts#1 #2 #3 #4 #5\relax{
    \marginpar{\small\textsf{\mbox{#1 #2 #3 #5}}}
}

\def\@lecture{}%
\newcommand{\lecture}[3]{
    \ifthenelse{\isempty{#3}}{%
        \def\@lecture{Lecture #1}%
    }{%
        \def\@lecture{Lecture #1: #3}%
    }%
    \subsection*{\@lecture}
    \marginpar{\small\textsf{\mbox{#2}}}
}
% Todonotes and inline notes in fancy boxes
\usepackage{todonotes}
\usepackage{tcolorbox}

% Make boxes breakable
\tcbuselibrary{breakable}
\newenvironment{correction}{\begin{tcolorbox}[
    arc=0mm,
    colback=white,
    colframe=green!60!black,
    title=Correction,
    fonttitle=\sffamily,
    breakable
]}{\end{tcolorbox}}
% These are the fancy headers
\usepackage{fancyhdr}
\pagestyle{fancy}

% LE: left even
% RO: right odd
% CE, CO: center even, center odd
% My name for when I print my lecture notes to use for an open book exam.
% \fancyhead[LE,RO]{Gilles Castel}

\fancyhead[RO,LE]{\@lecture} % Right odd,  Left even
\fancyhead[RE,LO]{}          % Right even, Left odd

\fancyfoot[RO,LE]{\thepage}  % Right odd,something additional 1  Left even
\fancyfoot[RE,LO]{}          % Right even, Left odd
\fancyfoot[C]{\leftmark}     % Center

\usepackage{import}
\usepackage{xifthen}
\usepackage{pdfpages}
\usepackage{transparent}
\newcommand{\incfig}[1]{%
    \def\svgwidth{\columnwidth}
    \import{./figures/}{#1.pdf_tex}
}
\usepackage{tikz}
\begin{document}
    \maketitle
    \section{Введение}
    курс будет идти 4 семестра. 

    1 лекций + 1 практика в неделю

    баллы: практика -- выходишь к доске и делаешь задание.

    практика -- до 30 баллов

    0-25 -- по 5 баллов
    25-40 -- по 3 балла
    40+ -- по 1 баллу

    лабораторные: 50 баллов

    экзамен: в каком-то виде будет. до 20 баллов.

    \chapter{I курс}
    \section{Алгоритмы}

    Алгоритм: входные данные $\to $ \tikz \draw (0,0) -- (0,1) -- (1,1) -- (1,0) -- cycle; $\to $ выходные данные

    входной массив $a[0\ldots n-1]$, выходная сумма $\sum a_i$

    \begin{verbatim}
        S = 0           \\ 1
        for i = 0 .. n-1\\ 1+2n
            S+=a[i]     \\ 3n

        print(S)        \\ 1

    \end{verbatim}

    Модель вычислений.

    RAM - модель. симулирует ПК. За единицу времени можно достать/положить в любое место памяти.

    Время работы (число операций)
    В примере выше  $T(n) = 3+5n$

    мотивация: 3 становится мальньким, а 5 -- не свойство алгоритма

    $T(n) = O(n)$

    $f(n) = O(g(n)) \iff \exists n_0, c\quad \forall n\geqslant n_0\quad f(n)\leqslant c\cdot g(n)$

    $n_0 = 4, c = 6\quad 3+5n\leqslant 6n, n\geqslant 4\quad 3\leqslant n$

    $f(n) = \Omega (g(n)) \iff --||-- f(n) \geqslant c g(n)$

    $3+5n = \Omega(n)\quad n_0 =1, c=1$

    $3+5n \geqslant n, n\geqslant 1$

    $T(n) = O(n), T(n) = \Omega(n) \iff T(n) = \Theta(n)$

    \begin{verbatim}
        for i = 0 .. n-1
            for j = 0 .. n-1
        \end{verbatim}
        -- $O(n^2)$

    \begin{verbatim}
    
        for i = 0 .. n-1
            for j = 0 .. i-1
    \end{verbatim}

    $\sum_{i=0}^{n-1} i =\sum  \frac{n\cdot (n-1)}{2} = \Theta(n^2)$

    \begin{verbatim}
        i=1
        while i\cdot i<n
            i++
        i=1
        while i < n
            i=i\cdot \cdot 2
    \end{verbatim}

    $O(\sqrt{n}), O(\ln n) $

    \begin{verbatim}
        f(n):
            if n=0
                ..
            else
                f(n-1)
    \end{verbatim}

    $n$ рекурсивных вызовов $O(n)$


    \begin{verbatim}
        f(n):
            if n=0
                ..
            else
                f(n/2)
                f(n/2)
    \end{verbatim}

    $2^{\ln  n} = n$

    если добавить третий вызов: $2^{\log_2 n} = n^{\log_2 3}$

    \section{Сортировки}
    \subsection{Сортировка вставками}

    Берём массив, идём слева направо: берём очередной элемент и двигаем в влево, пока он не упрётся

    \begin{verbatim}
        for i = 0 .. n-1
            j=i
            while j>0 and a[j]<a[j-1]
                swap(a[j-1], a[j])
                j--
    \end{verbatim}

    Докажем, что алгоритм работает. по индукции. Если часть отсортирована и мы рассматриваем новый элемент, то он будет двигаться, пока не вставиться на своё место и массив снова будет отсортированным.

    Если массив отсортирован $(1,2,\ldots, n)$ -- $O(n)$

    Если нет(n, n-1, \ldots, 1), то $O(n^2)$

    Рассматривать мы дальше будем худшие случаи.

    \subsection{Сортировка слияниями}

    Слияние: из двух отсортированных массивов делает один отсортированный.

   как найти первй элемент. Он наименьший, значит либо самый левые в массиве $a$, либо в массиве $b$. Мы забыли нужный первый элемент и свели к такой же задаче поменьше.

    \begin{verbatim}
        merge(a,b):
            n = a.size()
            m = b.size()
            i=0, j=0
            while i<n or j<m:
                if j==m or (i<n and a[i]<b[j]):
                    c[k++] = a[i++]
                else
                    c[k++] = b[j++]
            return c
    \end{verbatim}
    $O(n+m)$

    Сортировка: берём массив, делим его пополам, рекурсивно сортируем левую и правую часть, а потом сольём их в один отсортированный массив.

    \begin{verbatim}
        sort(a):
            n = a.size()
            if n<=1:
                return a
            al = [0, .. n/2-1]
            ar = [n/2 .. n-1]
            al = sort(al)
            ar = sort(ar)
            return merge(al, ar)
    \end{verbatim}

    порядка $n$ рекурсивных массивов.

    $T(n) = 2 \cdot T(\frac{n}{2}) + n$

    красиво и понятно: 

    математически и хардкорно: по индукции $T(n) \leqslant \ln  n$

    База: $n=1$ -- не взять из-за логарифма, но можоно на маленькие $n$ не обращать внимания

    Переход:  $$T(n) = 2T(\frac{n}{2}) + n \leqslant 2\cdot \frac{n}{2}\ln  \frac{n}{2} + n = n(\ln  n-1)+n = n \ln  n + n(1-1) \leqslant \ln  n$$

    \begin{theorem}
        [Мастер-теорема]

        $T(n) = aT\left( \frac{n}{b} \right) + f(n) $

        Если без $f(n) = O(n^{\log _ba-\varepsilon})$, то $a^{\log _b^n} = n^{\log _ba}$

        Если без $f(n) = O(n^{\log _ba+\varepsilon})$, тогда $T(n) = O(f(n))$

        Если без $f(n) = O(n^{\log _ba})$, то $T(n) = n^{\ln _ba}\cdot \ln  n$
    \end{theorem}
%
%    ---------------------------------
%
    \section{Структуры данных}

    Структура, которая хранит данные

    Операции: структура данных определляется операциями, которые она умеет исполнять

    Массив:
    \begin{itemize}
        \item get(i) (return a[i])
        \item put(i,v) (a[i] = v)
    \end{itemize}

    Время работы на каждую операцию

    \subsection{Двоичная куча}

    Куча:
    \begin{itemize}
        \item  храним множество ($x<y$)
        \item insert(x) $\qquad A = A\cup \{x\}$
        \item remove\_min() 
    \end{itemize}

    Варианты:
    \begin{enumerate}
        \item Массив
            \begin{itemize}
                \item  insert(x) $a[n++] = x$ (O(1))
                    
                \item remove\_min() (O(n))
                    \begin{verbatim}
                        j=0
                        for i=1 .. n-1
                            if a[i]<a[j]: j=i
                        swap(a[j], a[n-1])
                        return a[--n]
                    \end{verbatim}
            \end{itemize}
        \item Отсортированный массив (по убыванию)
            \begin{itemize}
                \item remove\_min()
                    \begin{verbatim}
                        return a[--n]
                    \end{verbatim}
                \item insert(x)
                    \begin{verbatim}
                        a[n++] = x
                        i=n-1
                        while i >0 and a[i-1]<a[i]
                            swap(a[i], a[i-1])
                            i--
                    \end{verbatim}
            \end{itemize}
            \item Куча. Двоичное дерево, каждого элемента -- 2 ребёнка. У каждого есть один родитель (кроме корня). В каждый узел положим по элементу. Заполняется по слоям. Правило: у дети больше родителя. Минимум в корне -- удобно находить.

                Занумеруем все элементы слева направо. Из узла i идёт путь в $2i+1$ и $2i+2$

                \begin{verbatim}
                    insert(x)
                        a[n++] = x
                        i=n-1
                        while i>0 and a[i]<a[(i-1)/2]
                            swap(a[i], a[(i-1)/2])
                            i = (i-1)/2     
                \end{verbatim}

                $O(\log n)$

                Идея убирания минимума: поставить вверх вместо минимума последний элемент и сделать просеивание вниз.
\newpage
                \begin{verbatim}
                    remove_min()
                        res = a[i]
                        a[0] = a[--n]
                        i=0
                        while True:
                            j=i
                            if 2i+1<n and a[2i+1]<a[j]:
                                j=2i+1
                            if 2i+2<n and a[2i+2]<a[j]:
                                k=2i+2
                            if j == i: break
                            swap(a[i], a[j])
                            i=j
                        return res
                \end{verbatim}
    \end{enumerate}

    \subsection{Сортировка Кучей (Heap Sort)}
    \begin{verbatim}
        sort(a):
            for i = 1 .. n-1: insert(a[i])
            for i = 1 .. n-1: remove_min()
                
        heap_sort(a)
            for i = 0 .. n-1
                sift_up(i)
            for i = n-1 .. 0
                swap(a[0], a[i])
                sift_down(0, i) // i -- размер кучи
    \end{verbatim}
    ---------------------

    \section{Быстрая сортировка}

    \subsection{Рандомизированные алгоритмы}

    Алгоритм: Пусть есть массив и все элементы различны. Давайте выберем случайный элемент
    Поделим массив на две части: $<x$ и $\geqslant x$ -- $O(n)$

    Рекурсивно запускаем от каждого куска.

    \begin{verbatim}
        a // глобальнный массив
        sort(l, r):
            x = a[random(l..r-1)]
            if r-l =1:
                return
            m=l
            for i = l .. r-1:
                if a[i]<x:
                    swap(a[i],a[m])
                    m++
            sort(l,m)
            sort(m,r)
    \end{verbatim}

    Вместо изучения худшего случая рандомизированного алгоритма мы изучаем мат ожидание.

    $E(T(n))$

    $E(x) = \sum t\cdot p(x=t) $ 

    Покажем, что мат ожидание времени работы нашего алгоритма $O(n\log n)$

    Подход №1: посмотрим. был массив, поделили на две части, от каждой части запустились. каждая часть примерно $\frac{n}{2}\qquad T(n) = n+2T(\frac{n}{2})\quad O(n\log n)$

    Скорее всего поделимся не ровно пополам. 

    Подход №2: поделим на 3 части. средняя -- хорошая часть, выбрав элемент в которой части получаются $\leqslant \frac{2}{3}n$. Каждый третий раз пилим пополам примерно. $E(T(n)) \leqslant 3\cdot \log _{\frac{3}{2}} n = O(\log n)$


    \begin{definition}
        К-я порядковая статистика: ровно $k$ элементов меньше выбранного.
    \end{definition}

    Сортировкой: $O(n\log n)$

    Можно быстрее: Аогоритм Хоара

    Возьмём массив $a$, выберем случайный элемент $x$. Распилим массив на 2 куска : $<x, \geqslant x$

    Если знаем $k$, которое ищем, то выбираем одну часть и смотрим там.

    \begin{verbatim}
        a // глобальнный массив
        find(l, r, k): // l<=k<r
            x = a[random(l..r-1)]
            if r-l = 1: // l=k, r = k+1
                return
            m=l
            for i = l .. r-1:
                if a[i]<x:
                    swap(a[i],a[m])
                    m++
            if k<m:
                find(l,m,k)
            else:
                find(m,r,k)
    \end{verbatim}

    \section{Алгоритм Блют-Флойд-Пратт-Ривест-Тарьян}

    Разобьём массив на блоки по 5 элементов. $\frac{n}{5}$ блоков. В каждом блоке выбираем медиану. Выбираем медиану среди всех медиан. Если брать медиану из медиан, то это будет неплохой средний элемент

    $T(n) = n + T(\frac{n}{5}) + T(\frac{7n}{10}) = O(n)$

    $T(n)\leqslant c\cdot n$

    $T(n)\leqslant n + c\cdot \frac{n}{5} + c\cdot \frac{7n}{10} = n\left( 1+\frac{9}{10}\cdot c \right) <c\cdot n\qquad 10\leqslant c$


    Recap: Изучили
    \begin{itemize}
        \item MergeSort
        \item HeapSort
        \item QuickSort
    \end{itemize}

    Все за $O(n\log n)$

    Что мы можем делать: сравнивать элементы и перекладывать их.

    Программа: запустилась, что-то делает, сравнилось: $a[i]<a[j]$. От неё две ветки на два случая. Потом появляются сравнения в подслучаях, дающие больше случаев.
\newpage
    Есть три элемента: $x, y, z$. Отсортируем их.

     $x<y$
      \begin{itemize}
          \item [<|] $x<z$
               \begin{itemize}
                   \item [<|] $x$ -- минимальный.

                       $y<z$
                        \begin{itemize}
                            \item [<|] $xyz$
                            \item [$\not <|$] $xzy$
                       \end{itemize}
                   \item [$\not <|$] $zxy$
              \end{itemize}
          \item [$\not <|$] $y<z$
              \begin{itemize}
                  \item [<|] $x<z$
                       \begin{itemize}
                           \item [<|] $yxz$
                           \item [$\not <|$]  $yzx$
                      \end{itemize}
                  \item [$\not <|$] $zyx$
              \end{itemize}

     \end{itemize}

     В листьях перестановки листьев. $n!$ листьев. Глубина хотя бы $\log n!$

     $T\geqslant \log n! = \sum_{i=1}^{n} \log i = \Omega(n\log n)$

     Можно ли сделать что-то быстрее не только сравнивая.

     Пусть есть массив $a[0\ldots n-1]$. Какие-то маленькие целые числа: $a[i]\in [0\ldots m-1]$, $m$ -- маленькое.

     \section{Сортировка подсчётом}

     $a = [2,0,2,1,1,1,0,2,1]$

      \begin{verbatim}
        cnt = [0,0,0] увеличиваем, проходя по массиву
     \end{verbatim}

     $\p a = [0,0,1,1,1,1,2,2,2]$

     $O(n+m)$

     Но часто сортируются не числа, а объекты, к которым приделаны числа.
      \begin{verbatim}
         class Item {
            int key; // in [0..m-1]
            Data data;
         }
     \end{verbatim} 
     Создадим ящики для объектов с ключами 0,1 и 2.

     В реальности лучше создавать один массив и просто раскладывать в блоки внутри этого массива. 

     А дальше заполняем эти ящики проходя по массиву. А потом соберём их в один большой массив.

     $x = [0\ldots m^2-1]\quad x = a\cdot m + b\quad a,b\in [0\ldots m-1]$ -- двухзначное число в $m$-ичное число

     $a[i] = b[i]\cdot m + c[i]$

     Если нужно остортировать $a[i]$, то это то же самое, что отсортировать пары $\left( b[i],c[i] \right) $ по первому элементу, а при равенстве по второму.

     Пусть есть числа: a = [02 21 01 11 21 20 02 00]


     Можно сортировать по первой цифре, а потом по второй. Но это работает довольно долго.

     А если сортировать слева направо, то лучше:

     \begin{verbatim}
         cnt = [2,4,2]
         a' = [20 00 | 21 01 11 21 | 02 02]
         cnt  = [4,1,3]
         a'' = [00 01 02 02  | 11 | 20 21 21]
     \end{verbatim}
     $O(n+m)$ 

     Дофиксим: если $a[i] \in [0 \ldots m^k-1]$

     $O(k\cdot (n+m))$

     \section{Сортирующие сети} 
     Идея: одна операция -- компаратор

     \begin{verbatim}
         cmp(i,j):
             if a[i] > a[j]
                 swap(a[i],a[j])
     \end{verbatim}

    $n=2\quad cmp(0,1)$

    $n=3\quad cmp(0,1)\quad cmp(0,2)\quad cmp(1,2)$

    сортировка выбором: на первую позицию ставится минимальный элемент, потом сортируются оставшиеся.

    Возьмём сортировку вставками

\begin{figure}[ht]
    \centering
    \incfig{net}
    \caption{net}
    \label{fig:net }
\end{figure}

Порядка $n^2$ компараторов нужно. Рзрешим делаться нескольким компараторам сразу: cmp(0,1), cmp(2,3) друг другу не мешают. Разрешим неконфликтующим компараторам скольким угодно выполняться одновременно.

\begin{figure}[ht]
    \centering
    \incfig{net2}
    \caption{net2}
    \label{fig:net2}
\end{figure}

Элементов уже порядка $n$

 \begin{statement}
    Если сеть сортирует любой массив из 0 и 1, то она сортирует любой массив
\end{statement}
\begin{proof}
    Пускай сеть сортируют любую последовательность 0 и 1. Дадим ей какой-то массив. Наименьший элемент отметим 0, он придёт наверх в сети. Возьмём следующий элемент, он вылезет следом за 1.
\end{proof}

\section{Bitonic sort}

Битонная последовательность -- сначала возрастает, потом убывает.

Рассмотрим только 0 и 1. 

a = [0 0 1 1 1 1 1 0 0 0]

\begin{figure}[ht]
    \centering
    \incfig{btonic}
    \caption{bitonic}
    \label{fig:btonic}
\end{figure}

Пилим последовательность пополам. Применяем сеть, где сравниваются соответствующие элементы. Обе части битонные, правая больше, чем левая.


В общем случае: сравниваем парами, каждый второй разворачиваем. Получаем битонные последовательности через 4. Применяем к ним Bitonic Sort.

\section{Двоичный поиск}

Есть массив $a$ отсортированный по возрастанию

Как писать \underline{НЕ} надо:
 \begin{verbatim}
    x = 8, i: a[i] = x
    a 2 5 8 13 21 27 35
      l              r
    x in a[l..r] -- инвариант. Будем сужать.
    m=floor((l+r)/2)    l<=m<=r
    a[m] >x => a[j] >x, j>m

    l=0, r=n-1
    while (r-l+1)>0:
        m=(l+r)/2
        if a[m]>x// a[m..r]>x
            r = m - 1
        else if a[m]<x // a[l..m]<x
            l = m + 1
        else
            return m
 \end{verbatim}

 $O(\log n)$

 Задача: найти минимальный $i:a[i]\geqslant x$
 Как писать надо:
 \begin{verbatim}
    l, r
    a[l]<x
    a[r]>=x
    l=-1, r=n
    while (l-r)>1:
        m = (l+r)/2
        if a[m]<x:
            l = m
        else
            r=m
     return r // if r=n: такого элемента вообще нет
 \end{verbatim}

 Задача: найти максимальный $i: a[i]\leqslant x$
 \begin{verbatim}
    l, r
    a[l]<=x
    a[r]>x
    l=-1, r=n
    while (l-r)>1:
        m = (l+r)/2
        if a[m]<=x:
            l = m
        else
            r=m
     return l // if l=-1: такого элемента вообще нет
 \end{verbatim}

$x\in \N  \begin{cases}
    \text{хорошие}\\
    \text{плохие}
\end{cases}$

$x$ -- хорошее $\implies x+1$ -- хорошее

\begin{problem}
    найти минимальное хорошее число.
\end{problem}

$n$ штук прямоугольничков  $w\cdot h$. Мы хотим запихать их в квадрат. Вопрос: какой наименьший квадрат подойдёт.

Если можно впихнуть в $x^2$, то и в $(x+1)^2$ тоже.

\begin{verbatim}
    l -- плохое
    r -- хорошее

    l = 0
    r = max(h,w)*n

    good(l) = 0
    good(r) = 1

    while (l-r)>1:
        m = (l+r)/2
        if good(m):
            r = m
        else
            l = m
    return r
    
    good(x):
      return (x/h)*(x/w)>=n
\end{verbatim}
$O(\log (l-r)) = O(\log \left( \max(h,w) \cdot  n \right) = O(\log (h+w+n))$

Нужно аккуратно брать правое число, чтобы не было переполнений.

$r = max(h,w)\cdot \left\lceil \sqrt{n}  \right\rceil $

$2^k$ -- плохое, $2^{k+1}$ -- хорошее

$\log (a+b) = O(\log  a + \log b)$

\begin{problem}
    Есть прямая 

    На неё в точках $x_i$ живут котики

    $v_i$ -- скорость

    Собраться в одной точке за  $min$ время

     $t$ -- хорошее, если за  $t$ можно собраться
\end{problem}

Как писать \underline{НЕ} надо:
\begin{verbatim}
    r = 0, l = 10^10, EPS = 10^-6
    while (r-l)>EPS:
        m = (l+r)/2
        //fix
        if m <= l || m >= r
            break
        //xif
        if good(m):
            r = m
        else:
            l = m
    good(x):
        x >= max(li)
        x <= max(ri)
        x -- существует, если max(li) <= min(ri)
\end{verbatim}

$O(\log \left( \frac{r-l}{EPS} \right) \cdot n$

так писать не надо, потому что не все числа с нужной точностью можно получить. Точности может не хватать и цикл станет вечным.

Как исправить: можно сделать for


\section{Троичный поиск}
\begin{figure}[ht]
    \centering
    \incfig{tri}
    \caption{tri}
    \label{fig:tri}
\end{figure}

$m_1 = \frac{2l+r}{3}\quad m_2 = \frac{l+2r}{3}$ 

\begin{verbatim}
    if f(m2)>f(m1):
        l = m1
    else:
        r = m2
\end{verbatim}

$O(\log _{\frac{3}{2}} \frac{r-l}{EPS} )$

\section{Стеки и очереди}
\begin{definition}
    Стек -- ``стаканчик''.

    \tikz [scale=0.3] \draw (0,5) -- (0,0) -- (2,0) -- (2,5); 

    .push(A) -- добавить сверху элемент А
    
    .push(B), .push(C)

    .pop() -> C -- вернуть самый верхний элемент

    LIFO -- last in first out

\end{definition}

\begin{verbatim}
    a = [A,B,C, \ldots] -- первые n элементов заполнены

    push(x):
        a[n++]=x
    pop(x):
        return a[--n]
\end{verbatim}

\begin{definition}
   Очередь -- ``очередь в магазин'' 

   голова очереди -- то место, которое обрабатывается

   хвост очереди -- то, куда добавляются новые элементы
\end{definition}

\begin{verbatim}
    a = [A,B,C,D,E,F, \ldots]
    head -- первый неудалённый элемент
    tail -- на первый свободный

    add(x):
        a[tail++] = x
    remove():
        return a[head++]
\end{verbatim}

\begin{definition}
    Дек -- можно класть и доставать с обоих концов. Когда точно не знаешь нужна тебе очередь или стек.
\end{definition}

В обоих реализациях мы считаем, что у нас бесконечно большой массив.


решение 1: Пусть мы знаем, что в очереди $n$ элементов.
зациклить массив в очереди, чтобы когда место справа закончится, добавлять в начало.

Но что если мы не знаем сколько элементов максимум.

Сделаем стек, не зная сколько в нём максимум элементов.

\begin{verbatim}
    a = [x, x, x]
    a'= [x, x, x, _, _, _]

    Если не влезает -- расширять в 2 раза.

    push(x):
        if n = a.size():
            a' = new int[2*n]
            copy(a[0..n-1] -> a'[0..n-1])
            a = a'
        a[n++] = x
\end{verbatim}

\section{Амортизационный анализ}
$o_1, o_2, \ldots, o_k$ -- операции, проделанные в таком порядке

$T(o_i)$ -- время работы операции

$\widetilde{T}\left(o_i \right) $ -- амортизированные время работы (выбирается нами)

Амортизированное время хорошее, если $\sum \widetilde{T}(o_i) \geqslant \sum T(0_i)$

Пусть мы хотим $\widetilde{T}(o_i)\leqslant c \implies \sum\limits_k T(o_i)\leqslant c\cdot k$

Пусть мы делаем $k$ пушей

$\sum T \leqslant  k + \underbrace{2^p}\limits_{\leqslant 2k} \leqslant 3k\quad \widetilde{T}(push) = 3$

\subsection{Метод потенциалов}:

$\Phi$ -- потенциал (выбираемый опять же нами)

$\widetilde{T} = T + \Delta \Phi$

 $\Phi_0 = 0, \quad \Phi\geqslant 0$

 $\sum \widetilde T = \sum T + \sum \underbrace{\Delta\Phi}\limits_{=\Phi_{\text{к}=\Phi_0}} \geqslant \sum T$

 Цель: $\Phi = ?\quad \widetilde T = O(1)$

 $\Phi=$ (число элементов правой половине массива)$\cdot 2$

 $\widetilde T = 1 +1 = O(1)$

 $\sum \widetilde T = n-n+1 = O(1)$

\subsection{Метод бухгалтерского учёта (метод с монетами)}
\begin{verbatim}
    put_coin(x) // ~T = x
    take_coin(x) // ~T = -x

    [2r, 2r, 2r, 2r] -> [2r, 2r, 2r, 2r, _, _, _, _]

    мы "тратим" 4 рубля, чтобы увеличить массив из 4 элементов в 2 раза. Для копирования: берём + тратим = n + (-n)  

\end{verbatim}

Теперь про $pop$

 \begin{verbatim}
    pop():
        return a[--n]
\end{verbatim}

Если сначала много запушить, а потом много запопить, то останется много свободного места, которое мы не используем

Если массив заполнен меньше, чем на половину, можно сужать его в два раза. Но тогда на границе половины, если чередовать, то он будет постоянно сужаться-расширяться.

\begin{verbatim}
    pop():
        if n <= a.size()/4: ..
\end{verbatim}

В правой половине массив лежат монетки по 2 рубля для расширения.

Когда делаем pop, то кладём по рублю и копим для сужения. 

\begin{verbatim}
    add(x):
        s2.push(x)
    remove(x):
        if s1.empty():
            while !s2.empty:
                s1.push(s2.pop())
        return s1.pop()
\end{verbatim}

\section{Фибоначчева куча}

\subsection{Биномиальная куча}

Куча: $add(x), remove\_min(x)$

Двоичная куча: обе операции за логарифм

Добавим третью операцию:  $merge(H_1, H_2)$ -- склеивает две кучи  $H_1$ и $H_2$  в одну

\subsubsection{Биномиальное дерево}

$B_0$ -- дерево из одной вершины

$B_1$ -- одна вершина, к которой подвешена друга

$B_2$ --
%
%\begin{figure}[ht]
%    \centering
%    \incfig{binom}
%    \caption{binom}
%    \label{fig:binom}
%\end{figure}

\section{Динамическое программирование}

$F_1 = , F_2=1\quad F_n = F_{n-1} + F_{n-2}$

$F_n = ?$

 \begin{verbatim}
    f(n):
        if n <= 2:
            return 1
        else:
            return f(n-1)+f(n-2)
\end{verbatim}

$T(f(n)) = ?$

$T(n) = 1+T(n-1)+T(n-2)$

$T(n)\approx F_n$

При вызове  $f(10)\quad f(8)$ считается 2 раза,  $f(7)$ -- три раза, и чем дальше, тем хуже

$res = [1..n]$

\begin{verbatim}
    f(n):
        if res[n] != null:
            return res[n]
        if n <= 2:
            res[n] = 1
        else:
            res[n] = f(n-1) + f(n-2)
\end{verbatim}

$T(n) = O(n)$

$res[1]=1\quad res[2]=1$

 \begin{verbatim}
    for i = 3 .. n:
        res[i] = res[i-1] + res[i-2]
    print(res[n])
\end{verbatim}

\begin{problem}
    Есть $n$ кочек и зайчик. 

    Может прыгнуть на след. кочку, может через одну. 

    Нужно посчитать число способов попасть на последнюю клетку
\end{problem}

Последняя клетка у всех путей -- 6. В неё можно было попасть из пятой или из четвёртой. 

$d = [1, 1, 2, 3, 5, 8]$

 \begin{verbatim}
    d[1] = d[2] = 1
    for i = 3 .. n:
        d[i] = d[i-1] + d[i-2]
\end{verbatim}

Хотим посчитать количеств комбинаторных объектов

Сколько разных векторов из 0 и 1 длины $n$, в которых нет двух 1 подряд.

Если в конце 0, то перед этим может быть что угодно, если 1, то перед ней стоит 0, а дальше снова что-угодно.

Если 0 в конце, то $d[n-1]$, а если 1, то  $d[n-2]$

Теперь, если кролик может прыгать ещё и на 3 кочки. Сколько тогда способов будет? Просто добавить её одно слагаемое $d[i-3]$

Но придётся добавить  $d[3]=3$

А что если, он может прыгать до  $k$ клеток вперёд?

 \begin{verbatim}
    d[1]=1
    for i = 2 .. n:
        for j = 1 .. k:
            if i-j >= 1:
                d[i] += d[i-j]
\end{verbatim}  

Пусть теперь за вставание на кочку нужно платить.

$\cos t = [0,1,3, 1, 2, 0]$

Все пути делятся на группы:

$\ldots \to 5\to 6$

$\ldots\to 4\to 6$

$d[i]$ -- минимальный штраф, чтобы дойти до $i$

$d = [0,1,3,2,4,2] $

\begin{verbatim}
    d[0]=1
    for i = 2 .. n:
        d[i] = min(d[i-1], d[i-2]) + cost[i]
\end{verbatim}

\begin{verbatim}
    d[1] = 0
    for i = 2 .. n:
        d[i] = +inf
        for j = 1 .. k:
            if i-j >= 1:
                d[i] = min(d[i], d[i-j] + cost[i])
\end{verbatim}

Восстановим путь: будем хранить ещё один массивчик с предыдущим числом для каждой кочки.

Другая задача: то же самое, только котик в табличке. вместо прямой, собирает баллы по алгосам. Нужно собрать как можно больше.

$d[i,j]$ -- максимум до клетки  $(i,j)$

В клетку  $(i,j)$ мы придём либо слева, либо сверху.

\begin{verbatim}
    for i = 1 .. n:
        for j = 1 .. n:
            if i = j = 1:
                d[i,j] = 0
            else:
                d[i,j] = -inf
                if j > 1:
                    d[i,j] = max(d[i,j], d[i,j-1] + cost[i,j])
                if i > 1:
                    d[i,j] = max(d[i,j], d[i-1,j] + cpst[i,j])
\end{verbatim}

Вернёмся к самой первой задаче про кролика. Но, он не может тормозить и следующий прыжок должен быть хотя бы на столько же.

$d[i,j]$ -- число способов попасть в  $i$ с последним прыжком  $\leqslant j$

Так никто не пишет:

\begin{verbatim}
    for j = 1 .. n:
        d[1,j] = 1
    for i = 2 .. n:
        for j = 2 .. i:
            for k = 1 .. j:
                d[i,j] += d[i-k,k]
\end{verbatim}  

Так по-человечески:
\begin{verbatim}
    for j = 1 .. n:
        d[1,j] = 1
    for i = 2 .. n:
        for j = 1.. n:
            if j > 1:
                d[i,j] += d[i,j-1]
            if i-j >= 1:
                d[i,j] += d[i-j,j]
\end{verbatim}

\subsection{Реальное применение динамического программирования}

diff

\begin{verbatim}
1    res = 0
2    for (int i = 1; i<n; i++) {
3        if (a[i] < res) {
4            res = a[i]
5            break
6        }
7    }
\end{verbatim}

А потом кто-то исправляет

\begin{verbatim}
1    res = INT_MAX
2    for (int i = 0; i < n; i++) {
3        if (a[i] < res) {
4            res = a[i]
5        }
6    }
\end{verbatim}

diff:
\begin{verbatim}
    1: -res = 0
       +res = INT_MAX
    2: - ..
\end{verbatim}

Минимально количество изменений, чтобы из А получить В

\begin{verbatim}
    s = котик
    t = коржик

    котик
    ко(т->р)(+ж)ик
    коржик
\end{verbatim}

хочется получить вторую из первой:
\begin{enumerate}
    \item добавить символ
    \item удалить символ
    \item заменить символ
\end{enumerate}

Минимальное количество действий называется расстоянием Левенштейна.

\begin{verbatim}
    s = .. .. x
    t = .. .. x
    s->t -- s[:-1] -> t[:-1]
    Откидываем общую буквц в конце (D[i-1,j-1])

    s = .. .. x
    t = .. .. y
    1) change(x,y) и снова сводим к меньшему (1 + D[i-1,j-1])
    2) Если мы сделали remove(x), тогда из s[:-1] -> t (1 + D[i-1,j])
    3) add(y), тогда s -> t[:-1] (1 + D[i,j-1])
\end{verbatim}

$D[i,j]$ -- минимальное количество действий, чтобы из $s[0..i-1]$ получить  $t[0..j-1]$

\begin{example}
    \begin{verbatim}
        s = котик
        t = коржик
    \end{verbatim}
    $\begin{array}{c|c|c|c|c|c|c|}
        &0&1&2&3&4&5\\\hline
        0&0&1&2&3&4&5\\ \hline
        1&1&0&1&2&3&4\\\hline
        2&2&1&0&1&2&3\\\hline
        3&3&2&1&1&2&3\\\hline
        4&4&3&2&2&2&3\\\hline
        5&5&4&3&3&2&3\\\hline
        6&6&5&4&4&3&2\\\hline
    \end{array}$

    $D[n,m] = 2$
\end{example}

Код:
\begin{verbatim}
    for i = 0 .. |s|
        for j = 0 .. |t|
            if i = 0 or j = 0:
                D[i,j] = i+j
                continue
            if s[i-1] = t[j-1]:
                D[i,j] = D[i-1,j-1]
            else:
                D[i,j] = min(
                    D[i-1,j-1] + 1,
                    D[i-1, j]  + 1,
                    D[i, j-1]  + 1
                )
\end{verbatim}

Так мы считаем само расстояние. Давайте для каждого состояния динамики помечать предыдущие состояние динамики. Так мы можем восстановить путь

На самом деле всё делается немножко не так. Так -- медленно $O(nm)$. Мысль: если мало изменений, от они находятся близко. Тогда можно хранить значения около главной диагонали. Можно хранить блок строки вокруг оптимума (мин. значения). Блок ширины $X$ --  $O(nX)$

Перейлём к другой задаче

Пусть у нас есть текст со словами разной ширины, их надо красиво акууратно разместить на листчке (допустим, пишем свой текстовый редактор)
 \begin{enumerate}
    \item Пихаем слова, пока пихается. Спецэффекты: длинные слова, которые могут некрасиво переносится, образуя здоровые конские пробелы. Здоровый конский пробел -- это некрасиво. 
    \item Давайте введём метрику-штраф. Допустим штраф за пробел размера $x$ --  $x^2$
\end{enumerate}

Пусть ширина листочка $L$

$bad(l,r)$ -- штраф, если слова с  $l$ по  $r$ мы разместим в одной строчке

$bad(l,r) = \left( L - \sum_{i=l}^{r-1} w_i \right)^2 $ 

$bad(l,r) = \left( \frac{L-\sum w_i}{r-l-1} \right) \cdot (r-l-1) \#  if r-l-1>1$

$D[n]$ --  $\min \sum bad$ для слов  $0\ldots n-1$

\begin{verbatim}
    for n = 1 .. n:
        for j = 0 ... i-1:
            if sum(w(j..i-1)) > L:
                break
            cost = bad(k,i) + D[j]
            D[i] = min(D[i],cost)
\end{verbatim}

\begin{problem}
    Есть RLE кодирование

    2A -> AA

    $3(2AB) \to A A B A A B A A B$

    Нужно по строчке получить компактное представление её в RLE
\end{problem}

\begin{verbatim}
    s = AAABAABAABABB

    A2(2AB)A2B

    Если первая буква записана как буква s -> 1 + s[:-1] (1 + D[i+1,j])

    Если повторяющаяся строка длины x

    Тогда будет k(s[0..x-1]) s[kx..n-1] (3 + D[i,i+x] + D[i+kx,nj]
\end{verbatim}

Пусть $D[i,j]$ -- самое короткое представление строки  $s[i..j-1]$

Когда восстанавливаем состояние, то находим не один путь, а дерево.
\section{Алгосики}

\begin{problem}
    $n$ предметов, веса  $w_i$, стоимости  $c_i$, рюкзак  $S$

     $\sum w_i\leqslant S\quad \sum c_i \to \max$

     $NP$-полная

      $c_i=w_i\quad \sum w_i\leqslant S\quad \sum w_i\to \max$

      $\sum w_i =S$
\end{problem}
\begin{proof}
    \begin{enumerate}
        \item $w_i, S$ -- маленькие, целые ( $S \approx 10^6$)

            $d[i,j]$ -- можно ли набрать сумму  $j$ из предметов  $(0\ldots i-1)$

            \begin{enumerate}
                \item Не берём $i-1\quad d[i-1,j]$
                \item Берём  $i-1\quad d[i-1,j-w_{i-1}]$
            \end{enumerate}

            База:

            \begin{verbatim}
                $d[0,0] = True$
                for i = 1..n:
                    for j = 0..S:
                        d[i,j] = d[i-1,j]
                        if j-w_{i-1} >= 0:
                            d[i,j] |= d[i-1,j-w_{i-1}] # |= -- or-равно
            \end{verbatim}      
    \end{enumerate}

    Теперь сделаем $d[i,j]$ --  $\max\sum c$, которую можно набрать, если  $\sum w = j$ из предметов  $(0,i-1)$
            \begin{verbatim}
                d[0,0] = 0, d[o,j] = infty
                for i = 1..n:
                    for j = 0..S:
                        d[i,j] = d[i-1,j]
                        if j-w_{i-1} >= 0:
                            d[i,j] = max(d[i,j], d[i-1,j-w_{i-1}] + c_{i-1}) # |= -- or-равно
            \end{verbatim}      
        \item $n$ -- очень маленькое (можем перебрать все  $2^n$ -- небольшое)

            Переберём все  $X\subseteq \{0\ldots n-1\}$

            Подмножество $\leftrightarrow$  числа.

        $n=5\quad x = \{0,2,3\}\quad 01101 = 13$ 

        Подмножества $\{0\ldots n-1\} \leftarrow$ числа $0\ldots 2^n-1$

        $x\cup y = x|y\quad x\cap y = x \& y$

        $x \setminus y\quad x\&\left( ~y \right) $

        $\{i\}\quad 1<<i$

        $i\in X\quad (x\&(1<<i))>0$
        \begin{verbatim}
            for x = 0 .. 2^n-1:
                sw=0, sc=0
                for i = 0 .. n-1:
                    if x&(1<<i) > 0:
                        sw += w(i)
                        sc += c(i)
                if sw <= S:
                    res = max(res,sc)
            O(2^n * n)
        \end{verbatim}

    Оптимизации:
    \begin{itemize}
        \item Meet in the middle.

            Поделим предметы на левые и правые. Получим $2^{\frac{n}{2}}$ способов с каждой стороны

            $sw[x] + sw[y] \leqslant S\quad sc[x]+sx[y] \to \max$

            \begin{verbatim}
                for x = 0 .. 2^{n-1}-1
                    sq[y] <= S*sq[x]
                    sc[y] -> max (бин поиск)
            \end{verbatim}  
        \item Мультирюкзак. Положить все предметы в минимальное количество рюкзак. 

            $d[x]$ -- минимальное число рюкзаков, чтобы положить подмножество  $x$ 

            \begin{verbatim}
                for x = 0 .. 2^{n-1}:
                    for y \in x:
                        if sum w[y] <= S:
                            d[x] = min(dx, 1 + d[x-y])

                O(4^n)
            \end{verbatim}

            Каждый предмет:
            \begin{enumerate}
                \item В $X$ и в  $Y$
                \item В  $X$, но не в  $Y$
                \item Ни там, ни там
            \end{enumerate}

            Так получается $3^n$ вариантов.

            $d[x] = (A,B)\quad A$ -- число рюкзаков,  $B$ -- заполненность последнего рюкзака.

            $(A_1,B_1)<(A_2,B_2)$, если $A_1<A_2 $ or $A_1=A_2\& B_1<B_2$
    \end{itemize}
\end{proof}

\section{Динамика подмножеством и динамика профилем.}

\begin{problem}
    [Задача Комивояжёра]

    Есть граф. Мы в одной точке. Нужно проехать в другую. За проезд между вершинами есть стоимость, разная для разных рёбер. Нужно найти кратчайший пусть.

\end{problem}
\begin{proof}
    [Решение]

    $s$ -- стартовая вершина

    $a[i,j]$ -- кратчайший путь от  $i$ до  $j$

    Путь: $\{s\} \to \{s,v\} \to \ldots \to \{0\ldots n-1\}$

    $d[x,u]$ --  $\min\sum$ длина пути, который обходит множество  $X$ и заканчивает в точке  $u$ -- последней вершине нашего пути.

    $d[x,u] = \min\limits_{\substack{v\in X \\ v\neq n}}d[x\setminus \{u\},v] + a[v,u]$
\end{proof}

\begin{problem}
    Надо замостить прямоугольник маленьким прямоугольниками $1\times 2$. Нужно найти сколько способов это сделать. 
\end{problem}
\begin{proof}
    [Решение]

    Пусть мы замостили столбцы с $0$ по  $i-1$. Какие-то доминошки могли вылезти наружу (отсюда, примерно, пошёл термин профиль). Если его запомнить, это поможет нам в заполнении остального

    Можно запоминать как двоичное число (0 -- не торчит, 1 -- торчит) 

    $d[i,p]$ -- число способов заполнить первые  $i$ столбцов, чтобы торчал профиль  $p$

    Переход: вот мы знаем профиль $i$ -- $p$, как будет устроен переход на  $i+1$ -- $\p p$? 

    $(i,p)  \to  (i+1, \p p)$ 

    Если в $p$  $0$, то в  $\p p$  $1$

    Иначе может быть либо  $1$, либо  $0$, если вертикальная доминошка, но для неё нужны два нолика подряд.

    $d[0,0] = 1$

     \begin{verbatim}
        for i = 0 .. m-1:
            for p = 0 .. 2^n-1:
                for p' = 0 .. 2^n-1:
                    if comp(p,p'):
                        d[i+1,p'] += d[i,p]
    \end{verbatim}  
    $O(m2^n2^n) = O(m\cdot 4^n)$

    \begin{verbatim}
        comp(p,q):
            # надо проверить, что нет 11
            if p & q != 0:
                return false
            # блоки с горизонтальными палками должны быть чётной длины. Хочется 1 там, где 00 -- nor
            x = (1<<n)-1-(p|q)
            # в числе х группы должны стоять блоками чётной длины.
            y = x / 3
            # мы поделили на 11. Казалось бы проверили на соседние единички, но 9 = 1001 также делится на 3
            if y&(y<<1)!=0:
                return false
    \end{verbatim}

    План: сделать, чтобы переходов было мало. Можно сделать переход небольшой -- по одной доминошке.

    Столбцы с 0 по i-1 заполнены. В i-ом столбце заполнено с 0 по j-1. Как-то торчат доминошки, снова составим профиль.

    $d[i,j,p]$ -- заполнено i столбцов, j строчек в последнем столбце и торчит профиль  $p$. Давайте попробуем заполнить новую клетку в последнем столбце.

    $d[i,j,p] \to d[i,j+1,\p p]$ Если в новоё клетке торчит доминошка, то она уже заполнена и ничего трогать не надо.

    Если она была не заполнена. Тогда мы всегда можем поставить горизонтальную штуку и профиль станет $1$

    Третий случай, когда мы ставим вертикальную доминошку Тогда она профиль  $00$ превратит в  $01$

    $d[0,0,0] = 1$

     \begin{verbatim}
        for i = 0 .. m-1:
            for j = 0 .. n-1:
                for p = 0 .. 2^n-1:
                    if p & (1<<j):
                        p' = p-(1<<j)
                        d[i,j+1,p'] += d[i,j,p]
                    else:
                        p' = p+(1<<j)
                        d[i,j+1,p'] += d[i,j,p]
                        if j < n-1 and p & (1<<(j-1)) = 0:
                            p' = p + (1<<(j+1))
                            d[i,j+1,p'] += d[i,j,p]
            for p = 0 .. 2^n-1:
                d[i+1,0,p] = d[i,n,p]
    \end{verbatim}      
    $O(nm2^n)$

    Есть поле, нужно покрасить его в чёрный и белый цвет, чтобы не было квадрата $2\times 2$ одного цвета.

    Пусть покрасили $i-1$ столбцов. Нужно помнить только самый правый. По нему можно составить профиль (цвета 2, профиль двоичный). Можно составить функцию проверки совместимости профилей.

    Можно сделать так же изломанный профиль. При переходе нужно проверить только один образовавшийся квадрат.

\end{proof}

\section{Хэш-таблицы}

Структуры до этого:
\begin{enumerate}
    \item Set
        \begin{itemize}
            \item insert(x)
            \item remove(x)
            \item contains(x)
        \end{itemize}
    \item Map
        \begin{itemize}
            \item k $\to $ v
            \item put(k,v)
            \item get(k)
        \end{itemize}
\end{enumerate}

\begin{enumerate}
    \item map: $k\in [0\ldots m-1], m$ -- маленькое

        $a[0\ldots m-1]$ 

        \begin{verbatim}
            put(k,v):
                a[k] = v
            get(k):
                return a[k]
        \end{verbatim}
    \item $k\in [0\ldots u-1], u$ -- большое

        $h(k):[0\ldots i-1] \to [0 \ldots m-1]$

        $h(k) = k \% m$

         \begin{verbatim}
            put(k,v):
                a[h(k)] = v
            get(k):
                return a[h(k)]
        \end{verbatim}

        \begin{example}
            \begin{verbatim}
                put(17,5)
                get(17) -> 5
                get(32) -> 5 (обращаемся к a[2],  оно относится к 17)
            \end{verbatim}
        \end{example}

        Будем класть пару из ключа и значения в массив

        \begin{verbatim}
            put(k,v):
                a[h(k)] = (k,v)
            get(k):
                if a[h(k)].first == k:
                    return a[h(k)].second
                return null
        \end{verbatim}

        \begin{example}
            \begin{verbatim}
                put(17,5)
                put(32,5) (первая пара потрётся)
            \end{verbatim}
            Такое называется коллизия: два разных ключа, но одинаковое значение функции.

            $x\neq y,h(x) = h(y)$
        \end{example}

        $h(x)$ -- случайная функция из $m^u$ функций  $[0\ldots u-1] \to [0\ldots m-1]$

        $n$ элементов

         $m$ -- размер массива

         вероятность коллизий  $<\varepsilon\quad m\sim n^2$

         Если есть $n$ элементов, есть  $n^2$ пар $(x,y)$

         $p(h(x) = h(y)) = \frac{1}{m}\sim \frac{1}{n^2} \implies m\sim n^2$ 

         \begin{example}
             \begin{verbatim}
                 put(17,5) -> a[2] = (17,5)
                 put(32,7) -> a[2] = [(17,5), (32,7)]
                 get(17) -> a[2] -> (17,5) -> 5
             \end{verbatim}
         \end{example}

         \begin{verbatim}
             put(k,v): O(1)
                a[h(k)].add(k,v)
            get(k): O(?)
                for p in a[h(k)]:
                    if p.first = k:
                        return p.second
                return null
         \end{verbatim}

         $p\left( h(y) = h(x) \right)  = \frac{1}{m}$

         $E\left( T\left( get(x) \right)  \right)  = E\left( \text{числа } y: h(y) = h(x) \right) = \frac{n}{m} $

         $m=n \implies  E\left( T\left( get(x) \right)  \right)  = O(1)$

         $h_{AP}(k) = \left( \left( K\cdot A \right) \%p \right) \%m$ 

         $p$ -- случайное большое простое число,  $A$ -- случайное от 1 до  $p-1$

         $h(y) = h(x), \quad x\neq y$

         $\left( \left( xA \right) \%p \right) \%m = \left( \left( yA \right) \%p \right) \%m$

         $\left( \left( \left( x-y \right) A \right) \%p \right) \%m = 0$

         $\left( \left( x-y \right) A \right) \%p = t\cdot m\quad t = 0\ldots\left\lfloor \frac{p}{m} \right\rfloor$ 

         Левая штука от 0 до $p-1$. Хочется  $A = \frac{tm}{x-y}\%p$

         $\left\lfloor \frac{p}{m} \right\rfloor$ значений $A$ для которых будет выполнятся.  $p(A\text{ -- плохое}) = \dfrac{\frac{p}{m}}{p} = \frac{1}{m}$

         Свойства функций $h$

         Можно подумать, что достаточно  $p\left( h(x) = i \right)  \approx \frac{q}{m}$ 

         $h_A(k) = A\quad A = 0\ldots m-1$

         $h_0(x) = 0\quad h_1(x) = 1\quad \ldots \quad h_{n-1}(x) = n-1$

         Ели кто-то узнает 
\end{enumerate}

\subsection{Открытая адресация}

\begin{example}
    \begin{verbatim}
        put(17,5) -> a[2] = (17,5)
        put(32,5) -> a[2] занято, положимся рядом в a[3] = (32,5)
    \end{verbatim}

    Кот:
    \begin{verbatim}
        put(k,v):
            i = h(k)
            while a[i] != null:
                i = (i++)%m 
            a[i] = (k,v)
        get(k):
            i = h(k)
            while a[i] != null:
                if a[i].first = k:
                    return a[i].second
                i = (i++)%m
            return null
    \end{verbatim}

    $m = ?\quad E(T) = ?$

    $m=2n\quad E(T) = O(1)$
\end{example}

\subsection{Удаление}

Пусть надо удалить ячейку $c$. Удалим, но это могло вызвать проблемы, давайте их обнаружим. Будем идти вправо, пока не найдём  $x:h(x)$ находится левее  $x$. Поставим его на освободившуюся ячейку. Повторим операцию с новой.

Более универсальный способ: положить какой-то мусорный символ. чтобы о него не спотыкался while. Его может накопится много. И каждые $n$ операций можно чистить таблицу (перекладывать всё в новую таблицу)

\section{Идеальное хэширование}

$x \to h(x) \to a(h(x))$

$m=n$

$a(i)$ -- хэш-таблица с хэш-функцией  $g_i(x)$

$n_i$ -- число элементов  $h(x) = i$

Размер  $a(i) = n_i^2$ -- всё ещё квадрат, но здесь именно от  $n_i$, а оно наверное маленькое.

Суммарный размер  $\sum n_i^2$

$E\left( \sum n_i^2 \right)  = E\left( \sum\limits_i | (x,y): h(x) = i, h(y) = i \right)  = \left( \sum\limits_i n_i + | (x,y): x\neq y, h(x) = i, h(y)=i| \right) = n + \sum|(x,y):x\neq y, h(x) = i, h(y) = i| = n+n^2\cdot \frac{1}{n} = O(n) $ 

$\frac{1}{n}$ -- вероятность коллизии.

$E(x) = cn$

$p\left( x>2cn \right) \leqslant \frac{1}{2}$ 

Но на практике так не делают..

\section{Хэширование кукушки}

Возьмём две хэш-функции $h_1(x), h_2(x)$

Возьмём да массива $a_1, a_2$

Инвариант $x$ лежит в  $a_1[h(x_1)]$, либо в $a_2[h(x_2)]$

Инсерт: Если хотя бы одна ячейка свободная -- кладём и не паримся. Если обе заняты, то выпихиваем А и садимся вместо него. Теперь надо поставить А. Но у А было запасное второе место. Если там тоже занято, выпихиваем его, а затем ищем куда поставить его. Проблема: оно может замкнуться. Бывает, что элементы точно нельзя разместить. Задедектить это несложно (можно поставить ограничение на количество перекладываний). Если такая ситуация, берём новые функции и строим хэш-таблицу заново. Есть хорошая  вероятность, что всё сойдётся.

\begin{definition}
    [$k$-универсальные]

    $p\left( \left( h(x_1), h(x_2), \ldots, g(x_{n} ) \right) =(y_1, y_2, \ldots, y_{n}) \right) \leqslant \frac{c}{m^k}$
\end{definition}


\section{Фильтр Блума}

\begin{itemize}
    \item insert
    \item contains
\end{itemize}

Хотим по минимуму памяти.

Для начала хотим функцию  equals(y) //$x == y$

$h(y) = h(x)$ -- они наверное равны.

$p$(ошибки) $=\frac{1}{m}$, $m = 2^k$

 $k$ бит, ошибка с вероятностью  $\frac{1}{2^k}$ 

 \begin{verbatim}
     insert(x):
        for i = 1 .. k:
            a[h_i(x)] = 1

    contains(x):
        for i = 1 .. k:
            if a[h_i(x)]=0:
                return false
        return true.
 \end{verbatim}

 $k = \log \frac{1}{\varepsilon}\quad p$(ошибки$\leqslant \varepsilon$

$m \approx 2kn $

$n=10000\quad \varepsilon=\frac{1}{2^{20}}$ 

$k = 20\quad m \approx 400000$ бит $50000$ байт

\section{Фингерпринт = Кукушка + Блум}

fingerprint $fp(x):[0\ldots 2^k-1]$

$h_1(x) = hash(x)$

$h_2(x) = h_1(x)\oplus hash(fp(x))$ 

По элементу и его фингерпринту можно понять следующую позицию для него.

\end{document}

\documentclass{book}
%nerd stuff here
\pdfminorversion=7
\pdfsuppresswarningpagegroup=1
% Languages support
\usepackage[utf8]{inputenc}
\usepackage[T2A]{fontenc}
\usepackage[english,russian]{babel}
% Some fancy symbols
\usepackage{textcomp}
\usepackage{stmaryrd}
% Math packages
\usepackage{amsmath, amssymb, amsthm, amsfonts, mathrsfs, dsfont, mathtools}
\usepackage{cancel}
% Bold math
\usepackage{bm}
% Resizing
%\usepackage[left=2cm,right=2cm,top=2cm,bottom=2cm]{geometry}
% Optional font for not math-based subjects
%\usepackage{cmbright}

\author{Коченюк Анатолий}
\title{Линейный Анализ}

\usepackage{url}
% Fancier tables and lists
\usepackage{booktabs}
\usepackage{enumitem}
% Don't indent paragraphs, leave some space between them
\usepackage{parskip}
% Hide page number when page is empty
\usepackage{emptypage}
\usepackage{subcaption}
\usepackage{multicol}
\usepackage{xcolor}
% Some shortcuts
\newcommand\N{\ensuremath{\mathbb{N}}}
\newcommand\R{\ensuremath{\mathbb{R}}}
\newcommand\Z{\ensuremath{\mathbb{Z}}}
\renewcommand\O{\ensuremath{\emptyset}}
\newcommand\Q{\ensuremath{\mathbb{Q}}}
\renewcommand\C{\ensuremath{\mathbb{C}}}
\newcommand{\p}[1]{#1^{\prime}}
\newcommand{\pp}[1]{#1^{\prime\prime}}
% Easily typeset systems of equations (French package) [like cases, but it aligns everything]
\usepackage{systeme}
\usepackage{lipsum}
% limits are put below (optional for int)
\let\svlim\lim\def\lim{\svlim\limits}
%\let\svlim\int\def\int{\svlim\limits}
% Command for short corrections
% Usage: 1+1=\correct{3}{2}
\definecolor{correct}{HTML}{009900}
\newcommand\correct[2]{\ensuremath{\:}{\color{red}{#1}}\ensuremath{\to }{\color{correct}{#2}}\ensuremath{\:}}
\newcommand\green[1]{{\color{correct}{#1}}}
% Hide parts
\newcommand\hide[1]{}
% si unitx
\usepackage{siunitx}
\sisetup{locale = FR}
% Environments
% For box around Definition, Theorem, \ldots
\usepackage{mdframed}
\mdfsetup{skipabove=1em,skipbelow=0em}
\theoremstyle{definition}
\newmdtheoremenv[nobreak=true]{definition}{Определение}
\newmdtheoremenv[nobreak=true]{theorem}{Теорема}
\newmdtheoremenv[nobreak=true]{lemma}{Лемма}
\newmdtheoremenv[nobreak=true]{problem}{Задача}
\newmdtheoremenv[nobreak=true]{property}{Свойство}
\newmdtheoremenv[nobreak=true]{statement}{Утверждение}
\newmdtheoremenv[nobreak=true]{corollary}{Следствие}
\newtheorem*{note}{Замечание}
\newtheorem*{example}{Пример}
\renewcommand\qedsymbol{$\blacksquare$}
\newcommand\vect[1]{\overset{\longrightarrow}{#1}}
% Fix some spacing
% http://tex.stackexchange.com/questions/22119/how-can-i-change-the-spacing-before-theorems-with-amsthm
\makeatletter
\def\thm@space@setup{%
  \thm@preskip=\parskip \thm@postskip=0pt
}
\usepackage{xifthen}
\def\testdateparts#1{\dateparts#1\relax}
\def\dateparts#1 #2 #3 #4 #5\relax{
    \marginpar{\small\textsf{\mbox{#1 #2 #3 #5}}}
}

\def\@lecture{}%
\newcommand{\lecture}[3]{
    \ifthenelse{\isempty{#3}}{%
        \def\@lecture{Lecture #1}%
    }{%
        \def\@lecture{Lecture #1: #3}%
    }%
    \subsection*{\@lecture}
    \marginpar{\small\textsf{\mbox{#2}}}
}
% Todonotes and inline notes in fancy boxes
\usepackage{todonotes}
\usepackage{tcolorbox}

% Make boxes breakable
\tcbuselibrary{breakable}
\newenvironment{correction}{\begin{tcolorbox}[
    arc=0mm,
    colback=white,
    colframe=green!60!black,
    title=Correction,
    fonttitle=\sffamily,
    breakable
]}{\end{tcolorbox}}
% These are the fancy headers
\usepackage{fancyhdr}
\pagestyle{fancy}

% LE: left even
% RO: right odd
% CE, CO: center even, center odd
% My name for when I print my lecture notes to use for an open book exam.
% \fancyhead[LE,RO]{Gilles Castel}

\fancyhead[RO,LE]{\@lecture} % Right odd,  Left even
\fancyhead[RE,LO]{}          % Right even, Left odd

\fancyfoot[RO,LE]{\thepage}  % Right odd,something additional 1  Left even
\fancyfoot[RE,LO]{}          % Right even, Left odd
\fancyfoot[C]{\leftmark}     % Center

\usepackage{import}
\usepackage{xifthen}
\usepackage{pdfpages}
\usepackage{transparent}
\newcommand{\incfig}[1]{%
    \def\svgwidth{\columnwidth}
    \import{./figures/}{#1.pdf_tex}
}
\usepackage{tikz}
\makeatletter
\renewcommand*\env@matrix[1][*\c@MaxMatrixCols c]{%
  \hskip -\arraycolsep
  \let\@ifnextchar\new@ifnextchar
  \array{#1}}
\makeatother
\begin{document}
    \maketitle
    \section{Введение}
    Трифанов Александр Игоревич

    Два модуля: аналитическая геометрия, линейная алгебра

    Отчётность: дз, кр, лабы, рубежное тестирование, экзамен

    дз (16 штук(8 в модуль) по 2 баллв)
    
    кр (4 (2 в модуль) 5 баллов)

    лаба (1-2 по 5 баллов)

    рубежный тест (1)



    \chapter{I курс}
    \section{Матрицы и операции над ними}

    \begin{definition}
        Матрица -- прямоугольная таблицв чисел
    \end{definition}

    $\begin{bmatrix} 
        a_{11}&a_{12}&\ldots & a_{1n}\\
        a_{21}&a_{22}&\ldots &a_{2n}\\
        \vdots & \vdots & \ldots, \vdots\\
        a_{m 1}& a_{m 2 } & \ldots & a_{m n}
     \end{bmatrix} $

     $a_{i,j}\in \R$ -- элементы матрицы

     $a_{11}, a_{12}, \ldots, a_{1n}$ -- строка 1
     
     $a_{12}, a_{22}, a_{32}, \ldots, a_{m 2}$ -- столбец 2

     $a_{ij}$ -- элемент на пересечении $i$-той строки и $j$-того столбца

     В матрице выше $m$ строк и $n$ столюцов. $A_{m\times n}$ -- обозначение

     \begin{note}
         $n=m \implies A_{n\times n}$ -- квадратная матрица 

         $\{a_{ii}\}_{i=1}^n$ -- диагональ матрицы $A_{n\times n}$
     \end{note}

     \begin{note}
         $A = \| a_{ij} \|$ $B = \| b_{ij} \|$
     \end{note}
    
     \begin{note}
         $A=B \iff $
         \begin{itemize}
             \item одинаковые размеры
             \item $\forall i, j a_{ij} = b_{ij}$
         \end{itemize}
     \end{note}

     Операции с матрицами:
     \begin{enumerate}
         \item Умножение на число:

             $B = \alpha \cdot A \iff  \forall i, j\quad b_{ij} = \alpha \cdot a_{ij}$
         \item Сложение:

             Пусть $A,B$ -- одинакового размера

             $A+B = C :\quad c_{ij} = a_{ij} + b_{ij} \forall i,j$

             \begin{note}
                 $\sphericalangle \mathbb{O}: \mathbb{O} + A = A+ \mathbb{O} = A$

                 $\mathbb{O}$ -- полностью состоит из нулей
             \end{note}
     \end{enumerate}

     Свойства:
     \begin{itemize}
         \item коммутативность сложения (следует из коммутативности сложения чисел) $A+B = B+A$
         \item ассоциативность (--||--) $(A+B)+C = A+(B+C)$
         \item дистрибутивность $\alpha(A+B) = \alpha A + \alpha B$
         \item $\forall A\quad -A = -1\cdot A:\quad A+(-A)= \mathbb{O}$ противоположный элемент по сложению
     \end{itemize}

     \begin{enumerate}
         \item [3] Умножение матриц
             
             Пусть $A_{m\times l}, B_{l\times n}$

             $C = A\cdot B\quad c_{ij} = \sum_{k=1}^{l} a_{ik}\cdot b_{kj}= C_{m\times n}$
     \end{enumerate}
\begin{note}
    $A\cdot B \neq B\cdot A$

    Для квадратных матриц вводится такое понятие, как коммутатор 

    $[AB] = A\cdot B - B\cdot A$
\end{note}

\begin{example}
    $A\cdot D = \begin{bmatrix} 1&2&0\\1&-1&1 \end{bmatrix} \cdot \begin{bmatrix} 1&2&3\\4&5&6\\7&8&9 \end{bmatrix}  = \begin{bmatrix} 9&12&15\\4&5&6 \end{bmatrix} $


\end{example}

    \begin{note}
        $\sphericalangle I: A\cdot I = I\cdot A = A$

        $I = \begin{bmatrix} 1&0&\ldots&0\\
        0&1&\ldots&0\\ 
        \vdots&\vdots&\vdots&\vdots\\
        0&0&\ldots&1 \end{bmatrix} $
    \end{note}

    Свойства:
    \begin{itemize}
        \item некоммутативность $A\cdot B \neq B\cdot A$
        \item ассоциативность $(A\cdot B)\cdot C = A\cdot (B\cdot C) $
        \item дистрибутивность1  $A(B+C) = AB + AC$
        \item дистрибутивноcть2 $\alpha (AB) = (\alpha A)B = A(\alpha B)$
    \end{itemize}
    \begin{definition}
        
    $\sphericalangle N\neq \mathbb{O}:\quad N^k = N\cdot N\cdot \ldots\cdot N = \mathbb{O}$

    N -- нильпотентная матрица, $k$ -- её порядок нильпотентности
    \end{definition}
    \begin{example}
        $\begin{bmatrix} 0& 1\\0&0\end{bmatrix} $
    \end{example}

    \begin{definition}
        Идемпотентной матрица называется, если $N^k = I$

        $k$ -- порядок идемпотентности

        
    \end{definition}

    \begin{example}
        $\begin{bmatrix} 1&0\\0&-1 \end{bmatrix} $
    \end{example}

    \begin{enumerate}
        \item [4] Транспонирование

            $A_{m\times n} = \| a_{ij} \|   $ Пусть $B = A^T = \| b_{ij} \| \implies  b_{ij} = a_{ji}$
    \end{enumerate}

    Свойства:
    \begin{itemize}
        \item $\left( \alpha \cdot A \right)^T = \alpha\cdot A^T $
        \item $(A+B)^T = A^T + B^T$
        \item $(AB)^T = B^T\cdot A^T$ -- проверить для себя
    \end{itemize}

    \begin{note}
        $A:\quad A=A^T$ -- симметричная/симметрическая матрица.Любая квадратная матрица с симметричными относительно диагонали элементами.

        $A:\quad A = -A^T$ -- антисимметричная матрица. На главной диагонали стоят нули

        $\begin{bmatrix} a_1&a_2&a_3\\ 0&a_4&a_5\\0&0&a_6 \end{bmatrix} $ -- верхняя треугольная. Транспонированная -- нижняя треугольная матрица
    \end{note}
   

    \section{Определитель}

    $\sqsupset A = \begin{bmatrix} a_{11}&a_{12}&\ldots & a_{1n}\\ a_{21} & a_{22} & \ldots & a_{2n}\\ \ldots &\ldots &\ldots & \ldots\\ a_{n_1} & a_{n_2} & .. & a_{nn} \end{bmatrix}  $
    \begin{definition}
        Определитель -- это число

        $\sqsupset A_{1x 1} = (a_{11})\qquad \det A \equiv \left| A \right| = a_{11}$

        $\sqsupset A_{2x 2} = \begin{bmatrix} a_{11} & a_{12}\\ a_{21} & a_{22} \end{bmatrix} \det A = a_{11}\cdot a_{22}-a_{12}\cdot a_{21} $

        $\sqsupset A_{3x 3} = \begin{bmatrix} a_{11} & a_{12} & a_{13} \\ a_{21} & a_{22} & a_{23}\\ a_{31} & a_{32} & a_{33} \end{bmatrix}  \quad \\ \det A = a_{11}\cdot a_{22}\cdot a_{33}+a_{12}\cdot a_{23}\cdot a_{31} - a_{13}\cdot a_{22}\cdot a_{31}-a_{11}\cdot a_{23}\cdot a_{32} - a_{33}\cdot a_{21}\cdot a_{12}$

        Мнемоническое правило для случая с тремя: 

        берём диагонали в одну сторону с + в другую с -.
    \end{definition}
    \begin{example}
        $\sqsupset A = \begin{bmatrix} 1 & -3 & -1\\ -2 & 7 & 2 \\ 3 & 2 & -4 \end{bmatrix} $ 

        $\det A = -28-18+4 - \left( -21 +4 -24 \right) = -1 $
    \end{example}

    $\sphericalangle \begin{bmatrix} a_{11} & a_{12} & \ldots & a_{1k} & \ldots & a_{1n}\\ a_{21} & a_{22} & \ldots & a_{2k} & \ldots & a_{2n}\\ \ldots & \ldots & \ldots & \ldots & \ldots & \ldots \\ a_{m_1} & a_{m_2} & \ldots & a_{mk} & \ldots & a_{m k} \end{bmatrix} $ 

    \begin{definition}
        Дополнительный минор элемента $a_{ij}$ -- определитель матрицы, полученной из исходно1 вычёркиванием $i$-ой строки и $j$-го столбца. 

        Обозначение: $M_{ij}$
    \end{definition}

    \begin{statement}
        [Рекуррентная формула вычисления определителя]

        $\det A = \sum\limits_{i=1}^{n} (-1)^{i+j}a_{i+j} \cdot M_{ij}$, где $j$ -- номер любого столбца. Эта формула называется разложением определителя по $j$-ому столбцу.

        $\det A = \sum_{i=1}^{n} (-1)^{i+j}a_{ij}M_{ij}$ -- разложение по $i$-ой строке
    \end{statement}

    $\sphericalangle (-1)^{i+j}M_{ij} = \mathcal{A}_{ij}$ -- алгебраическое дополнение элемента $a_{ij}$

    \begin{example}
        $\begin{bmatrix} 1 & -3 & -1\\ -2 & 7 & 2 \\ 3 & 2 & -4 \end{bmatrix} = A\quad \\ \det A = (-1)^{1+3}\cdot (-1)\cdot 
        \begin{vmatrix}
            -2 & 7\\ 3 & 2
        \end{vmatrix} + (-1)^{2+3}\cdot 2\cdot 
        \begin{vmatrix}
            1 & -3 \\ 3 & 2\\
        \end{vmatrix} + (-1)^{3+3}\cdot (-4)\cdot 
        \begin{vmatrix}
            1 & -3\\ -2 & 7\\
        \end{vmatrix} =\\ = 25 - 22  - 4 = -1$
    \end{example}

    \begin{example}
        $ \\ 
        \begin{vmatrix}
            1 & -1 & 3 & 4\\ -1 & 4 & 0 & -1\\ 3 & 0 & 0 & -3\\ 4 & -1 & -3 & 1\\ 
        \end{vmatrix} = (-1)^{3+1} \cdot  3 \cdot  
        \begin{vmatrix}
            -1 & 3 & 4\\ 4 & 0 & -1\\ -1 & -3 & 1
        \end{vmatrix} + (-1)^{3+4} \cdot  (-3) \cdot  
        \begin{vmatrix}
            1 & -1 & 3\\
            -1 & 4 & 0\\
            4 & -1 & -3\\
        \end{vmatrix} = \ldots$
    \end{example}

    \begin{example}
        $\begin{vmatrix} 1 & 2 & 3 & 4\\ 0 & 2 & 5 & 6 \\ 0 & 0 & 3 & 7\\ 0 & 0 & 0 & 2\\ \end{vmatrix}  = 1\cdot 2\cdot 3\cdot 2 = 12$

        В нижних (верхних) треугольных матрицах определитель -- просто произведение элементов
    \end{example}

    Свойства определителя:
    \begin{enumerate}
        \item $\begin{vmatrix} 0& 0 & \ldots & 0 \\ \ldots&\ldots&\ldots&\ldots\\\ldots&\ldots&\ldots&\ldots&\\\ldots&\ldots&\ldots&\ldots& \end{vmatrix} = 0$ 
        \item $\begin{vmatrix} a_1 & a_2 & \ldots & a_n\\ b_1 & b_2 & \ldots & b_n\\ \ldots&\ldots&\ldots&\ldots\\ \end{vmatrix}  =  - \begin{vmatrix} b_1 & b_2 & \ldots & b_n\\ a_1 & a_2 & \ldots & a_n\\ \ldots&\ldots&\ldots&\ldots\\ \end{vmatrix}$ 
        \item $\begin{vmatrix} 
                    \ldots&\ldots&\ldots&\ldots\\
                    a_1 & a_2 & \ldots & a_n\\
                    b_1 & b_2 & \ldots & b_n\\ 
                    \ldots & \ldots & \ldots & \ldots\\  
                \end{vmatrix}  = 
                \begin{vmatrix} 
                    \ldots&\ldots&\ldots&\ldots\\
                    a_1 & a_2 & \ldots & a_n\\
                    \alpha_1\cdot a_1 + b_1 & \alpha_2\cdot a_2 + b_2 & \ldots & \alpha_n \cdot  a_n + b_n\\ 
                \ldots & \ldots & \ldots & \ldots\\  \end{vmatrix} $
        \item $\det(A^T) = \det(A)$ (всё, что работает со строчками, работает и со столбцами)
        \item $
            \begin{vmatrix}
                a_{11} & a_{12} & \ldots & a_{1i}+c_1 & \ldots\\
                a_{21} & a_{22} & \ldots & a_{2i}+c_2 & \ldots\\
                \ldots& \ldots &\ldots&\ldots&\ldots\\
            \end{vmatrix} = 
            \begin{vmatrix}
                a_{11} & a_{12} & \ldots & a_{1i} & \ldots\\
                a_{21} & a_{22} & \ldots & a_{2i} & \ldots\\
                \ldots& \ldots &\ldots&\ldots&\ldots\\
            \end{vmatrix} + 
            \begin{vmatrix}
                a_{11} & a_{12} & \ldots & c_{1} & \ldots\\
                a_{21} & a_{22} & \ldots & c_{2} & \ldots\\
                \ldots& \ldots &\ldots&\ldots&\ldots\\
            \end{vmatrix}$ 

            $\det (A+B) \neq \det (A) + \det(B)$
        \item $
            \begin{vmatrix}
                \ldots&\ldots&\ldots&\ldots\\
                \alpha a_1 & \alpha a_2 & \ldots & \alpha a_n\\
                \ldots&\ldots&\ldots&\ldots\\
            \end{vmatrix} = \alpha 
            \begin{vmatrix}
                \ldots&\ldots&\ldots&\ldots\\
                a_1 & a_2 & \ldots & a_n\\
                \ldots&\ldots&\ldots&\ldots\\
            \end{vmatrix}$
    \end{enumerate}

    \begin{example}
         $\left. 
             \begin{matrix}
                 1280\\ 2848\\ 1184\\ 3072\\
             \end{matrix}\right| \vdots~32$ 


    \end{example}
TODO
    -----

    \section{Лекция 1 - Метод аналитической геометрии}

      метод -- метод координат.

      \begin{tikzpicture}
          \draw (0,0) -- (2.5,1) node [above] {$\ell$ };
          \fill (1.75,0.7) circle [radius=.03cm] node [below right] {\tiny $\mathcal X \leftrightarrow x\in \R$};
          \fill (.5, .2) circle [radius=.03cm] node [below right] {$0$};
          \fill (1, .4) circle [radius=.03] node [above] {$E$};
      \end{tikzpicture} -- координатная линия

      \begin{note}
          Система координат на плоскости -- две координатные линии

          \begin{tikzpicture}[scale = 0.8]
              \draw (-1,0) -- (4, 0) node [above right] {$x$};
              \draw (-0.6,-1) -- (2.4,4)  node [right] {$y$};
              \fill (0,0) circle [radius = .03] node [below right] {$0$};
          \end{tikzpicture}
          \begin{tikzpicture}[scale=3]
              \draw[->] (0,0,0) -- (0,0,1) node [below] {$x$};
              \draw[->] (0,0,0) -- (0,1,0) node [right] {$z$};
              \draw[->] (0,0,0) -- (1,0,0) node [below] {$y$};
          \end{tikzpicture}
      \end{note}

      \begin{definition}
          Система координат называется декартовой, если
          \begin{enumerate}
              \item Углы между координатными линиями прямые
              \item Масштабы на осях одинаковые
          \end{enumerate}
      \end{definition}

      \begin{tikzpicture}
          \draw[<->] (1,2) -- (0,0) -- (3, 0);
          \fill (1.5, 1) circle [radius=.04] node [above right] {P};
          \fill (.5, 0) circle [radius=.04] node [below] {$E_x$};
          \fill (.25, .5) circle [radius=.04] node [above left] {$E_y$};
          \draw [dotted] (.25,.5) -- (3,.5);
          \draw [dotted] (.5,1) node [left] {$y_p$}-- (3,1);
          \draw [dotted] (.75,1.5) -- (3,1.5);
          \draw [dotted] (.5,0) -- (1.5, 2);
          \draw [dotted] (1,0) node [below] {$x_p$} -- (2, 2);
          \draw [dotted] (1.5,0) -- (2.5, 2);
          \draw [dotted] (2,0) -- (3, 2);
      \end{tikzpicture}
      $(x_p, y_p)$ -- координаты точки  $P$.

      Полярная система координат:

      \begin{tikzpicture}
          \draw (3,0) -- (0,0) node [below] {$0$} --++ (30:2cm) node [above right] {$P$};
          \draw (1,.3) node {$\varphi$}; 
      \end{tikzpicture}


      (TODO)
      -------------------------------------------
      \section{Практика 3}

      \subsection{Обратная матрица}

      $\sqsupset A,B$ -- квадратные матрицы $n\times n$ 

      Матричная алгебра:
      \begin{enumerate}
          \item $A+B$
          \item  $\lambda\cdot A$
          \item $A\cdot  B$
      \end{enumerate}
        \begin{definition}
            [Обратная матрица]
      $A^{-1}\qquad A^{-1}A = A A^{-1} = I$
        \end{definition}
      \begin{statement}
          Обратная к $A$ матрица существует, если и только если  $\det A \neq 0$
      \end{statement}
   
        \begin{problem}
            $A = \begin{bmatrix} 5 & 1 & -2\\ 1 & 3 & -1\\ 8 & 4 & -1 \end{bmatrix}\qquad A^{-1} = ? $
        \end{problem}
        \begin{enumerate}
            \item Метод союзной матрицы $A^{-1} = \frac{1}{\det A}\cdot \overset{\sim}A^T$ 
                \begin{definition}
                    $\overset{\sim }A$ союзная матрица -- матрица алгебраических дополнений элементов матрицы $A$
                \end{definition}

                $\det A = (-1)\cdot 7 + 3\cdot 11 + 1\cdot 12 = 38$

                $\overset{\sim }A = \begin{bmatrix} 1 & -7 & -20\\-7 & 11 & -12\\ 5 & 3 & 14 \end{bmatrix} $

                $A^{-1} = \frac{1}{38} \cdot \begin{bmatrix} 1 & -7 & 5 \\ -7 & 11 & 3\\ -20 & -12 & 14 \end{bmatrix} $ 

                $\sphericalangle A\cdot A^{-1} = \frac{1}{38}\begin{bmatrix} 1 & -7 & 5 \\ -7 & 11 & 3\\ -20 & -12 & 14 \end{bmatrix}\begin{bmatrix} 5 & 1 & -2\\ 1 & 3 & -1\\ 8 & 4 & -1 \end{bmatrix} = \begin{bmatrix} 1 & 0 & 0\\ 0 & 1 & 0\\ 0 & 0 & 1 \end{bmatrix} $
            \item Метод Гаусса 
                $[A|I]\sim [I|A^{-1}]$
                
                Элементарные преобразования:
                \begin{enumerate}
                    \item Умножение на число $\lambda\neq 0$ одной строчки
                    \item Перестановка двух любых строк
                    \item Составление линейной комбинации строк
                \end{enumerate}

                $\begin{bmatrix}[ccc|ccc]
                    5 & 1 & -2 & 1 & 0 & 0\\ 1 & 3 & -1 & 0 & 1 & 0\\ 8 & 4 & -1 & 0 & 0 & 1\\
                    \end{bmatrix} \sim \begin{bmatrix} [ccc|ccc] 
                1 & 3 & -1 & 0 & 1 & 0\\
                5 & 1 & -2 & 1 & 0 & 0\\
                8 & 4 & -1 & 0 & 0 & 1\\
            \end{bmatrix} \sim \begin{bmatrix} [ccc|ccc] 
                1 & 3 & -1 & 0 & 1 & 0\\
                0 & -14 & 3 & 1 & -5 & 0\\
                0 & -20 & 7 & 0 & -8 & 1\\
            \end{bmatrix}   \sim \begin{bmatrix} [ccc|ccc] 
                1 & 3 & -1 & 0 & 1 & 0\\
                0 & -140 & 30 & 10 & -50 & 0\\
                0 & -140 & 49 & 0 & -56 & 7
            \end{bmatrix}  \sim \begin{bmatrix} [ccc|ccc] 
                1 & 3 & -1 & 0 & 1 & 0\\
                0 & -14 & 3 & 1 & -5 & 0\\
                0 & 0 & 19 & -10 & -6 & 7\\
            \end{bmatrix} \sim \ldots$
        \end{enumerate}

        Системы уравнений:

        $\begin{cases}
            2x+y+4z = -5\\
            x+3y-6z=2\\
            3x-2y+2z=9\\
        \end{cases}$

         Заметим, что стандартный способ решения таких систем: упрощение, складывание строк, домножение на число строк, их комбинация -- напоминают элементарные преобразования. Т.е. элементарные преобразования -- преобразования, приводящие к эквивалентной системе.

         \begin{enumerate}
             \item Метод Гаусса $\qquad \begin{bmatrix} [c c c|c] 2 & 1 & 4 & -5\\ 1 & 3 & -6 & 2\\ 3 & -2 & 2 & 9\\ \end{bmatrix} \sim \begin{bmatrix} [c c c|c] 1 & 3 & -6 & 2 \\ 2 & 1 & 4 & -5\\ 3 & -2 & 2 & 9 \end{bmatrix} \sim \begin{bmatrix} [c c c|c] 1 & 3 & -6 & 2\\ 0 & -5 & 16 & -9\\ 0 & -11 & 20 & 3\\ \end{bmatrix}\sim \begin{bmatrix} [c c c|c] 1 & 3 & -6 & 2\\ 0 & -55 & 176 & -99\\ 0 & -55 & 100&15\\ \end{bmatrix} \sim \begin{bmatrix} [c c c|c] 1 & 3 & -6 & 2\\ 0 & -5 & 16 & -9\\ 0 & 0 & -76 & 114\\ \end{bmatrix}   $ 

                 $\begin{cases}
                     x+3y-6z=2 & x =2\\
                     -5y+16z=-9 & y=-3\\
                     -76z=114 & z=-\frac{3}{2}\\
                 \end{cases}$ 
             \item Метод Крамера

                 $\Delta = \begin{vmatrix} 2 & 1 & 4\\ 1 &3 & -6\\ 3 & -2 & 2\\  \end{vmatrix}  = 2\cdot (-6) + 1\cdot (-20) + 4\cdot (-11)  = -12 -20-44 = -76$

                 $\Delta_x = \begin{vmatrix} -5 & 1 & 4\\ 2 & 3 & -6 \\ 9 & -2 & 2\\  \end{vmatrix} = (-5)\cdot (-6) - 1 \cdot 58 + 4\cdot (-22) = -152 $ 

                 $x = \frac{\Delta_x}{\Delta} = \frac{-152}{-76} = 2$

                 $\Delta_y = \begin{vmatrix} 2 & -5 & 4\\ 1 & 2 & -6\\ 3 & 9 & 2\\ \end{vmatrix} $
             \item метод обратной матрицы

                 $A = \begin{bmatrix} 2 & 1 & 4\\ 1 & 3 & 6\\ 3 & -2 & 2\\ \end{bmatrix}\quad X = \begin{bmatrix} x\\y\\z \end{bmatrix} \quad B = \begin{bmatrix} -5\\2\\9 \end{bmatrix}  $

                 $AX = B\quad X = A^{-1}B$

                 $A^{-1} = -\frac{1}{76}\begin{bmatrix} -6 & 20 & -11\\ -10 & -8 & 7\\ -18 & 16 & 5  \end{bmatrix}^T = \frac{1}{76}\begin{bmatrix} 6 & 10 & 18 \\ -20 & 8 & -16 \\ 11 & -7 & -5 \end{bmatrix}  $

                 $A^{-1}B = X$
        \end{enumerate}


        \section{Лекция 3}

        $V / \sim = W$ 

        $\vec a + \vec b$

        Напоминание:
        \begin{enumerate}
            \item $\exists -(\vec a)$
            \item $\exists \vec 0$
            \item $\lambda \vec a\quad \lambda (\vec a  + \vec b) = \lambda \vec a + \lambda \vec b$
        \end{enumerate}

        \subsection{Проекция вектора на ось}

\begin{figure}[ht]
    \centering
    \incfig{os}
    \caption{Проекция}
    \label{fig:os}
\end{figure}


\begin{definition}
    Проекцией вектора $\vec a$ на ось  $\gamma$ называется класс эквивалентности $a_l^{\parallel \gamma}$, содержащий вектор, начало которого совпадает с точкой пересечения оси  $l$ и прямой парралельной $\gamma$ и проходящей через начало вектора $\vec a$, а конец -- с точкой пересечения оси $l$ и прямой, параллельной  $\gamma$ и проходящей через конец вектора  $\vec a$
\end{definition}

Свойства проекции:
\begin{enumerate}
    \item $(\vec a + \vec b)_l^{\parallel \gamma} = \vec a_l^{\parallel \gamma} + \vec b_l^{\parallel \gamma}$
    \item  $(\lambda \vec a)_l^{\parallel \gamma} = \lambda \vec a_l^{\parallel \gamma}$
    \item [\underline{$\lim$}:]  $\left( \sum\limits_{i=1}^{m} \lambda_i\vec a_i \right)_l^{\parallel \gamma} = \sum\limits_{i=1}^{m} \lambda_i \vec a_{i_l}^{\parallel \gamma} $

\end{enumerate}

$\sqsupset \vec e$ -- вектор, параллельный $l$. Положим  $\left| \vec e \right| = 1 \implies  $ орт оси $l$

$\sphericalangle \vec a_l^{\parallel \gamma} = \alpha \cdot \vec e\quad \alpha = (\text{Пр}_l^{\parallel \gamma}\vec a) \cdot  \vec e\qquad \alpha$ -- длина проекции $\vec a_l^{\parallel \gamma}$ на ось  $l$.

\begin{lemma}
$\text{Пр}_l^{\parallel \gamma} \left[ \sum_{i=1}^{m} \lambda_i \vec a_i \right] = \sum_{i=1}^{m} \lambda_i \text{Пр}_l^{\parallel \gamma} \vec a_i$
\end{lemma}
\begin{proof}
    $\left( \sum_{i=1}^{m} \lambda_i \vec a_i \right) _l^{\parallel \gamma} = \sum_{i=1}^{m} \lambda_i \vec a_{i_l}^{\parallel \gamma}$

    $\text{Пр}_l^{\parallel \gamma}\left[\sum_{i=1}^{m} \lambda_i \vec a_i\right] = \sum_{i=1}^{m} \lambda _i \text{Пр}_l^{\parallel \gamma} \vec a_i \vec e $
\end{proof}

$\R^1$
\begin{figure}[ht]
    \centering
    \incfig{r1}
    \caption{r1}
    \label{fig:r1}
\end{figure}

$\forall \vec a\qquad \vec a = x_a \vec e\quad x_a$ -- координата вектора $\vec a$ на ось $l$

 $\R^2$

\begin{figure}[ht]
    \centering
    \incfig{r2}
    \caption{r2}
    \label{fig:r2}
\end{figure}

$\vec a = \vec a_x^{\parallel y} + \vec a_y^{\parallel x} = \text{Пр}_y^{\parallel x}\vec a\vec \vec e_1 + \text{Пр}_y^{\parallel x} \vec a\vec e_2 = x_x\vec e_1 + a_y\vec e_2$

\begin{definition}
    Говорят, что в базисе $\{e_1, e_2\}$ вектор $\vec a$ имеет координаты  $\vec a(a_x,a_y)$
\end{definition}

$\R^3$

\begin{figure}[ht]
    \centering
    \incfig{r3}
    \caption{r3}
    \label{fig:r3}
\end{figure}

$\vec a(a_x, a_y, a_z)$

 \begin{note}
     Если угол между осью $l$ и прямой  $\gamma$ прямой ( $l\perp \gamma ) \implies \vec a_l^{\parallel \gamma} = \vec a_l^{\perp} = \text{Пр}_l^{\perp} \vec a\cdot \vec e$ 

     $l \perp \Gamma \implies \vec a+l^{\perp \Gamma} = \vec a_l^{\perp} = -||-$
\end{note}
В декартовой системе координат орты координатных осей обозначаются следующим образом:
\begin{itemize}
    \item [x] $\vec  e_1 = \vec i$ 
    \item [y] $\vec e_2 = \vec j$
    \item [z] $\vec e_3 = \vec k$
\end{itemize}

\begin{note}
    $\vec a(1,2,3) \implies \vec a = 1\cdot \vec i + 2\cdot \vec j + 3\cdot \vec k$
\end{note}

\begin{lemma}
    $\left( \sum\limits_{i=1}^{m} \lambda_i \vec a_i \right) _x = \sum\limits_{i=1}^{m} \lambda _i\vec a_{i_x}$
\end{lemma}

\section{Лекция 4}

\subsection{Скалярное произведение}

$W = V / \sim $

\begin{figure}[ht]
    \centering
    \incfig{scal}
    \caption{скалярное произведение}
    \label{fig:scal}
\end{figure}

\begin{definition}
    $(\vec a\vec b) = \left| \vec a \right| \text{Пр}_{\vec a}^{\perp}\vec b$
\end{definition}

\begin{note}
\begin{figure}[ht]
    \centering
    \incfig{note}
    \caption{note}
    \label{fig:note}
\end{figure}

$(\vec a\vec b) = \left| \vec a \right| \cdot  \left| \vec b \right| \cos gamma $
\end{note}


Алгебраические свойства:
\begin{enumerate}
    \item $\left( \vec a\vec b \right)  = \left( \vec b\vec a \right) $
    \item $\left( \vec a+\vec b, \vec c \right)  = \left( \vec a, \vec c \right) + \left( \vec b, \vec c \right) $
        \begin{proof}
            $(\vec a + \vec b, \vec c) = \left| \vec c  \right| \text{Пр}_{\vec a}^{\perp} \left( \vec a + \vec b \right)  = \left| \vec c \right| \left( \text{Пр}_{\vec c}^{\perp} \vec a \right) $
        \end{proof}
    \item $(\lambda \vec a, \vec b) = \lambda (\vec a, \vec b)$
\end{enumerate}

Геометрические свойства:
\begin{enumerate}
    \item $\vec a \perp \vec b\quad \vec a\neq \vec 0 \neq \vec b \implies  (\vec a, \vec b) = 0
        $
    \item $(\vec a, \vec a) = \left| \vec a \right| ^2$
    \item $\sqsupset \left| \vec a \right|  = 1 \implies  (\vec a, \vec b) = \text{Пр}_{\vec a}^{\perp}\vec b$ 
        \begin{note}
            $\text{Пр}_{\vec i}^{\perp}\vec b = (\vec i, \vec b)$
        \end{note}
\end{enumerate}

Скалярное произведение в координатах:
\begin{itemize}
    \item Декартова Прямоугольная Система Координат.
        \begin{itemize}
            \item [] $\vec a = a_x\vec i + a_y\vec j + a_z\vec k$
            \item [] $\vec b = b_x\vec i + b_y\vec j + b_z\vec k$
        \end{itemize}
        $(\vec a \vec b) = (a_x\vec i + a_y\vec j + a_z\vec k)(b_x\vec i + b_y\vec j + b_z\vec k) = a_xb_x + a_yb_y + a_zb_z$

         $\cos \varphi = \frac{a_xb_x + a_yb_y + a_zb_z}{\sqrt{a_x^2 + a_y^2 + a_z^2}\cdot \sqrt{b_x^2 + b_y^2 + b_z^2}  }$
     \item Произвольная Система Координат. 
         
         $\vec a = a_1\vec e_1 + a_2\vec e_2 + a_3\vec e_3 = \sum_{j=1}^{3} a_j\vec e_j$
         
         $\vec b = b_1\vec e_1 + b_2\vec e_2 + b_3\vec e_3 = \sum_{k=1}^{3} b_k\vec e_k$

         $(\vec a, \vec b) = \left( \sum_{j=1}^{3} a_k\vec e_j \cdot \sum_{k=1}^{3} b_k\vec e_k \right)  = \sum_{j, k=1}^{3} a_jb_k(\vec e_j \vec e_k)$ -- достаточно знать скалярное произведение базисных векторов. $g_{jk} = \vec e_j\vec e_k$ называется метрическим тензором.

         $(\vec a, \vec b) = \sum_{j, k=1}^{3} a_jb_kg_{jk}$

         \begin{note}
             В ДПСК $g_{jk} = \delta{jk} = \begin{cases}
                 0 &, j\neq k\\
                 1 &, j=k
             \end{cases}\qquad \delta $ -- символ Кронекера
         \end{note}

         $g_{jk} = \begin{bmatrix} g_{11} & g_{12} & g_{13}\\ g_{21} & g_{22} & g_{23} \\ g_{31} & g_{32} & g_{33}\\ \end{bmatrix} $ -- матрица Грама

         $a^T\cdot g\cdot b = (\vec a, \vec b)$
\end{itemize}

\subsection{Векторное произведение}

$W = V / \sim $

$\vec a, \vec b \in W$

 $\vec a \times  \vec b = \left[ \vec a \vec b \right]  = \vec c$ -- вектор:
 \begin{enumerate}
     \item $\left| \vec c \right| = \left| \vec a \right| \cdot \left| \vec b \right| \cdot \sin \varphi  = \left| \vec a_{\perp} \right| \left| \vec b \right|  = \left| \vec a \right| \left| \vec b_{\perp} \right| $
     \item $\vec c \perp \vec a\quad \vec c \perp \vec b\quad \{ \vec a, \vec b, \vec c\}$ -- правая тройка (Буравчик, штопор)

\begin{figure}[ht]
    \centering
    \incfig{shtopor}
    \caption{shtopor}
    \label{fig:shtopor}
\end{figure}
 \end{enumerate}

 Алгебраические свойства:
 \begin{enumerate}
     \item $\left[ \vec a, \vec b \right] = -\left[ \vec b, \vec a \right]  $
     \item $\left[ \vec a + \vec b, \vec c \right]  = \left[ \vec a, \vec c \right]  + \left[ \vec b, \vec c \right] $
         \begin{proof}
             $[\vec a ,\vec c] = \left[ \vec a_{\perp}, \vec b \right] $ 

             $\left[ \vec a_{\perp} + \vec b_{\perp}, \vec c \right] = \left[ \vec a_{\perp}, \vec c \right] + \left[ \vec b_{\perp}, \vec c \right]  $

              Если нарисовать картиночку, там будут углы с перпендикулярными сторонами и всё будет хорошо.
         \end{proof}
     \item $\left[ \alpha \vec a, \vec b \right]  = \alpha \left[ \vec a, \vec b \right] $
         \begin{proof}
             $\vec m = \left[ \alpha\vec a, \vec b \right] \quad \vec n = \left[ \vec a, ]vec b \right] $ 

             $\left| \vec m \right| = \left| \alpha \right| \left| \vec a \right| \left| \vec b \right| \sin (\alpha \vec a, \vec b)\qquad \left| n \right|  = \left| \vec a \right| \left| \vec b \right| \sin (\vec a, \vec b)$ 

             Если $\alpha<0$, то у одного другая ориентация тройки, а у другого отрицательный множитель
         \end{proof}
 \end{enumerate}

 Геометрические свойства:
 \begin{enumerate}
     \item $\vec a \parallel \vec b \Longleftrightarrow \left[ \vec a, \vec b \right] = \vec 0 $ 
     \item $\left| \vec a \times \vec b \right| = \left| \vec a \right| \left| \vec b \right| \sin \varphi  = S_{\tikz[scale=0.1] \draw (0,0) -- (1,1) -- (3,1) -- (2,0) -- cycle;}$

\begin{figure}[ht]
    \centering
    \incfig{couple}
    \caption{couple}
    \label{fig:couple}
\end{figure}
 \end{enumerate}

 \begin{note}
     $\vec i \times  \vec i = \vec j \times \vec j = k \times \vec k = \vec 0$
     
     $\vec i \times \vec j = \vec k\quad \vec i \times  \vec k = -\vec j\quad \vec j\times  \vec k = \vec i$

     $[\vec a, \vec b] = (a_x\vec i + a_y\vec j + a_z\vec k) \times (b_x\vec i + b_y\vec j + b_z\vec k) = (a_xb_y - a_yb_x)\vec k + (a_zb_x-a_xb_z)\vec j  + (a_yb_z - a_zb_y)\vec i$

     Можно упростить запоминание: $a_xb_y-a_yb_x = 
     \begin{vmatrix}
         a_x&a_y\\
         b_x&b_y\\
     \end{vmatrix}$ 

     $=
     \begin{vmatrix}
         \vec i & \vec j & \vec k\\
         a_x & a_y & a_x\\
         b_x & b_y & b_z\\
     \end{vmatrix}$

     Хотя формально так делать плохо, потому что в матрице объекты разной природы: скаляры и векторы.
 \end{note}

Общий случай: 

$\vec a = \sum_{m=1}^{3} a_m\vec e_m$

$\vec b = \sum_{n=1}^{3} b_n\vec e_n$

Точно: $e_m \times  e_m = \vec 0 \forall m$

$\vec a \times  \vec b = 
\begin{vmatrix}
    a_1 & a_2\\ b_1 & b_2\\
\end{vmatrix} \vec e_1 \times \vec e_2 - 
\begin{vmatrix}
    a_1 & a_3\\ b_1 & b_3
\end{vmatrix} \vec e_1\times  \vec e_3 + 
\begin{vmatrix}
    a_2 & a_3\\ b_2 & b_3\\
\end{vmatrix} \vec e_2 \times \vec e_3$

$\vec e_1\vec e_2 = \sum\limits_{i=1}^{3} \alpha_ie_i$

\section{Практика 4: Векторная Алгебра}
    \begin{example}
\begin{figure}[ht]
    \centering
    \incfig{vec}
    \caption{vec}
    \label{fig:vec}
\end{figure}
$\frac{\left| AC \right| }{\left| CB \right| } = \frac{n}{m}$ 

$\vec r_C = ?$

$\vec r_C = \frac{n}{m+n} \left( \vec r_B - \vec r_A \right)  + \vec r_A$
    \end{example}

    \begin{itemize}
        \item [п1] Скалярное произведение $(\vec a, \vec b) = \left| \vec a \right| \Pr_{\vec a}^{\perp}\vec b$
    \end{itemize}

    \begin{example}
        $\sqsupset \left| \vec a \right|  = 1\quad \left| \vec b \right|  = 2\qquad (\vec a - \vec b)^2 + (\vec a + 2\vec b)^2=20$

        $\left| \vec  a\right|^2 + 2(\vec a, \vec b) + \left| \vec b \right|^2 + \left| \vec a \right| ^2 + 4(\vec a, \vec b) + 4\left| \vec b \right| ^2 = 20  $

        $2(\vec a, \vec b) = -2 \implies (\vec a, \vec b) = -1\quad \cos  \varphi = -\frac{1}{2}\quad \varphi = \frac{2\pi }{3}$
    \end{example}

    \begin{example}
        $\vec a(1, 2, 1)\quad \vec b\left( 2, 3, 1,  \right) $ 

        $\Pr_{\vec a}(\vec a + \vec b) = \Pr_{\vec a}\vec a + \Pr_{\vec a} \vec b = \left| a \right| + \frac{(\vec a,\vec b)}{\left| \vec a \right| }$
    
    Теперь посчитаем то же самое. но пусть вектора заданы в косоугольном базисе.

    $\vec a = 2\vec m + \vec n\quad \vec b = 3\vec m - 2\vec n\qquad \left| \vec m \right| =2 \quad \left| \vec n \right|  = 3\quad \angle(\vec m, \vec n) = \frac{\pi }{4}$

    Проекция и скалярное произведение не зависят от базиса.

    $\left| \vec a \right|^2  = (2\vec m + 2\vec n)^2 = 4\vec |m|^2 + 4(\vec m, \vec n) + \vec |n|^2$
    \end{example}
    \begin{example}
\begin{figure}[ht]
    \centering
    \incfig{trive}
    \caption{trive}
    \label{fig:trive}
\end{figure}

$\vec{AB} = 2\vec e_1 - 6\vec e_2\quad \vec{AC} = 3\vec e_1 + \vec e_2\quad \left| \vec e_1 \right|  = \left| \vec e_2 \right| =1\quad \angle\left( \vec e_1, \vec e_2 \right)  = \frac{\pi}{2} $ 

$\cos\alpha = \frac{(\vec{AB}, \vec {AC}}{\left| \vec{AB} \right| \cdot \left| \vec{AC} \right| } = 0\quad \alpha = \frac{\pi }{2}$

Аналогично остальные, вектор третеьй стороны можно выразить из двух других, как радиус векторов из $A$
    \end{example}

\begin{example}
    $A(1, 2, 3)\quad B(1, 1, 1)\quad C(1, 0, 1)$ -- три точки параллелограмма

    $O(1, 1, 2)\quad \vect{OB}(0,0,-1)\quad \vect{OC}(0,-1,-1)\quad \cos\varphi = \frac{\left( \vect{OB}, \vec{OC} \right) }{\left| \vect{OB} \right| \left| \vect{OC} \right| } = \frac{1}{1\cdot \sqrt{2} }$
\end{example}

\begin{example}
    $\vec a = \alpha \vec m + \beta \vec n\quad \vec b = \gamma \vec m + \delta \vec n\quad \vec a \not\parallel \vec b\quad \left| \vec n \right|,\left| \vec m \right| , \left| \vec m \vec n \right| $ -- знаем

    $\cos\varphi = \frac{\left| \vec a \right| ^2 - \left| \vec b \right| ^2}{\left| \vec a + \vec b \right| \cdot \left| \vec a - \vec b \right| }$ 

    $\left| \vec a \right| ^2 = (\vec a, \vec a) = \left( \alpha \vec m + \beta \vec n \right) ^2$, для $\vec b$ так же

    Проверка того, что параллелограмм $\frac{\alpha}{\gamma} \neq \frac{\beta}{\delta}$ 


\end{example}
 \begin{example}
\begin{figure}[ht]
    \centering
    \incfig{pr}
    \caption{pr}
    \label{fig:pr}
\end{figure}

$\vec x(1, 2, 2)\quad \vec e_1(1, -1, 1)\quad \vec e_2(-2, 0, 1)$

$\p {\vec x} = \alpha \vec e_1 + \beta \vec e_2$

$\vec x = \p {\vec x} + \vec z$

$\vec x = \alpha \vec e_1 + \beta \vec e_2 + \vec z\quad \begin{cases}
   (\vec x, \vec e_1) = \alpha(\vec e_1, \vec e_1) + \beta (\vec e_1, \vec e_2)\\
   (\vec x, \vec e_2) = \alpha (\vec e_1, \vec e_2) + \beta (\vec e_2, \vec e_2)
\end{cases} $

$\begin{cases}
    3\alpha-\beta =1\\
    -\alpha+5\beta = 0\\
\end{cases}
\quad \begin{cases}
    \alpha = \frac{5}{14}\\
    \beta = \frac{1}{14}\\
\end{cases}$
 \end{example}

 \subsection{Векторное произведение}

 $\left| \vec c \right| = \left| \vec a \right| \cdot \left| \vec b \right|\sin(\varphi)  $

 При векторном произведении можно не учитывать компоненту 

 \begin{example}
     $[\vec a-\vec b, \vec a + \vec b] = [\vec a, \vec a]+[\vec a, \vec b] - [\vec b, \vec a] + [\vec b, \vec b] = 0 + [\vec a, \vec b] + [\vec a, \vec b] + 0  = 2[\vec a, \vec b]$

     Получается площадь образуемого параллелограмма.
 \end{example}

\begin{example}
\begin{figure}[ht]
    \centering
    \incfig{paral}
    \caption{paral}
    \label{fig:paral}
\end{figure}
\end{example}

\section{Лекция 5}
\subsection{Смешанное произведение}

\begin{definition}
    $\sqsupset \vec a, \vec b, \vec c$

    Смешанным произведением векторов $\vec a, \vec b$ и  $\vec c$ называется число  $w$

     \[
         w = (\vec a, [\vec b, \vec c])
     .\] 
    $w$ -- псевдоскаляр
\end{definition}

\begin{note}
    Алгебраические свойства:
    \begin{enumerate}
        \item Общие свойства векторного и скалярного произведений
        \item $\vec a, \vec b, \vec c$ -- компланарны  $\iff (\vec a, [\vec b, \vec c]) = 0$
        \item $\left( \vec a, \left[ \vec b, \vec c \right]  \right)  = \left| \vec a \right| \left| \left[ \vec b, \vec c \right]  \right|\cdot \cos \left( \vec a, [\vec b , \vec c] \right)  = \left| \vec a \right| \left| \vec b \right| \left| \vec c \right|  \sin (\vec b, \vec c)\cos \left( \vec a, \left[ \vec b, \vec c \right]  \right) $

            получается объём следующего параллелограмма (рисунок)
            \begin{note}
                $V_{\text{Пар}}$ -- ориентированный объём 

                $\begin{cases}
                    V>0&,(\vec a, \vec b, \vec c) \text{ -- правая тройка }\\
                    V<0&,(\vec a, \vec b, \vec c) \text{ -- левая тройка }\\
                \end{cases}$
            \end{note}
        \item $\sphericalangle \left( [\vec a, \vec b], \vec c \right) = \left( \vec a, \left[ \vec b, \vec c \right]  \right)  = (\vec a, \vec b, \vec c) = (\vec c, \vec a, \vec b) = (\vec b, \vec c, \vec a $

            $\sphericalangle (\vec b, \vec a, \vec c) = \left( \left[ \vec b, \vec c \right] , \vec c \right) = -\left( \vec a, \vec b, \vec c \right) $
\begin{figure}[ht]
    \centering
    \incfig{mixed}
    \caption{mixed}
    \label{fig:mixed}
\end{figure}
    \end{enumerate}
\end{note}

\subsection{Смешанное произведение а координатах}

\begin{enumerate}
    \item Базис $\vec i, \vec j, \vec k\quad \vec i\vec j\vec k=1$

        $\vec a(a_x, a_y, a_z),\quad \vec b(b_x, b_y, b_z)\quad \vec c(c_x, c_y, c_z)$

        $\vec a \vec b\vec c = (a_x\vec i + a_y\vec j + a_z\vec k)\left( b_x\vec i + b_y\vec j + b_z\vec k \right) \left( c_x\vec i + c_y\vec j + c_z\vec k \right) = (a_x\vec i + a_y\vec j + a_z\vec k) \left( \begin{vmatrix} b_x&b_y\\c_x&c_y \end{vmatrix} \vec k - \begin{vmatrix} b_x&b_z\\c_x&c_z \end{vmatrix}\vec j  + \begin{vmatrix} b_y&b_z\\c_y&c_z \end{vmatrix} \vec i \right)  = a_x \begin{vmatrix} b_y&b_z\\c_y&c_z \end{vmatrix} - a_y \begin{vmatrix} b_x&b_z\\c_x&c_z \end{vmatrix} + a_z \begin{vmatrix} b_x&b_y\\c_x&c_y \end{vmatrix}  = \begin{vmatrix} a_x&a_y&a_z\\b_x&b_y&b_z\\c_x&c_y&c_z \end{vmatrix} (\vec i \vec j \vec k) $ -- число элементарных объёмов, которые помещаются в задаваемый параллелепипед
    \item Произвольный базис. $\vec e_1, \vec e_2, \vec e_3$

        $\vec a(a_1, a_2, a_3)\quad \vec b(b_1, b_2, b_3)\quad \vec c (c_1, c_2, c_3)$

        $(\vec a, \vec b,\vec c) = (a_1\vec e_1 + a_2\vec e_2 + a_3\vec e_3)(b_1\vec e_1 + b_2\vec e_2 + b_3\vec e_3)(c_1\vec e_1 + c_2\vec e_2 + c_3\vec e_3) = \begin{bmatrix} a_1&a_2&a_3\\b_1&b_2&b_3\\c_1&c_2&c_3 \end{bmatrix} \cdot \left( \vec e_1, \vec e_2, \vec e_3 \right)$, где $\vec e_1 \vec e_2 \vec e_3$ -- объём параллелепипеда для меры
\end{enumerate}

\subsection{Двойное векторное произведение}

$\vec a, \vec b, \vec c\quad \vec v = \left[ \vec a, \left[ \vec b, \vec c \right]  \right] $

\begin{figure}[ht]
    \centering
    \incfig{dvec}
    \caption{dvec}
    \label{fig:dvec}
\end{figure}

$\vec v = \alpha \vec b + \beta \vec c$

 \begin{enumerate}
     \item $\vec v = \vec b\left( \vec a \vec c \right)  - \vec c(\vec a\vec b)$
         \begin{proof}
             $\left[ \vec b, \vec c \right]  = (b_yb_z-c_yc_z)\vec i - \left( b_xc_z - c_xb_z \right) \vec j + \left( b_yc_z-c_yb_z \right) \vec k$

             $[\vec a, [\vec b, \vec c] ]= \left(a_y(b_xc_y-c_xb_y) + a_z\left( b_xc_z-c_xb_z \right) \right)\vec i - \left( a_x\left( b_xc_y-c_xb_y \right) - a_z\left( b_yc_z - b_zc_y \right)  \right) \vec j + \left( a_x\left( b_zc_x - c_zb_x \right)  - a_y\left( b_yc_z - b_zc_y \right)  \right) \vec k$ 

             $v_x = b_x\left( a_xc_x+a_yc_y+a_zc_z \right)  - c_x\left( a_xb_x + a_yb_y+a_zc_z \right) $

             $\sphericalangle a_yb_xc_y-a_yc_xb_y + a_zb_xc_z - a_zc_xb_z = b_x(a_yc_y+a_zc_z)-c_x \left( a_yb_y+a_zb_z \right)  + a_xb_xc_x - a_xb_xc_x = \vec v_x$ 

             Дальше то же самое для второй и третьей компонент (Упражнение: дома проделать для одной из оставшихся координат)
         \end{proof}
\end{enumerate}


\begin{theorem}
    [Тождество Бьянки]

    $\left[ \vec a, \left[ \vec b , \vec c \right]  \right] + \left[ \vec b , \left[ \vec c, \vec a \right]  \right] + \left[ \vec c, \left[ \vec a, \vec b \right]  \right] = \vec 0$
\end{theorem}
\begin{proof}
    $\vec b( \vec a\vec c)-\vec c(\vec a\vec b) + \vec c(\vec b\vec a)  - \vec a(\vec b \vec c) - \vec b(\vec c\vec a) = \vec 0$
\end{proof}

$d(f,g) = (df)(g) + (f)(dg)$

 $\left[ \vec a, \left[ \vec b, \vec c \right]  \right]  = \left[ \left[ \vec a, \vec b \right] \vec c \right] +\left[ \vec b, \left[ \vec a, \vec c \right]  \right] $

 Позволяет считать двойное векторное произведение некоторым аналогом дифференцирования.
 \section{Лекция 6}

 \subsection{Аналитическая геометрия}
 \subsubsection{Прямые и плоскости}

 \begin{enumerate}
     \item Прямая на плоскости.

         $P_1, P_2$ -- произвольные различные точки на плоскости.

         Прямая -- геометрическое место точек, равноудалённых от этих двух.

         $\left| PP_1 \right|  = \left| PP_2 \right| $

         Пусть $O$ -- точка отсчёта.  $\vec r_1 = \vect{OP_1},\vec r_2 =  \vect{OP_2}$ -- два радиус вектора

         $\left| \vec r - \vec r_1 \right|^2 = \left| \vec r - \vec r_2 \right|  $

         $\left|\vec  r_2 \right| - 2\left( \vec r\vec r_1 \right) + \left| \vec r_1 \right| ^2 = \left| \vec r \right| -2\left( \vec r\vec r_2 \right) +\left| \vec r_2 \right| ^2$

         $2\left( \vec r,-\vec r_1-\vec r_2 \right) = \left| \vec r_2 \right| ^2-\left| r_1 \right| ^2 = \left( \vec r_2-\vec r_1, \vec r_2+\vec r_1 \right)  $

         $\left( \vec r - \frac{\vec r_2+\vec r_1}{2}, \vec r_2-\vec r_1 \right) =0 \iff \vec r - \underbrace{\frac{\vec r_2+\vec r_1}{2}}\limits_{\vec r_0} \perp \underbrace{\vec r_2-\vec r_1}\limits_{\vec n}$

         $\vec r-\vec r_0\perp \vec n\quad (\vec r-\vec r_0, \vec n) = 0\quad\\ (\vec r, \vec n) = (\vec r_0, \vec n)$ -- нормальное уравнение прямой
\begin{figure}[ht]
    \centering
    \incfig{line}
    \caption{line}
    \label{fig:line}
\end{figure}
 \[
     \left( \vec r\vec n \right) = \left( \vec r_0 \vec n \right)  
.\] 
Способы задания:
\begin{enumerate}
    \item В ДПСК $\vec r(x,y), \vec n(a,b),\vec r_0(x_0,y_0)$

         \[
             ax+by = ax_0+by_0 = -c \qquad a(x-x_0)+b(y-y_0) = 0
        .\] 
        \[
        ax+by+c=0
        .\]
        Последнее -- общее уравнение прямой.


    \item $\vec r-\vec r_0 \parallel \vec s \implies \exists t: \vec r-\vec r_0 = \vec s t$

        $\vec r = \vec r_0 + \vec st\quad\vec s$ -- направляющий вектор, любой ненулевой, смотрящий вдоль прямой
    \item $\vec z(x,y)\quad \vec r_0(x_0,y_0)\quad \vec s(\alpha, \beta)$

        $\begin{cases}
            x = x_0+\alpha t\\
            y = y_0+\beta t\\
        \end{cases}$ -- параметрическое уравнение прямой.
    \item $\frac{x-x_0}{\alpha} = \frac{y-y_0}{\beta}$ -- каноническое уравнение прямой.
\end{enumerate}
\begin{example}
    $l:\quad2x+3y-1=0$ Рассмотрим  $A(1,1)$, которая не лежит на прямой

    $\p l:\quad2(x-1) + 3(y-1) = 0$ -- уравнение прямой, проходящей через  $A$ параллельно  $\ell$

    $\pp l:\quad \frac{x-1}{2} = \frac{y-1}{3}$ -- уравнение прямой, проходящей через $A$ перпендикулярно  $\ell $ (её направляющая -- нормаль изначальной)
\end{example}
\begin{example}
    $\frac{x-5}{2} = \frac{y-1}{4}\quad A(1,1)$ 

    $\p l:\quad \frac{x-1}{2} = \frac{y-1}{4}$

    $\pp l: 2(x-1) + 4(y-1) = 0$
\end{example}
\item Уравнение прямой с произвольным параметром

    $ax+by+c=0\quad | \cdot \frac{1}{\sqrt{a^2+b^2} }$ 

    $Ax+By+C=0$

\begin{figure}[ht]
    \centering
    \incfig{radvec}
    \caption{radvec}
    \label{fig:radvec}
\end{figure}

$C = \frac{c}{\sqrt{a^2+b^2} } = -(\vec r_0, \vec n^0)$

$C$ -- прицельный параметр -- (по модулю) расстояние от прямой до начала координат.
\item Уравнение прямой в отрезках.

     \[
    ax+by=-c
    .\] 

    $\frac{ax}{-c} + \frac{by}{-c} = 1$ 

    $\frac{x}{p} + \frac{y}{q} = 1$ 

\begin{figure}[ht]
    \centering
    \incfig{отрезок}
    \caption{отрезок}
    \label{fig:отрезок}
\end{figure}
\item Уравнение с угловыми коэффициентами.

    $\frac{a}{b}x + y = -\frac{c}{b}$ 

    $y = kx+l$

\begin{figure}[ht]
    \centering
    \incfig{coeff}
    \caption{coeff}
    \label{fig:coeff}
\end{figure}
$\tg \varphi = k$
\item Уравнение прямой через две точки

     \[
     \frac{x-x_1}{x_2-x_1} = \frac{y-y_1}{y_2-y_1}
     .\] 

     $\vec r = \vec r_1 + (\vec r_2-\vec r_1)t$
\end{enumerate}

Взаимное расположение прямых:
\begin{enumerate}
    \item Найти угол между прямыми:

        $l_1:\quad (\vec r\vec n_1) = D_1$

        $l_2:\quad (\vec r, \vec n_2) = D_2$

        $\angle \left( l_1, l_2 \right)  = \angle\left( \vec n_1, \vec n_2 \right)  = \varphi$

        $\cos\varphi = \frac{(\vec n_1, \vec n_2)}{\left| \vec n_1 \right| \left| \vec n_2 \right| }$

    \item Точка пересечения
        $\left( \vec r_1, \vec n_1 \right)  = D_1$
        
        $\left( \vec r_2, \vec n_2 \right)  = D_2$

        $\vec n_1(a_1, b_1)\quad \vec n_2\left( a_2, b_2 \right) \quad \frac{a_1}{a_2}\neq \frac{b_1}{b_2}$ 

        $\vec r_0$ -- радиус вектор токи пересечения прямых

        $a_1x+b_1y=D_1\quad a_2x+b_2y = D_2$

        $\triangle = a_1b_2-a_2b_1\neq 0$

        $\triangle_x = D_1b_2-b_1D_2$

        $\triangle_y = a_1D_2-a_2D_1$

        $x = \frac{D_1b_2-b_1D_2}{a_1b_2-a_2b_1}$ 

        $y = \frac{a_1D_2-a_2D_1}{a_1b_2-a_2b_1}$ 

        Подойдём с другой стороны:

        $l_1: \vec r = \vec r_1 + \vec s_1 t$

        $l_2: \vec r = \vec r_2 + \vec s_2 t$

        $\vec s_1 \not\parallel s_2\quad \frac{\alpha_1}{\alpha_2}\neq \frac{\beta_1}{\beta_2}$ 

        $\vec r_1 + \vec s_1t = \vec r_2+\vec s_2 t$

        $\vec s_1t-\vec s_2t = \vec r_2-\vec r_1$

        $\alpha_1t_1-\alpha_2t_1 = x_2-x_1$

        $\beta_1t_1-\beta_2t_1 = y_2-y_1$

        $\triangle = \alpha_1\beta_2-\alpha_2\beta_1\neq 0$
    \item Есть прямая $\ell $ и точка $P\not\in \ell $

        $(\vec r, \vec n)=D = \left( \vec r_0, \vec n \right) \quad \ell, P_0 $

        $\vec r_p\quad P$

        Сначала найдём ортогональную проекцию точки  $P$ -- точку  $Q$

        $\vec r_Q = \vec r_P+\vect{PQ} = \vec r_P-\alpha\vec n$

        $\left| QP \right| = (\vect{P_0P}, \vec n_0) = \frac{ \left(\vect{P_0P}, \vec n \right)}{\left| \vec n \right| } $ 

        $\vect{QP} = \left| QP \right| \vec n = \frac{\left( \vect{P_0P}, \vec n \right) }{\left| \vec n \right| ^2}\vec n = \frac{\left( \vec r_p-\vec r_0, \vec n \right) }{\left| \vec n \right| ^2}\vec n$ 
        $\vec r_q = \vec r_p-\vect{QP} = \vec r_p - \frac{\left( \vec r_p, \vec n \right) -D}{\left| \vec n \right| ^2}\vec n$ 

        $II: \quad \left( \vec r_q, \vec n \right)  = \left( \vec r_p, \vec n \right) - \alpha \left| \vec n \right| ^2$

        $D = \left( \vec r_p, \vec n \right) -\alpha \left| \vec n \right| ^2 \implies \alpha = \frac{\left( \vec r_0, \vec n \right) -D}{\left| \vec n \right| }$

        Расстояние от $P$ до  $\ell \quad \rho(P, \ell ) = \left| \vect{QP} \right| = \frac{\left( \vec r_p, \vec n \right) -D}{\left| \vec n \right| }$

        В координатах: $\rho(P, \ell ) = \frac{ax_p+by_p-D}{\sqrt{a^2+b^2} }$
\end{enumerate}

Плоскость -- множество точек пространство, равноудалённых от фиксированных точек.

$\left| P_1P \right|  = \left| P_2P \right| $

$\left( \vec r_0- \frac{\vec r_1+\vec r_2}{2}, \vec r_1-\vec r_2 \right) = 0 $ 

$\left( \vec r , \vec n \right) = \left( \vec r_0, \vec n \right) =D $ -- нормальное векторное уравнение плоскости

ДПСК: $\vec n(A,B,C)$

$\left( \vec r-\vec r_0, \vec n \right) = 0\quad A(x-x_0) + B(y-y_0) + C(z-z_0) = 0 $

$\left( \vec r, \vec n \right)  = D\quad Ax+By+Cz=D$ -- общее уравнение плоскости

Параметрическое уравнение плоскости

$\vec r = \vec r_0 + \vect{P_0P} = \vec r_0 + \alpha\vec m + \beta \vec q$ -- векторное параметрическое уравнение плоскости

ДПСК: $\begin{cases}
    x = x_0+\alpha m_1+\beta q_1\\
    y = y_0+\alpha m_2 + \beta q_2\\
    z = z_0+\alpha m_2 + \beta q_3\\
\end{cases}\quad \vec m(m_1, m_2, m_3)\quad \vec q(q_1, q_2, q_3)$

Плоскость можно задать через три точки. $P_1, P_2, P_3$

Если нам нужна нормаль, образуем два вектора $\vec r_2-\vec r_1\quad \vec r_3 - \vec r_1$

$\vec n = \left[ \vec r_2-\vec r_1, \vec r_3-\vec r_1 \right]\quad \vec r_0 = \vec r_1 $ 

$(\vec r, \vec n) = (\vec r_1, \vec n)$

$\left( \vec r-\vec r_1, \vec n \right) $

$\left( \vec r-\vec r_1, \vec r_2-\vec r_1, \vec r_3-\vec r_1 \right) = 0$ 

ДПСК:  $
\begin{vmatrix}
    x-x_1&y-y_1&z-z_1\\
    x_2-x_1&y_2-y_1&z_1-z_1\\
    x_3-x_1&y_3-y_1&z_3-z_1\\
\end{vmatrix} = 0$

Для паремтрического уравнения: $\vec r = \vec r_0 + \alpha \left( \vec r_2-\vec r_1 \right)  + \beta \left( \vec r_3-\vec r_1 \right) $ 

Уравнение плоскости с прицельным параметром. $Ax+By+Cz+D = 0\quad \frac{1}{\sqrt{A^2+B^2+C^2} }$ 

$\frac{A}{\sqrt{A^2+B^2+C^2} }x + \frac{B}{\sqrt{A^2+B^2+C^2} }y + \frac{C}{\sqrt{A^2+B^2+C^2} }z + \frac{D}{\sqrt{A^2+B^2+C^2} } = 0\quad \frac{D}{\sqrt{A^2+B^2+C^2} }$ -- прицельный параметр

$D = \left( \vec r_0, \vec n \right)\quad $ Прицельный параметр: $ \left( \vec r_0, \vec n \right) / \left| \vec n \right| $

Взаимное расположение плоскостей:
$(\vec r, \vec n_1) = D_1\quad (\vec r, \vec n_2)=D_2$ -- две плоскости

\begin{figure}[ht]
    \centering
    \incfig{плоскости}
    \caption{плоскости}
    \label{fig:плоскости}
\end{figure}
\begin{enumerate}
    \item Угол

        $\angle \left( \mathscr L_1, \mathscr L_2 \right)  = \angle\left( \vec n_1, \vec n_2 \right) $
    \item Пересечение

        $\left[ \vec n_1, \vec n_2 \right] \neq 0$
    \item Параллельность

        $\vec n_1 \parallel \vec n_2\implies \vec n_2 = \lambda \vec n_1$

        $\left( \vec r_2-\vec r_1, \vec n \right) =0\quad \left( \vec r_2-\vec r_1, \vec n_1 \right)  = \left( \vec r_2, \vec n_1 \right) -D = \frac{D_2}{\lambda} - D_1 \left( = \frac{\left| \vec n_1 \right| }{\left| \vec n_2 \right| }D_2-D_1\neq 0  \right) $ 

        $\left( \vec r, \lambda \vec n_1 \right) =D_2\quad D_1 = \frac{D_2}{\lambda}\quad \lambda = \frac{D_2}{D_1} = \frac{\left| \vec n_2 \right| }{\left| \vec n_1 \right| }$
    \item Совпадение
        $\frac{D_2}{\left| \vec n_2 \right| } = \frac{D_1}{\left| \vec n_1 \right| }$
\end{enumerate}

\begin{example}
    Даны плоскость и точка, не лежащая на плоскости. найдём точку проекции.

\begin{figure}[ht]
    \centering
    \incfig{проекция}
    \caption{проекция}
    \label{fig:проекция}
\end{figure}

$\mathscr L\quad \left( \vec r, \vec n \right) =D$

$P\quad \vec r_1$

$\vec r_Q -?$

$\vec r_Q = \vec r_1 + \vect{PQ} = \vec r_1 + \frac{\left( \vec r_0-\vec r_1, \vec n \right) }{\left| \vec n \right| ^2}\vec n = \vec r_1 + \frac{D - \left( \vec r_1, \vec n \right) }{\left| \vec n \right| ^2}\vec n$ 

$\rho\left( P, \mathscr L \right)  = \left| \vect{PQ} \right|  = \left| \frac{D-\left( \vec r, \vec n \right) }{|\vec n|^2} \right| $ 


\end{example}

Прямая в пространстве.

Три точки задают прямую, равноудалённую от них всех

\begin{figure}[ht]
    \centering
    \incfig{прямая-через-3-точки}
    \caption{прямая-через-3-точки}
    \label{fig:прямая через 3 точки}
\end{figure}

$\left| PP_1 \right| = \left| PP_2 \right|  = \left| PP_3 \right|  $

(вывод будет позже)

$\vec s$ -- нормаль к плоскости, образованной тремя точками.

$\vec r = \vec r_0 + \vect{P_0P} = \vec r_0 + \vec st$ -- векторное параметрическое уравнение прямой

ДПСК:
$\begin{cases}
    x = x_0+\alpha t\\
    y = y_0+\beta t\\
    z = z_0+\gamma t\\
\end{cases}\quad \vec s\left( \alpha, \beta, \gamma \right) \quad \vec r_0\left( x_0, y_0, z_0 \right) \quad \vec r\left( x, y, z \right) $ 

Векторное уравнение: $\vec r = \vec r_0 + \vec st\quad \times \vec s$

$\left[ \vec r, \vec s \right] = \left[ \vec r_0, \vec s \right]  = \vec b$

$\frac{x-x_0}{\alpha} = \frac{y-y_0}{\beta} = \frac{z-z_0}{\gamma}$ -- каноническое уравнение прямой (выразили $t$ и приравняли)

Через две точки $\frac{x-x_0}{x_1-x_0} = \frac{y-y_0}{y_1-y_0} = \frac{z-z_0}{z_1-z_0}$ 

Через две пересекающиеся плоскости. $\begin{cases}
    \left( \vec r, \vec n_1 \right) =D_1\\
    \left( \vec r, \vec n_2 \right) =D_2\\
\end{cases}\quad [\vec n_1, \vec n_2]\neq 0$

$\vec s = \left[ \vec n_1, \vec n_2 \right] $ 

Прямая(взаимное расположение)

$\ell_1\quad [\vec r, \vec s_1] = \vec b_1$

$\ell_2\quad [\vec r, \vec s_2] = \vec b_2$

\begin{enumerate}
    \item $\ell_1 \parallel \ell_2  $

        $\left[ \vec s_1, \vec s_2 \right] = 0 \implies \vec s_2 = \lambda \vec s_1$

        $\left[ \vec r_1-\vec r_2, \vec s \right] \neq 0$ 
    \item $\ell_1 = \ell_2  $

        $[\vec s_1, \vec s_2] = 0$

        $\left[ \vec r_1-\vec r_2, \vec s \right] = 0$

    \item [] Плоскость через $\ell_1 \parallel \ell_2 $

        $\vec n = \left[ \vec r_1-\vec r_2, \vec s\right] $

        $\left( \vec r, \vec n \right)  = \left( \vec r_0, \vec n \right) $

        $\left( \vec r-\vec r_0, \vec r_1-\vec r_2, \vec s \right) = 0$ -- уравнение плоскости
    \item Скрещивание $\ell_1 \times \ell_2 $

        $(\vec n = ) \left[ \vec s_1, \vec s_2 \right] \neq 0$ 

        $\left( \vec r_2-\vec r_1, \vec n \right) \neq 0\quad \left( \vec r_2-\vec r_1, \vec s_1, \vec s_2 \right) \neq 0$
\end{enumerate}

\begin{example}
    Есть прямая, есть точка. Найти отогональную проекцию этой точки на прямую.

    \begin{tikzpicture}
        \draw (0,0) node {$\ell$} -- (2, 1);
        \draw (1,.5) node {$Q$} -- ++(-.5,-1) node {$P$};
    \end{tikzpicture}

    $[\vec r, \vec s] = \vec b$

    $\vec r_p\quad P$

     $\vec r_Q -?$

     $\vec r_Q = \vec r_p + \vect{PQ}$

     $(\vec r, \vec s) = (\vec r_p, \vec s)$ -- плоскость

     $\left[ \vec s, \left[ \vec r, \vec s \right]  \right]  = \vec r(\vec s \vec s) - \vec s(\vec s \vec r) = \left[ \vec b, \vec s \right]  $ 

     $\vec r_Q\left| \vec s \right| ^2 - \vec s\left( \vec r_p\vec s \right) - \left[ \vec b, \vec s \right] $
\end{example}
\section{Практика }

\begin{example}
    Даны две прямые, заданные параметрическими уравнениями:

    $l_1:\quad \begin{cases}
        x = 1+2t\\y=1-t
    \end{cases}$ 

    $l_2:\quad \begin{cases}
        x = 2-t\\
        y = 2+t
    \end{cases}$ 

    \begin{enumerate}
        \item Параллельны ли они?

            Набавляющие вектора $\vec s_1(2, -1)\quad \vec s_2 (-1, 1) \implies \vec s_1 \not\parallel s_2 \implies $ пересекаеются
        \item Найти точку пересечения

            $1+2t=2-t \implies t = \frac{1}{3} \implies x = \frac{5}{3}$

            $1-t = 2+t \implies t = -\frac{1}{2} \implies y = \frac{3}{2}$ 

            $l_1:\quad \frac{x-1}{2} = \frac{y-1}{-1}\quad \begin{cases}
                x+2y = +3\\
                x+y = 4\\
            \end{cases}$

            $l_2:\quad \frac{x-2}{-1} = \frac{y-2}{1}$

            $y = -1, x = 5$ -- точка пересечение. верхний способ не работает, там разные  $t$
    \end{enumerate}
\end{example}

\begin{example}
    $l_1:\quad 2x+y+2 = 0\\ l_2:4x+y+1=0\\ $

    $A(1, 2)\quad AB=AC\quad l=?$, чтобы проходила через  $BC$

     Можно найти точку пересечения прямых  $O = l_1\cap l_2$

     $\begin{cases}
         2x+y = -2\\
         4x+y = -1
     \end{cases} \implies  2x=1\quad x = \frac{1}{2}\quad y = -3\quad O(\frac{1}{2}, -3)\quad OA(\frac{1}{2}, 5) \implies \p O(\frac{3}{2}, 7)$ 

     $2(x-\frac{3}{2}) + (y-7) = 0$ 

     $2x+y -10 = 0$

    $C = \p l\cap l_2\quad \begin{cases}
        2x+y=10\\ 4x+y = -1
    \end{cases}\quad \begin{cases}
        x = -\frac{11}{2}\\
        y = 21\\
    \end{cases} \implies l: \frac{x-1}{-\frac{11}{2}-1} = \frac{y-2}{21-2} $ -- ответ
\end{example}

\begin{example}
    $A(1, 3)\quad B(2, 5)$ -- две вершины треугольника.  $H(1,4)$ -- точка пересечения высот этого треугольника. Найти треться точку треугольника $C$

    $\vect{AB} = (1, 2) \implies l = 1(x-1) +2(y-4) = 0\quad x+2y=3$

    $\vect{AH}(0,1) \implies \p l: y=5$ 

    $C = l\cap \p l \implies C(-1, 5)$
\end{example}

Плоскости:

$(\vec r\vec n) = (\vec r_0, \vec n) = \alpha$

$Ax+By+Cz+\alpha = 0\quad \vec r = \vec r_0 + ]alpha \vec q + \beta \vec m$

\begin{example}
    $A(1, 1, 1)\quad B(2, 3, -1)\quad \vec a(0, -1, 2)$

    $\vect{AB}(1, 2, -2)$

    $\vec r = \vec r_a + \alpha \vec a + \beta \vect{AB}$
\end{example}

\begin{example}
    $A(1, 1, -1)$

     $\alpha_1: 2x-y+5z+3 = 0\\ \alpha_2: x+3y-z-7=0$

      $\alpha: \alpha \perp \alpha 1, \alpha\perp \alpha_2\quad A\in \alpha\quad \alpha ?$

      $\vec n_1(2, -1, 5)\quad \vec n_2(1, 3, -1) \implies \vec n_1 \not\parallel \vec n_2 \implies  \vec r = \vec r_a + \alpha \vec n_1 + \beta \vec n_2$
\end{example}

\begin{example}
    $\alpha: 2x+3y+6z-12=0$

    Вычислить объём фигуры, ограниченной тремя координатными и заданной плоскостями. 

     $\frac{x}{6} + \frac{y}{4} + \frac{z}{2} = 1 \implies $ Знаем пересечения плоскости с координатными плоскостями.

     $V = 8$
\end{example}

\begin{example}
    $L = \begin{cases}
        x+2y-3z-5=0\\
        2x-y+z+2=0
    \end{cases}$ -- уравнение прямой через пересечение непараллельных плоскостей.

    Найти каноническое уравнение прямой.

    $\vec n_1(1, 2, -3)\quad \vec n_2(2, -1, 1)$

    $\vec n = \vec n_1 \times \vec n_2$

    $\vec n =
    \begin{vmatrix}
        \vec i&\vec j &\vec k\\ 1 & 2 & -3\\ 2 & -1 & 1\\
    \end{vmatrix} = -\vec i - u\vec j - 5\vec k$

    $\sqsupset z = 0\quad \begin{cases}
        x+2y=5\\ 2+-y=-2
    \end{cases}\quad -5y = -12\quad \begin{cases}
        y = \frac{12}{5}\\ x = \frac{1}{5}
    \end{cases}$ 

    $\frac{x-\frac{1}{5}}{-1} = \frac{y - \frac{12}{5}}{-7} = \frac{z-0}{-5}$
\end{example}

\begin{example}
    $L:\quad \frac{x}{1} = \frac{y-3}{-2} = \frac{z-3}{-1} ( = t)\quad \alpha: x++2y+3z+3 = 0$ 

    $\sqsupset P = L\cap \alpha\quad P -?$

    $\vec s(1, -2, -1)\quad \vec n (1, 2, 3)$

     $\begin{cases}
         x = t\\
         y = 3-2t\\
         z = 3-t\\
     \end{cases}$ -- вставляем в урванение для плоскости

     $t + 2(3-2t)+3(3-t)+3 = 0$

     $-6t = -18 \implies t = 3 \implies \begin{cases}
         x=3\\
         y = -3\\
         z = 0\\
     \end{cases}$


\end{example}

\begin{example}
    $L:\quad \frac{x-2}{6} = \frac{y+1}{-5} = \frac{z-5}{4}$ 

    $\alpha: x-3y + 2z-7 = 0$

    Найти проекцию прямой  $L$ на плоскость  $\alpha$

     $\vec n \times \vec s$
\end{example}



\end{document}

